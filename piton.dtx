% \iffalse -*- coding: utf-8 ; -*- \fi 
% \iffalse meta-comment
%
% Copyright (C) 2022-2025 by F. Pantigny
% -----------------------------------
%
% This file may be distributed and/or modified under the
% conditions of the LaTeX Project Public License, either version 1.3
% of this license or (at your option) any later version.
% The latest version of this license is in:
%
% http://www.latex-project.org/lppl.txt
%
% and version 1.3 or later is part of all distributions of LaTeX
% version 2005/12/01 or later.
%
% \fi
% \iffalse
% 
%<*batchfile> 
\begingroup
\input l3docstrip.tex
\endgroup
%</batchfile>
%
%<@@=piton>
%<*driver>
\documentclass{l3doc}
\VerbatimFootnotes
\usepackage{geometry}
\geometry{left=2.8cm,right=2.8cm,top=2.5cm,bottom=2.5cm,papersize={21cm,29.7cm}}
\usepackage{fontspec}
\usepackage[dvipsnames,svgnames]{xcolor}
\usepackage{caption,tabularx,luacolor,lua-ul,upquote,varwidth}
\usepackage[framemethod=tikz]{mdframed}
\usepackage{tcolorbox}
\tcbuselibrary{skins,breakable}
\tcbset{enhanced}
\def\emphase{\bgroup\color{RoyalPurple}\let\next=}
\fvset{commandchars=\~\#\@,formatcom=\color{gray}}
\captionsetup{labelfont = bf}
\usepackage{ragged2e}
\usepackage[footnotehyper]{piton} 
\PitonOptions
  { 
    splittable = 4 ,
    math-comments ,
    begin-escape = ! ,
    end-escape = ! ,
    begin-escape-math = \( , 
    end-escape-math = \) ,
    detected-commands = { highLight, footnote } ,
    vertical-detected-commands = newpage ,
    raw-detected-commands = NameTable
  }

\parindent 0pt
\skip\footins = 2\bigskipamount

\def\CC{{C\nolinebreak[4]\hspace{-.05em}\raisebox{.4ex}{\tiny\bfseries ++}}}

\usepackage{makeidx}
\makeindex

\NewDocumentCommand{\indexcommand}{m}{\index{#1@\texttt{\textbackslash #1}}}

\NewDocumentCommand{\indexenv}{m}{\index{#1@\texttt{\{#1\}}}}

\NewDocumentCommand{\Definition}{m}
  {{\setlength{\fboxsep}{1pt}\colorbox{gray!20}{\ttfamily \vphantom{gl}#1}}}

\NewDocumentCommand{\DefinitionCommand}{m}
  {{\setlength{\fboxsep}{1pt}\colorbox{gray!20}{\ttfamily
        \vphantom{gl}\textbackslash #1}}}



  
\PitonOptions{gobble=2} 

\EnableCrossrefs

\begin{document}
\DocInput{piton.dtx}
\end{document}
%</driver>
% \fi
% \iffalse
%<*STY>
% \fi
\def\PitonFileVersion{4.7x}
\def\PitonFileDate{2025/07/10}
% \iffalse
%</STY>
%<*LUA>
piton_version = "4.7x" -- 2025/07/10
%</LUA>
%\fi
%
% \catcode`\" = 11 
% 
% \title{The package \pkg{piton}\thanks{This document corresponds to the
% version~\PitonFileVersion\space of \pkg{piton}, at the date of~\PitonFileDate.}} 
% \author{F. Pantigny \\ \texttt{fpantigny@wanadoo.fr}}
%
% \maketitle
%
% \begin{abstract}
% The package \pkg{piton} provides tools to typeset computer listings, with
% syntactic highlighting, by using the Lua library LPEG. It requires LuaLaTeX.
% \end{abstract}
% 
%
% \section{Presentation}
% 
%
% The package \pkg{piton} uses the Lua library LPEG\footnote{LPEG is a
% pattern-matching library for Lua, written in C, based on \emph{parsing
% expression grammars}:
% \url{http://www.inf.puc-rio.br/~roberto/lpeg/}} for parsing informatic
% listings and typesets them with syntactic highlighting. Since it uses the Lua
% of LuaLaTeX, it works with |lualatex| only (and won't work with the other
% engines: |latex|, |pdflatex| and |xelatex|). It does not use external program
% and the compilation does not require |--shell-escape|. The compilation is very
% fast since all the parsing is done by the library LPEG, written in C. 
% 
% \bigskip
% Here is an example of code typeset by \pkg{piton}, with the environment |{Piton}|.
%
% \bigskip
%
%    \begin{Piton}
% from math import pi
%
% def arctan(x,n=10):
%    """Compute the mathematical value of arctan(x)
%
%    n is the number of terms in the sum
%    """
%     if x < 0:
%         return -arctan(-x) # recursive call
%     elif x > 1: 
%         return pi/2 - arctan(1/x) 
%         #> (we have used that $\arctan(x)+\arctan(1/x)=\frac{\pi}{2}$ for $x>0$)\footnote{This LaTeX escape has been done by beginning the comment by \ttfamily\#>.}
%     else: 
%         s = 0
%         for k in range(n):
%             s += (-1)**k/(2*k+1)*x**(2*k+1)
%         return s 
%    \end{Piton}
%
%
% \medskip
% The main alternatives to the package \pkg{piton} are probably the packages
% \pkg{listings} and \pkg{minted}.
% 
% \medskip
% The name of this extension (\pkg{piton}) has been chosen arbitrarily by
% reference to the pitons used by the climbers in alpinism.
% 
% 
% \section{Installation}
% 
% The package \pkg{piton} is contained in two files: |piton.sty| and |piton.lua|
% (the LaTeX file |piton.sty| loaded by |\usepackage| will load the Lua file
% |piton.lua|). Both files must be in a repertory where LaTeX will be able to
% find them, for instance in a |texmf| tree. However, the best is to install
% \pkg{piton} with a TeX distribution such as MiKTeX, TeX Live or MacTeX.
% 
% \section{Use of the package}
%
% The package \pkg{piton} must be used with \textbf{LuaLaTeX exclusively}: if
% another LaTeX engine (|latex|, |pdflatex|, |xelatex|,\dots ) is used, a fatal
% error will be raised.
%
% \subsection{Loading the package}
% 
% The package \pkg{piton} should be loaded by: |\usepackage{piton}|.
%
% \smallskip
% The package \pkg{piton} uses and \emph{loads} the package \pkg{xcolor}. It
% does not use any exterior program.
%
% \subsection{Choice of the computer language}
%
% The package \pkg{piton} supports two kinds of languages:
% \begin{itemize}
% \item the languages natively supported by \pkg{piton}, which are Python, OCaml,
% C~(in fact \CC), SQL and two special languages called |minimal| and |verbatim|;
% \item the languages defined by the final user by using the built-in command
% |\NewPitonLanguage| described p.~\pageref{NewPitonLanguage} (the parsers of
% those languages can't be as precise as those of the languages supported
% natively by \pkg{piton}).
% \end{itemize}
% 
%
% 
% \smallskip
% By default, the language used is Python.
%
% \smallskip
% \index{language (key)}
% It's possible to change the current language with the command |\PitonOptions|
% and its key \Definition{language}: |\PitonOptions{language = OCaml}|.
%
% \smallskip
% In fact, for \pkg{piton}, the names of the computer languages are always
% \textbf{case-insensitive}. In this example, we might have written |Ocaml| or
% |ocaml|. 
%
% \smallskip
% For the developers, let's say that the name of the current language is stored
% (in lower case) in the L3 public variable |\l_piton_language_str|.
%
% \smallskip
% In what follows, we will speak of Python, but the features described also
% apply to the other languages.
% 
% \subsection{The tools provided to the user}
%
% \indexenv{Piton}
% 
% The package \pkg{piton} provides several tools to typeset informatic codes: the
% command |\piton|, the environment |{Piton}| and the command |\PitonInputFile|.
%
% \begin{itemize}
% \item The command \DefinitionCommand{piton} should be used to typeset small
% pieces of code inside a paragraph. For example:
% 
% {\color{gray}\verb|\piton{def square(x): return x*x}|}\qquad 
% \piton{def square(x): return x*x}
%
% The syntax and particularities of the command |\piton| are detailed below.
%
% \item The environment \Definition{\{Piton\}} should be used to
% typeset multi-lines code. Since it takes its argument in a verbatim mode, it
% can't be used within the argument of a LaTeX command. For sake of
% customization, it's possible to define new environments similar to the
% environment |{Piton}| with the command |\NewPitonEnvironment|:
% cf.~\ref{NewPitonEnvironment} p.~\pageref{NewPitonEnvironment}.
%
% \item The command \DefinitionCommand{PitonInputFile} is used to insert and
% typeset an external file: cf.~\ref{PitonInputFile}
% p.~\pageref{PitonInputFile}.
%
%
% \end{itemize}
%
% \subsection{The double syntax of the command \textbackslash piton}
% 
% \indexcommand{piton}
% 
% In fact, the command |\piton| is provided with a double syntax. It may be used
% as a standard command of LaTeX taking its argument between curly braces
% (|\piton{...}|) but it may also be used with a syntax similar to the syntax of
% the LaTeX command |\verb|, that is to say with the argument delimited by two identical
% characters (e.g.: \verb!\piton|...|! or \verb!\piton+...+!). 
% 
% \begin{itemize}
% \item {\color{blue} \textsf{Syntax} \verb|\piton{...}|}
%
% \nobreak
% When its argument is given between curly braces, the command |\piton| does not
% take its argument in verbatim mode. In particular:
% \begin{itemize}
% \item several consecutive spaces will be replaced by only one space (and also
% the character of end of line),
% 
% {\color{cyan} but the command |\|␣ is provided to force the insertion of a space};
%
% \item it's not possible to use |%| inside the argument,
%
% {\color{cyan} but the command |\%| is provided to insert a |%|};
%
% \item the braces must be appear by pairs correctly nested
%
% {\color{cyan} but the commands |\{| and |\}| are provided for individual braces};
%
% \item the LaTeX commands\footnote{That concerns the commands beginning with a
% backslash but also the active characters (with catcode equal to 13). } of the
% argument are fully expanded and not executed,
%
% {\color{cyan} so, it's possible to use |\\| to insert a backslash}.
% \end{itemize}
%
%
% The other characters (including |#|, |^|, |_|, |&|, |$| and |@|)
% must be inserted without backslash.
% 
% \bigskip
%\begin{tabular}{>{\color{gray}}w{l}{75mm}@{\hspace*{1cm}}l}
% \omit Examples : \hfil \\
% \noalign{\vskip1mm}
% \verb|\piton{MyString = '\\n'}| & 
% \piton{MyString = '\\n'} \\
% \verb|\piton{def even(n): return n\%2==0}| & 
% \piton{def even(n): return n\%2==0} \\
% \verb|\piton{c="#"    # an affectation }| & 
% \piton{c="#"     # an affectation } \\
% \verb|\piton{c="#" \ \ \ # an affectation }| & 
% \piton{c="#" \ \ \ # an affectation } \\
% \verb|\piton{MyDict = {'a': 3, 'b': 4 }}| &
% \piton{MyDict = {'a': 3, 'b': 4 }}
% \end{tabular}
%
% \bigskip
% It's possible to use the command |\piton| with that syntax in the arguments of
% a LaTeX command.\footnote{For example, it's possible to use the command
% |\piton| in a footnote. Example : \piton{s = 123}.}
%
% However, since the argument is expanded (in the TeX sens), one should take
% care not using in its argument \emph{fragile} commands (that is to say
% commands which are neither \emph{protected} nor \emph{fully expandable}).
%
% \bigskip
% \item {\color{blue} \textsf{Syntax} \verb!\piton|...|!}
%
% \nobreak
% When the argument of the command |\piton| is provided between two identical
% characters (all the characters are allowed except |%|, |\|, |#|, |{|, |}| and
% the space), that argument is taken in a \emph{verbatim mode}. Therefore, with
% that syntax, the command |\piton| can't be used within the argument of another
% command. 
%
% \medskip
%
% \begin{tabular}{>{\color{gray}}w{l}{75mm}@{\hspace*{1cm}}l}
% \omit Examples : \hfil \\
% \noalign{\vskip1mm}
% \verb!\piton|MyString = '\n'|! & 
% \piton|MyString = '\n'| \\
% \verb|\piton!def even(n): return n%2==0!| & 
% \piton!def even(n): return n%2==0! \\
% \verb|\piton+c="#"    # an affectation +| & 
% \piton+c="#"     # an affectation + \\
% \verb|\piton?MyDict = {'a': 3, 'b': 4}?| &
% \piton!MyDict = {'a': 3, 'b': 4}!
% \end{tabular}
%
% \end{itemize}
% 
% \section{Customization}
%
%
% \subsection{The keys of the command \textbackslash PitonOptions}

%
% \indexcommand{PitonOptions}
% 
% The command |\PitonOptions| takes in as argument a comma-separated list of
% \textsl{key=value} pairs. The scope of the settings done by that command is
% the current TeX group.\footnote{We remind that a LaTeX environment is, in
% particular, a TeX group.}
%
% These keys may also be applied to an individual environment |{Piton}| (between
% square brackets).
%
% \begin{itemize}
% \item The key \Definition{language} specifies which computer language is
% considered (that key is case-insensitive). It's possible to use the name of
% the six built-in languages (|Python|, |OCaml|, |C|, |SQL|, |minimal| and
% |verbatim|) or the name of a language defined by the user with
% |\NewPitonLanguage| (cf.~part~\ref{NewPitonLanguage},
% p.~\pageref{NewPitonLanguage}).   
%
% The initial value is |Python|.
%
% \item \index{font-command}
% The key \Definition{font-command} contains instructions of font which will be
% inserted at the beginning of all the elements composed by \pkg{piton} (without
% surprise, these instructions are not used for the so-called ``LaTeX comments'').
%
% The initial value is |\ttfamily| and, thus, \pkg{piton} uses by default the
% current monospaced font.
% 
% \item \index{gobble}\label{gobble} The key \Definition{gobble} takes in as
% value a positive integer~$n$: the first $n$ characters are discarded (before
% the process of highlighting of the code) for each line of the environment
% |{Piton}|. These characters are not necessarily spaces.
%
% When the key |gobble| is used without value, it is equivalent to the key
% |auto-gobble|, that we describe now.
%
% \item \index{auto-gobble}\index{gobble!auto-gobble} When the key
% \Definition{auto-gobble} is in force, the extension \pkg{piton} computes the
% minimal value $n$ of the number of consecutive spaces beginning each (non
% empty) line of the environment |{Piton}| and applies |gobble| with that value
% of~$n$.
%
% \item \index{env-gobble}\index{gobble!env-gobble}
% When the key \Definition{env-gobble} is in force, \pkg{piton} analyzes the last
% line of the environment |{Piton}|, that is to say the line which contains
% |\end{Piton}| and determines whether that line contains only spaces followed
% by the |\end{Piton}|. If we are in that situation, \pkg{piton} computes the
% number~$n$ of spaces on that line and applies |gobble| with that value of~$n$.
% The name of that key comes from \emph{environment gobble}: the effect of
% gobble is set by the position of the commands |\begin{Piton}| and
% |\end{Piton}| which delimit the current environment.
%
% \item \index{write} The key \Definition{write} takes in as argument a name of
% file (with its extension) and write the content\footnote{In fact, it's not
% exactly the body of the environment but the value of |piton.get_last_code()|
% which is the body without the overwritten LaTeX formatting instructions (cf.
% the part~\ref{API}, p.~\pageref{API}).} of the current environment in
% that file. At the first use of a file by \pkg{piton} (during a given
% compilation done by LaTeX), it is erased. In fact, the file is written once at
% the end of the compilation of the file by LuaLaTeX.
%
% \item \index{path-write} The key \Definition{path-write} specifies a path
% where the files written by the key |write| will be written.
%
% \item The key \Definition{join} is similar to the key |write| but the files which
% are created are joined (as \emph{joined files}) in the \textsc{pdf}. Be
% careful: Some \textsc{pdf} readers don't provide any tool to access to these
% joined files. Among the applications wich provide an access to those joined
% files, we will mention the free application Foxit~PDF~Reader, which is
% available on all the platforms.
%
% \item \index{print}
% The key \Definition{print} controls whether the content of the environment is 
% actually printed (with the syntactic formating) in the \textsc{pdf}. Of
% course, the initial value of |print| is |true|. However, it may be useful to
% use |print=false| in some circonstancies (for example, when the key |write| or
% the key |join| is used).
%
% \item \index{line-numbers} The key \Definition{line-numbers} activates the
% line numbering in the environments |{Piton}| and in the listings resulting
% from the use of |\PitonInputFile|.
% 
% In fact, the key |line-numbers| has several subkeys.
% \begin{itemize}
% \item With the key \Definition{line-numbers/skip-empty-lines}, the empty lines
% (which contains only spaces) are considered as non existent for the line
% numbering (if the key |/absolute|, described below, is in force, the key
% |/skip-empty-lines| is no-op in |\PitonInputFile|). The initial value of that
% key is |true| (and not |false|).\footnote{For the language Python, the empty
% lines in the docstrings are taken into account (by design).}
% \item With the key \Definition{line-numbers/label-empty-lines}, the labels
% (that is to say the numbers) of the empty lines are displayed. If the key
% |/skip-empty-line| is in force, the clé |/label-empty-lines| is no-op. The
% initial value of that key is~|true|.\footnote{When the key
% |split-on-empty-lines| is in force, the labels of the empty lines are never printed.}
% \item With the key \Definition{line-numbers/absolute}, in the listings
% generated in |\PitonInputFile|, the numbers of the lines displayed are
% \emph{absolute} (that is to say: they are the numbers of the lines in the
% file). That key may be useful when |\PitonInputFile| is used to insert only a
% part of the file (cf. part~\ref{part-of-a-file}, p.~\pageref{part-of-a-file}).
% The key |/absolute| is no-op in the environments |{Piton}| and those created
% by |\NewPitonEnvironment|.
% \item The key \Definition{line-numbers/start} requires that the line numbering
% begins to the value of the key. 
% \item With the key \Definition{line-numbers/resume}, the counter of lines is
% not set to zero at the beginning of each environment |{Piton}| or use of
% |\PitonInputFile| as it is otherwise. That allows a numbering of the lines
% across several environments.
% \item The key \Definition{line-numbers/sep} is the horizontal distance between
% the numbers of lines (inserted by |line-numbers|) and the beginning of the
% lines of code. The initial value is 0.7~em.
% \item The key \Definition{line-numbers/format} is a list of tokens which are inserted before
% the number of line in order to format it. It's possible to put, \emph{at the
% end} of the list, a LaTeX command with one argument, such as, for example,
% |\fbox|. 
%
% The initial value is |\footnotesize\color{gray}|. 
% \end{itemize}
%
% For convenience, a mechanism of factorisation of the prefix |line-numbers| is
% provided. That means that it is possible, for instance, to write:
% \begin{Verbatim}
% \PitonOptions
%   {
%     line-numbers = 
%       { 
%         skip-empty-lines = false ,
%         label-empty-lines = false ,
%         sep = 1 em ,
%         format = \footnotesize \color{blue}
%       }
%   }
% \end{Verbatim}
%
%
% \item \index{left-margin} The key \Definition{left-margin} corresponds to a
% margin on the left. That key may be useful in conjunction with the key
% |line-numbers| if one does not want the numbers in an overlapping position on
% the left. 
%
% It's possible to use the key |left-margin| with the special value |auto|. With that
% value, if the key |line-numbers| is in force, a margin will be automatically
% inserted to fit the numbers of lines. See an example part
% \ref{example-numbering} on page~\pageref{example-numbering}.
%
% \item \index{background-color} The key \Definition{background-color} sets the
% background color of the environments |{Piton}| and the listings produced by
% |\PitonInputFile| (it's possible to fix the width of that background with the
% key |width| or the key |max-width| described below).

% The key |background-color| accepts a color
% defined «on the fly». For example, it's possible to write
% |background-color = [cmyk]{0.1,0.05,0,0}|.
%
% \smallskip
% The key |background-color| supports also as value a \emph{list} of colors. In
% this case, the successive rows are colored by using the colors of the list in
% a cyclic way.
%
% \smallskip
% \colorbox{yellow!50}{\textbf{New 4.6}}\enskip
% In that list, the special color \Definition{none} may be use to specify no
% color at all.
%
% \emph{Example} : |\PitonOptions{background-color = {gray!15,none}}|
%
% \item \index{rounded-corners} \colorbox{yellow!50}{\textbf{New 4.7}}
%
% \nobreak
% It's possible to use the key \Definition{rounded-corners} to require rounded
% corners for the colored panels drawn by the key |background-color| The initial
% value of that is $0$~pt, which means that the corners are not rounded. If the
% key |rounded-corners| is used, the extension \pkg{tikz} must be loaded because
% those rounded corners are drawn by using \pkg{tikz}. If \pkg{tikz} is not
% loaded, an error will be raised at the first use of the key |rounded-corners|.
%
% The default value of the key |rounded-corners| is $4$~pt.\footnote{This value is the
% initial value of the \emph{rounded corners} of TikZ.}
% 
% \item \index{prompt-background-color} With the key
% \Definition{prompt-background-color}, \pkg{piton} adds a 
% color background to the lines beginning with the prompt ``|>>>|'' (and its
% continuation ``|...|'') characteristic of the Python consoles with
% \textsc{repl} (\emph{read-eval-print loop}).
%
% \item \index{width} \label{width} The key \Definition{width} fixes the width
% of the listing in the \textsc{pdf}. The initial value of that parameter
% is the current value of |\linewidth|.
%
% That parameter is used for:
% \begin{itemize}
% \item the breaking the lines which are too long (except, of course, when the key
% |break-lines| is set to false: cf.~p.~\pageref{line-breaks});
% \item the color of the backgrounds specified by the keys |background-color| and
% |prompt-background-color| described below;
% \item the width of the LaTeX box created by the key |box| when that key is used
% (cf. p.~\pageref{box});
% \item the width of the graphical box created by the key |tcolorbox| when that
% key is used.
% \end{itemize}
%
% \item \index{max-width} \colorbox{yellow!50}{\textbf{New 4.6}}
%
% \nobreak

% The key \Definition{max-width} is similar to the key |width| but it fixes the
% \emph{maximal} width of the lines. If all the lines of the listing are shorter
% than the value provided to |max-width|, the parameter |width| will be equal to
% the maximal length of the lines of the listing, that is to say the natural width of
% the listing.
%
% For legibility of the code, |width=min| is a shortcut for |max-width=\linewidth|.
% 
%
% \item \index{show-spaces-in-strings} When the key
% \Definition{show-spaces-in-strings} is activated, the spaces in the 
% strings of characters\footnote{With the language Python that feature applies
% only to the short strings (delimited by~\verb|'| or~\verb|"|) and, in
% particular, it does not apply for the \emph{doc strings}. In OCaml, that
% feature does not apply to the \emph{quoted strings}.} are replaced by the
% character~␣ (U+2423 : \textsc{open box}). Of course, that character~U+2423
% must be present in the monospaced font which is used.\footnote{The initial
% value of |font-command| is |\ttfamily| and, thus,  by default, \pkg{piton}
% merely uses the current monospaced font.}
%
% \nobreak
% \begingroup
% \PitonOptions{show-spaces-in-strings} Example : 
% \piton|my_string = 'Very good answer'| \endgroup
%
% With the key \index{show-spaces} \Definition{show-spaces}, all the spaces are
% replaced by U+2423 (and no line break can occur on those ``visible spaces'',
% even when the key |break-lines|\footnote{cf. \ref{line-breaks}
% p.~\pageref{line-breaks}} is in force). By the way, one should remark that all
% the trailing spaces (at the end of a line) are deleted by \pkg{piton} --- and,
% therefore, won't be represented by~␣. Moreover, when the key |show-spaces| is
% in force, the tabulations at the beginning of the lines are represented by arrows.
% \end{itemize}
%
% \bigskip
% \begingroup
% \fvset{commandchars=\~\&\@,formatcom=\small\color{gray}}
% \begin{Verbatim}
% ~emphase&\begin{Piton}[language=C,line-numbers,gobble,background-color=gray!15@
% ~emphase&              rounded-corners,width=min,splittable=4]@
%     void bubbleSort(int arr[], int n) {
%         int temp;
%         int swapped;
%         for (int i = 0; i < n-1; i++) {
%             swapped = 0;
%             for (int j = 0; j < n - i - 1; j++) {
%                 if (arr[j] > arr[j + 1]) {
%                     temp = arr[j];
%                     arr[j] = arr[j + 1];
%                     arr[j + 1] = temp;
%                     swapped = 1; 
%                 }
%             }
%             if (!swapped) break;
%         }
%     }   
% ~emphase&~textbackslash&@end{Piton}@
% \end{Verbatim}
% \endgroup
%
% \begingroup
% \PitonOptions{language=C,line-numbers,gobble=6,background-color = gray!15,
%    rounded-corners,width=min}
%     \begin{Piton}
%     void bubbleSort(int arr[], int n) {
%         int temp;
%         int swapped;
%         for (int i = 0; i < n-1; i++) {
%             swapped = 0;
%             for (int j = 0; j < n - i - 1; j++) {
%                 if (arr[j] > arr[j + 1]) {
%                     temp = arr[j];
%                     arr[j] = arr[j + 1];
%                     arr[j + 1] = temp;
%                     swapped = 1; 
%                 }
%             }
%             if (!swapped) break;
%         }
%     }   
%     \end{Piton}
% \endgroup
%
% 
% \bigskip
% The command |\PitonOptions| provides in fact several other keys which will be
% described further (see in particular the ``Pages breaks and line breaks''
% p.~\pageref{breakable}). 
%
% \subsection{The styles}
%
% \label{styles}
%
% \subsubsection{Notion of style}
%
% The package \pkg{piton} provides the command
% \DefinitionCommand{SetPitonStyle} to customize the different styles used to
% format the syntactic elements of the informatic listings. The customizations done
% by that command are limited to the current TeX group.\footnote{We remind that
% a LaTeX environment is, in particular, a TeX group.}
%
% \bigskip
% \indexcommand{SetPitonStyle}
% The command |\SetPitonStyle| takes in as argument a comma-separated list of
% \textsl{key=value} pairs. The keys are names of styles and the value are LaTeX
% formatting instructions.
%
% \bigskip
% These LaTeX instructions must be formatting instructions such as
% |\color{...}|, |\bfseries|, |\slshape|, etc. (the commands of this kind are
% sometimes called \emph{semi-global} commands). It's also possible to put,
% \emph{at the end of the list of instructions}, a LaTeX command taking exactly
% one argument.
% 
% \bigskip
% Here an example which changes the style used to highlight, in the definition
% of a Python function, the name of the function which is defined. That code
% uses the command |\highLight| of \pkg{lua-ul} (that package requires also the
% package \pkg{luacolor}).
%
% \begin{verbatim}
% \SetPitonStyle{ Name.Function = \bfseries \highLight[red!30] }
% \end{verbatim}
%
% In that example, |\highLight[red!30]| must be considered as the name of a
% LaTeX command which takes in exactly one argument, since, usually, it is used
% with |\highLight[red!30]{...}|.
%
% \medskip
% \begingroup
% \SetPitonStyle 
%   { Name.Function = \bfseries \highLight[red!30] }
% With that setting, we will have : \piton{def cube(x) : return x * x * x }
% \endgroup
% 
% \bigskip
% The different styles, and their use by \pkg{piton} in the different languages
% which it supports (Python, OCaml, C, SQL, ``|minimal|'' and ``|verbatim|''),
% are described in the part \ref{Semantic}, starting at the page
% \pageref{Semantic}.  
% 
%
% \bigskip
% \indexcommand{PitonStyle}
% The command \DefinitionCommand{PitonStyle} takes in as argument the name of a
% style and allows to retrieve the value (as a list of LaTeX instructions) of
% that style. That command is \emph{fully expandable} (in the TeX sens).
%
% \smallskip
% For example, it's possible to write |{\PitonStyle{Keyword}{function}}| and we
% will have the word {\PitonStyle{Keyword}{function}} formatted as a keyword.
%
% \smallskip
% The syntax |{\PitonStyle{|\textsl{\texttt{style}}|}{...}}| is mandatory in
% order to be able to deal both with the semi-global commands and the commands
% with arguments which may be present in the definition of the style
% \texttt{\textsl{style}}. 
%
% \bigskip
% \subsubsection{Global styles and local styles}
%
% A style may be defined globally with the command |\SetPitonStyle|. That means
% that it will apply to all the computer languages that use that style.
%
% \bigskip
% For example, with the command
% \begin{Verbatim}
% \SetPitonStyle{Comment = \color{gray}}
% \end{Verbatim}
% all the comments will be composed in gray in all the listings, whatever
% computer language they use (Python, C, OCaml, etc. or a language defined by
% the command |\NewPitonLanguage|).
% 
% \bigskip
% But it's also possible to define a style locally for a given informatic
% language by providing the name of that language as optional argument (between
% square brackets) to the command |\SetPitonStyle|.\footnote{We recall, that, in
% the package \pkg{piton}, the names of the computer languages are
% case-insensitive.}
%
% \bigskip
% For example, with the command
% \begin{Verbatim}
% \SetPitonStyle~emphase#[SQL]@{Keyword = \color[HTML]{006699} \bfseries \MakeUppercase}
% \end{Verbatim}
% the keywords in the SQL listings will be composed in capital letters, even if
% they appear in lower case in the LaTeX source (we recall that, in SQL, the
% keywords are case-insensitive).
%
% \bigskip
% As expected, if a computer language uses a given style and if that style has
% no local definition for that language, the global version is used. That notion
% of ``global style'' has no link with the notion of global definition in TeX
% (the notion of \emph{group} in TeX).\footnote{As regards the TeX groups, the
% definitions done by \texttt{\textbackslash SetPitonStyle} are always local.}
% 
% \bigskip
% The package \pkg{piton} itself (that is to say the file |piton.sty|) defines
% all the styles globally. 
%
% \bigskip
%
% \subsubsection{The style UserFunction}
%
% \index{UserFunction (style)}
% 
% The extension \pkg{piton} provides a special style called~|UserFunction|. That
% style applies to the names of the functions previously defined by the user
% (for example, in Python, these names are those following the keyword
% \piton{def} in a previous Python listing). The initial value of that style
% |\PitonStyle{Identifier}| and, therefore, the names of the functions are
% formatted like the other identifiers (that is to say, by default, with no
% special formatting except the features provided in |font-command|). However,
% it's possible to change the value of that style, as any other style, with the
% command |\SetPitonStyle|.
%
% \medskip
% In the following example, we tune the styles |Name.Function| and |UserFunction|
% so as to have clickable names of functions linked to the (informatic)
% definition of the function.
%
% \begingroup
%
% \NewDocumentCommand{\MyDefFunction}{m}{\hypertarget{piton:#1}{\color[HTML]{CC00FF}{#1}}}
% \NewDocumentCommand{\MyUserFunction}{m}{\hyperlink{piton:#1}{#1}}
%
% \SetPitonStyle
%   {
%     Name.Function = \MyDefFunction ,
%     UserFunction = \MyUserFunction
%   }
%
% \medskip
% \begin{Verbatim}
% \NewDocumentCommand{\MyDefFunction}{m}
%    {\hypertarget{piton:~#1}{\color[HTML]{CC00FF}{~#1}}}
% \NewDocumentCommand{\MyUserFunction}{m}{\hyperlink{piton:~#1}{~#1}}
%
% \SetPitonStyle{Name.Function = \MyDefFunction, ~emphase#UserFunction@ = \MyUserFunction}
% \end{Verbatim}
%
% \smallskip
%
% \begin{Piton}
% def transpose(v,i,j):
%     x = v[i]
%     v[i] = v[j]
%     v[j] = x 
%
% def passe(v):
%     for in in range(0,len(v)-1):
%         if v[i] > v[i+1]:
%             transpose(v,i,i+1)
% \end{Piton}
%
% \endgroup
%
% \medskip
% (Some \textsc{pdf} viewers display a frame around the clickable word
% |transpose| but other do not.)
%
% \PitonClearUserFunctions[Python]
%
% \bigskip
% \begin{small}
% Of course, the list of the names of Python functions previously défined is
% kept in the memory of LuaLaTeX (in a global way, that is to say independently
% of the TeX groups). The extension \pkg{piton} provides a command to clear that
% list : it's the command \DefinitionCommand{PitonClearUserFunctions}. When it
% is used without argument, that command is applied to all the informatic
% languages used by the user but it's also possible to use it with an optional
% argument (between square brackets) which is a list of computer languages to
% which the command will be applied.\footnote{We remind that, in \pkg{piton},
% the name of the computer languages are case-insensitive.}
% \end{small}
% 
% \subsection{Creation of new environments}
%
% \label{NewPitonEnvironment}
% \indexcommand{NewPitonEnvironment}
% 
% Since the environment |{Piton}| has to catch its body in a special way (more
% or less as verbatim text), it's not possible to construct new environments
% directly over the environment |{Piton}| with the classical commands
% |\newenvironment| (of standard LaTeX) or |\NewDocumentEnvironment| (of
% LaTeX3). 
%
% With a LaTeX kernel newer than 2025-06-01, it's possible to use
% |\NewEnvironmentCopy| on the environment |{Piton}| but it's not very powerful.
%
% That's why \pkg{piton} provides a command
% \DefinitionCommand{NewPitonEnvironment}. That command takes in three mandatory
% arguments.
%
% That command has the same syntax as the classical environment
% |\NewDocumentEnvironment|.\footnote{However, the specifier of argument |b|
% (used to catch the body of the environment as a LaTeX argument) is
% not allowed (of course)}
%
% \medskip
% There also exist the commands \DefinitionCommand{RenewPitonEnvironment},
% \DefinitionCommand{DeclarePitonEnvironment} and
% \DefinitionCommand{ProvidePitonEnvironment}, similar to the corresponding
% commands of L3.
% 
% \bigskip
% With the following instruction, a new environment |{Python}| will be
% constructed with the same behaviour as |{Piton}|:
%
% \begin{Verbatim}
% ~emphase#\NewPitonEnvironment@{Python}{O{}}{\PitonOptions{~#1}}{} 
% \end{Verbatim}
% 
% \bigskip
% If one wishes to format Python code in a box of \pkg{mdframed}, it's possible
% to define an environment |{Python}| with the following code (of course, the
% package \pkg{mdframed} must be loaded, and, in this document, it has been loaded
% with the key |framemethod=tikz|).
%
%\begin{Verbatim}
% ~emphase#\NewPitonEnvironment@{Python}{}
%   {\begin{mdframed}[roundcorner=3mm]}
%   {\end{mdframed}}
% \end{Verbatim}
%
% \bigskip
% With this new environment |{Python}|, it's possible to write:
%
% \begin{Verbatim}
% ~emphase#\begin{Python}@
% def square(x):
%     """Compute the square of x"""
%     return x*x
% ~emphase#\end{Python}@
% \end{Verbatim}
%
% \NewPitonEnvironment{Python}{}
%   {\begin{mdframed}[roundcorner=3mm]}
%   {\end{mdframed}}
%
% \begin{Python}
% def square(x):
%     """Compute the square of x"""
%     return x*x
% \end{Python}
% 
% \bigskip
% It's possible to a similar construction with an environment of \pkg{tcolorbox}.
% However, for a better cooperation between \pkg{piton} and \pkg{tcolorbox}, the
% extension \pkg{piton} provides a key \pkg{tcolorbox}: cf.~p.~\pageref{tcolorbox}.
% 
%
% \section{Definition of new languages with the syntax of listings}
%
% \indexcommand{NewPitonLanguage}
% \label{NewPitonLanguage}
% \index{listings@\pkg{listings} (extension)}
%
% \medskip
% The package \pkg{listings} is a famous LaTeX package to format informatic
% listings.
%
% That package provides a command |\lstdefinelanguage| which allows the user to
% define new languages. That command is also used by \pkg{listings} itself to
% provide the definition of the predefined languages in \pkg{listings} (in fact,
% for this task, \pkg{listings} uses a command called |\lst@definelanguage| but
% that command has the same syntax as |\lstdefinelanguage|). 
%
% \medskip
% The package \pkg{piton} provides a command
% \DefinitionCommand{NewPitonLanguage} to define new languages (available in
% |\piton|, |{Piton}|, etc.) with a syntax which is almost the same as the
% syntax of~|\lstdefinelanguage|.
%
% Let's precise that \pkg{piton} does \emph{not} use that command to define the
% languages provided natively (Python, OCaml, C, SQL, |minimal| and |verbatim|),
% which allows more powerful parsers.
%
% \medskip
% For example, in the file |lstlang1.sty|, which is one of the definition files
% of \pkg{listings}, we find the following instructions (in version 1.10a).
% 
%\begin{Verbatim}[formatcom=\small\color{gray}]
% \lst~@definelanguage{Java}%
%   {morekeywords={abstract,boolean,break,byte,case,catch,char,class,%
%       const,continue,default,do,double,else,extends,false,final,%
%       finally,float,for,goto,if,implements,import,instanceof,int,%
%       interface,label,long,native,new,null,package,private,protected,%
%       public,return,short,static,super,switch,synchronized,this,throw,%
%       throws,transient,true,try,void,volatile,while},%
%    sensitive,%
%    morecomment=[l]//,%
%    morecomment=[s]{/*}{*/},%
%    morestring=[b]",%
%    morestring=[b]',%
%   }[keywords,comments,strings]
% \end{Verbatim}
%
% \medskip
% In order to define a language called |Java| for \pkg{piton}, one has only to
% write the following code {\bfseries where the last argument of
% |\lst@definelanguage|, between square brackets, has been discarded} (in fact,
% the symbols \verb+%+ may be deleted without any problem).
%
% \begin{Verbatim}[formatcom=\small\color{gray}]
% ~emphase#\NewPitonLanguage@{Java}%
%   {morekeywords={abstract,boolean,break,byte,case,catch,char,class,%
%       const,continue,default,do,double,else,extends,false,final,%
%       finally,float,for,goto,if,implements,import,instanceof,int,%
%       interface,label,long,native,new,null,package,private,protected,%
%       public,return,short,static,super,switch,synchronized,this,throw,%
%       throws,transient,true,try,void,volatile,while},%
%    sensitive,%
%    morecomment=[l]//,%
%    morecomment=[s]{/*}{*/},%
%    morestring=[b]",%
%    morestring=[b]',%
%   }
% \end{Verbatim}
%
% 
% \NewPitonLanguage{Java}
%   {morekeywords={abstract,boolean,break,byte,case,catch,char,class,
%       const,continue,default,do,double,else,extends,false,final,
%       finally,float,for,goto,if,implements,import,instanceof,int,
%       interface,label,long,native,new,null,package,private,protected,
%       public,return,short,static,super,switch,synchronized,this,throw,
%       throws,transient,true,try,void,volatile,while},
%    sensitive,
%    morecomment=[l]//,
%    morecomment=[s]{/*}{*/},
%    morestring=[b]",
%    morestring=[b]',
%   }
%
% \medskip
% It's possible to use the language Java like any other language defined by
% \pkg{piton}.
%
% Here is an example of code formatted in an environment |{Piton}| with
% the key |language=Java|.\footnote{We recall that, for \pkg{piton}, the names
% of the computer languages are case-insensitive. Hence, it's possible to
% write, for instance, |language=java|.}
%
% 
% \bigskip
% \begingroup
% \small
% \PitonOptions{splittable-on-empty-lines}
% \begin{Piton}[language = java]
% public class Cipher {  // Caesar cipher
%     public static void main(String[] args) {
%         String str = "The quick brown fox Jumped over the lazy Dog";
%         System.out.println( Cipher.encode( str, 12 ));
%         System.out.println( Cipher.decode( Cipher.encode( str, 12), 12 ));
%     }
%
%     public static String decode(String enc, int offset) {
%         return encode(enc, 26-offset);
%     }
%
%     public static String encode(String enc, int offset) {
%         offset = offset % 26 + 26;
%         StringBuilder encoded = new StringBuilder();
%         for (char i : enc.toCharArray()) {
%             if (Character.isLetter(i)) {
%                 if (Character.isUpperCase(i)) {
%                     encoded.append((char) ('A' + (i - 'A' + offset) % 26 ));
%                 } else {
%                     encoded.append((char) ('a' + (i - 'a' + offset) % 26 ));
%                 }
%             } else {
%                 encoded.append(i);
%             }
%         }
%         return encoded.toString();
%     }
% }
% \end{Piton}
% \endgroup
%
% \bigskip

% The keys of the command |\lstdefinelanguage| of \pkg{listings} supported by |\NewPitonLanguage| are:
% |morekeywords|, |otherkeywords|, |sensitive|, |keywordsprefix|, |moretexcs|,
% |morestring| (with the letters |b|,
% |d|, |s| and |m|), |morecomment| (with the letters |i|, |l|, |s| and |n|),
% |moredelim| (with the letters |i|,
% |l|, |s|, |*| and |**|), |moredirectives|, |tag|, |alsodigit|, |alsoletter|
% and |alsoother|.
%
% For the description of those keys, we redirect the reader to the documentation
% of the package \pkg{listings} (type |texdoc| |listings| in a terminal). 
%
% \bigskip
% For example, here is a language called ``LaTeX'' to format LaTeX chunks of codes:

% \begin{Verbatim}
% \NewPitonLanguage{LaTeX}{keywordsprefix = \ , alsoletter = ~@_ }
% \end{Verbatim}
%
% Initially, the characters |@| and |_| are considered as letters because, in
% many computer languages, they are allowed in the keywords and the names of
% the identifiers. With \verb|alsoletter = @_|, we retrieve them from the
% category of the letters. 
% 
% \section{Advanced features}
%
% \subsection{The key ``box''}
%
% \label{box}
% \index{box}
% 
% \colorbox{yellow!50}{\textbf{New 4.6}}
%
% \nobreak
% \smallskip
% If one wishes to compose a listing in a box of LaTeX, he should use the key
% |box|. That key takes in as value |c|, |t| or |b| corresponding to the parameter
% of vertical position (as for the envionment |{minipage}| of LaTeX). The default value
% is |c| (as for |{minipage}|).
%
% When the  key |box| is used, |width=min| is activated (except, of course, when the
% key |width| or the key |max-width| is explicitely used). For the keys |width| and
% |max-width|, cf.~p.~\pageref{width}.
%
%
% \smallskip
% \begin{Verbatim}
% \begin{center}
% \PitonOptions{~emphase#box@,background-color=gray!15}
% \begin{Piton}
% def square(x):
%     return x*x
% \end{Piton}
% \hspace{1cm}
% def cube(x):
%     return x*x*x
% \end{Piton}
% \end{center}
% \end{Verbatim}
%
% \begin{center}
% \PitonOptions{box,background-color=gray!15}
% \begin{Piton}
% def square(x):
%     return x*x
% \end{Piton}
% \hspace{1cm}
% \begin{Piton}
% def cube(x):
%     return x*x*x
% \end{Piton}
% \end{center}
%
% \vspace{1cm}
% It's possible to use the key |box| with a numerical value for the key |width|.
% 
%
% \smallskip
%
% \begin{Verbatim}
% \begin{center}
% \PitonOptions{~emphase#box,width=5cm@,background-color=gray!15}
% \begin{Piton}
% def square(x):
%     return x*x
% \end{Piton}
% \hspace{1cm}
% def cube(x):
%     return x*x*x
% \end{Piton}
% \end{center}
% \end{Verbatim}
%
% \begin{center}
% \PitonOptions{box,width=5cm,background-color=gray!15}
% \begin{Piton}
% def square(x):
%     return x*x
% \end{Piton}
% \hspace{1cm}
% \begin{Piton}
% def cube(x):
%     return x*x*x
% \end{Piton}
% \end{center}
%
% \vspace{1cm}
% Here is an exemple with the key |max-width|.
%
% \begin{Verbatim}
% \begin{center}
% \PitonOptions{~emphase#box=t,max-width=7cm@,background-color=gray!15}
% \begin{Piton}
% def square(x):
%     return x*x
% \end{Piton}
% \hspace{1cm}
% \begin{Piton}
% def P(x):
%     return 24*x**8 - 7*x**7 + 12*x**6 -4*x**5 + 4*x**3 + x**2 - 5*x + 2
% \end{Piton}
% \end{center}
% \end{Verbatim}
%
% \begin{center}
% \PitonOptions{box=t,max-width=7cm,background-color=gray!15}
% \begin{Piton}
% def square(x):
%     return x*x
% \end{Piton}
% \hspace{1cm}
% \begin{Piton}
% def P(x):
%     return 24*x**8 - 7*x**7 + 12*x**6 -4*x**5 + 4*x**3 + x**2 - 5*x + 2
% \end{Piton}
% \end{center}
%
% \subsection{The key ``tcolorbox''}
%
% \label{tcolorbox}
% \index{tcolorbox (key)}
%
% The extension \pkg{piton} provides a key \Definition{tcolorbox} in order to
% ease the use of the extension \pkg{tcolorbox} in conjunction with the extension
% \pkg{piton}. However, the extension \pkg{piton} does not load \pkg{tcolorbox} and
% the final user should have loaded it. Moreover, he must load the library |breakable|
% of \pkg{tcolorbox} with |\tcbuselibrary{breakable}| in the preamble of the LaTeX document.
% If this is not the case, an error will be raised at the first use of the key
% |tcolorbox|.
% 
% \bigskip
% When the key |tcolorbox| is used, the listing formated by \pkg{piton} is
% included in an environment |{tcolorbox}|. That applies both to the command
% |\PitonInputFile| and the environment |{Piton}| (or, more generally, an
% environment created by the dedicated command |\NewPitonEnvironment|:
% cf.~p.~\pageref{NewPitonEnvironment}). If the key |splittable| of \pkg{piton}
% is used (cf. p.~\pageref{page breaks}), the graphical box created by
% \pkg{tcolorbox} will be splittable by a change of page.
%
% \bigskip
% In the present document, we have loaded, besides \pkg{tcolorbox} and its library
% |breakable|, the library |skins| of \pkg{tcolorbox} and we have activated the
% ``\emph{skin}'' |enhanced|, in order to have a better appearance at the page break.
%
%
% \begin{Verbatim}[formatcom = \footnotesize\color{gray}]
% \tcbuselibrary{skins,breakable} % in the preamble
% \tcbset{enhanced}               % in the preamble 
% \end{Verbatim}
%
% \begin{Verbatim}
% \begin{Piton}~emphase#[tcolorbox,splittable=3]@
% def carré(x):
%     """Computes the square of x"""
%     return x*x
% ...
% def carré(x):
%     """Computes the square of x"""
%     return x*x
% \end{Piton}
% \end{Verbatim}
%
% \begin{Piton}[tcolorbox,splittable=3]
% def carré(x):
%     """Computes the square of x"""
%     return x*x
% def carré(x):
%     """Computes the square of x"""
%     return x*x
% def carré(x):
%     """Computes the square of x"""
%     return x*x
% def carré(x):
%     """Computes the square of x"""
%     return x*x
% def carré(x):
%     """Computes the square of x"""
%     return x*x
% def carré(x):
%     """Computes the square of x"""
%     return x*x
% def carré(x):
%     """Computes the square of x"""
%     return x*x
% def carré(x):
%     """Computes the square of x"""
%     return x*x
% def carré(x):
%     """Computes the square of x"""
%     return x*x
% def carré(x):
%     """Computes the square of x"""
%     return x*x
% def carré(x):
%     """Computes the square of x"""
%     return x*x
% def carré(x):
%     """Computes the square of x"""
%     return x*x
% def carré(x):
%     """Computes the square of x"""
%     return x*x
% def carré(x):
%     """Computes the square of x"""
%     return x*x
% def carré(x):
%     """Computes the square of x"""
%     return x*x
% def carré(x):
%     """Computes the square of x"""
%     return x*x
% def carré(x):
%     """Computes the square of x"""
%     return x*x
% def carré(x):
%     """Computes the square of x"""
%     return x*x
% def carré(x):
%     """Computes the square of x"""
%     return x*x
% def carré(x):
%     """Computes the square of x"""
%     return x*x
% def carré(x):
%     """Computes the square of x"""
%     return x*x
% def carré(x):
%     """Computes the square of x"""
%     return x*x
% def carré(x):
%     """Computes the square of x"""
%     return x*x
% def carré(x):
%     """Computes the square of x"""
%     return x*x
% def carré(x):
%     """Computes the square of x"""
%     return x*x
% \end{Piton}
%
% Of course, if we want to change the color of the background, we won't use the key
% |background-color| of \pkg{piton} but the tools provided by \pkg{tcolorbox}
% (the key |colback| for the color of the background).
%
% \bigskip
% If we want to adjust the width of the graphical box to its content, we only
% have to use the key |width=min| provided by \pkg{piton} (cf.
% p.~\pageref{width}). It's also possible to use |width| or |max-width| with a
% numerical value. The environment is splittable if the key |splittable| is used (cf.
% p.~\pageref{page breaks}).
%
% \begin{Verbatim}
% \begin{Piton}~emphase#[tcolorbox,width=min,splittable=3]@
% def square(x):
%     """Computes the square of x"""
%     return x*x
% ...
% def square(x):
%     """Computes the square of x"""
%     return x*x
% \end{Piton}
% \end{Verbatim}
%
% \begin{Piton}[tcolorbox,width=min,splittable=3]
% def square(x):
%     """Computes the square of x"""
%     return x*x
% def square(x):
%     """Computes the square of x"""
%     return x*x
% def square(x):
%     """Computes the square of x"""
%     return x*x
% def square(x):
%     """Computes the square of x"""
%     return x*x
% def square(x):
%     """Computes the square of x"""
%     return x*x
% def square(x):
%     """Computes the square of x"""
%     return x*x
% def square(x):
%     """Computes the square of x"""
%     return x*x
% def square(x):
%     """Computes the square of x"""
%     return x*x
% def square(x):
%     """Computes the square of x"""
%     return x*x
% def square(x):
%     """Computes the square of x"""
%     return x*x
% def square(x):
%     """Computes the square of x"""
%     return x*x
% def square(x):
%     """Computes the square of x"""
%     return x*x
% def square(x):
%     """Computes the square of x"""
%     return x*x
% def square(x):
%     """Computes the square of x"""
%     return x*x
% def square(x):
%     """Computes the square of x"""
%     return x*x
% def square(x):
%     """Computes the square of x"""
%     return x*x
% def square(x):
%     """Computes the square of x"""
%     return x*x
% def square(x):
%     """Computes the square of x"""
%     return x*x
% def square(x):
%     """Computes the square of x"""
%     return x*x
% def square(x):
%     """Computes the square of x"""
%     return x*x
% def square(x):
%     """Computes the square of x"""
%     return x*x
% def square(x):
%     """Computes the square of x"""
%     return x*x
% def square(x):
%     """Computes the square of x"""
%     return x*x
% def square(x):
%     """Computes the square of x"""
%     return x*x
% def square(x):
%     """Computes the square of x"""
%     return x*x
% \end{Piton}
%
% \bigskip
% If we want an output composed in a LaTeX box (despites its name, an environment
% of \pkg{tcolorbox} does not always create a LaTeX box), we only have to use,
% in conjunction with the key |tcolorbox|, the key |box| provided by \pkg{piton}
% (cf. p.~\pageref{box}). Of course, such LaTeX box can't be broken by a change of page.
%
% We recall that, when the key |box| is used, |width=min| is activated (except,
% when the key |width| or the key |max-width| is explicitely used).
%
%
% \smallskip
% \begin{Verbatim}
% \begin{center}
% \PitonOptions{~emphase#tcolorbox,box=t@}
% \begin{Piton}
% def square(x):
%     return x*x
% \end{Piton}
% \hspace{1cm}
% \begin{Piton}
% def cube(x):
%     """The cube of x"""
%     return x*x*x
% \end{Piton}
% \end{center}
% \end{Verbatim}
%
% \begin{center}
% \PitonOptions{tcolorbox,box=t}%
% \begin{Piton}
% def square(x):
%     return x*x
% \end{Piton}
% \hspace{1cm}
% \begin{Piton}
% def cube(x):
%     """The cube of x"""
%     return x*x*x
% \end{Piton}
% \end{center}
%
%
%
% \bigskip
% For a more sophisticated example of use of the key |tcolorbox|, see the example
% given at the page~\pageref{examples-tcolorbox}.
% 
% \subsection{Insertion of a file}
%
% \label{PitonInputFile}
%
% \subsubsection{The command \textbackslash PitonInputFile}
%
% \indexcommand{PitonInputFile}
%
% The command \DefinitionCommand{PitonInputFile} includes the content of the
% file specified in argument (or only a part of that file: see below). The
% extension \pkg{piton} also provides the commands
% \DefinitionCommand{PitonInputFileT}, \DefinitionCommand{PitonInputFileF} and
% \DefinitionCommand{PitonInputFileTF} with supplementary arguments
% corresponding to the letters~|T| and~|F|. Those arguments will be executed if
% the file to include has been found (letter~|T|) or not found (letter~|F|).
% 
% \smallskip
% The syntax for the pathes (absolute or relative) is the following one:
% \begin{itemize}
% \item The paths beginning by |/| are absolute.
%
% \emph{Example} : |\PitonInputFile{/Users/joe/Documents/program.py}|
%
% \item The paths which do not begin with |/| are relative to the current
% repertory. 
%
% \emph{Example} : |\PitonInputFile{my_listings/program.py}|
% \end{itemize}
%
% \index{path}
% The key \Definition{path} of the command |\PitonOptions| specifies a
% \emph{list} of paths where the files included by |\PitonInputFile| will be
% searched. That list is comma separated.
%
% As previously, the absolute paths must begin with |/|.
%
% \bigskip
% \subsubsection{Insertion of a part of a file}
%
% \label{part-of-a-file}
% \indexcommand{PitonInputFile}
%
% The command |\PitonInputFile| inserts (with formatting) the content of a file.
% In fact, it's possible to insert only \emph{a part} of that file. Two
% mechanisms are provided in this aim.
% \begin{itemize}
% \item It's possible to specify the part that we want to insert by the numbers
% of the lines (in the original file).
% \item It's also possible to specify the part to insert with textual markers.
% \end{itemize}
% In both cases, if we want to number the lines with the numbers of the
% lines in the file, we have to use the key |line-numbers/absolute|.
%
% \bigskip
% \textbf{With line numbers}
%
% \nobreak
% The command |\PitonInputFile| supports the keys \Definition{first-line} and
% \Definition{last-line} in order to insert only the part of file between the
% corresponding lines. Not to be confused with the key |line-numbers/start|
% which fixes the first line number for the line numbering. In a sens,
% |line-numbers/start| deals with the output whereas |first-line| and
% |last-line| deal with the input.
%
% \bigskip
% \textbf{With textual markers}
%
% \nobreak
% \index{marker/beginning}
% \index{marker/end}
%
% In order to use that feature, we first have to specify the format of the
% markers (for the beginning and the end of the part to include) with the keys
% \Definition{marker-beginning} and \Definition{marker-end} (usually with the
% command |\PitonOptions|). 
%
%
% \medskip
% Let us take a practical example.
%
% \medskip
% We assume that the file to include contains solutions to exercises of
% programmation on the following model.
%
% \begin{Verbatim}[formatcom=\small\color{gray}]
% ~#[Exercise 1] Iterative version
% def fibo(n):
%     if n==0: return 0 
%     else:
%         u=0
%         v=1
%         for i in range(n-1):
%             w = u+v
%             u = v
%             v = w
%         return v
% ~#<Exercise 1>
% \end{Verbatim}
%
% The markers of the beginning and the end are the strings |#[Exercise 1]| and
% |#<Exercise 1>|. The string ``|Exercise 1|'' will be called the \emph{label}
% of the exercise (or of the part of the file to be included).
%
% In order to specify such markers in \pkg{piton}, we will use the keys
% |marker/beginning| and |marker/end| with the following instruction (the
% character |#| of the comments of Python must be inserted with the protected
% form |\#|). 
%
% \begin{Verbatim}
% \PitonOptions{ ~emphase#marker/beginning@ = \~#[~#1] , ~emphase#marker/end@ = \~#<~#1> } 
% \end{Verbatim}
%
% As one can see, |marker/beginning| is an expression corresponding to the
% mathematical function which transforms the label (here |Exercise 1|) into the
% the beginning marker (in the example |#[Exercise 1]|). The string |#1|
% corresponds to the occurrences of the argument of that function, which the
% classical syntax in TeX. Idem for |marker/end|.\footnote{In regard to LateX,
% both functions must be \emph{fully expandable}.}
%
% \bigskip 
% Now, you only have to use the key \Definition{range} of |\PitonInputFile| to
% insert a marked content of the file.
%
% \smallskip
% \begin{Verbatim}
% \PitonInputFile[~emphase#range = Exercise 1@]{~textsl#file_name@}
% \end{Verbatim}
%
% \medskip
% \begin{Piton}
% def fibo(n):
%     if n==0: return 0 
%     else:
%         u=0
%         v=1
%         for i in range(n-1):
%             w = u+v
%             u = v
%             v = w
%         return v
% \end{Piton}
%
% \vspace{1cm}
% \index{marker/include-lines}
% The key \Definition{marker/include-lines} requires the insertion of the lines
% containing the markers.
%
% \begin{Verbatim}
% \PitonInputFile[~emphase#marker/include-lines@,range = Exercise 1]{~textsl#file_name@}
% \end{Verbatim}
%
% \begin{Piton}
% #[Exercise 1] Iterative version
% def fibo(n):
%     if n==0: return 0 
%     else:
%         u=0
%         v=1
%         for i in range(n-1):
%             w = u+v
%             u = v
%             v = w
%         return v
% #<Exercise 1>
% \end{Piton}
%
%
% \bigskip
% \index{begin-range}
% \index{end-range}
% In fact, there exist also the keys \Definition{begin-range} and
% \Definition{end-range} to insert several marked contents at the same time.
%
% For example, in order to insert the solutions of the exercises~3 to~5, we will
% write (if the file has the correct structure!):
%
%
% \begin{Verbatim}
% \PitonInputFile[~emphase#begin-range = Exercise 3, end-range = Exercise 5@]{~textsl#file_name@}
% \end{Verbatim}
%
%
%
% \bigskip
% \subsection{Page breaks and line breaks}
%
% \label{breakable}
% 
% \subsubsection{Line breaks}
% 
% \label{line-breaks}
% 
% By default, the elements produced by \pkg{piton} can't be broken by an end on
% line. However, there are keys to allow such breaks (the possible breaking
% points are the spaces, even the spaces which appear in the strings of the
% computer languages).
% \begin{itemize}
% \item \index{break-lines!break-lines-in-piton} With the key
% \Definition{break-lines-in-piton}, the line breaks are allowed in the command
% |\piton{...}| (but not in the command \verb+\piton|...|+, that is to say the
% command |\piton| in verbatim mode).
% \item \index{break-lines!break-lines-in-Piton} With the key
% \Definition{break-lines-in-Piton}, the line breaks are allowed in the
% environment |{Piton}| (hence the capital letter |P| in the name) and in the
% listings produced by |\PitonInputFile|. The initial value of that parameter is
% |true| (and not |false|).
% \item \index{break-lines} The key \Definition{break-lines} is a conjunction of
% the two previous keys. 
% \end{itemize}
% 
% \bigskip 
% The package \pkg{piton} provides also several keys to control the appearance
% on the line breaks allowed by |break-lines-in-Piton|.
%
% \begin{itemize}
% \item \index{indent-broken-lines} With the key
% \Definition{indent-broken-lines}, the indentation of a 
% broken line is respected at carriage return (on the condition that the used
% font is a monospaced font and this is the case by default since the initial
% value of |font-command| is |\ttfamily|).
%
% \item The key \Definition{end-of-broken-line} corresponds to the symbol placed
% at the end of a broken line. The initial value is:
% |\hspace*{0.5em}\textbackslash|.
%
% \item \index{continuation-symbol} The key \Definition{continuation-symbol}
% corresponds to the symbol placed at each carriage return. The initial value
% is: |+\;| (the command |\;| inserts a small horizontal space).
%
% \item \index{continuation-symbol-on-indentation} 
% The key \Definition{continuation-symbol-on-indentation} corresponds to
% the symbol placed at each carriage return, on the position of the indentation
% (only when the key |indent-broken-line| is in force). The initial value is:
% |$\hookrightarrow\;$|.
% \end{itemize}
%
%
% \bigskip
% The following code has been composed with the following tuning:
%
% \begin{Verbatim}
% \PitonOptions{width=12cm,break-lines,indent-broken-lines,background-color=gray!15}
% \end{Verbatim}
%
% \begin{center}
% \PitonOptions{width=12cm,break-lines,indent-broken-lines,background-color=gray!15}
% \begin{Piton}
% def dict_of_list(l):
%     """Converts a list of subrs and descriptions of glyphs in a dictionary"""
%     our_dict = {}
%     for list_letter in l:
%         if (list_letter[0][0:3] == 'dup'): # if it's a subr
%             name = list_letter[0][4:-3]
%             print("We treat the subr of number " + name)
%         else:
%             name = list_letter[0][1:-3] # if it's a glyph
%             print("We treat the glyph of number " + name)
%         our_dict[name] = [treat_Postscript_line(k) for k in list_letter[1:-1]]
%     return dict
% \end{Piton}
% \end{center}
%
% \bigskip
% With the key \Definition{break-strings-anywhere}, the strings may be broken
% anywhere (and not only on the spaces). 
%
% \bigskip
% With the key \Definition{break-numbers-anywhere}, the numbers may be broken
% anywhere.
%
% \subsubsection{Page breaks}
% \label{page breaks}
% \index{splittable}
% \index{splittable-on-empty-lines}
%
% By default, the listings produced by the environment |{Piton}| and the command
% |\PitonInputFile| are not breakable.
%
% However, \pkg{piton} provides the keys |splittable-on-empty-lines| and
% |splittable| to allow such breaks.
%
% \begin{itemize}
% \item The key \Definition{splittable-on-empty-lines} allows breaks on the empty
% lines. The ``empty lines'' are in fact the lines which contains only spaces.
%
% \item Of course, the key |splittable-on-empty-lines| may not be sufficient and
% that's why \pkg{piton} provides the key \Definition{splittable}.
%
% When the key |splittable| is used with the numeric value~$n$ (which must be a
% positive integer) the listing, or each part of the listing delimited by empty
% lines (when |split-on-empty-lines| is in force) may be broken anywhere with
% the restriction that no break will occur within the $n$ first lines of
% the listing or within the $n$ last lines.\footnote{Remark that we speak of the
% lines of the original informatic listing and such line may be composed on
% several lines in the final \textsc{pdf} when the key |break-lines-in-Piton| is
% in force.} 
%
% For example, a tuning with |splittable = 4| may be a good choice.
%
% When used without value, the key |splittable| is equivalent to 
% |splittable = 1| and the listings may be broken anywhere (it's probably 
% not recommandable).
%
% The initial value of the key |splittable| is equal to 100 (by default, the
% listings are not breakable at all).
% \end{itemize} 
%
% \medskip
% \begin{tcolorbox}[breakable]
% Even with a background color (set by the key |background-color|), the pages
% breaks are allowed, as soon as the key |split-on-empty-lines| or the key
% |splittable| is in force.
%
% \medskip
% With the key |splittable|, the environments
% \texttt{\{Piton\}} are breakable, even within a (breakable) environment of
% \pkg{tcolorbox}. Remind that an environment of \pkg{tcolorbox} included in
% another environment of \pkg{tcolorbox} is \emph{not} breakable, even when both
% environments use the key |breakable| of \pkg{tcolorbox}.
%
% \medskip
% We illustrate that point with the following code (the current environment
% |{tcolorbox}| uses the key |breakable|).
%
% \begin{Verbatim}
% ~emphase#\begin{Piton}[background-color=gray!30,rounded-corners,width=min,splittable=4]@
% def square(x):
%     """Computes the square of x"""
%     return x*x
% ...
% def square(x):
%     """Computes the square of x"""
%     return x*x
% ~emphase#\end{Piton}@
% \end{Verbatim}
% 
% \bigskip
% \begin{Piton}[background-color=gray!30,rounded-corners,width=min,splittable=4]
% def square(x):
%     """Computes the square of x"""
%     return x*x
% def square(x):
%     """Computes the square of x"""
%     return x*x
% def square(x):
%     """Computes the square of x"""
%     return x*x
% def square(x):
%     """Computes the square of x"""
%     return x*x 
% def square(x):
%     """Computes the square of x"""
%     return x*x
% def square(x):
%     """Computes the square of x"""
%     return x*x
% def square(x):
%     """Computes the square of x"""
%     return x*x
% def square(x):
%     """Computes the square of x"""
%     return x*x
% def square(x):
%     """Computes the square of x"""
%     return x*x
% def square(x):
%     """Computes the square of x"""
%     return x*x
% def square(x):
%     """Computes the square of x"""
%     return x*x
% def square(x):
%     """Computes the square of x"""
%     return x*x
% \end{Piton}
% 
% \end{tcolorbox}
% 
% \subsection{Splitting of a listing in sub-listings}
%
% \index{split-on-empty-lines}
% \label{split-on-empty-lines}
% \index{split-separation}
%
% The extension \pkg{piton} provides the key \Definition{split-on-empty-lines},
% which should not be confused with the key |splittable-on-empty-lines|
% previously defined.
%
% \smallskip
% In order to understand the behaviour of the key |split-on-empty-lines|, one
% should imagine that he has to compose an informatic listing which contains
% several definitions of informatic functions. Usually, in the informatic
% languages, those definitions of functions are separated by empty lines.
%
% \smallskip
% The key |split-on-empty-lines| splits the listings on the empty lines. Several
% empty lines are deleted and replaced by the content of the parameter
% corresponding to the key \Definition{split-separation}.
% \begin{itemize}
% \item That parameter must contain elements allowed to be inserted in
% \emph{vertical mode} of TeX. For example, it's possible to put the TeX
% primitive |\hrule|.
%
% \item The initial value of this parameter is
% |\vspace{\baselineskip}\vspace{-1.25pt}| which corresponds eventually to an 
% empty line in the final \textsc{pdf} (this vertical space is deleted if it
% occurs on a page break). If the key |background-color| is in force, no
% background color is added to that empty line.
% \end{itemize}
%
%
% Each chunk of the informatic listing is composed in an environment whose name
% is given by the key \Definition{env-used-by-split}. The initial value of that
% parameter is, not surprisingly, |Piton| and, hence, the different chunks are
% composed in several environments |{Piton}|. If one decides to change the value
% of |env-used-by-split|, he should use the name of an environment created by
% |\NewPitonEnvironment| (cf.~part~\ref{NewPitonEnvironment},
% p.~\pageref{NewPitonEnvironment}).
%
% Each chunk of the informatic listing is formated in its own environment.
% Therefore, it has its own line numbering (if the key
% |line-numbers| is in force) and its own colored background (when the key
% |background-color| is in force), separated from the background color of the
% other chunks. When used, the key |splittable| applies in each chunk
% (independently of the other chunks). Of course, a page break may occur between
% the chunks of code, regardless of the value of |splittable|.
%
% \bigskip
% \begin{Verbatim}
% \begin{Piton}[~emphase#split-on-empty-lines@,background-color=gray!15,line-numbers]
% def square(x):
%     """Computes the square of x"""
%     return x*x
%
% def cube(x):
%     """Calcule the cube of x"""
%     return x*x*x
% \end{Piton}
% \end{Verbatim}
%
%
% \begin{Piton}[split-on-empty-lines,background-color=gray!15,line-numbers]
% def square(x):
%     """Computes the square of x"""
%     return x*x
%
% def cube(x):
%     """Calcule the cube of x"""
%     return x*x*x
% \end{Piton}
%
% \bigskip
% \textbf{Caution}: Since each chunk is treated independently of the others, the
% commands specified by |detected-commands| or |raw-detected-commands|
% (cf.~p.~\pageref{detected-commands}) and the commands and environments of
% Beamer automatically detected by \pkg{piton} must not cross the empty lines of
% the original listing.
%
% \subsection{Highlighting some identifiers}
%
% \label{SetPitonIdentifier}
%
% The command \DefinitionCommand{SetPitonIdentifier} allows to automatically
% change the formatting of some identifiers. That change is only based on the
% name of those indentifiers.
%
% \smallskip
% That command takes in three arguments: 
%
% \begin{itemize}
% \item The optional argument (within square brackets) specifies the informatic
% language. If this argument is not present, the tunings done by
% |\SetPitonIdentifier| will apply to all the computer languages of
% \pkg{piton}.\footnote{We recall, that, in the package \pkg{piton}, the
% names of the computer languages are case-insensitive.}
%
% \item The first mandatory argument is a comma-separated list of names of
% identifiers.
%
% \item The second mandatory argument is a list of LaTeX instructions of the
% same type as \pkg{piton} ``styles'' previously presented (cf.~\ref{styles}
% p.~\pageref{styles}). 
% \end{itemize}
% 
% \emph{Caution}: Only the identifiers may be concerned by that key. The
% keywords and the built-in functions won't be affected, even if their name
% appear in the first argument of the command |\SetPitonIdentifier|.
% 
% \begin{Verbatim}
% ~emphase#\SetPitonIdentifier{l1,l2}{\color{red}}@
% \begin{Piton}
% def tri(l):
%     """Segmentation sort"""
%     if len(l) <= 1:
%         return l
%     else:
%         a = l[0]
%         l1 = [ x for x in l[1:] if x < a  ]
%         l2 = [ x for x in l[1:] if x >= a ]
%         return tri(l1) + [a] + tri(l2)
% \end{Piton}
% \end{Verbatim}
%
%
% \bigskip
%
% \begingroup
%
% \SetPitonIdentifier{l1,l2}{\color{red}}
%
% \begin{Piton}
% def tri(l):
%     """Segmentation sort"""
%     if len(l) <= 1:
%         return l
%     else:
%         a = l[0]
%         l1 = [ x for x in l[1:] if x < a  ]
%         l2 = [ x for x in l[1:] if x >= a ]
%         return tri(l1) + [a] + tri(l2)
% \end{Piton}
%
% \endgroup
% 
% \bigskip
% By using the command |\SetPitonIdentifier|, it's possible to add other
% built-in functions (or other new keywords, etc.) that will be detected by
% \pkg{piton}.
%
%
% \begin{Verbatim}
% ~emphase#\SetPitonIdentifier[Python]@
%   {cos, sin, tan, floor, ceil, trunc, pow, exp, ln, factorial}
%   {\PitonStyle{Name.Builtin}}
%
% \begin{Piton}
% from math import *
% cos(pi/2) 
% factorial(5)
% ceil(-2.3) 
% floor(5.4) 
% \end{Piton}
% \end{Verbatim}
%
% \begingroup
%
% \SetPitonIdentifier[Python]
%   {cos, sin, tan, floor, ceil, trunc, pow, exp, ln, factorial}
%   {\PitonStyle{Name.Builtin}}
%
% \begin{Piton}
% from math import *
% cos(pi/2) 
% factorial(5)
% ceil(-2.3) 
% floor(5.4) 
% \end{Piton}
%
%
% \endgroup
%
%
% \subsection{Mechanisms to escape to LaTeX}
%
% \index{escapes to LaTeX}
% 
% The package \pkg{piton} provides several mechanisms for escaping to LaTeX:
% \begin{itemize}
% \item It's possible to compose comments entirely in LaTeX.
% \item It's possible to have the elements between \texttt{\$} in the comments
% composed in LateX mathematical mode.
% \item It's possible to ask \pkg{piton} to detect automatically some LaTeX
% commands, thanks to the keys |detected-commands|, |raw-detected-commands| and
% |vertical-detected-commands|.
% \item It's also possible to insert LaTeX code almost everywhere in a Python listing.
% \end{itemize}
%
% One should also remark that, when the extension \pkg{piton} is used with the
% class \cls{beamer}, \pkg{piton} detects in |{Piton}| many commands and
% environments of Beamer: cf. \ref{beamer} p.~\pageref{beamer}.
%
% \subsubsection{The ``LaTeX comments''}
%
% \index{comment-latex}
%
% In this document, we call ``LaTeX comments'' the comments which begins by
% |#>|. The code following those characters, until the end of the line, will be
% composed as standard LaTeX code. There is two tools to customize those
% comments.
%
% \begin{itemize}
% \item It's possible to change the syntactic mark (which, by default, is |#>|).
% For this purpose, there is a key \Definition{comment-latex} available only in
% the preamble of the document, allows to choice the characters which, preceded
% by |#|, will be the syntactic marker.
%
% For example, if the preamble contains the following instruction:
% 
% \quad \verb|\PitonOptions{comment-latex = LaTeX}|
%
% the LaTeX comments will begin by |#LaTeX|.
%
% If the key |comment-latex| is used with the empty value, all the Python
% comments (which begins by |#|) will, in fact, be ``LaTeX comments''.
%
% \smallskip
% \item It's possible to change the formatting of the LaTeX comment itself by
% changing the \pkg{piton} style |Comment.LaTeX|.
%
% For example, with |\SetPitonStyle{Comment.LaTeX = \normalfont\color{blue}}|,
% the LaTeX comments will be composed in blue.
%
% If you want to have a character |#| at the beginning of the LaTeX comment in
% the \textsc{pdf}, you can use set |Comment.LaTeX| as follows:
%
% \begin{Verbatim}
% \SetPitonStyle{Comment.LaTeX = \color{gray}\~#\normalfont\space }
% \end{Verbatim}
% 
% For other examples of customization of the LaTeX comments, see the part
% \ref{example-comments} p.~\pageref{example-comments}
% \end{itemize}
%
% \bigskip
% If the user has required line numbers (with the key |line-numbers|), it's
% possible to refer to a number of line with the command |\label| used in a
% LaTeX comment.\footnote{That feature is implemented by using a redefinition of
% the standard command \texttt{\textbackslash label} in the environments
% \texttt{\{Piton\}}. Therefore, incompatibilities may occur with extensions
% which redefine (globally) that command \texttt{\textbackslash label} (for
% example: \pkg{varioref}, \pkg{refcheck}, \pkg{showlabels}, etc.)}
%
% \subsubsection{The key ``math-comments''}
%
% \index{math-comments}
%
% It's possible to request that, in the standard Python comments (that is to say
% those beginning by |#| and not |#>|), the elements between \texttt{\$} be
% composed in LaTeX mathematical mode (the other elements of the comment being
% composed verbatim).
%
% That feature is activated by the key \Definition{math-comments}, \emph{which is
% available only in the preamble of the document}.
%
% \medskip
% Here is a example, where we have assumed that the preamble of the document
% contains the instruction |\PitonOptions{math-comment}|:
% 
% \begin{Verbatim}
% \begin{Piton}
% def square(x):
%     return x*x ~# compute $x^2$
% \end{Piton}
% \end{Verbatim}
%
% \begin{Piton}
% def square(x):
%     return x*x # compute $x^2$
% \end{Piton}
%
% \subsubsection{The key ``detected-commands'' and its variants}
%
% \index{raw-detected-commands (key)}
% \label{raw-detected-commands}
% \index{vertical-detected-commands (key)}
% \label{vertical-detected-commands}
% \index{detected-commands (key)}
% \label{detected-commands}
%
% The key \Definition{detected-commands} of |\PitonOptions| allows to specify a
% (comma-separated) list of names of LaTeX commands that will be detected 
% directly by \pkg{piton}.
%
% \begin{itemize}
% \item The key |detected-commands| must be used in the preamble of the LaTeX document.
% 
% \item The names of the LaTeX commands must appear without the leading
% backslash (eg. |detected-commands = { emph, textbf }|). 
%
% \item These commands must be LaTeX commands with only one (mandatory) argument
% between braces (and these braces must appear explicitly in the informatic
% listing). 
%
% \item These commands must be \textbf{protected}\footnote{We recall that the
% command \texttt{\textbackslash NewDocumentCommand} creates protected commands,
% unlike the historical LaTeX command \texttt{\textbackslash newcommand} (and
% unlike the command \texttt{\textbackslash def} of TeX).} against expansion in
% the TeX sens (because the command |\piton| expands its arguments before
% throwing it to Lua for syntactic analysis).
% \end{itemize}
%
% \bigskip
% In the following example, which is a recursive programmation of the factorial
% function, we decide to highlight the recursive call. The command |\highLight|
% of \pkg{lua-ul}\footnote{The package \pkg{lua-ul} requires itself the package
% \pkg{luacolor}.} directly does the job with the easy syntax |\highLight{...}|.
%
% \medskip 
% We assume that the preamble of the LaTeX document contains the following line:
% \begin{Verbatim}
% \PitonOptions{~emphase#detected-commands@ = highLight}
% \end{Verbatim}
%
% Then, it's possible to write directly:
% \begin{Verbatim}
% \begin{Piton}
% def fact(n):
%     if n==0:
%         return 1
%     else:
%         ~emphase#\highLight@{return n*fact(n-1)}
% \end{Piton}
% \end{Verbatim}
%
% \begin{Piton}
% def fact(n):
%     if n==0:
%         return 1
%     else:
%         \highLight{return n*fact(n-1)}
% \end{Piton}
%
% \bigskip
% The key \Definition{raw-detected-commands} is similar to the key
% |detected-commands| but \pkg{piton} won't do any syntactic analysis of the
% arguments of the LaTeX commands which are detected.
%
% If there is a line break within the argument of a command detected by the mean
% of |raw-detected-commands|, that line break is replaced by a space (as does
% LaTeX by default).
%
% \medskip
% Imagine, for example, that we wisth, in the main text of a document, introduce
% some specifications of tables of the language \textsc{sql} by the the name of the table,
% followed, between brackets, by the names of its fields (ex. : |client(name,town)|).
%
% If we insert that element in a command |\piton|, the word \emph{client} won't be
% recognized as a name of table but as a name of field. It's possible to define
% a personal command |\NomTable|
% which we will apply by hand to the names of the tables. In that aim, we declare that
% command with |raw-detected-commands| and, thus, its argument won't be re-analyzed
% by \pkg{piton} (that second analysis would format it as a name of field).
%
% \medskip
% In the preamble of the LaTeX document, we insert the following lines:
%
% \begin{Verbatim}
% \NewDocumentCommand{\NameTable}{m}{{\PitonStyle{Name.Table}{~#1}}}
% \PitonOptions{language=SQL, ~emphase#raw-detected-commands = NameTable@}
% \end{Verbatim}
% 
% In the main document, the instruction:
%
% \begin{Verbatim}
% Exemple : \piton{\NameTable{client} (name, town)}
% \end{Verbatim}
%
% \NewDocumentCommand{\NameTable}{m}{{\PitonStyle{Name.Table}{#1}}}
%
% produces the following output :
%
% \smallskip
% \begingroup
% \PitonOptions{language=SQL}%
% Exemple : \piton{\NameTable{client} (nom, prénom)}
% \endgroup
%
% \bigskip
% \colorbox{yellow!50}{\textbf{New 4.6}}
%
% \nobreak
% \smallskip
% The key \Definition{vertical-detected-commands} is similar to the key |raw-detected-commands|
% but the commands which are detected by this key must be LaTeX commands (with one argument)
% which are executed in \emph{vertical} mode between the lines of the code.
%
% \smallskip
% For example, it's possible to detect the command |\newpage| by
%
% \begin{Verbatim}
% \PitonOptions{vertical-detected-commands = newpage}
% \end{Verbatim}
%
% and ask in a listing a mandatory break of page with |\newgage{}|:
%
% \begin{Verbatim}
% \begin{Piton}
% def square(x):
%     return x*x  ~emphase#\newpage{}@
% def cube(x):
%     return x*x*x
% \end{Piton}
% \end{Verbatim}
% 
% \subsubsection{The mechanism ``escape''}
%
% \label{escape}
%
% It's also possible to overwrite the informatic listings to insert LaTeX code
% almost everywhere (but between lexical units, of course). By default,
% \pkg{piton} does not fix any delimiters for that kind of escape.
%
% In order to use this mechanism, it's necessary to specify the delimiters which
% will delimit the escape (one for the beginning and one for the end) by using
% the keys \Definition{begin-escape} and \Definition{end-escape}, \emph{available only
% in the preamble of the document}.
%
% \medskip
% We consider once again the previous example of a recursive programmation of
% the factorial. We want to highlight in pink the instruction containing the
% recursive call. With the package \pkg{lua-ul}, we can use the syntax
% |\highLight[LightPink]{...}|. Because of the optional argument between square
% brackets, it's not possible to use the key |detected-commands| but it's
% possible to achieve our goal with the more general mechanism ``escape''.
%
% \medskip
% We assume that the preamble of the document contains the following
% instruction:
%
% \begin{Verbatim}
% \PitonOptions{~emphase#begin-escape=!,end-escape=!@}
% \end{Verbatim}
%
% \medskip
% Then, it's possible to write:
% \begin{Verbatim}
% \begin{Piton}
% def fact(n):
%     if n==0:
%         return 1
%     else:
%         ~emphase#!\highLight[LightPink]{!@return n*fact(n-1)~emphase#!}!@
% \end{Piton}
% \end{Verbatim}
%
%    \begin{Piton}
% def fact(n):
%     if n==0:
%         return 1
%     else:
%         !\highLight[LightPink]{!return n*fact(n-1)!}!
%    \end{Piton}
%
%
%
% \bigskip
% \emph{Caution} : The mechanism ``escape'' is not active in the strings nor in
% the Python comments (however, it's possible to have a whole Python comment
% composed in LaTeX by beginning it with |#>|; such comments are merely called
% ``LaTeX comments'' in this document).
% 
%
% \subsubsection{The mechanism ``escape-math''}
% 
% The mechanism ``escape-math'' is very similar to the mechanism ``escape''
% since the only difference is that the elements sent to LaTeX are composed in
% the math mode of LaTeX.
%
% This mechanism is activated with the keys \Definition{begin-escape-math} and
% \Definition{end-escape-math} (\emph{which are available only in the preamble of the
% document}). 
%
% Despite the technical similarity, the use of the the mechanism
% ``escape-math'' is in fact rather different from that of the mechanism
% ``escape''. Indeed, since the elements are composed in a mathematical mode
% of LaTeX, they are, in particular, composed within a TeX group and, therefore,
% they can't be used to change the formatting of other lexical units.
%
% In the languages where the character \verb|$| does not play a important role,
% it's possible to activate that mechanism ``escape-math'' with the character
% \verb|$|:
% \begin{Verbatim}
% \PitonOptions{~emphase#begin-escape-math=$,end-escape-math=$@}
% \end{Verbatim}
% Note: the character \verb|$| must \emph{not} be protected by a backslash.
%
% \bigskip
% However, it's probably more prudent to use |\(| et |\)|, which are delimiters
% of the mathematical mode provided by LaTeX.
% \begin{Verbatim}
% \PitonOptions{~emphase#begin-escape-math=\(,end-escape-math=\)@}
% \end{Verbatim}
% 
% \bigskip
% Here is an example of use.
%
% \medskip
% \begin{Verbatim}
% \begin{Piton}[line-numbers]
% def arctan(x,n=10):
%     if ~emphase#\(x < 0\)@ :
%         return ~emphase#\(-\arctan(-x)\)@ 
%     elif ~emphase#\(x > 1\)@ : 
%         return ~emphase#\(\pi/2 - \arctan(1/x)\)@ 
%     else: 
%         s = ~emphase#\(0\)@
%         for ~emphase#\(k\)@ in range(~emphase#\(n\)@): s += ~emphase#\(\smash{\frac{(-1)^k}{2k+1} x^{2k+1}}\)@
%         return s
% \end{Piton}
% \end{Verbatim}
%
%
% \bigskip
%
% \begin{Piton}[line-numbers]
% def arctan(x,n=10):
%     if \(x < 0\) :
%         return \(-\arctan(-x)\) 
%     elif \(x > 1\) : 
%         return \(\pi/2 - \arctan(1/x)\) 
%     else: 
%         s = \(0\)
%         for \(k\) in range(\(n\)): s += \(\smash{\frac{(-1)^k}{2k+1} x^{2k+1}}\)
%         return s
% \end{Piton}
% 
%
% \subsection{Behaviour in the class Beamer}
%
% \label{beamer}
% \index{Beamer@\cls{Beamer} (class)}
%
% \emph{First remark}
%
% \nobreak
% Since the environment |{Piton}| catches its body with a verbatim mode, it's
% necessary to use the environments |{Piton}| within environments |{frame}| of
% Beamer protected by the key |fragile|, i.e. beginning with
% |\begin{frame}[fragile]|.\footnote{Remind that for an environment
% |{frame}| of Beamer using the key |fragile|, the instruction
% |\end{frame}| must be alone on a single line (except
% for any leading whitespace).}
%
%
% \bigskip
% When the package \pkg{piton} is used within the class
% \cls{beamer}\footnote{The extension \pkg{piton} detects the class \cls{beamer}
% and the package \pkg{beamerarticle} if it is loaded previously
% but, if needed, it's also possible to activate that mechanism with the key
% |beamer| provided by \pkg{piton} at load-time:
% |\usepackage[beamer]{piton}|}, the behaviour of \pkg{piton} is slightly 
% modified, as described now.
%
% \subsubsection{\{Piton\} and \textbackslash PitonInputFile are
% ``overlay-aware''}
%
% When \pkg{piton} is used in the class \cls{beamer}, the environment |{Piton}|
% and the command |\PitonInputFile| accept the optional argument |<...>| of
% Beamer for the overlays which are involved.
%
% For example, it's possible to write:
%
% \begin{Verbatim}
% \begin{Piton}~emphase#<2-5>@
% ...
% \end{Piton}
% \end{Verbatim}
%
% and 
%
% \begin{Verbatim}
% \PitonInputFile~emphase#<2-5>@{my_file.py}
% \end{Verbatim}
% 
% \subsubsection{Commands of Beamer allowed in \{Piton\} and \textbackslash PitonInputFile}
%
% \index{detected-beamer-commands}
%
% When \pkg{piton} is used in the class \cls{beamer}, the following commands of
% \cls{beamer} (classified upon their number of arguments) are automatically
% detected in the environments |{Piton}| (and in the listings processed by
% |\PitonInputFile|):
% \begin{itemize}
% \item no mandatory argument : |\pause|\footnote{One should remark that it's
% also  possible to use the command |\pause| in a ``LaTeX
% comment'', that is to say by writing |#> \pause|. By
% this way, if the Python code is copied, it's still executable by Python}.  ;
% \item one mandatory argument : |\action|, |\alert|, |\invisible|, |\only|,
% |\uncover| and |\visible| ; \newline
% % It's possible to add new commands to that list with the key
% \Definition{detected-beamer-commands} (the names of the commands must
% \emph{not} be preceded by a backslash). 
% \item two mandatory arguments : |\alt| ; 
% \item three mandatory arguments  : |\temporal|.
% \end{itemize}

% \medskip 
% These commands must be used preceded and following by a space. In the
% mandatory arguments of these commands, the braces must be balanced. However,
% the braces included in short strings\footnote{The short strings of Python are
% the strings delimited by characters |'| or the characters |"|
% and not |'''| nor |"""|. In Python, the short strings can't
% extend on several lines.} of Python are not considered.
%
% \medskip
% Regarding the functions |\alt| and |\temporal| there should be no carriage
% returns in the mandatory arguments of these functions.
%
% \medskip
% Here is a complete example of file:
%
% \begin{Verbatim}[formatcom = \small\color{gray}]
% \documentclass{beamer}
% \usepackage{piton}
% \begin{document}
% \begin{frame}[fragile]
% \begin{Piton}
% def string_of_list(l):
%     """Convert a list of numbers in string"""
% ~emphase#    \only<2->{s = "{" + str(l[0])}@
% ~emphase#    \only<3->{for x in l[1:]: s = s + "," + str(x)}@
% ~emphase#    \only<4->{s = s + "}"}@
%     return s
% \end{Piton}
% \end{frame}
% \end{document}
% \end{Verbatim}
%
% In the previous example, the braces in the Python strings |"{"| and |"}"| are
% correctly interpreted (without any escape character).
% 
% 
%
%
% \bigskip
% \subsubsection{Environments of Beamer allowed in \{Piton\} and \textbackslash PitonInputFile}
%
% When \pkg{piton} is used in the class \pkg{beamer}, the following environments
% of Beamer are directly detected in the environments |{Piton}| (and in the
% listings processed by |\PitonInputFile|): |{actionenv}|, |{alertenv}|,
% |{invisibleenv}|, |{onlyenv}|, |{uncoverenv}| and |{visibleenv}|.
%
% \smallskip
% \index{detected-beamer-environments}
% It's possible to add new environments to that list with the key
% \Definition{detected-beamer-environments}. 
%
% \smallskip
%
% However, there is a restriction: these environments must contain only \emph{whole
% lines of Python code} in their body. The instructions |\begin{...}| and
% |\end{...}| must be alone on their lines.
%
%\medskip
% Here is an example:
%
% \begin{Verbatim}[formatcom = \small\color{gray}]
% \documentclass{beamer}
% \usepackage{piton}
% \begin{document}
% \begin{frame}[fragile]
% \begin{Piton}
% def square(x):
%     """Compure the square of its argument"""
%     ~emphase#\begin{uncoverenv}<2>@
%     return x*x
%     ~emphase#\end{uncoverenv}@
% \end{Piton}
% \end{frame}
% \end{document}
% \end{Verbatim}
%
%
% \vspace{1cm}
% \textbf{Remark concerning the command \textbackslash alert and the environment
% \{alertenv\} of Beamer}
%
% \nobreak
% \smallskip
% Beamer provides an easy way to change the color used by the environment
% |{alertenv}| (and by the command |\alert| which relies upon it) to highlight
% its argument. Here is an example:
%
% \begin{Verbatim}
% \setbeamercolor{~emphase#alerted text@}{fg=blue} 
% \end{Verbatim}
%
% However, when used inside an environment |{Piton}|, such tuning will probably
% not be the best choice because \pkg{piton} will, by design, change (most of
% the time) the color the different elements of text. One may prefer an environment
% |{alertenv}| that will change the background color for the elements to be
% highlighted. 
%
% \smallskip
% Here is a code that will do that job and add a yellow background. That code
% uses the command |\@highLight| of \pkg{lua-ul} (that extension requires also
% the package \pkg{luacolor}).
%
% \begingroup
% \fvset{commandchars=\~\#\+,formatcom=\color{gray}}
% \begin{Verbatim}
% \setbeamercolor{alerted text}{bg=yellow!50}
% \makeatletter
% \AddToHook{env/Piton/begin}
%   {\renewenvironment<>{alertenv}{\only~#1{~emphase#\@highLight+[alerted text.bg]}}{}}
% \makeatother
% \end{Verbatim}
% \endgroup
%
% That code redefines locally the environment |{alertenv}| within the
% environments |{Piton}| (we recall that the command |\alert| relies upon that
% environment |{alertenv}|).
%
% 
% \subsection{Footnotes in the environments of piton}
%
%  \index{footnote@\pkg{footnote} (extension)}
% \index{footnote (key)}
% \index{footnotehyper@\pkg{footnotehyper} (extension)}
% \index{footnotehyper (key)}
%
% \label{footnote}
% If you want to put footnotes in an environment |{Piton}| or
% (or, more unlikely, in a listing produced by |\PitonInputFile|), you can use a
% pair |\footnotemark|--|\footnotetext|. 
%
% \smallskip
% However, it's also possible to extract the footnotes with the help of the
% package \pkg{footnote} or the package \pkg{footnotehyper}.
%
% \smallskip
% If \pkg{piton} is loaded with the option \Definition{footnote} (with
% |\usepackage[footnote]{piton}| or with |\PassOptionsToPackage|), the
% package \pkg{footnote} is loaded (if it is not yet loaded) and it is used to
% extract the footnotes.
%
% \smallskip
% If \pkg{piton} is loaded with the option \Definition{footnotehyper}, the
% package \pkg{footnotehyper} is loaded (if it is not yet loaded) ant it is used
% to extract footnotes.
%
% \smallskip
% Caution: The packages \pkg{footnote} and \pkg{footnotehyper} are incompatible.
% The package \pkg{footnotehyper} is the successor of the package \pkg{footnote}
% and should be used preferently. The package \pkg{footnote} has some drawbacks,
% in particular: it must be loaded after the package \pkg{xcolor} and it is not
% perfectly compatible with \pkg{hyperref}.
%
%
% \medskip
% \textbf{Important remark} : If you use Beamer, you should know taht Beamer has
% its own system to extract the footnotes. Therefore, \pkg{piton} must be loaded
% in that class without the option |footnote| nor the option |footnotehyper|.
% 
%
% \bigskip
% By default, in an environment |{Piton}|, a command |\footnote| may appear only
% within a ``LaTeX comment''. But it's also possible to add the command
% |\footnote| to the list of the ``\emph{detected-commands}''
% (cf.~part~\ref{detected-commands}, p.~\pageref{detected-commands}). 

% \medskip 
% In this document, the package \pkg{piton} has been loaded with the
% option |footnotehyper| dans we added the command |\footnote| to the list of
% the ``\emph{detected-commands}'' with the following instruction in the
% preamble of the LaTeX document.
%
% \qquad \verb|\PitonOptions{detected-commands = footnote}|
%
%
% \begingroup
% \fvset{commandchars=\~\&\@,formatcom=\small\color{gray}}
% \begin{Verbatim}
% \PitonOptions{background-color=gray!15}
% \begin{Piton}
% def arctan(x,n=10):
%     if x < 0:
%         return -arctan(-x)~emphase&\footnote{First recursive call.}]@
%     elif x > 1: 
%         return pi/2 - arctan(1/x)~emphase&\footnote{Second recursive call.}@
%     else: 
%         return sum( (-1)**k/(2*k+1)*x**(2*k+1) for k in range(n) ) 
% \end{Piton}
% \end{Verbatim}
% \endgroup
%
% \begingroup
% \PitonOptions{background-color=gray!15}
% \begin{Piton}
% def arctan(x,n=10):
%     if x < 0:
%         return -arctan(-x)\footnote{First recursive call.}
%     elif x > 1: 
%         return pi/2 - arctan(1/x)\footnote{Second recursive call.}
%     else: 
%         return sum( (-1)**k/(2*k+1)*x**(2*k+1) for k in range(n) ) 
% \end{Piton}
% \endgroup
%
%
% \vspace{1cm}
%
% If an environment |{Piton}| is used in an environment |{minipage}| of LaTeX,
% the notes are composed, of course, at the foot of the environment
% |{minipage}|. Recall that such |{minipage}| can't be broken by a page break.
%
%
% \begingroup
% \fvset{commandchars=\~\&\@,formatcom=\small\color{gray}}
% \begin{Verbatim}
% \PitonOptions{background-color=gray!15}
% ~emphase&\begin{minipage}{\linewidth}@
% \begin{Piton}
% def arctan(x,n=10):
%     if x < 0:
%         return -arctan(-x)~emphase&\footnote{First recursive call.}@
%     elif x > 1: 
%         return pi/2 - arctan(1/x)~emphase&\footnote{Second recursive call.}@
%     else: 
%         return sum( (-1)**k/(2*k+1)*x**(2*k+1) for k in range(n) ) 
% \end{Piton}
% ~emphase&\end{minipage}@
% \end{Verbatim}
% \endgroup
%
% \begingroup
% \PitonOptions{background-color=gray!15}
% \begin{minipage}{\linewidth}
% \begin{Piton}
% def arctan(x,n=10):
%     if x < 0:
%         return -arctan(-x)\footnote{First recursive call.}
%     elif x > 1: 
%         return pi/2 - arctan(1/x)\footnote{Second recursive call.}
%     else: 
%         return sum( (-1)**k/(2*k+1)*x**(2*k+1) for k in range(n) ) 
% \end{Piton}
% \end{minipage}
% \endgroup
%
%
% \subsection{Tabulations}
%
% \index{tabulations}
% \index{tab-size}
%
% \smallskip 
% Even though it's probably recommended to indent the informatics listings with
% spaces and not tabulations\footnote{For the language Python, see the note %
% PEP~8.}, \pkg{piton} accepts the characters of tabulation (that is to say the 
% characters U+0009) at the beginning of the lines. Each character U+0009 is
% replaced by $n$~spaces. The initial value of $n$ is $4$ but it's possible to
% change it with the key \Definition{tab-size} of |\PitonOptions|.
%
% \smallskip
% There exists also a key \Definition{tabs-auto-gobble} which computes the
% minimal value $n$ of the number of consecutive characters U+0009 beginning
% each (non empty) line of the environment |{Piton}| and applies |gobble| with
% that value of~$n$ (before replacement of the tabulations by spaces, of
% course). Hence, that key is similar to the key |auto-gobble| but acts on
% U+0009 instead of U+0020 (spaces).
%
% The key |env-gobble| is not compatible with the tabulations.
%
%
% \bigskip
% \section{API for the developpers}
%
% \label{API}
% 
% The L3 variable \DefinitionCommand{l_piton_language_str} contains the name of
% the current language of \pkg{piton} (in lower case).
%
% \bigskip
%
% The extension \pkg{piton} provides a Lua function
% \Definition{piton.get_last_code} without argument which returns the code in
% the latest environment of \pkg{piton}.
% \begin{itemize}
% \item The carriage returns (which are present in the initial environment)
% appears as characters |\r| (i.e. U+000D).
%
% \item The code returned by |piton.get_last_code()| takes into account the
% potential application of a key |gobble|, |auto-gobble| or |env-gobble|
% (cf.~p.~\pageref{gobble}). 
%
% \item The extra formatting elements added in the code are deleted in the code
% returned by |piton.get_last_code()|. That concerns the LaTeX commands declared
% by the key |detected-commands| and its variants (cf.
% part~\ref{detected-commands}) and the elements inserted by the mechanism
% ``|escape|'' (cf. part~\ref{escape}).
%
% \item |piton.get_last_code| is a Lua function and not a Lua string: the
% treatments outlined above are executed when the function is called. Therefore,
% it might be judicious to store the value returned by |piton.get_last_code()|
% in a variable of Lua if it will be used several times.
% \end{itemize}
% 
% \medskip
% For an example of use, see the part concerning |pyluatex|, part~\ref{pyluatex},
% p.~\pageref{pyluatex}. 
%
%
% \section{Examples}
%
%
% \subsection{An example of tuning of the styles}
%
% The graphical styles have been presented in the section \ref{styles},
% p.~\pageref{styles}.
%
% \smallskip
% We present now an example of tuning of these styles adapted to the documents
% in black and white.
%
% That tuning uses the command |\highLight| of \pkg{lua-ul} (that package
% requires itself the package \pkg{luacolor}).
%
% \begin{Verbatim}
% \SetPitonStyle
%   {
%     Number = ,
%     String = \itshape , 
%     String.Doc = \color{gray} \slshape ,
%     Operator = , 
%     Operator.Word = \bfseries ,
%     Name.Builtin = ,
%     Name.Function = \bfseries \highLight[gray!20] ,
%     Comment = \color{gray} ,
%     Comment.LaTeX = \normalfont \color{gray},
%     Keyword = \bfseries ,
%     Name.Namespace = ,
%     Name.Class = , 
%     Name.Type = ,
%     InitialValues = \color{gray}
%   }
% \end{Verbatim}
%
% In that tuning, many values given to the keys are empty: that means that the
% corresponding style won't insert any formatting instruction (the element will
% be composed in the standard color, usually in black, etc.). Nevertheless,
% those entries are mandatory because the initial value of those keys in
% \pkg{piton} is \emph{not} empty.
%
% \begingroup
%  \PitonOptions{splittable}
%
% \SetPitonStyle
%   {
%     Number = ,
%     String = \itshape , 
%     String.Doc = \color{gray} \slshape ,
%     Operator.Word = \bfseries ,
%     Operator = , 
%     Name.Builtin = ,
%     Name.Function = \bfseries \highLight[gray!20] ,
%     Comment = \color{gray} ,
%     Comment.LaTeX = \normalfont \color{gray} ,
%     Keyword = \bfseries ,
%     Name.Namespace = ,
%     Name.Class = , 
%     Name.Type = ,
%     InitialValues = \color{gray}
%   }
%
%
% \bigskip
%
% \begin{Piton}
% from math import pi
%
% def arctan(x,n=10):
%     """Compute the mathematical value of arctan(x)
%                                               
%     n is the number of terms in the sum
%     """
%     if x < 0:
%         return -arctan(-x) # recursive call
%     elif x > 1: 
%         return pi/2 - arctan(1/x) 
%         #> (we have used that $\arctan(x)+\arctan(1/x)=\pi/2$ for $x>0$)
%     else: 
%         s = 0
%         for k in range(n):
%             s += (-1)**k/(2*k+1)*x**(2*k+1)
%         return s 
% \end{Piton}
%
% \endgroup
%
%
% \subsection{Line numbering}
%
% \label{example-numbering}
% \index{numbers of the lines de code|emph}
%
% We remind that it's possible to have an automatic numbering of the lines in
% the informatic listings by using the key |line-numbers| (used without value).
%
% By default, the numbers of the lines are composed by \pkg{piton} in an
% overlapping position on the left (by using internally the command |\llap| of LaTeX).
%
% In order to avoid that overlapping, it's possible to use the option |left-margin=auto|
% which will insert automatically a margin adapted to the numbers of lines that
% will be written (that margin is larger when the numbers are greater than~10).
%
%
% \begingroup
% \fvset{commandchars=\~\&\@,formatcom=\small\color{gray}}
% \begin{Verbatim}
% ~emphase&\PitonOptions{background-color=gray!15, left-margin = auto, line-numbers}@
% \begin{Piton}
% def arctan(x,n=10):
%     if x < 0:
%         return -arctan(-x)        #> (recursive call)
%     elif x > 1: 
%         return pi/2 - arctan(1/x) #> (other recursive call) 
%     else: 
%         return sum( (-1)**k/(2*k+1)*x**(2*k+1) for k in range(n) ) 
% \end{Piton}
% \end{Verbatim}
% \endgroup
%
%
%
% \begingroup
% \PitonOptions{background-color=gray!15,left-margin = auto, line-numbers}
% \begin{Piton}
% def arctan(x,n=10):
%     if x < 0:
%         return -arctan(-x)        #> (recursive call)
%     elif x > 1: 
%         return pi/2 - arctan(1/x) #> (other recursive call) 
%     else: 
%         return sum( (-1)**k/(2*k+1)*x**(2*k+1) for k in range(n) ) 
% \end{Piton}
% \endgroup
%
%
%
% \bigskip
% \subsection{Formatting of the LaTeX comments}
%
% \label{example-comments}
%
% It's possible to modify the style |Comment.LaTeX| (with |\SetPitonStyle|) in
% order to display the LaTeX comments (which begin with |#>|) aligned on the
% right margin.
%
%
% \begingroup
% \fvset{commandchars=\~\&\@,formatcom=\small\color{gray}}
% \begin{Verbatim}
% \PitonOptions{background-color=gray!15}
% ~emphase&\SetPitonStyle{Comment.LaTeX = \hfill \normalfont\color{gray}}@
% \begin{Piton}
% def arctan(x,n=10):
%     if x < 0:
%         return -arctan(-x)        #> recursive call
%     elif x > 1: 
%         return pi/2 - arctan(1/x) #> other recursive call 
%     else: 
%         return sum( (-1)**k/(2*k+1)*x**(2*k+1) for k in range(n) ) 
% \end{Piton}
% \end{Verbatim}
% \endgroup
%
% \begingroup
% \PitonOptions{background-color=gray!15}
% \SetPitonStyle{Comment.LaTeX = \hfill \normalfont\color{gray}}
% \begin{Piton}
% def arctan(x,n=10):
%     if x < 0:
%         return -arctan(-x)        #> recursive call
%     elif x > 1: 
%         return pi/2 - arctan(1/x) #> another recursive call 
%     else: 
%         return sum( (-1)**k/(2*k+1)*x**(2*k+1) for k in range(n) ) 
% \end{Piton}
% \endgroup
%
%
% \vspace{1cm}
% It's also possible to display these LaTeX comments in a kind of second column
% by limiting the width of the Python code with the key |width|. 
%
% \begingroup
% \fvset{commandchars=\~\&\@,formatcom=\small\color{gray}}
% \begin{Verbatim}
% \PitonOptions{background-color=gray!15, width=9cm}
% ~emphase&\NewDocumentCommand{\MyLaTeXCommand}{m}{\hfill \normalfont\itshape\rlap{\quad #1}}@
% ~emphase&\SetPitonStyle{Comment.LaTeX = \MyLaTeXCommand}@
% \begin{Piton}
% def arctan(x,n=10):
%     if x < 0:
%         return -arctan(-x) #> recursive call
%     elif x > 1: 
%         return pi/2 - arctan(1/x) #> another recursive call 
%     else: 
%         s = 0
%         for k in range(n):
%              s += (-1)**k/(2*k+1)*x**(2*k+1)
%         return s 
% \end{Piton}
% \end{Verbatim}
% \endgroup
%
%
%
% \begingroup
% \PitonOptions{background-color=gray!15, width=9cm}
% \NewDocumentCommand{\MyLaTeXCommand}{m}{\hfill \normalfont\itshape\rlap{\quad #1}}
% \SetPitonStyle{Comment.LaTeX = \MyLaTeXCommand}
% \begin{Piton}
% def arctan(x,n=10):
%     if x < 0:
%         return -arctan(-x) #> recursive call
%     elif x > 1: 
%         return pi/2 - arctan(1/x) #> another recursive call 
%     else: 
%         s = 0
%         for k in range(n):
%              s += (-1)**k/(2*k+1)*x**(2*k+1)
%         return s 
% \end{Piton}
% \endgroup
%
%
% 
%
%
% \subsection{Use with tcolorbox}
%
% \label{examples-tcolorbox}
%
% The key |tcolorbox| of \pkg{piton} has been presented at the page
% \pageref{tcolorbox}.
%
% If, when that key is used, we wish to customize the graphical box created by
% \pkg{tcolorbox} (with the keys provided by \pkg{tcolorbox}), we should use the
% command |\tcbset| provided by \pkg{tcolorbox}. In order to limit the scope of
% the settings done by that command, the best way is to create a new environment
% with the dedicated command |\NewPitonEnvironment|
% (cf.~p.~\pageref{NewPitonEnvironment}). That environment with contain the
% settings done by \pkg{piton} (with |\PitonOptions|) and those done by
% \pkg{tcolorbox} (with |\tcbset|).
%
%
% \bigskip Here is an example of such environment |{Python}| with a colored
% column on the left for the numbers of lines. That example requires the library
% |skins| of \pkg{tcolorbox} to be loaded in the preamble of the LaTeX document
% with the instruction |\tcbuselibrary{skins}| (in order to be able to use the
% key |enhanced|).
% 
%
% \begin{Verbatim}
% \NewPitonEnvironment{Python}{m}
%   {%
%     \PitonOptions
%       {
%         ~emphase#tcolorbox@,
%         splittable=3,
%         width=min,
%         line-numbers,       % activate the numbers of lines
%         line-numbers =      % tuning for the numbers of lines
%          {
%            format = \footnotesize\color{white}\sffamily ,
%            sep = 2.5mm
%          }
%       }%
%     \tcbset
%       {
%         enhanced,
%         title=~#1,
%         fonttitle=\sffamily,
%         left = 6mm,
%         top = 0mm,
%         bottom = 0mm,
%         overlay=
%          {%
%             \begin{tcbclipinterior}%
%                 \fill[gray!80]
%                     (frame.south west) rectangle
%                     ([xshift=6mm]frame.north west); 
%             \end{tcbclipinterior}%
%          }
%       } 
%   } 
%   { }
% \end{Verbatim}
%
% \bigskip
% In the following example of use, we have illustrated the fact that it is possible
% to impose a break of page in such environment with |\newpage{}| if we have required the
% detection of the LaTeX command |\newpage| with the key |vertical-detected-commands|
% (cf.~p.~\pageref{vertical-detected-commands}) in the premable of the LaTeX document.
%
% Remark that we must use |\newpage{}| and not |\newpage| because the LaTeX commands
% detected by \pkg{piton} are meant to be commands with one argument (between curly
% braces).
%
% \begin{Verbatim}[formatcom=\footnotesize\color{gray}]
% \PitonOptions{vertical-detected-commands = newpage} % in the premable
% \end{Verbatim}
%
% \medskip
% \begin{Verbatim}
% ~emphase#\begin{Python}@{My example}
% def square(x):
%     """Computes the square of x"""
%     return x*x
% def square(x):
%     """Computes the square of x"""
%     return x*x
% def square(x):
%     """Computes the square of x"""
%     return x*x
% def square(x):
%     """Computes the square of x"""
%     return x*x ~emphase#\newpage{}@
% def square(x):
%     """Computes the square of x"""
%     return x*x
% ...    
% def square(x):
%     """Computes the square of x"""
%     return x*x
% ~emphase#\end{Python}@
% \end{Verbatim}
%
% \bigskip
%
% \RenewPitonEnvironment{Python}{m}
%   {%
%     \PitonOptions
%       {
%         tcolorbox,
%         splittable=3,
%         width=min,
%         line-numbers,
%         line-numbers =
%          {
%            format = \footnotesize\color{white}\sffamily ,
%            sep = 2.5mm
%          }
%       }%
%     \tcbset
%       {
%         enhanced,
%         title=#1,
%         fonttitle=\sffamily,
%         left = 6mm,
%         top = 0mm,
%         bottom = 0mm,
%         overlay=
%          {%
%             \begin{tcbclipinterior}%
%                 \fill[gray!80]
%                     (frame.south west) rectangle
%                     ([xshift=6mm]frame.north west); 
%             \end{tcbclipinterior}%
%          }
%       } 
%   } 
%   { }
%  
%
% \begin{Python}{My example}
% def square(x):
%     """Computes the square of x"""
%     return x*x
% def square(x):
%     """Computes the square of x"""
%     return x*x
% def square(x):
%     """Computes the square of x"""
%     return x*x
% def square(x):
%     """Computes the square of x"""
%     return x*x \newpage{}
% def square(x):
%     """Computes the square of x"""
%     return x*x
% def square(x):
%     """Computes the square of x"""
%     return x*x
% def square(x):
%     """Computes the square of x"""
%     return x*x
% def square(x):
%     """Computes the square of x"""
%     return x*x
% def square(x):
%     """Computes the square of x"""
%     return x*x
% def square(x):
%     """Computes the square of x"""
%     return x*x
% def square(x):
%     """Computes the square of x"""
%     return x*x
% def square(x):
%     """Computes the square of x"""
%     return x*x
% def square(x):
%     """Computes the square of x"""
%     return x*x
% def square(x):
%     """Computes the square of x"""
%     return x*x
% def square(x):
%     """Computes the square of x"""
%     return x*x
% def square(x):
%     """Computes the square of x"""
%     return x*x
% def square(x):
%     """Computes the square of x"""
%     return x*x
% def square(x):
%     """Computes the square of x"""
%     return x*x
% def square(x):
%     """Computes the square of x"""
%     return x*x
% def square(x):
%     """Computes the square of x"""
%     return x*x
% def square(x):
%     """Computes the square of x"""
%     return x*x
% def square(x):
%     """Computes the square of x"""
%     return x*x
% def square(x):
%     """Computes the square of x"""
%     return x*x
% def square(x):
%     """Computes the square of x"""
%     return x*x
% def square(x):
%     """Computes the square of x"""
%     return x*x
% def square(x):
%     """Computes the square of x"""
%     return x*x
% def square(x):
%     """Computes the square of x"""
%     return x*x
% def square(x):
%     """Computes the square of x"""
%     return x*x
% \end{Python}
%
%
%
% \bigskip
% \subsection{Use with pyluatex}
%
% \index{pyluatex@{\pkg{pyluatex}} (extension)}
% \label{pyluatex}
%
% The package \pkg{pyluatex} is an extension which allows the execution of some
% Python code from |lualatex| (as long as Python is installed on the machine
% and that the compilation is done with |lualatex| and |--shell-escape|).
%
% Here is, for example, an environment |{PitonExecute}| which formats a Python
% listing (with \pkg{piton}) but also displays the output of the execution of the
% code with Python.
% 
% \medskip
% \begin{Verbatim}
% \NewPitonEnvironment{~emphase#PitonExecute@}{!O{}}
%   {\PitonOptions{~#1}}
%   {\begin{center}
%    \directlua{pyluatex.execute(piton.get_last_code(), false, true, false, true)}%
%    \end{center}
%    \ignorespacesafterend}
% \end{Verbatim}
%
% \medskip
% We have used the Lua function |piton.get_last_code| provided in the API of
% \pkg{piton} : cf.~part~\ref{API}, p.~\pageref{API}.
%
% \medskip
% This environment |{PitonExecute}| takes in as optional argument (between
% square brackets) the options of the command |\PitonOptions|.
%
%
%
% \clearpage
% \section{The styles for the different computer languages}
%
% \label{Semantic}
%
%
% \subsection{The language Python}
%
% In \pkg{piton}, the default language is Python. If necessary, it's possible to
% come back to the language Python with |\PitonOptions{language=Python}|.
%
% \bigskip
%
% The initial settings
% done by \pkg{piton} in |piton.sty| are inspired by the style \pkg{manni} of
% Pygments, as applied by Pygments to the language Python.\footnote{See:
% \url{https://pygments.org/styles/}. Remark that, by default, Pygments provides
% for its style \pkg{manni} a colored background whose color is the HTML color
% |#F0F3F3|. It's possible to have the same color in |{Piton}|
% with the instruction |\PitonOptions{background-color = [HTML]{F0F3F3}}|.} 
%
% \vspace{1cm}
%
% \begin{center}
% \begin{tabularx}{\textwidth}{@{}>{\ttfamily}l>{\raggedright\arraybackslash}X@{}}
% \toprule
% \normalfont Style  & Use \\
% \midrule
% Number & the numbers \\
% String.Short & the short strings (entre \texttt{'} ou \texttt{"}) \\
% String.Long & the long strings (entre \texttt{'''} ou \texttt{"""}) excepted
%               the doc-strings (governed by |String.Doc|)\\
% String & that key fixes both |String.Short| et |String.Long| \\
% String.Doc & the doc-strings (only with |"""| following PEP~257) \\
% String.Interpol & the syntactic elements of the fields of the f-strings
% (that is to say the characters \texttt{\{} et \texttt{\}}); that style
% inherits for the styles |String.Short| and |String.Long| (according the kind
% of string where the interpolation appears) \\
% Interpol.Inside & the content of the interpolations in the f-strings (that
% is to say the elements between \texttt{\{} and~\texttt{\}}); if the final
% user has not set that key, those elements will be formatted by \pkg{piton}
% as done for any Python code. \\
% Operator & the following operators: \texttt{!= == << >> - \~{} + / * \% = < >
% \& .} \verb+|+ \verb|@| \\ 
% Operator.Word & the following operators: |in|, |is|, |and|, |or| et |not| \\
% Name.Builtin & almost all the functions predefined by Python \\
% Name.Decorator & the decorators (instructions beginning by \verb|@|) \\
% Name.Namespace & the name of the modules \\
% Name.Class & the name of the Python classes defined by the user \emph{at their
% point of definition} (with the keyword |class|) \\  
% Name.Function & the name of the Python functions defined by the user \emph{at their
% point of definition} (with the keyword |def|) \\ 
% UserFunction & the name of the Python functions previously defined by the user
%                (the initial value of that parameter is
%                {\ttfamily \textbackslash PitonStyle\{Identifier\}} and,
%                therefore, the names of that functions are formatted like the
%                identifiers).  \\
% Exception & les exceptions prédéfinies (ex.: \texttt{SyntaxError}) \\
% InitialValues & the initial values (and the preceding symbol |=|) of the
% optional arguments in the definitions of functions; if the final
% user has not set that key, those elements will be formatted by \pkg{piton}
% as done for any Python code. \\
% Comment & the comments beginning with \texttt{\#} \\
% Comment.LaTeX & the comments beginning with \texttt{\#>}, which are composed by
% \pkg{piton} as LaTeX code (merely named ``LaTeX comments'' in this document) \\
% Keyword.Constant & |True|, |False| et |None| \\
% Keyword & the following keywords:
%           \ttfamily assert, break, case, continue, del,
%           elif, else, except, exec, finally, for, from, 
%           global, if, import, in, lambda, non local,
%           pass, raise, return, try, while,
%           with, yield et yield from.\\
% Identifier & the identifiers. \\
% \bottomrule
% \end{tabularx}
% \end{center}
%
%
% \newpage
%
% \subsection{The language OCaml}
%
% It's possible to switch to the language |OCaml| with the key |language|:
% |language = OCaml|. 
%
% 
% \bigskip
%
%
% \begin{center}
% \begin{tabularx}{0.9\textwidth}{@{}>{\ttfamily}l>{\raggedright\arraybackslash}X@{}}
%   \toprule
%   \normalfont Style  & Use \\
%   \midrule
%   Number & the numbers \\
%   String.Short & the characters (between \texttt{'}) \\
%   String.Long & the strings, between |"| but also the \emph{quoted-strings} \\
%   String & that key fixes both |String.Short| and |String.Long| \\
%   Operator & les opérateurs, en particulier |+|, |-|, |/|, |*|, |@|, |!=|, |==|, |&&| \\
%   Operator.Word & les opérateurs suivants : |asr|, |land|, |lor|, |lsl|, |lxor|, |mod| et |or| \\
%   Name.Builtin & les fonctions |not|, |incr|, |decr|, |fst| et |snd| \\
%   Name.Type & the name of a type of OCaml \\
%   Name.Field & the name of a field of a module \\
%   Name.Constructor & the name of the constructors of types (which begins by a capital) \\
%   Name.Module & the name of the modules \\
%   Name.Function & the name of the Python functions defined by the user \emph{at their
%   point of definition} (with the keyword |let|) \\ 
%   UserFunction & the name of the Python functions previously defined by the user
%                  (the initial value of that parameter is
%                  {\ttfamily \textbackslash PitonStyle\{Identifier\}} and,
%                  therefore, the names of that functions are formatted like the identifiers). \\
%   Exception & the predefined exceptions (eg : |End_of_File|) \\
%   TypeParameter &  the parameters of the types \\
%   Comment & the comments, between |(*| et |*)|; these comments may be nested \\
%   Keyword.Constant & |true| et |false| \\
%   Keyword & the following keywords: |assert|, |as|, |done|, |downto|, |do|,
%   |else|, |exception|, |for|, |function| , |fun|, |if|, |lazy|, |match|,
%   |mutable|, |new|, |of|, |private|, |raise|, |then|, |to|, |try| , |virtual|,
%   |when|, |while| and |with| \\  
%   Keyword.Governing & the following keywords: |and|, |begin|, |class|, |constraint|,
%                      |end|, |external|, |functor|, |include|, |inherit|, |initializer|,
%                      |in|, |let|, |method|, |module|, |object|, |open|, |rec|, |sig|,
%                      |struct|, |type| and |val|. \\
%  Identifier & the identifiers. \\
%   \bottomrule
% \end{tabularx}
% \end{center}
%
% \newpage
%
% \subsection[The language C (and C++)]{The language C (and \CC)}
%
% 
% It's possible to switch to the language |C| with the key |language|: |language = C|.
%
% \bigskip
% 
% \begin{center}
% \begin{tabularx}{0.9\textwidth}{@{}>{\ttfamily}l>{\raggedright\arraybackslash}X@{}}
% \toprule
% \normalfont Style  & Use \\
% \midrule
% Number & the numbers \\
% String.Short & the characters (between \texttt{'}) \\
% String.Long & the strings (between \texttt{"}) \\
% String.Interpol &  the elements \texttt{\%d}, \texttt{\%i}, \texttt{\%f},
% \texttt{\%c}, etc. in the strings; that style inherits from the style |String.Long| \\
% Operator & the following operators : 
% \texttt{!= == << >> - \~{} + / * \% = < > \& .} \verb+|+ \verb|@| \\ 
% Name.Type & the following predefined types:
%   |bool|, |char|, |char16_t|, |char32_t|, |double|, |float|, |int|, |int8_t|,
%   |int16_t|, |int32_t|, |int64_t|, |long|, |short|, |signed|, |unsigned|,
%   |void| et |wchar_t| \\  
% Name.Builtin & the following predefined functions: |printf|, |scanf|,
% |malloc|, |sizeof| and |alignof|  \\ 
% Name.Class & le nom des classes au moment de leur définition, c'est-à-dire
%   après le mot-clé \verb|class| \\
% Name.Function & the name of the Python functions defined by the user \emph{at their
%   point of definition} (with the keyword |let|) \\ 
% UserFunction & the name of the Python functions previously defined by the user
%                  (the initial value of that parameter is
%                  {\ttfamily \textbackslash PitonStyle\{Identifier\}} and,
%                  therefore, the names of that functions are formatted like the
%                  identifiers). \\ 
% Preproc & the instructions of the preprocessor (beginning par |#|) \\
% Comment & the comments (beginning by \texttt{//} or between |/*| and |*/|) \\
% Comment.LaTeX & the comments beginning by  \texttt{//>} which are composed by
% \pkg{piton} as LaTeX code (merely named ``LaTeX comments'' in this document) \\
% Keyword.Constant & |default|, |false|, |NULL|, |nullptr| and |true| \\
% Keyword & the following keywords:
% |alignas|, |asm|, |auto|, |break|, |case|, |catch|, |class|, 
% |constexpr|, |const|, |continue|, |decltype|, |do|, |else|, |enum|,
% |extern|, |for|, |goto|, |if|, |nexcept|, |private|, |public|, |register|,
% |restricted|, |try|, |return|, |static|, |static_assert|, |struct|, |switch|,
% |thread_local|, |throw|, |typedef|, |union|, |using|, |virtual|, |volatile|
% and |while| \\
%  Identifier & the identifiers. \\
% \bottomrule
% \end{tabularx}
% \end{center}
%
% \newpage
%
% \subsection{The language SQL}
%
% 
% It's possible to switch to the language |SQL| with the key |language|: |language = SQL|.
%
% \bigskip
%
% \begin{center}
% \begin{tabularx}{0.9\textwidth}{@{}>{\ttfamily}l>{\raggedright\arraybackslash}X@{}}
% \toprule
% \normalfont Style  & Use \\
% \midrule
% Number & the numbers \\
% String.Long & the strings (between \texttt{'} and not \texttt{"} because the
% elements between \texttt{"} are names of fields and formatted with |Name.Field|) \\
% Operator & the following operators : \texttt{= != <> >= > < <= * + / } \\
% Name.Table & the names of the tables \\
% Name.Field & the names of the fields of the tables \\
% Name.Builtin & the following built-in functions (their names are \emph{not} case-sensitive): 
%    |avg|, |count|, |char_length|, |concat|, |curdate|, |current_date|,
%    |date_format|, |day|, |lower|, |ltrim|, |max|, |min|, |month|, |now|,
%    |rank|, |round|, |rtrim|, |substring|, |sum|, |upper| and |year|. \\
% Comment & the comments (beginning by \texttt{--} or between |/*| and |*/|) \\
% Comment.LaTeX & the comments beginning by  \texttt{-->} which are composed by
% \pkg{piton} as LaTeX code (merely named ``LaTeX comments'' in this document) \\
% Keyword & the following keywords (their names are \emph{not} case-sensitive):
% |abort|, |action|, |add|, |after|, |all|, |alter|, |always|, |analyze|,
% |and|, |as|, |asc|, |attach|, |autoincrement|, |before|, |begin|, |between|,
% |by|, |cascade|, |case|, |cast|, |check|, |collate|, |column|, |commit|,
% |conflict|, |constraint|, |create|, |cross|, |current|, |current_date|,
% |current_time|, |current_timestamp|, |database|, |default|, |deferrable|,
% |deferred|, |delete|, |desc|, |detach|, |distinct|, |do|, |drop|, |each|,
% |else|, |end|, |escape|, |except|, |exclude|, |exclusive|, |exists|,
% |explain|, |fail|, |filter|, |first|, |following|, |for|, |foreign|, |from|,
% |full|, |generated|, |glob|, |group|, |groups|, |having|, |if|, |ignore|,
% |immediate|, |in|, |index|, |indexed|, |initially|, |inner|, |insert|,
% |instead|, |intersect|, |into|, |is|, |isnull|, |join|, |key|, |last|,
% |left|, |like|, |limit|, |match|, |materialized|, |natural|, |no|, |not|,
% |nothing|, |notnull|, |null|, |nulls|, |of|, |offset|, |on|, |or|, |order|,
% |others|, |outer|, |over|, |partition|, |plan|, |pragma|, |preceding|,
% |primary|, |query|, |raise|, |range|, |recursive|, |references|, |regexp|,
% |reindex|, |release|, |rename|, |replace|, |restrict|, |returning|, |right|,
% |rollback|, |row|, |rows|, |savepoint|, |select|, |set|, |table|, |temp|,
% |temporary|, |then|, |ties|, |to|, |transaction|, |trigger|, |unbounded|,
% |union|, |unique|, |update|, |using|, |vacuum|, |values|, |view|, |virtual|,
% |when|, |where|, |window|, |with|, |without| \\
% \bottomrule
% \end{tabularx}
% \end{center}
%
%
% \bigskip
% It's possible to automatically capitalize the keywords by modifying locally
% for the language SQL the style |Keywords|.
% \begin{Verbatim}
% \SetPitonStyle~emphase#[SQL]@{Keywords = \bfseries \MakeUppercase}
% \end{Verbatim}
%
% \newpage
% \subsection{The languages defined by \textbackslash NewPitonLanguage}
%
% \vspace{1cm}
% The command |\NewPitonLanguage|, which defines new computer languages with the
% syntax of the extension \pkg{listings}, has been described p.~\pageref{NewPitonLanguage}.
%
% All the languages defined by the command |\NewPitonLanguage| use the same styles.
% \vspace{1cm}
%
% \begin{center}
% \begin{tabularx}{0.9\textwidth}{@{}>{\ttfamily}l>{\raggedright\arraybackslash}X@{}}
% \toprule
% \normalfont Style  & Use \\
% \midrule
% Number & the numbers \\
% String.Long & the strings defined in |\NewPitonLanguage| by the key |morestring| \\
% Comment & the comments defined in |\NewPitonLanguage| by the key |morecomment|  \\
% Comment.LaTeX & the comments which are composed by \pkg{piton} as LaTeX code (merely named ``LaTeX comments'' in this document) \\
% Keyword & the keywords defined in |\NewPitonLanguage| by the keys |morekeywords|
%           and |moretexcs| (and also the  key |sensitive| which specifies whether
%           the keywords are case-sensitive or not) \\ 
% Directive & the directives defined in |\NewPitonLanguage| by the key
% |moredirectives| \\
% Tag & the ``tags'' defined by the key |tag| (the lexical units detected within
% the tag will also be formatted with their own style) \\
%  Identifier & the identifiers. \\
% \bottomrule
% \end{tabularx}
% \end{center}
%
%
%\vspace{1cm}
% Here is for example a definition for the language HTML, obtained with a slight
% adaptation of the definition done by \pkg{listings} (file |lstlang1.sty|).
%
% \medskip
% \begin{Verbatim}[formatcom=\small]
% \NewPitonLanguage{HTML}%
%   {morekeywords={A,ABBR,ACRONYM,ADDRESS,APPLET,AREA,B,BASE,BASEFONT,%
%       BDO,BIG,BLOCKQUOTE,BODY,BR,BUTTON,CAPTION,CENTER,CITE,CODE,COL,%
%       COLGROUP,DD,DEL,DFN,DIR,DIV,DL,DOCTYPE,DT,EM,FIELDSET,FONT,FORM,%
%       FRAME,FRAMESET,HEAD,HR,H1,H2,H3,H4,H5,H6,HTML,I,IFRAME,IMG,INPUT,%
%       INS,ISINDEX,KBD,LABEL,LEGEND,LH,LI,LINK,LISTING,MAP,META,MENU,%
%       NOFRAMES,NOSCRIPT,OBJECT,OPTGROUP,OPTION,P,PARAM,PLAINTEXT,PRE,%
%       OL,Q,S,SAMP,SCRIPT,SELECT,SMALL,SPAN,STRIKE,STRING,STRONG,STYLE,%
%       SUB,SUP,TABLE,TBODY,TD,TEXTAREA,TFOOT,TH,THEAD,TITLE,TR,TT,U,UL,%
%       VAR,XMP,%
%       accesskey,action,align,alink,alt,archive,axis,background,bgcolor,%
%       border,cellpadding,cellspacing,charset,checked,cite,class,classid,%
%       code,codebase,codetype,color,cols,colspan,content,coords,data,%
%       datetime,defer,disabled,dir,event,error,for,frameborder,headers,%
%       height,href,hreflang,hspace,http-equiv,id,ismap,label,lang,link,%
%       longdesc,marginwidth,marginheight,maxlength,media,method,multiple,%
%       name,nohref,noresize,noshade,nowrap,onblur,onchange,onclick,%
%       ondblclick,onfocus,onkeydown,onkeypress,onkeyup,onload,onmousedown,%
%       profile,readonly,onmousemove,onmouseout,onmouseover,onmouseup,%
%       onselect,onunload,rel,rev,rows,rowspan,scheme,scope,scrolling,%
%       selected,shape,size,src,standby,style,tabindex,text,title,type,%
%       units,usemap,valign,value,valuetype,vlink,vspace,width,xmlns},%
%    tag=<>,%
%    alsoletter = - ,%
%    sensitive=f,%
%    morestring=[d]",
%   }
% \end{Verbatim}
%
%
%
% \subsection{The language ``minimal''}
%
% \label{minimal}
%
% It's possible to switch to the language ``|minimal|'' with the key |language|:
% |language = minimal|.
%
% \bigskip
%
% \begin{center}
% \begin{tabularx}{0.9\textwidth}{@{}>{\ttfamily}l>{\raggedright\arraybackslash}X@{}}
% \toprule
% \normalfont Style  & Usage \\
% \midrule
% Number & the numbers \\
% String & the strings (between \texttt{"}) \\
% Comment & the comments (which begin with |#|) \\
% Comment.LaTeX &  the comments beginning with \texttt{\#>}, which are composed by
%   \pkg{piton} as LaTeX code (merely named ``LaTeX comments'' in this document)
%   \\
%  Identifier & the identifiers. \\
% \bottomrule
% \end{tabularx}
% \end{center}
%
% \bigskip
% That language is provided for the final user who might wish to add keywords in
% that language (with the command |\SetPitonIdentifier|: cf. \ref{SetPitonIdentifier},
% p.~\pageref{SetPitonIdentifier}) in order to create, for example, a language
% for pseudo-code.
%
% \vspace{1cm}
% 
% \subsection{The language ``verbatim''}
%
% \label{verbatim}
%
% It's possible to switch to the language ``|verbatim|'' with the key |language|:
% |language = verbatim|.
%
% \bigskip
%
% \begin{center}
% \begin{tabularx}{0.9\textwidth}{@{}>{\ttfamily}l>{\raggedright\arraybackslash}X@{}}
% \toprule
% \normalfont Style  & Usage \\
% \midrule
% None... & \\
% \bottomrule
% \end{tabularx}
% \end{center}
%
% The language |verbatim| doen't provide any style and, thus, does not do any
% syntactic formating. However, it's possible to use the mechanism
% |detected-commands| (cf. part~\ref{detected-commands},
% p.~\pageref{detected-commands}) and the detection of the commands and
% environments of Beamer.
%
%
% \newpage
%
% \section{Implementation}
%
% \medskip
% The development of the extension \pkg{piton} is done on the following GitHub
% depot:
%
% \verb|https://github.com/fpantigny/piton|
%
% \subsection{Introduction}
% 
% The main job of the package \pkg{piton} is to take in as input a Python
% listing and to send back to LaTeX as output that code \emph{with interlaced LaTeX
% instructions of formatting}.
%
% In fact, all that job is done by a \textsc{lpeg} called |python|. That
% \textsc{lpeg}, when matched against the string of a Python listing,
% returns as capture a Lua table containing data to send to LaTeX. 
% The only thing to do after will be to apply |tex.tprint| to each element of
% that table.\footnote{Recall that |tex.tprint| takes in as argument a Lua table whose
% first component is a ``catcode table'' and the second element a string. The
% string will be sent to LaTeX with the regime of catcodes specified by the
% catcode table. If no catcode table is provided, the standard catcodes of LaTeX
% will be used.}
% 
% \bigskip
% Consider, for example, the following Python code:
%
% \begin{Piton}
% def parity(x):
%     return x%2
% \end{Piton}
%
% The capture returned by the \pkg{lpeg} |python| against that code is the
% Lua table containing the following elements :
%
% \bigskip
% \begin{minipage}{\linewidth}
% \color{gray}
% 
% |{ "\\__piton_begin_line:" }|\footnote{Each line of the Python listings will
% be encapsulated in a pair: \texttt{\textbackslash_@@_begin_line:} --
% \texttt{\textbackslash@@_end_line:}. The token
% \texttt{\textbackslash@@_end_line:} must be explicit because it will be used as
% marker in order to delimit the argument of the command \texttt{\textbackslash
% @@\_begin\_line:}. Both tokens \texttt{\textbackslash_@@_begin_line:} and
% \texttt{\textbackslash@@_end_line:} will be nullified in the command
% \texttt{\textbackslash piton} (since there can't be lines breaks in the
% argument of a command \texttt{\textbackslash piton}).}  
% 
% \texttt{\{ "\{\textbackslash PitonStyle\{Keyword\}\{" \}}\footnote{The
% lexical elements of Python for which we have a \pkg{piton} style will be
% formatted via the use of the command \texttt{\textbackslash PitonStyle}.
% Such an element is typeset in LaTeX via the syntax \texttt{\{\textbackslash
% PitonStyle\{\textsl{style}\}\{...\}\}} because the instructions inside an \texttt{\textbackslash
% PitonStyle} may be both semi-global declarations like
% \texttt{\textbackslash bfseries} and commands with one argument like
% \texttt{\textbackslash fbox}.}
%
% \texttt{\{
% luatexbase.catcodetables.CatcodeTableOther\footnote{\texttt{luatexbase.catcodetables.CatcodeTableOther} is a mere number which corresponds to the ``catcode table'' whose all characters have the catcode ``other'' (which means that they will be typeset by LaTeX verbatim).}, "def" \} }
%
% |{ "}}" }|
%
% |{ luatexbase.catcodetables.CatcodeTableOther, " " }|
%
% |{ "{\PitonStyle{Name.Function}{" }|
%
% |{ luatexbase.catcodetables.CatcodeTableOther, "parity" }|
%
% |{ "}}" }|
% 
% |{ luatexbase.catcodetables.CatcodeTableOther, "(" }|
%
% |{ luatexbase.catcodetables.CatcodeTableOther, "x" }|
%
% |{ luatexbase.catcodetables.CatcodeTableOther, ")" }|
%
% |{ luatexbase.catcodetables.CatcodeTableOther, ":" }|
% 
% |{ "\\__piton_end_line: \\__piton_newline: \\__piton_begin_line:" }|
%
% |{ luatexbase.catcodetables.CatcodeTableOther, "    " }|
%
% |{ "{\PitonStyle{Keyword}{" }|
%
% |{ luatexbase.catcodetables.CatcodeTableOther, "return" }|
%
% |{ "}}" }|
%
% |{ luatexbase.catcodetables.CatcodeTableOther, " " }|
%
% |{ luatexbase.catcodetables.CatcodeTableOther, "x" }|
%
% |{ "{\PitonStyle{Operator}{" }|
%
% |{ luatexbase.catcodetables.CatcodeTableOther, "&" }|
%
% |{ "}}" }|
%
% |{ "{\PitonStyle{Number}{" }|
%
% |{ luatexbase.catcodetables.CatcodeTableOther, "2" }|
%
% |{ "}}" }|
% 
% |{ "\\__piton_end_line:" }|
% 
% \end{minipage}
%
% \bigskip
% We give now the LaTeX code which is sent back by Lua to TeX (we have written
% on several lines for legibility but no character |\r| will be sent to LaTeX). The
% characters which are greyed-out are sent to LaTeX with the catcode ``other''
% (=12). All the others characters are sent with the regime of catcodes of L3
% (as set by |\ExplSyntaxOn|)
%
% 
% \begingroup
% \def\gbox#1{\colorbox{gray!20}{\strut #1}}
% \setlength{\fboxsep}{1pt}
% 
% \begin{Verbatim*}[formatcom = \color{black}]
% \__piton_begin_line:{\PitonStyle{Keyword}{~gbox#def@}}
% ~gbox# @{\PitonStyle{Name.Function}{~gbox#parity@}}~gbox#(x):@\__piton_end_line:\__piton_newline:
% \__piton_begin_line:~gbox#    @{\PitonStyle{Keyword}{~gbox#return@}}
% ~gbox# x@{\PitonStyle{Operator}{~gbox#%@}}{\PitonStyle{Number}{~gbox#2@}}\__piton_end_line:
% \end{Verbatim*}
% \endgroup
%
%
% 
%
% \subsection{The L3 part of the implementation}
%
% \subsubsection{Declaration of the package}
%    \begin{macrocode}
%<*STY>
\NeedsTeXFormat{LaTeX2e}
\ProvidesExplPackage
  {piton}
  {\PitonFileDate}
  {\PitonFileVersion}
  {Highlight informatic listings with LPEG on LuaLaTeX}
%    \end{macrocode}
%
%    \begin{macrocode}
\msg_new:nnn { piton } { latex-too-old }
  { 
    Your~LaTeX~release~is~too~old. \\
    You~need~at~least~the~version~of~2023-11-01 
  }
%    \end{macrocode}
%
%    \begin{macrocode}
\IfFormatAtLeastTF
  { 2023-11-01 } 
  { } 
  { \msg_fatal:nn { piton } { latex-too-old } }
%    \end{macrocode}
% \bigskip
% The command |\text| provided by the package \pkg{amstext} will be used to
% allow the use of the command |\pion{...}| (with the standard syntax) in
% mathematical mode.
%    \begin{macrocode}
\RequirePackage { amstext }
%    \end{macrocode}
%
% \bigskip
%    \begin{macrocode}
\cs_new_protected:Npn \@@_error:n { \msg_error:nn { piton } }
\cs_new_protected:Npn \@@_warning:n { \msg_warning:nn { piton } }
\cs_new_protected:Npn \@@_warning:nn { \msg_warning:nnn { piton } }
\cs_new_protected:Npn \@@_error:nn { \msg_error:nnn { piton } }
\cs_new_protected:Npn \@@_error:nnn { \msg_error:nnnn { piton } }
\cs_new_protected:Npn \@@_fatal:n { \msg_fatal:nn { piton } }
\cs_new_protected:Npn \@@_fatal:nn { \msg_fatal:nnn { piton } }
\cs_new_protected:Npn \@@_msg_new:nn { \msg_new:nnn { piton } }
%    \end{macrocode}
%
% With Overleaf (and also TeXPage), by default, a document is compiled in
% non-stop mode. When there is an error, there is no way to the user to use the
% key H in order to have more information. That's why we decide to put that
% piece of information (for the messages with such information) in the main part
% of the message when the key |messages-for-Overleaf| is used (at load-time).
%    \begin{macrocode}
\cs_new_protected:Npn \@@_msg_new:nnn #1 #2 #3
  {
    \bool_if:NTF \g_@@_messages_for_Overleaf_bool
      { \msg_new:nnn { piton } { #1 } { #2 \\ #3 } }
      { \msg_new:nnnn { piton } { #1 } { #2 } { #3 } }
  }
%    \end{macrocode}
% 
% \bigskip
% We also create commands which will generate usually an error but only a
% warning on Overleaf. The argument is given by curryfication.
%    \begin{macrocode}
\cs_new_protected:Npn \@@_error_or_warning:n 
  { \bool_if:NTF \g_@@_messages_for_Overleaf_bool \@@_warning:n \@@_error:n }
\cs_new_protected:Npn \@@_error_or_warning:nn 
  { \bool_if:NTF \g_@@_messages_for_Overleaf_bool \@@_warning:nn \@@_error:nn }
%    \end{macrocode}
% 
% We try to detect whether the compilation is done on Overleaf. We use
% |\c_sys_jobname_str| because, with Overleaf, the value of |\c_sys_jobname_str|
% is always ``|output|''.
%    \begin{macrocode}
\bool_new:N \g_@@_messages_for_Overleaf_bool
\bool_gset:Nn \g_@@_messages_for_Overleaf_bool
  { 
       \str_if_eq_p:on \c_sys_jobname_str { _region_ }  % for Emacs
    || \str_if_eq_p:on \c_sys_jobname_str { output }   % for Overleaf
  }
%    \end{macrocode}
%
% \bigskip
%    \begin{macrocode}
\@@_msg_new:nn { LuaLaTeX~mandatory }
  { 
    LuaLaTeX~is~mandatory.\\
    The~package~'piton'~requires~the~engine~LuaLaTeX.\\
    \str_if_eq:onT \c_sys_jobname_str { output }
      { If~you~use~Overleaf,~you~can~switch~to~LuaLaTeX~in~the~"Menu"~and~
        if~you~use~TeXPage,~you~should~go~in~"Settings". \\ }
    \IfClassLoadedTF { beamer }
      {
        Since~you~use~Beamer,~don't~forget~to~use~piton~in~frames~with~
        the~key~'fragile'.\\
      }
      { }
    That~error~is~fatal.
  }
\sys_if_engine_luatex:F { \@@_fatal:n { LuaLaTeX~mandatory } }
%    \end{macrocode}
% 
% \bigskip
%    \begin{macrocode}
\RequirePackage { luacode }
%    \end{macrocode}
% 
% \bigskip
%    \begin{macrocode}
\@@_msg_new:nnn { piton.lua~not~found }
  { 
    The~file~'piton.lua'~can't~be~found.\\ 
    This~error~is~fatal.\\
    If~you~want~to~know~how~to~retrieve~the~file~'piton.lua',~type~H~<return>.
  }
  {
    On~the~site~CTAN,~go~to~the~page~of~'piton':~https://ctan.org/pkg/piton.~
    The~file~'README.md'~explains~how~to~retrieve~the~files~'piton.sty'~and~
    'piton.lua'.
  }
%    \end{macrocode}
%
% \bigskip
%    \begin{macrocode}
\file_if_exist:nF { piton.lua } { \@@_fatal:n { piton.lua~not~found } }
%    \end{macrocode}
%
%
% \bigskip
% The boolean |\g_@@_footnotehyper_bool| will indicate if the option
% |footnotehyper| is used.
%    \begin{macrocode}
\bool_new:N \g_@@_footnotehyper_bool
%    \end{macrocode}
%
% \medskip
% The boolean |\g_@@_footnote_bool| will indicate if the option |footnote| is
% used, but quickly, it will also be set to |true| if the option |footnotehyper|
% is used.
%    \begin{macrocode}
\bool_new:N \g_@@_footnote_bool
%    \end{macrocode}
%
%
% \medskip
%    \begin{macrocode}
\bool_new:N \g_@@_beamer_bool
%    \end{macrocode}
%
%
% 
% \bigskip
% We define a set of keys for the options at load-time.
%    \begin{macrocode}
\keys_define:nn { piton }
  { 
    footnote .bool_gset:N = \g_@@_footnote_bool ,
    footnotehyper .bool_gset:N = \g_@@_footnotehyper_bool ,
    footnote .usage:n = load , 
    footnotehyper .usage:n = load ,

    beamer .bool_gset:N = \g_@@_beamer_bool ,
    beamer .default:n = true , 
    beamer .usage:n = load ,

    unknown .code:n = \@@_error:n { Unknown~key~for~package }
  }
%    \end{macrocode}
% 
%    \begin{macrocode}
\@@_msg_new:nn { Unknown~key~for~package }
  {
    Unknown~key.\\
    You~have~used~the~key~'\l_keys_key_str'~when~loading~piton~
    but~the~only~keys~available~here~
    are~'beamer',~'footnote',~and~'footnotehyper'.~
    Other~keys~are~available~in~\token_to_str:N \PitonOptions.\\ 
    That~key~will~be~ignored.
  }
%    \end{macrocode}
%
%
% \bigskip
% We process the options provided by the user at load-time.
%    \begin{macrocode}
\ProcessKeyOptions 
%    \end{macrocode}
%
% 
% \bigskip
%    \begin{macrocode}
\IfClassLoadedTF { beamer } 
  { \bool_gset_true:N \g_@@_beamer_bool } 
  {
    \IfPackageLoadedTF { beamerarticle } 
    { \bool_gset_true:N \g_@@_beamer_bool } 
    { }
  }
%    \end{macrocode}
%
%    \begin{macrocode}
\lua_now:e
  { 
    piton = piton~or~{ } 
    piton.last_code = ''
    piton.last_language = ''
    piton.join = ''
    piton.write = ''
    piton.path_write = ''
    \bool_if:NT \g_@@_beamer_bool { piton.beamer = true } 
  }
%    \end{macrocode}
%
% 
% \bigskip
%    \begin{macrocode}
\RequirePackage { xcolor } 
%    \end{macrocode}
%
%
%    \begin{macrocode}
\@@_msg_new:nn { footnote~with~footnotehyper~package }
  { 
    Footnote~forbidden.\\
    You~can't~use~the~option~'footnote'~because~the~package~
    footnotehyper~has~already~been~loaded.~
    If~you~want,~you~can~use~the~option~'footnotehyper'~and~the~footnotes~  
    within~the~environments~of~piton~will~be~extracted~with~the~tools~
    of~the~package~footnotehyper.\\
    If~you~go~on,~the~package~footnote~won't~be~loaded.
  }
%    \end{macrocode}
%
%    \begin{macrocode}
\@@_msg_new:nn { footnotehyper~with~footnote~package }
  { 
    You~can't~use~the~option~'footnotehyper'~because~the~package~
    footnote~has~already~been~loaded.~
    If~you~want,~you~can~use~the~option~'footnote'~and~the~footnotes~  
    within~the~environments~of~piton~will~be~extracted~with~the~tools~
    of~the~package~footnote.\\ 
    If~you~go~on,~the~package~footnotehyper~won't~be~loaded.
  }
%    \end{macrocode}
%
% \medskip
%    \begin{macrocode}
\bool_if:NT \g_@@_footnote_bool 
  { 
%    \end{macrocode}
% The class \cls{beamer} has its own system to extract footnotes and that's why
% we have nothing to do if \cls{beamer} is used. 
%    \begin{macrocode}
    \IfClassLoadedTF { beamer }
      { \bool_gset_false:N \g_@@_footnote_bool }
      { 
        \IfPackageLoadedTF { footnotehyper }
          { \@@_error:n { footnote~with~footnotehyper~package } }
          { \usepackage { footnote } }
      }
  }
%    \end{macrocode}
%
%    \begin{macrocode}
\bool_if:NT \g_@@_footnotehyper_bool 
  { 
%    \end{macrocode}
% The class \cls{beamer} has its own system to extract footnotes and that's why
% we have nothing to do if \cls{beamer} is used. 
%    \begin{macrocode}
    \IfClassLoadedTF { beamer }
      { \bool_gset_false:N \g_@@_footnote_bool }
      { 
        \IfPackageLoadedTF { footnote }
          { \@@_error:n { footnotehyper~with~footnote~package } }
          { \usepackage { footnotehyper } }
        \bool_gset_true:N \g_@@_footnote_bool 
      }
  }
%    \end{macrocode}
% The flag |\g_@@_footnote_bool| is raised and so, we will only have to test
% |\g_@@_footnote_bool| in order to know if we have to insert an environment
% |{savenotes}|.
%
% 
% \bigskip
% \subsubsection{Parameters and technical definitions}
%
%    \begin{macrocode}
\dim_new:N \l_@@_rounded_corners_dim
%    \end{macrocode}
% 
%    \begin{macrocode}
\bool_new:N \l_@@_in_label_bool
%    \end{macrocode}
% 
%    \begin{macrocode}
\dim_new:N \l_@@_tmpc_dim
%    \end{macrocode}
% \medskip
%
% The listing that we have to format will be stored in |\l_@@_listing_tl|.
% That applies both for the command |\PitonInputFile| and the environment |{Piton}|
% (or another environment defined by |\NewPitonEnvironment|).
%    \begin{macrocode}
\tl_new:N \l_@@_listing_tl
%    \end{macrocode}
%
% \medskip
% The content of an environment such as |{Piton}| will be composed first in the
% following box, but that box be \emph{unvboxed} at the end.
%    \begin{macrocode}
\box_new:N \g_@@_output_box
%    \end{macrocode}
%
% \medskip
%    \begin{macrocode}
\box_new:N \l_@@_line_box
%    \end{macrocode}
%
% The following string will contain the name of the computer language
% considered (the initial value is |python|).
%
%    \begin{macrocode}
\str_new:N \l_piton_language_str
\str_set:Nn \l_piton_language_str { python }
%    \end{macrocode}
% 
% \medskip
% Each time an environment of \pkg{piton} is used, the informatic code in the
% body of that environment will be stored in the following global string.
%    \begin{macrocode}
\tl_new:N \g_piton_last_code_tl
%    \end{macrocode}
%
% \medskip
% The following parameter corresponds to the key |path| (which is the path used
% to include files by |\PitonInputFile|). Each component of that sequence will
% be a string (type |str|).
%    \begin{macrocode}
\seq_new:N \l_@@_path_seq
%    \end{macrocode}
%
% \medskip
% The following parameter corresponds to the key |path-write| (which is the path
% used when writing files from listings inserted in the environments of
% \pkg{piton} by use of the key |write|).
%    \begin{macrocode}
\str_new:N \l_@@_path_write_str
%    \end{macrocode}
%
% \medskip
% The following parameter corresponds to the key |tcolorbox|.
%    \begin{macrocode}
\bool_new:N \l_@@_tcolorbox_bool
% \end{macrocode}
%
% \medskip
% The following parameter corresponds to the key |box|.
%    \begin{macrocode}
\str_new:N \l_@@_box_str
% \end{macrocode}
%
% \medskip
% In order to have a better control over the keys.
%    \begin{macrocode}
\bool_new:N \l_@@_in_PitonOptions_bool 
\bool_new:N \l_@@_in_PitonInputFile_bool 
%    \end{macrocode}
%
% \medskip
% The following parameter corresponds to the key |font-command|.
%    \begin{macrocode}
\tl_new:N \l_@@_font_command_tl
\tl_set:Nn \l_@@_font_command_tl { \ttfamily }
%    \end{macrocode}
% 
% \medskip
% We will compute (with Lua) the numbers of lines of the listings (or
% \emph{chunks} of listings when |split-on-empty-lines| is in force) and store
% it in the following counter.
%    \begin{macrocode}
\int_new:N \g_@@_nb_lines_int
%    \end{macrocode}
% 
% The same for the number of non-empty lines of the listings.
%    \begin{macrocode}
\int_new:N \l_@@_nb_non_empty_lines_int
%    \end{macrocode}
%
% 
% \medskip 
% The following counter will be used to count the lines during the composition.
% It will take into account all the lines, empty or not empty. It won't be used
% to print the numbers of the lines but will be used to allow or disallow line
% breaks (when |splittable| is in force) and for the color of the background
% (when |background-color| is used with a \emph{list} of colors).
%    \begin{macrocode}
\int_new:N \g_@@_line_int
%    \end{macrocode}
% 
% \medskip
% The following counter corresponds to the key |splittable| of |\PitonOptions|.
% If the value of |\l_@@_splittable_int| is equal to $n$, then no line break can
% occur within the first $n$~lines or the last $n$~lines of a listing (or a
% \emph{chunk} of listings when the key |split-on-empty-lines| is in force).
%    \begin{macrocode}
\int_new:N \l_@@_splittable_int
%    \end{macrocode}
%
% \medskip
% An initial value of |splittable| equal to 100 is equivalent to say that the
% environments |{Piton}| are unbreakable.
%    \begin{macrocode}
\int_set:Nn \l_@@_splittable_int { 100 }
%    \end{macrocode}
%
% \medskip
% When the key |split-on-empty-lines| will be in force, then the following token
% list will be inserted between the chunks of code (the informatic code provided
% by the final user is split in chunks on the empty lines in the code). 
%    \begin{macrocode}
\tl_new:N \l_@@_split_separation_tl
\tl_set:Nn \l_@@_split_separation_tl 
  { \vspace { \baselineskip } \vspace { -1.25pt } }
%    \end{macrocode}
% That parameter must contain elements to be inserted in \emph{vertical} mode by
% TeX. 
%
% 
% \medskip
% The following string corresponds to the key |background-color| of |\PitonOptions|.
%    \begin{macrocode}
\clist_new:N \l_@@_bg_color_clist
%    \end{macrocode}
%
% \medskip
% The package \pkg{piton} will also detect the lines of code which correspond to
% the user input in a Python console, that is to say the lines of code beginning
% with |>>>| and |...|. It's possible, with the key |prompt-background-color|,
% to require a background for these lines of code (and the other lines of code
% will have the standard background color specified by |background-color|).
%    \begin{macrocode}
\tl_new:N \l_@@_prompt_bg_color_tl
%    \end{macrocode}
%
% \medskip
%    \begin{macrocode}
\tl_new:N \l_@@_space_in_string_tl
%    \end{macrocode}
% 
% \medskip
% The following parameters correspond to the keys |begin-range| and |end-range| of
% the command |\PitonInputFile|.
%    \begin{macrocode}
\str_new:N \l_@@_begin_range_str
\str_new:N \l_@@_end_range_str
%    \end{macrocode}
%
% \medskip
% The following boolean corresponds to the key |math-comments| (available only
% in the preamble of the LaTeX document). 
%    \begin{macrocode}
\bool_new:N \g_@@_math_comments_bool
%    \end{macrocode}
%
% \medskip
% The argument of |\PitonInputFile|.
%    \begin{macrocode}
\str_new:N \l_@@_file_name_str 
%    \end{macrocode}
%
% \medskip
% We will count the environments |{Piton}| (and, in fact, also the commands
% |\PitonInputFile|, despite the name |\g_@@_env_int|).
%    \begin{macrocode}
\int_new:N \g_@@_env_int 
%    \end{macrocode}
%
% \medskip
% The following flag corresponds to the key |print|. The initial value of that
% parameter will be |true| (and not |false|) since, of course, by default, we
% want to print the content of the environment |{Piton}|
%    \begin{macrocode}
\bool_new:N \l_@@_print_bool
\bool_set_true:N \l_@@_print_bool
%    \end{macrocode}
%
% \medskip
% The parameter |\l_@@_write_str| corresponds to the key |write|. 
%    \begin{macrocode}
\str_new:N \l_@@_write_str
%    \end{macrocode}
%
% \medskip
% The parameter |\l_@@_join_str| corresponds to the key |join|. In fact, |\l_@@_join_str|
% won't contain the exact value used the final user but its conversion in ``|utf16/hex|''.
%    \begin{macrocode}
\str_new:N \l_@@_join_str
%    \end{macrocode}
%
% \medskip
% The following boolean corresponds to the key |show-spaces|.
%    \begin{macrocode}
\bool_new:N \l_@@_show_spaces_bool
%    \end{macrocode}
%
% \medskip
% The following booleans correspond to the keys |break-lines| and
% |indent-broken-lines|.
%    \begin{macrocode}
\bool_new:N \l_@@_break_lines_in_Piton_bool
\bool_set_true:N \l_@@_break_lines_in_Piton_bool
\bool_new:N \l_@@_indent_broken_lines_bool
%    \end{macrocode}
%
% \medskip
% The following token list corresponds to the key |continuation-symbol|.
%    \begin{macrocode}
\tl_new:N \l_@@_continuation_symbol_tl
\tl_set:Nn \l_@@_continuation_symbol_tl { + }
%    \end{macrocode}
%
% \medskip
% The following token list corresponds to the key
% |continuation-symbol-on-indentation|. The name has been shorten to |csoi|.
%    \begin{macrocode}
\tl_new:N \l_@@_csoi_tl 
\tl_set:Nn \l_@@_csoi_tl { $ \hookrightarrow \; $  }
%    \end{macrocode}
%
% \medskip
% The following token list corresponds to the key |end-of-broken-line|.
%    \begin{macrocode}
\tl_new:N \l_@@_end_of_broken_line_tl 
\tl_set:Nn \l_@@_end_of_broken_line_tl { \hspace* { 0.5em } \textbackslash }
%    \end{macrocode}
%
% \medskip
% The following boolean corresponds to the key |break-lines-in-piton|.
%    \begin{macrocode}
\bool_new:N \l_@@_break_lines_in_piton_bool
%    \end{macrocode}
% 
% \bigskip
% The following dimension will be the width of the listing constructed by
% |{Piton}| or |\PitonInputFile|. If the user uses the key |width| of
% |\PitonOptions| with a numerical value, that value will be stored in
% |\l_@@_width_dim|.
% 
%    \begin{macrocode} 
\dim_new:N \l_@@_width_dim
% \end{macrocode}
%
% \bigskip
% |\g_@@_width_dim| will be the width of the environment, after construction.
% In particular, if |max-width| is used, |\g_@@_width_dim| has to be computed
% from the actual content of the environment.
%    \begin{macrocode}
\dim_new:N \g_@@_width_dim 
%    \end{macrocode}
% 
% \medskip
% We will also use another dimension called |\l_@@_code_width_dim|. That
% will the width of the actual lines of code. That dimension may be lower than
% the whole |\l_@@_width_dim| because we have to take into account the value of
% |\l_@@_left_margin_dim| (for the numbers of lines when |line-numbers| is in
% force) and another small margin when a background color is used (with the key
% |background-color|). 
%    \begin{macrocode}
\dim_new:N \l_@@_code_width_dim
%    \end{macrocode}
% 
% \medskip
% The following flag will be raised when the key |max-width| (and 
% when |width| is used with the key |min|, which is equivalent to
% |max-width=\linewidth|). 
%    \begin{macrocode}
\bool_new:N \l_@@_minimize_width_bool
%    \end{macrocode}
%
% \medskip
% The following dimension corresponds to the key |left-margin| of |\PitonOptions|. 
%    \begin{macrocode}
\dim_new:N \l_@@_left_margin_dim
%    \end{macrocode}
%
% \medskip
% The following boolean will be set when the key |left-margin=auto|
% is used.
%    \begin{macrocode}
\bool_new:N \l_@@_left_margin_auto_bool
%    \end{macrocode}
%
% \medskip
% The following dimension corresponds to the key |numbers-sep| of
% |\PitonOptions|.
%    \begin{macrocode}
\dim_new:N \l_@@_numbers_sep_dim 
\dim_set:Nn \l_@@_numbers_sep_dim { 0.7 em }
%    \end{macrocode}
% 
%
% \medskip
% Be careful. The following sequence |\g_@@_languages_seq| is not the list of
% the languages supported by \pkg{piton}. It's the list of the languages for
% which at least a user function has been defined. We need that sequence only
% for the command |\PitonClearUserFunctions| when it is used without its
% optional argument: it must clear all the list of languages for which at least
% a user function has been defined.
%    \begin{macrocode}
\seq_new:N \g_@@_languages_seq 
%    \end{macrocode}
% 
%
% \medskip
%    \begin{macrocode}
\int_new:N \l_@@_tab_size_int 
\int_set:Nn \l_@@_tab_size_int { 4 }
%    \end{macrocode}
%
%    \begin{macrocode}
\cs_new_protected:Npn \@@_tab:
  { 
    \bool_if:NTF \l_@@_show_spaces_bool
      { 
        \hbox_set:Nn \l_tmpa_box 
          { \prg_replicate:nn \l_@@_tab_size_int { ~ } } 
        \dim_set:Nn \l_tmpa_dim { \box_wd:N \l_tmpa_box }
        \( \mathcolor { gray } 
             { \hbox_to_wd:nn \l_tmpa_dim { \rightarrowfill } } \)
      }
      { \hbox:n { \prg_replicate:nn \l_@@_tab_size_int { ~ } } }
    \int_gadd:Nn \g_@@_indentation_int \l_@@_tab_size_int 
  }
%    \end{macrocode}
%
% 
% \medskip
% The following integer corresponds to the key |gobble|.
%    \begin{macrocode}
\int_new:N \l_@@_gobble_int 
%    \end{macrocode}
%
% \medskip
% The following token list will be used only for the spaces in the strings.
%    \begin{macrocode}
\tl_set_eq:NN \l_@@_space_in_string_tl \nobreakspace 
%    \end{macrocode}
% When the key |break-lines-in-piton| is set, that parameter will be replaced by
% |\space| (in |\piton| with the standard syntax) and when the key
% |show-spaces-in-strings| is set, it will be replaced by ␣ (U+2423).
% 
%
% \medskip
% At each line, the following counter will count the spaces at the beginning.
%    \begin{macrocode}
\int_new:N \g_@@_indentation_int
%    \end{macrocode}
% 
% \medskip
% Be careful: when executed, the following command does \emph{not} create a
% space (only an incrementation of the counter).
%    \begin{macrocode}
\cs_new_protected:Npn \@@_leading_space:
  { \int_gincr:N \g_@@_indentation_int }
%    \end{macrocode}
%
% 
% \bigskip
% In the environment |{Piton}|, the command |\label| will be linked to the
% following command.
%    \begin{macrocode}
\cs_new_protected:Npn \@@_label:n #1
  {
    \bool_if:NTF \l_@@_line_numbers_bool
      {
        \@bsphack 
        \protected@write \@auxout { }
          { 
            \string \newlabel { #1 } 
            { 
%    \end{macrocode}
% Remember that the content of a line is typeset in a box \emph{before} the
% composition of the potential number of line.
%    \begin{macrocode}
              { \int_eval:n { \g_@@_visual_line_int + 1 } } 
              { \thepage } 
            } 
          } 
        \@esphack
      }
      { \@@_error:n { label~with~lines~numbers } }
  }
%    \end{macrocode}
%
% \bigskip
% The following commands corresponds to the keys |marker/beginning| and
% |marker/end|. The values of that keys are functions that will be applied to
% the ``\emph{range}'' specified by the final user in an individual
% |\PitonInputFile|. They will construct the markers used to find textually in
% the external file loaded by \pkg{piton} the part which must be included (and
% formatted).
%
% These macros must \emph{not} be protected.
%    \begin{macrocode}
\cs_new:Npn \@@_marker_beginning:n #1 { }
\cs_new:Npn \@@_marker_end:n #1 { }
%    \end{macrocode}
% 
% 
% 
% \bigskip
% The following token list will be evaluated at the beginning of
% |\@@_begin_line:|... |\@@_end_line:| and cleared at the end. It will be used
% by LPEG acting between the lines of the Python code in order to add
% instructions to be executed at the beginning of the line.
%    \begin{macrocode} 
\tl_new:N \g_@@_before_line_tl
%    \end{macrocode}
%
% \bigskip
%    \begin{macrocode}
\tl_new:N \g_@@_after_line_tl
%    \end{macrocode}
%
% \smallskip
% For example, the LPEG |Prompt| will trigger the following command which will
% insert an instruction in the hook |\g_@@_begin_line_hook| to specify that a
% background must be inserted to the current line of code.
%    \begin{macrocode}
\cs_new_protected:Npn \@@_prompt: 
  { 
    \tl_gset:Nn \g_@@_before_line_tl
      { 
        \tl_if_empty:NF \l_@@_prompt_bg_color_tl 
          { \clist_set:No \l_@@_bg_color_clist { \l_@@_prompt_bg_color_tl } }
      } 
  }
%    \end{macrocode}
%
%
% \bigskip
% The spaces at the end of a line of code are deleted by \pkg{piton}.
% However, it's not actually true: they are replace by |\@@_trailing_space:|.
%    \begin{macrocode}
\cs_new_protected:Npn \@@_trailing_space: { }
%    \end{macrocode}
% When we have to rescan some pieces of code, we will use |\@@_piton:n|, which
% we will set |\@@_trailing_space:| equal to |\space|.
%
% \bigskip
%    \begin{macrocode}
\bool_new:N \g_@@_color_is_none_bool
\bool_new:N \g_@@_next_color_is_none_bool
%    \end{macrocode}
% 
%
% \bigskip
% \subsubsection{Detected commands}
%
% There are four keys for ``detected commands and environments'':
% |detected-commands|, |raw-detected-commands|, |beamer-commands| and
% |beamer-environments|. 
%
% For each of those keys, we keep a clist of the names of such detected
% commands and environments. For the commands, the corresponding |clist| will
% contain the name of the commands \emph{wihtout} the backlash.
%
%    \begin{macrocode}
\clist_new:N \l_@@_detected_commands_clist
\clist_new:N \l_@@_raw_detected_commands_clist
\clist_new:N \l_@@_beamer_commands_clist
\clist_set:Nn \l_@@_beamer_commands_clist
  { uncover, only , visible , invisible , alert , action}
\clist_new:N \l_@@_beamer_environments_clist
\clist_set:Nn \l_@@_beamer_environments_clist
  { uncoverenv , onlyenv , visibleenv , invisibleenv , alertenv , actionenv }
%    \end{macrocode}
% Remark that, since we have used clists, these clists, as token lists are
% ``purified'': there is no empty component and for each component, there is no
% space on both sides.
%
% \bigskip
% Of course, the value of those clists may be modified during the preamble of
% the document by using the corresponding key ('detected-commands', etc.).
%
% However, after the |\begin{document}|, it's no longer possible to modify those
% clists because their contents will be used in the construction of the main
% \textsc{lpeg} for each computer language.
%
% \bigskip
% However, in a |\AtBeginDocument|, we will convert those clists into
% ``toks registers'' of TeX.
%    \begin{macrocode}
\hook_gput_code:nnn { begindocument } { . }
  { 
    \newtoks \PitonDetectedCommands
    \newtoks \PitonRawDetectedCommands
    \newtoks \PitonBeamerCommands
    \newtoks \PitonBeamerEnvironments
%    \end{macrocode}
% L3 does \emph{not} support those ``toks registers'' but it's still possible to
% affect to the ``toks registers'' the content of the clists with a L3-like syntax.
%    \begin{macrocode}
    \exp_args:NV \PitonDetectedCommands \l_@@_detected_commands_clist 
    \exp_args:NV \PitonRawDetectedCommands \l_@@_raw_detected_commands_clist
    \exp_args:NV \PitonBeamerCommands \l_@@_beamer_commands_clist
    \exp_args:NV \PitonBeamerEnvironments \l_@@_beamer_environments_clist
  }
%    \end{macrocode}
%
% Then at the beginning of the document, when we will load the Lua file
% |piton.lua|, we will read those ``toks registers'' within Lua (with
% |tex.toks|) and convert them into Lua tables (and, then, use those tables to
% construct \textsc{lpeg}).
%
%
% 
% \bigskip
% When the key |vertical-detected-commands| is used, we will have to redefine
% the corresponding commands in |\@@_pre_composition:|.
%
% The instructions for these redefinitions will be put in the following token list.
%    \begin{macrocode}
\tl_new:N \g_@@_def_vertical_commands_tl
%    \end{macrocode}
%
% \bigskip
%    \begin{macrocode}
\cs_new_protected:Npn \@@_vertical_commands:n #1
  {
    \clist_put_right:Nn \l_@@_raw_detected_commands_clist { #1 } 
    \clist_map_inline:nn { #1 }
      {
        \cs_set_eq:cc { @@ _ old _ ##1 : } { ##1 }
        \cs_new_protected:cn { @@ _ new _ ##1 : n }
          {
            \bool_if:nTF
              { \l_@@_tcolorbox_bool || ! \str_if_empty_p:N \l_@@_box_str }
              {
                \tl_gput_right:Nn \g_@@_after_line_tl
                  { \use:c { @@ _old _ ##1 : } { ####1 } }
              }
              {
                \cs_if_exist:cTF { g_@@_after_line _ \int_use:N \g_@@_line_int _ tl }
                  { \tl_gput_right:cn }
                  { \tl_gset:cn }
                  { g_@@_after_line _ \int_eval:n { \g_@@_line_int + 1 } _ tl }
                  { \use:c { @@ _old _ ##1 : } { ####1 } }
              }
          }
        \tl_gput_right:Nn \g_@@_def_vertical_commands_tl
          { \cs_set_eq:cc { ##1 } { @@ _ new _ ##1 : n } }
      }
  }
%    \end{macrocode}
%
%
% 
% \bigskip
% \subsubsection{Treatment of a line of code}
% 
% 
%    \begin{macrocode}
\cs_new_protected:Npn \@@_replace_spaces:n #1 
  {
    \tl_set:Nn \l_tmpa_tl { #1 } 
    \bool_if:NTF \l_@@_show_spaces_bool
      { 
        \tl_set:Nn \l_@@_space_in_string_tl { ␣ } % U+2423
        \tl_replace_all:NVn \l_tmpa_tl \c_catcode_other_space_tl { ␣ } % U+2423
      } 
      {
%    \end{macrocode}
% If the key |break-lines-in-Piton| is in force, we replace all the characters
% U+0020 (that is to say the spaces) by |\@@_breakable_space:|. Remark that,
% except the spaces inserted in the LaTeX comments (and maybe in the math
% comments), all these spaces are of catcode ``other'' (=12) and are
% unbreakable.
%    \begin{macrocode}
        \bool_if:NT \l_@@_break_lines_in_Piton_bool
          { 
            \tl_if_eq:NnF \l_@@_space_in_string_tl { ␣ }
              { \tl_set_eq:NN \l_@@_space_in_string_tl \@@_breakable_space: }
%    \end{macrocode}
% 
% In the following code, we have to replace all the spaces in the token list
% |\l_tmpa_tl|. That means that this replacement must be ``recursive'': even the
% spaces which are within brace groups (|{...}|) must be replaced. For instance,
% the spaces in long strings of Python are within such groups since there are
% within a command |\PitonStyle{String.Long}{...}|. That's why the
% use of |\tl_replace_all:Nnn| is not enough. 
%
% The first implementation was using |\regex_replace_all:nnN|
%
% |\regex_replace_all:nnN { \x20 } { \c { @@_breakable_space: } } \l_tmpa_tl|
%
% but that programmation was certainly slow.
%
% Now, we use |\tl_replace_all:NVn| \emph{but}, in the styles
% |String.Long.Internal| we replace the spaces with
% |\@@_breakable_space:| by another use of the same technic with
% |\tl_replace_all:NVn|. We do the same jog for the \emph{doc strings} of Python and
% for the comments.
%    \begin{macrocode}
            \tl_replace_all:NVn \l_tmpa_tl 
              \c_catcode_other_space_tl 
              \@@_breakable_space:
          }
      }
   \l_tmpa_tl
  }
\cs_generate_variant:Nn \@@_replace_spaces:n { o }
%    \end{macrocode}
% 
% \bigskip
% In the contents provided by Lua, each line of the Python code will be
% surrounded by |\@@_begin_line:| and |\@@_end_line:|. 

% |\@@_begin_line:| is a
% TeX command with a delimited argument (|\@@_end_line:| is the marker for the
% end of the argument).
%
% However, we define also |\@@_end_line:| as no-op, because, when the last line
% of the listing is the end of an environment of Beamer (eg |\end{uncoverenv}|),
% we will have a token |\@@_end_line:| added at the end without any
% corresponding |\@@_begin_line:|).
%    \begin{macrocode}
\cs_set_protected:Npn \@@_end_line: { }
%    \end{macrocode}
% 
% \bigskip
%    \begin{macrocode}
\cs_set_protected:Npn \@@_begin_line: #1 \@@_end_line:
  { 
    \group_begin:
    \g_@@_before_line_tl
    \int_gzero:N \g_@@_indentation_int
%    \end{macrocode}
%
% \medskip
% We put the potential number of line, the potential left margin and the
% potential background.
%    \begin{macrocode}
    \hbox_set:Nn \l_@@_line_box 
      { 
        \skip_horizontal:N \l_@@_left_margin_dim
        \bool_if:NT \l_@@_line_numbers_bool
          {
%    \end{macrocode}
% |\l_tmpa_int| will be equal to $1$ when the current line is not empty.
%    \begin{macrocode}
            \int_set:Nn \l_tmpa_int 
              {
                \lua_now:e 
                  { 
                    tex.sprint
                      ( 
                        luatexbase.catcodetables.expl ,
%    \end{macrocode}
% Since the argument of |tostring| will be a integer of Lua (\emph{integer} is a
% sub-type of \emph{number} introduced in Lua 5.3), the output will be of the
% form |"3"| (and not |"3.0"|) which is what we want for |\int_set:Nn|.
%    \begin{macrocode}
                        tostring 
                          ( piton.empty_lines
                              [ \int_eval:n { \g_@@_line_int + 1 } ] 
                          ) 
                      ) 
                  } 
              }
            \bool_lazy_or:nnT
              { \int_compare_p:nNn \l_tmpa_int = \c_one_int }
              { ! \l_@@_skip_empty_lines_bool }
              { \int_gincr:N \g_@@_visual_line_int }
            \bool_lazy_or:nnT
              { \int_compare_p:nNn \l_tmpa_int = \c_one_int }
              { ! \l_@@_skip_empty_lines_bool && \l_@@_label_empty_lines_bool }
              { \@@_print_number: }
          }
%    \end{macrocode}
% If there is a background, we must remind that there is a left margin of 0.5~em
% for the background...
%    \begin{macrocode}
        \clist_if_empty:NF \l_@@_bg_color_clist 
          { 
%    \end{macrocode}
% ... but if only if the key |left-margin| is not used !
%    \begin{macrocode}
            \dim_compare:nNnT \l_@@_left_margin_dim = \c_zero_dim 
              { \skip_horizontal:n { 0.5 em } }
          }
%    \end{macrocode}
%
% \bigskip
%    \begin{macrocode}
        \bool_if:NTF \l_@@_minimize_width_bool
          {
            \hbox_set:Nn \l_tmpa_box
              {
                \language = -1
                \raggedright 
                \strut 
                \@@_replace_spaces:n { #1 }
                \strut \hfil 
              } 
             \dim_compare:nNnTF { \box_wd:N \l_tmpa_box } < \l_@@_code_width_dim 
               { \box_use:N \l_tmpa_box }
               { \@@_vtop_of_code:n { #1 } }               
          }
          { \@@_vtop_of_code:n { #1 } }
      }
%    \end{macrocode}
% Now, the line of code is composed in the box |\l_@@_line_box|.
% 
% \bigskip
%    \begin{macrocode}
    \box_set_dp:Nn \l_@@_line_box { \box_dp:N \l_@@_line_box + 1.25 pt } 
    \box_set_ht:Nn \l_@@_line_box { \box_ht:N \l_@@_line_box + 1.25 pt }
%    \end{macrocode}
%
%    \begin{macrocode}
    \dim_gset_eq:NN \g_@@_width_dim \l_@@_width_dim
%    \end{macrocode}
%
%    \begin{macrocode}
    \box_use_drop:N \l_@@_line_box
%    \end{macrocode}
%
%    \begin{macrocode}
    \group_end:
    \g_@@_after_line_tl
    \tl_gclear:N \g_@@_after_line_tl
    \tl_gclear:N \g_@@_before_line_tl
  }
%    \end{macrocode}
%
% \bigskip
%    \begin{macrocode}
\cs_new_protected:Npn \@@_vtop_of_code:n #1
  {
    \vbox_top:n
      {
        \hsize = \l_@@_code_width_dim 
        \language = -1
        \raggedright 
        \strut 
        \@@_replace_spaces:n { #1 }
        \strut \hfil 
      }  
  }
%    \end{macrocode}
% 
% \bigskip
% Of course, the following command will be used when the key |background-color|
% is used.
%
% The content of the line has been previously set in |\l_@@_line_box|.
%    \begin{macrocode}
\cs_new_protected:Npn \@@_add_background_to_line_and_use:
  {
    \vtop
      {
        \offinterlineskip
        \hbox
          {
%    \end{macrocode}
% |\@@_color:N| sets the current color but also sets the booleans
% |\g_@@_color_is_none_bool| and |\g_@@_next_color_is_none_bool|. It uses the value of
% |\g_@@_line_int| which is the number of the current line.
%    \begin{macrocode}
            \@@_color:N \l_@@_bg_color_clist
%    \end{macrocode}
% The colored panels are overlapping. However, if the special color |none| is used
% we must not put such overlapping.
%    \begin{macrocode}
            \dim_set:Nn \l_tmpa_dim { \box_dp:N \l_@@_line_box }
            \bool_if:NT \g_@@_next_color_is_none_bool
              { \dim_sub:Nn \l_tmpa_dim { 2.5 pt } }
%    \end{macrocode}
% When |\g_@@_color_is_none_bool| is in force, we will compose a |\vrule| of
% width 0~pt. We need that |\vrule| because it will be a strut.
%    \begin{macrocode}
            \bool_if:NTF \g_@@_color_is_none_bool
              { \dim_zero:N \l_tmpb_dim }
              { \dim_set_eq:NN \l_tmpb_dim \g_@@_width_dim }
            \dim_set:Nn \l_@@_tmpc_dim { \box_ht:N \l_@@_line_box }
%    \end{macrocode}
% Now, the colored panel.
%    \begin{macrocode}
            \dim_compare:nNnTF \l_@@_rounded_corners_dim > \c_zero_dim
              {
                \int_compare:nNnTF \g_@@_line_int = \c_one_int
                  {
                    \begin{tikzpicture}[baseline = 0cm]
                    \fill (0,0)
                          [rounded~corners = \l_@@_rounded_corners_dim]
                          -- (0,\l_@@_tmpc_dim)
                          -- (\l_tmpb_dim,\l_@@_tmpc_dim)
                          [sharp~corners] -- (\l_tmpb_dim,-\l_tmpa_dim)
                          -- (0,-\l_tmpa_dim)
                          -- cycle ;
                    \end{tikzpicture}                    
                  }
                  {
                    \int_compare:nNnTF \g_@@_line_int = \g_@@_nb_lines_int
                      {
                        \begin{tikzpicture}[baseline = 0cm]
                        \fill (0,0) -- (0,\l_@@_tmpc_dim)
                              -- (\l_tmpb_dim,\l_@@_tmpc_dim)
                              [rounded~corners = \l_@@_rounded_corners_dim]
                              -- (\l_tmpb_dim,-\l_tmpa_dim)
                              -- (0,-\l_tmpa_dim)
                              -- cycle ;
                        \end{tikzpicture}                    
                      }
                      {
                        \vrule height \l_@@_tmpc_dim
                        depth \l_tmpa_dim
                        width \l_tmpb_dim
                      }
                  }
              }
              {
                \vrule height \l_@@_tmpc_dim
                depth \l_tmpa_dim
                width \l_tmpb_dim
              }
          }
        \bool_if:NT \g_@@_next_color_is_none_bool
          { \skip_vertical:n { 2.5 pt } }
        \skip_vertical:n { - \box_ht_plus_dp:N \l_@@_line_box } 
        \box_use_drop:N \l_@@_line_box
     }
  }
%    \end{macrocode}
%
%
% \bigskip
% The command |\@@_color:N| will take in as argument a reference to a
% comma-separated list of colors. A color will be picked by using the value of
% |\g_@@_line_int| (modulo the number of colors in the list).
%    \begin{macrocode}
\cs_set_protected:Npn \@@_color:N #1 
  {
    \int_set:Nn \l_tmpa_int { \clist_count:N #1 }
    \int_set:Nn \l_tmpb_int { \int_mod:nn \g_@@_line_int \l_tmpa_int + 1 }
    \tl_set:Ne \l_tmpa_tl { \clist_item:Nn #1 \l_tmpb_int }
    \tl_if_eq:NnTF \l_tmpa_tl { none }
      { \bool_gset_true:N \g_@@_color_is_none_bool }
      {
        \bool_gset_false:N \g_@@_color_is_none_bool
        \@@_color_i:o \l_tmpa_tl
      }
% \end{macrocode}
% We are looking for the next color because we have to know whether that
% color is the special color |none|.
% \begin{macrocode}
    \int_compare:nNnTF { \g_@@_line_int + 1 } = \g_@@_nb_lines_int
      { \bool_gset_false:N \g_@@_next_color_is_none_bool }
      {
        \int_set:Nn \l_tmpb_int
          { \int_mod:nn { \g_@@_line_int + 1 } \l_tmpa_int + 1 }
        \tl_set:Ne \l_tmpa_tl { \clist_item:Nn #1 \l_tmpb_int }
        \tl_if_eq:NnTF \l_tmpa_tl { none }
          { \bool_gset_true:N \g_@@_next_color_is_none_bool }
          { \bool_gset_false:N \g_@@_next_color_is_none_bool }
      }
  }
%    \end{macrocode}
%
%
% The following command |\@@_color:n| will accept both the instruction
% |\@@_color:n { red!15 }| and the instruction |\@@_color:n { [rgb]{0.9,0.9,0} }|.
%    \begin{macrocode}
\cs_set_protected:Npn \@@_color_i:n #1
  {
    \tl_if_head_eq_meaning:nNTF { #1 } [ 
      { 
        \tl_set:Nn \l_tmpa_tl { #1 }
        \tl_set_rescan:Nno \l_tmpa_tl { } \l_tmpa_tl 
        \exp_last_unbraced:No \color \l_tmpa_tl
      }
      { \color { #1 } }
  }
\cs_generate_variant:Nn \@@_color_i:n { o }
%    \end{macrocode}
% 
% \bigskip
% The command |\@@_par:| will be inserted by Lua between two lines of the
% informatic listing.
% \begin{itemize}
% \item In fact, it will be inserted between two commands
% |\@@_begin_line:|...|\@@_end_of_line:|. 
% \item When the key |break-lines-in-Piton| is in force, a line of the
% informatic code (the \emph{input}) may result in several lines in the
% \textsc{pdf} (the \emph{output}). 
% \item Remind that |\@@_par:| has a rather complex behaviour because it will
% finish and start paragraphs.
% \end{itemize}
%    \begin{macrocode}
\cs_new_protected:Npn \@@_par: 
  { 
%    \end{macrocode}
% We recall that |\g_@@_line_int| is \emph{not} used for the number of line
% printed in the \textsc{pdf} (when |line-numbers| is in force)...
%    \begin{macrocode}
    \int_gincr:N \g_@@_line_int
%    \end{macrocode}
% ... it will be used to allow or disallow page breaks.
%
%
% Each line in the listing is composed in a box of TeX (which may contain
% several lines when the key |break-lines-in-Piton| is in force) put in a
% paragraph. 
%    \begin{macrocode}
    \par 
%    \end{macrocode}
% We now add a |\kern| because each line of code is overlapping vertically by a
% quantity of 2.5~pt in order to have a good background (when |background-color|
% is in force). We need to use a |\kern| (in fact |\par\kern...|) and not a
% |\vskip| because page breaks should \emph{not} be allowed on that kern.
%    \begin{macrocode}
    \kern -2.5 pt 
%    \end{macrocode}
% Now, we control page breaks after the paragraph. 
%    \begin{macrocode}
    \@@_add_penalty_for_the_line:
  } 
%    \end{macrocode}
% After the command |\@@_par:|, we will usually have a command |\@@_begin_line:|.
%
% \bigskip
% The following command |\@@_breakable_space:| is for breakable spaces in the
% environments |{Piton}| and the listings of |\PitonInputFile| and \emph{not} for the
% commands |\piton|.
%    \begin{macrocode}
\cs_set_protected:Npn \@@_breakable_space:
  { 
    \discretionary 
      { \hbox:n { \color { gray } \l_@@_end_of_broken_line_tl } } 
      { 
        \hbox_overlap_left:n 
          { 
            { 
              \normalfont \footnotesize \color { gray } 
              \l_@@_continuation_symbol_tl 
            } 
            \skip_horizontal:n { 0.3 em }  
            \clist_if_empty:NF \l_@@_bg_color_clist 
              { \skip_horizontal:n { 0.5 em } }
          }  
        \bool_if:NT \l_@@_indent_broken_lines_bool
          { 
            \hbox:n
              { 
                \prg_replicate:nn { \g_@@_indentation_int } { ~ } 
                { \color { gray } \l_@@_csoi_tl }
              } 
          } 
      } 
      { \hbox { ~ } }
  }
%    \end{macrocode}
% 
% \bigskip
% \subsubsection{PitonOptions}
%
%
%
% \medskip
%    \begin{macrocode}
\bool_new:N \l_@@_line_numbers_bool 
\bool_new:N \l_@@_skip_empty_lines_bool 
\bool_set_true:N \l_@@_skip_empty_lines_bool 
\bool_new:N \l_@@_line_numbers_absolute_bool
\tl_new:N \l_@@_line_numbers_format_bool
\tl_new:N \l_@@_line_numbers_format_tl
\tl_set:Nn \l_@@_line_numbers_format_tl { \footnotesize \color { gray } }
\bool_new:N \l_@@_label_empty_lines_bool
\bool_set_true:N \l_@@_label_empty_lines_bool 
\int_new:N \l_@@_number_lines_start_int
\bool_new:N \l_@@_resume_bool 
\bool_new:N \l_@@_split_on_empty_lines_bool
\bool_new:N \l_@@_splittable_on_empty_lines_bool
%    \end{macrocode}
% 
%
%  \bigskip
%    \begin{macrocode}
\keys_define:nn { PitonOptions / marker }
  {
    beginning .code:n = \cs_set:Nn \@@_marker_beginning:n { #1 } ,   
    beginning .value_required:n = true , 
    end .code:n = \cs_set:Nn \@@_marker_end:n { #1 } ,   
    end .value_required:n = true ,
    include-lines .bool_set:N = \l_@@_marker_include_lines_bool ,
    include-lines .default:n = true ,
    unknown .code:n = \@@_error:n { Unknown~key~for~marker } 
  }
%    \end{macrocode}
%
% \bigskip
%    \begin{macrocode}
\keys_define:nn { PitonOptions / line-numbers }
  {
    true .code:n = \bool_set_true:N \l_@@_line_numbers_bool , 
    false .code:n = \bool_set_false:N \l_@@_line_numbers_bool , 

    start .code:n = 
      \bool_set_true:N \l_@@_line_numbers_bool 
      \int_set:Nn \l_@@_number_lines_start_int { #1 }  ,
    start .value_required:n = true ,

    skip-empty-lines .code:n = 
      \bool_if:NF \l_@@_in_PitonOptions_bool
        { \bool_set_true:N \l_@@_line_numbers_bool }
      \str_if_eq:nnTF { #1 } { false }
        { \bool_set_false:N \l_@@_skip_empty_lines_bool } 
        { \bool_set_true:N \l_@@_skip_empty_lines_bool } ,
    skip-empty-lines .default:n = true , 

    label-empty-lines .code:n = 
      \bool_if:NF \l_@@_in_PitonOptions_bool
        { \bool_set_true:N \l_@@_line_numbers_bool }
      \str_if_eq:nnTF { #1 } { false }
        { \bool_set_false:N \l_@@_label_empty_lines_bool } 
        { \bool_set_true:N \l_@@_label_empty_lines_bool } ,
    label-empty-lines .default:n = true ,

    absolute .code:n = 
      \bool_if:NTF \l_@@_in_PitonOptions_bool
        { \bool_set_true:N \l_@@_line_numbers_absolute_bool }
        { \bool_set_true:N \l_@@_line_numbers_bool }
      \bool_if:NT \l_@@_in_PitonInputFile_bool 
        { 
          \bool_set_true:N \l_@@_line_numbers_absolute_bool 
          \bool_set_false:N \l_@@_skip_empty_lines_bool 
        } ,
    absolute .value_forbidden:n = true ,

    resume .code:n = 
      \bool_set_true:N \l_@@_resume_bool 
      \bool_if:NF \l_@@_in_PitonOptions_bool 
        { \bool_set_true:N \l_@@_line_numbers_bool } , 
    resume .value_forbidden:n = true ,

    sep .dim_set:N = \l_@@_numbers_sep_dim ,
    sep .value_required:n = true , 

    format .tl_set:N = \l_@@_line_numbers_format_tl ,
    format .value_required:n = true ,

    unknown .code:n = \@@_error:n { Unknown~key~for~line-numbers } 
  }
%    \end{macrocode}
%
%
% \bigskip
% Be careful! The name of the following set of keys must be considered as
% public! Hence, it should \emph{not} be changed.
% 
%    \begin{macrocode}
\keys_define:nn { PitonOptions }
  {
    rounded-corners .code:n = 
      \IfPackageLoadedTF { tikz }
        { \dim_set:Nn \l_@@_rounded_corners_dim { #1 } }
        { \@@_err_rounded_corners_without_Tikz: } ,
    rounded-corners .default:n = 4 pt ,
    tcolorbox .code:n =
      \IfPackageLoadedTF { tcolorbox }
        {
          \pgfkeysifdefined { / tcb / libload / breakable }
            {
              \str_if_eq:eeTF { #1 } { true }
                { \bool_set_true:N \l_@@_tcolorbox_bool }
                { \bool_set_false:N \l_@@_tcolorbox_bool }
            }
            { \@@_error:n { library~breakable~not~loaded } }

        }
        { \@@_error:n { tcolorbox~not~loaded } } ,
    tcolorbox .default:n = true ,
    box .choices:nn = { c , t , b , m }
      { \str_set_eq:NN \l_@@_box_str \l_keys_choice_tl } , 
    box .default:n = c ,
    break-strings-anywhere .bool_set:N = \l_@@_break_strings_anywhere_bool ,
    break-strings-anywhere .default:n = true , 
    break-numbers-anywhere .bool_set:N = \l_@@_break_numbers_anywhere_bool ,
    break-numbers-anywhere .default:n = true  ,
%    \end{macrocode}
% First, we put keys that should be available only in the preamble.
%    \begin{macrocode}
    detected-commands .code:n = 
      \clist_put_right:Nn \l_@@_detected_commands_clist { #1 } ,
    detected-commands .value_required:n = true , 
    detected-commands .usage:n = preamble ,
    vertical-detected-commands .code:n = \@@_vertical_commands:n { #1 } ,
    vertical-detected-commands .value_required:n = true , 
    vertical-detected-commands .usage:n = preamble , 
    raw-detected-commands .code:n = 
      \clist_put_right:Nn \l_@@_raw_detected_commands_clist { #1 } ,
    raw-detected-commands .value_required:n = true , 
    raw-detected-commands .usage:n = preamble , 
    detected-beamer-commands .code:n = 
      \@@_error_if_not_in_beamer:
      \clist_put_right:Nn \l_@@_beamer_commands_clist { #1 } ,
    detected-beamer-commands .value_required:n = true , 
    detected-beamer-commands .usage:n = preamble , 
    detected-beamer-environments .code:n = 
      \@@_error_if_not_in_beamer:
      \clist_put_right:Nn \l_@@_beamer_environments_clist { #1 } ,
    detected-beamer-environments .value_required:n = true , 
    detected-beamer-environments .usage:n = preamble , 
%    \end{macrocode}
%
% 
% Remark that the command |\lua_escape:n| is fully expandable. That's why we use
% |\lua_now:e|.
%    \begin{macrocode}
    begin-escape .code:n = 
      \lua_now:e { piton.begin_escape = "\lua_escape:n{#1}" } ,
    begin-escape .value_required:n = true , 
    begin-escape .usage:n = preamble , 

    end-escape   .code:n = 
      \lua_now:e { piton.end_escape = "\lua_escape:n{#1}" } ,
    end-escape   .value_required:n = true ,
    end-escape .usage:n = preamble ,

    begin-escape-math .code:n = 
      \lua_now:e { piton.begin_escape_math = "\lua_escape:n{#1}" } ,
    begin-escape-math .value_required:n = true , 
    begin-escape-math .usage:n = preamble ,

    end-escape-math .code:n = 
      \lua_now:e { piton.end_escape_math = "\lua_escape:n{#1}" } ,
    end-escape-math .value_required:n = true ,
    end-escape-math .usage:n = preamble ,

    comment-latex .code:n = \lua_now:n { comment_latex = "#1" } ,   
    comment-latex .value_required:n = true ,
    comment-latex .usage:n = preamble ,

    math-comments .bool_gset:N = \g_@@_math_comments_bool ,
    math-comments .default:n  = true ,
    math-comments .usage:n = preamble , 
%    \end{macrocode}
% 
% \bigskip
% Now, general keys.
%    \begin{macrocode}
    language      .code:n = 
      \str_set:Ne \l_piton_language_str { \str_lowercase:n { #1 } } , 
    language      .value_required:n  = true ,
    path          .code:n = 
      \seq_clear:N \l_@@_path_seq
      \clist_map_inline:nn { #1 }
        { 
          \str_set:Nn \l_tmpa_str { ##1 } 
          \seq_put_right:No \l_@@_path_seq { \l_tmpa_str }
        } ,
    path             .value_required:n  = true , 
%    \end{macrocode}
% The initial value of the key |path| is not empty: it's |.|, that is to say a
% comma separated list with only one component which is |.|, the current directory.
%    \begin{macrocode}
    path             .initial:n         = . ,
    path-write       .str_set:N         = \l_@@_path_write_str ,
    path-write       .value_required:n  = true , 
    font-command     .tl_set:N          = \l_@@_font_command_tl ,
    font-command     .value_required:n  = true ,
    gobble           .int_set:N         = \l_@@_gobble_int , 
    gobble           .default:n         = -1 ,
    auto-gobble      .code:n            = \int_set:Nn \l_@@_gobble_int { -1 } , 
    auto-gobble      .value_forbidden:n = true ,
    env-gobble       .code:n            = \int_set:Nn \l_@@_gobble_int { -2 } , 
    env-gobble       .value_forbidden:n = true ,
    tabs-auto-gobble .code:n            = \int_set:Nn \l_@@_gobble_int { -3 } , 
    tabs-auto-gobble .value_forbidden:n = true ,

    splittable-on-empty-lines .bool_set:N = \l_@@_splittable_on_empty_lines_bool ,
    splittable-on-empty-lines .default:n  = true ,

    split-on-empty-lines .bool_set:N = \l_@@_split_on_empty_lines_bool ,
    split-on-empty-lines .default:n  = true ,

    split-separation .tl_set:N         = \l_@@_split_separation_tl ,
    split-separation .value_required:n = true ,
 
    marker .code:n = 
      \bool_lazy_or:nnTF
        \l_@@_in_PitonInputFile_bool 
        \l_@@_in_PitonOptions_bool 
        { \keys_set:nn { PitonOptions / marker } { #1 } } 
        { \@@_error:n { Invalid~key } } , 
    marker .value_required:n = true , 

    line-numbers .code:n = 
      \keys_set:nn { PitonOptions / line-numbers } { #1 } ,
    line-numbers .default:n = true , 

    splittable       .int_set:N         = \l_@@_splittable_int ,
    splittable       .default:n         = 1 , 
    background-color .clist_set:N       = \l_@@_bg_color_clist ,
    background-color .value_required:n  = true ,
    prompt-background-color .tl_set:N         = \l_@@_prompt_bg_color_tl ,
    prompt-background-color .value_required:n = true ,
% \end{macrocode}
% With the tuning |write=false|, the content of the environment won't be parsed
% and won't be printed on the \textsc{pdf}. However, the Lua variables |piton.last_code|
% and |piton.last_language| will be set (and, hence, |piton.get_last_code| will be
% operationnal). The keys |join| and |write| will be honoured.
% \begin{macrocode}
    print .bool_set:N = \l_@@_print_bool ,
    print .value_required:n = true ,

    width .code:n = 
      \str_if_eq:nnTF  { #1 } { min } 
        { 
          \bool_set_true:N \l_@@_minimize_width_bool 
          \dim_zero:N \l_@@_width_dim 
        }
        { 
          \bool_set_false:N \l_@@_minimize_width_bool
          \dim_set:Nn \l_@@_width_dim { #1 } 
        } , 
    width .value_required:n  = true ,

    max-width .code:n =
      \bool_set_true:N \l_@@_minimize_width_bool
      \dim_set:Nn \l_@@_width_dim { #1 } ,
    max-width .value_required:n = true ,
    
    write .str_set:N = \l_@@_write_str ,
    write .value_required:n = true ,
%     \end{macrocode}
% For the key |join|, we convert immediatly the value of the key in utf16
% (with the \text{bom} big endian that will be automatically inserted)
% written in hexadecimal (what L3 calls the \emph{escaping}). Indeed, we will
% have to write that value in the key |/UF| of a |/Filespec| (between angular
% brackets |<| and |>| since it is in hexadecimal). It's prudent to do that
% conversion right now since that value will transit by the Lua of LuaTeX.
%     \begin{macrocode}
    join .code:n
      = \str_set_convert:Nnnn \l_@@_join_str { #1 } { } { utf16/hex } ,
    join .value_required:n = true ,

    left-margin      .code:n =
      \str_if_eq:nnTF { #1 } { auto }
        { 
          \dim_zero:N \l_@@_left_margin_dim 
          \bool_set_true:N \l_@@_left_margin_auto_bool
        }
        { 
          \dim_set:Nn \l_@@_left_margin_dim { #1 } 
          \bool_set_false:N \l_@@_left_margin_auto_bool
        } ,
    left-margin      .value_required:n  = true ,

    tab-size         .int_set:N         = \l_@@_tab_size_int ,
    tab-size         .value_required:n  = true , 
    show-spaces      .bool_set:N        = \l_@@_show_spaces_bool ,
    show-spaces      .value_forbidden:n = true ,
    show-spaces-in-strings .code:n      = 
        \tl_set:Nn \l_@@_space_in_string_tl { ␣ } , % U+2423
    show-spaces-in-strings .value_forbidden:n = true ,
    break-lines-in-Piton .bool_set:N    = \l_@@_break_lines_in_Piton_bool ,
    break-lines-in-Piton .default:n     = true ,
    break-lines-in-piton .bool_set:N    = \l_@@_break_lines_in_piton_bool ,
    break-lines-in-piton .default:n     = true ,
    break-lines .meta:n = { break-lines-in-piton , break-lines-in-Piton } , 
    break-lines .value_forbidden:n      = true ,
    indent-broken-lines .bool_set:N     = \l_@@_indent_broken_lines_bool ,
    indent-broken-lines .default:n      = true ,
    end-of-broken-line  .tl_set:N       = \l_@@_end_of_broken_line_tl ,
    end-of-broken-line  .value_required:n = true , 
    continuation-symbol .tl_set:N       = \l_@@_continuation_symbol_tl ,
    continuation-symbol .value_required:n = true , 
    continuation-symbol-on-indentation .tl_set:N = \l_@@_csoi_tl ,
    continuation-symbol-on-indentation .value_required:n = true , 

    first-line .code:n = \@@_in_PitonInputFile:n 
      { \int_set:Nn \l_@@_first_line_int { #1 } } ,
    first-line .value_required:n = true ,

    last-line .code:n = \@@_in_PitonInputFile:n 
      { \int_set:Nn \l_@@_last_line_int { #1 } } ,
    last-line .value_required:n = true , 

    begin-range .code:n = \@@_in_PitonInputFile:n 
      { \str_set:Nn \l_@@_begin_range_str { #1 } } ,
    begin-range .value_required:n = true ,

    end-range .code:n = \@@_in_PitonInputFile:n 
      { \str_set:Nn \l_@@_end_range_str { #1 } } ,
    end-range .value_required:n = true ,

    range .code:n = \@@_in_PitonInputFile:n 
      { 
        \str_set:Nn \l_@@_begin_range_str { #1 } 
        \str_set:Nn \l_@@_end_range_str { #1 } 
      } ,
    range .value_required:n = true ,

    env-used-by-split .code:n = 
      \lua_now:n { piton.env_used_by_split = '#1' } , 
    env-used-by-split .initial:n = Piton ,

    resume .meta:n = line-numbers/resume , 

    unknown .code:n = \@@_error:n { Unknown~key~for~PitonOptions } ,

    % deprecated
    all-line-numbers .code:n = 
      \bool_set_true:N \l_@@_line_numbers_bool 
      \bool_set_false:N \l_@@_skip_empty_lines_bool ,
  }
%    \end{macrocode}
%
% \bigskip
%    \begin{macrocode}
\cs_new_protected:Npn \@@_err_rounded_corners_without_Tikz:
  {
    \@@_error:n { rounded-corners~without~Tikz }
    \cs_gset:Npn \@@_err_rounded_corners_without_Tikz: { }
  }
%    \end{macrocode}
%
% \bigskip
%    \begin{macrocode}
\cs_new_protected:Npn \@@_in_PitonInputFile:n #1
  {
    \bool_if:NTF \l_@@_in_PitonInputFile_bool 
      { #1 }
      { \@@_error:n { Invalid~key } } 
  }
%    \end{macrocode}
%
% 
%
% \bigskip
%    \begin{macrocode}
\NewDocumentCommand \PitonOptions { m } 
  { 
    \bool_set_true:N \l_@@_in_PitonOptions_bool
    \keys_set:nn { PitonOptions } { #1 }
    \bool_set_false:N \l_@@_in_PitonOptions_bool
  }
%    \end{macrocode}
%
% 
% 
% \bigskip
% When using |\NewPitonEnvironment| a user may use |\PitonOptions| inside.
% However, the set of keys available should be different that in standard
% |\PitonOptions|. That's why we define a version of |\PitonOptions| with no
% restriction on the set of available keys and we will link that version to
% |\PitonOptions| in such environment.
%    \begin{macrocode}
\NewDocumentCommand \@@_fake_PitonOptions { } 
  { \keys_set:nn { PitonOptions } }
%    \end{macrocode}
% 
%
% 
% \bigskip
% \subsubsection{The numbers of the lines}
%
% \medskip
% The following counter will be used to count the lines in the code when the
% user requires the numbers of the lines to be printed (with |line-numbers|)
% whereas the counter |\g_@@_line_int| previously defined is \emph{not} used for
% that functionality.
%
%    \begin{macrocode}
\int_new:N \g_@@_visual_line_int 
%    \end{macrocode}
% 
%    \begin{macrocode}
\cs_new_protected:Npn \@@_incr_visual_line: 
  { 
    \bool_if:NF \l_@@_skip_empty_lines_bool
      { \int_gincr:N \g_@@_visual_line_int }
  }
%    \end{macrocode}
% 
%    \begin{macrocode}
\cs_new_protected:Npn \@@_print_number:
  {
    \hbox_overlap_left:n
      { 
        { 
          \l_@@_line_numbers_format_tl 
%    \end{macrocode}
% We put braces. Thus, the user may use the key |line-numbers/format| with a
% value such as |\fbox|. 
%    \begin{macrocode}
          { \int_to_arabic:n \g_@@_visual_line_int }
        } 
        \skip_horizontal:N \l_@@_numbers_sep_dim  
      }
  }
%    \end{macrocode}
% 
%
% 
% \bigskip
% \subsubsection{The main commands and environments for the final user}
%
% \bigskip
%    \begin{macrocode}
\NewDocumentCommand { \NewPitonLanguage } { O { } m ! o }
  {
    \tl_if_novalue:nTF { #3 }
%    \end{macrocode}
% The last argument is provided by curryfication.
%    \begin{macrocode}
      { \@@_NewPitonLanguage:nnn { #1 } { #2 } }
%    \end{macrocode}
% The two last arguments are provided by curryfication.
%    \begin{macrocode}
      { \@@_NewPitonLanguage:nnnnn { #1 } { #2 } { #3 } }
  }
%    \end{macrocode}
% 
% \bigskip
% The following property list will contain the definitions of the informatic
% languages as provided by the final user. However, if a language is defined
% over another base language, the corresponding list will contain the \emph{whole}
% definition of the language.
%    \begin{macrocode}
\prop_new:N \g_@@_languages_prop
%    \end{macrocode}
% 
% \bigskip
%    \begin{macrocode}
\keys_define:nn { NewPitonLanguage }
  {
    morekeywords .code:n = ,
    otherkeywords .code:n = , 
    sensitive .code:n = , 
    keywordsprefix .code:n = ,
    moretexcs .code:n = , 
    morestring .code:n = ,
    morecomment .code:n = ,
    moredelim .code:n = , 
    moredirectives .code:n = , 
    tag .code:n = , 
    alsodigit .code:n = , 
    alsoletter .code:n = , 
    alsoother .code:n = ,
    unknown .code:n = \@@_error:n { Unknown~key~NewPitonLanguage } 
  }
%    \end{macrocode}
%
% \bigskip
% The function |\@@_NewPitonLanguage:nnn| will be used when the language is
% \emph{not} defined above a base language (and a base dialect).
%    \begin{macrocode}
\cs_new_protected:Npn \@@_NewPitonLanguage:nnn #1 #2 #3 
  { 
%    \end{macrocode}
% We store in |\l_tmpa_tl| the name of the language with the potential dialect,
% that is to say, for example : |[AspectJ]{Java}|. We use |\tl_if_blank:nF|
% because the final user may have written |\NewPitonLanguage[ ]{Java}{...}|.
%    \begin{macrocode}
    \tl_set:Ne \l_tmpa_tl 
      { 
        \tl_if_blank:nF { #1 } { [ \str_lowercase:n { #1 } ] } 
        \str_lowercase:n { #2 } 
      }
%    \end{macrocode}
%
% The following set of keys is only used to raise an error when a key in unknown!
%    \begin{macrocode}
    \keys_set:nn { NewPitonLanguage } { #3 }
%    \end{macrocode}
% 
% We store in LaTeX the definition of the language because some languages may be
% defined with that language as base language.
%    \begin{macrocode}
    \prop_gput:Non \g_@@_languages_prop \l_tmpa_tl { #3 }
%    \end{macrocode}
% The Lua part of the package \pkg{piton} will be loaded in a
% |\AtBeginDocument|. Hence, we will put also in a |\AtBeginDocument| the
% use of the Lua function |piton.new_language| (which does the main job).
%    \begin{macrocode}
    \@@_NewPitonLanguage:on \l_tmpa_tl { #3 }
  }
%    \end{macrocode}
% 
%    \begin{macrocode}
\cs_new_protected:Npn \@@_NewPitonLanguage:nn #1 #2
  {
    \hook_gput_code:nnn { begindocument } { . }
      { \lua_now:e { piton.new_language("#1","\lua_escape:n{#2}") } }
  }
\cs_generate_variant:Nn \@@_NewPitonLanguage:nn { o }
%    \end{macrocode}
%
% \bigskip
% Now the case when the language is defined upon a base language.
%    \begin{macrocode}
\cs_new_protected:Npn \@@_NewPitonLanguage:nnnnn #1 #2 #3 #4 #5
  {
%    \end{macrocode}
% We store in |\l_tmpa_tl| the name of the base language with the dialect, that
% is to say, for example : |[AspectJ]{Java}|. We use |\tl_if_blank:nF| because
% the final user may have used |\NewPitonLanguage[Handel]{C}[ ]{C}{...}|
%    \begin{macrocode}
    \tl_set:Ne \l_tmpa_tl 
      { 
        \tl_if_blank:nF { #3 } { [ \str_lowercase:n { #3 } ] } 
        \str_lowercase:n { #4 }
      } 
%    \end{macrocode}
% We retrieve in |\l_tmpb_tl| the definition (as provided by the final user) of
% that base language. Caution: |\g_@@_languages_prop| does not contain all the
% languages provided by \pkg{piton} but only those defined by using
% |\NewPitonLanguage|. 
%    \begin{macrocode}
    \prop_get:NoNTF \g_@@_languages_prop \l_tmpa_tl \l_tmpb_tl 
%    \end{macrocode}
% We can now define the new language by using the previous function.
%    \begin{macrocode}
      { \@@_NewPitonLanguage:nnno { #1 } { #2 } { #5 } \l_tmpb_tl }
      { \@@_error:n { Language~not~defined } }
  }
%    \end{macrocode}
%
%  \bigskip
%    \begin{macrocode}
\cs_new_protected:Npn \@@_NewPitonLanguage:nnnn #1 #2 #3 #4
%    \end{macrocode}
% In the following line, we write |#4,#3| and not |#3,#4| because we want that the
% keys which correspond to base language appear before the keys which are added
% in the language we define.
%    \begin{macrocode}
  { \@@_NewPitonLanguage:nnn { #1 } { #2 } { #4 , #3 } }
\cs_generate_variant:Nn \@@_NewPitonLanguage:nnnn { n n n o }
%    \end{macrocode}
% 
% \bigskip
%    \begin{macrocode}
\NewDocumentCommand { \piton } { }
  { \peek_meaning:NTF \bgroup { \@@_piton_standard } { \@@_piton_verbatim } }
%    \end{macrocode}
%
%    \begin{macrocode}
\NewDocumentCommand { \@@_piton_standard } { m }
  {
    \group_begin:
    \tl_if_eq:NnF \l_@@_space_in_string_tl { ␣ }
      {
%    \end{macrocode}
% Remind that, when |break-strings-anywhere| is in force, multiple commands |\-|
% will be inserted between the characters of the string to allow the breaks. The
% |\exp_not:N| before |\space| is mandatory.
%    \begin{macrocode}
        \bool_lazy_or:nnT
          \l_@@_break_lines_in_piton_bool
          \l_@@_break_strings_anywhere_bool
          { \tl_set:Nn \l_@@_space_in_string_tl { \exp_not:N \space } }
      }
%    \end{macrocode}
%
% \bigskip
% The following tuning of LuaTeX in order to avoid all breaks of lines on the
% hyphens. 
%    \begin{macrocode}
    \automatichyphenmode = 1
%    \end{macrocode}
% Remark that the argument of |\piton| (with the normal syntax) is expanded in
% the TeX sens, (see the |\tl_set:Ne| below) and that's why we can provide the
% following escapes to the final user:
%    \begin{macrocode}
    \cs_set_eq:NN \\ \c_backslash_str
    \cs_set_eq:NN \% \c_percent_str  
    \cs_set_eq:NN \{ \c_left_brace_str
    \cs_set_eq:NN \} \c_right_brace_str
    \cs_set_eq:NN \$ \c_dollar_str
%    \end{macrocode}
% The standard command |\␣| is \emph{not} expandable and we need here expandable
% commands. With the following code, we define an expandable command.
%    \begin{macrocode}
    \cs_set_eq:cN { ~ } \space 
%    \end{macrocode}
%
%    \begin{macrocode}
    \cs_set_eq:NN \@@_begin_line: \prg_do_nothing:
    \tl_set:Ne \l_tmpa_tl 
      { 
        \lua_now:e 
          { piton.ParseBis('\l_piton_language_str',token.scan_string()) }
          { #1 } 
      }
    \bool_if:NTF \l_@@_show_spaces_bool
      { \tl_replace_all:NVn \l_tmpa_tl \c_catcode_other_space_tl { ␣ } } % U+2423
      {
        \bool_if:NT \l_@@_break_lines_in_piton_bool
%    \end{macrocode}
% With the following line, the spaces of catacode 12 (which were not breakable)
% are replaced by |\space|, and, thus, become breakable.
%    \begin{macrocode}
          { \tl_replace_all:NVn \l_tmpa_tl \c_catcode_other_space_tl \space }
      }
%    \end{macrocode}
% The command |\text| is provided by the package \pkg{amstext} (loaded by
% \pkg{piton}). 
%    \begin{macrocode}
    \if_mode_math:
       \text { \l_@@_font_command_tl \l_tmpa_tl }
    \else:
       \l_@@_font_command_tl \l_tmpa_tl
    \fi: 
    \group_end:
  }
%    \end{macrocode}
%
% 
% \bigskip
%    \begin{macrocode}
\NewDocumentCommand { \@@_piton_verbatim } { v }
  {
    \group_begin:
    \automatichyphenmode = 1
    \cs_set_eq:NN \@@_begin_line: \prg_do_nothing:
    \tl_set:Ne \l_tmpa_tl 
      { 
        \lua_now:e 
          { piton.Parse('\l_piton_language_str',token.scan_string()) } 
          { #1 } 
      }
    \bool_if:NT \l_@@_show_spaces_bool
      { \tl_replace_all:NVn \l_tmpa_tl \c_catcode_other_space_tl { ␣ } } % U+2423
    \if_mode_math:
       \text { \l_@@_font_command_tl \l_tmpa_tl }
    \else:
       \l_@@_font_command_tl \l_tmpa_tl
    \fi: 
    \group_end:
  }
%    \end{macrocode}
%

%  \bigskip
%
%
%  \bigskip 
%  The following command does \emph{not} correspond to a user command. It will
%  be used when we will have to ``rescan'' some chunks of informatic code. For
%  example, it will be the initial value of the Piton style |InitialValues| (the
%  default values of the arguments of a Python function).
%    \begin{macrocode}
\cs_new_protected:Npn \@@_piton:n #1
  { \tl_if_blank:nF { #1 } { \@@_piton_i:n { #1 } } }

\cs_new_protected:Npn \@@_piton_i:n #1
  { 
    \group_begin:
    \cs_set_eq:NN \@@_begin_line: \prg_do_nothing:
    \cs_set:cpn { pitonStyle _ \l_piton_language_str  _ Prompt } { }
    \cs_set:cpn { pitonStyle _ Prompt } { }
    \cs_set_eq:NN \@@_trailing_space: \space
    \tl_set:Ne \l_tmpa_tl 
      { 
        \lua_now:e 
          { piton.ParseTer('\l_piton_language_str',token.scan_string()) } 
          { #1 } 
      }
    \bool_if:NT \l_@@_show_spaces_bool
      { \tl_replace_all:NVn \l_tmpa_tl \c_catcode_other_space_tl { ␣ } } % U+2423
    \@@_replace_spaces:o \l_tmpa_tl 
    \group_end:
  }
%    \end{macrocode}
%
%
% \bigskip
% |\@@_pre_composition:| will be used both in |\PitonInputFile| and
% in the environments such as |{Piton}|.
%    \begin{macrocode}
\cs_new:Npn \@@_pre_composition:
  {
    \legacy_if:nT { @inlabel } { \bool_set_true:N \l_@@_in_label_bool }
    \automatichyphenmode = 1
    \int_gincr:N \g_@@_env_int
    \dim_compare:nNnT \l_@@_width_dim = \c_zero_dim
      {
        \dim_set_eq:NN \l_@@_width_dim \linewidth
        \str_if_empty:NF \l_@@_box_str
          { \bool_set_true:N \l_@@_minimize_width_bool }
      }
    \bool_if:NF \l_@@_resume_bool { \int_gzero:N \g_@@_visual_line_int }
    \g_@@_def_vertical_commands_tl
    \int_gzero:N \g_@@_line_int
    \int_gzero:N \g_@@_nb_lines_int
    \dim_zero:N \parindent 
    \dim_zero:N \lineskip 
    \cs_set_eq:NN \label \@@_label:n
    \dim_zero:N \parskip
    \l_@@_font_command_tl
  }
%    \end{macrocode}
% 
%
% \bigskip
% If the final user has used both |left-margin=auto| and |line-numbers|, we have
% to compute the width of the maximal number of lines at the end of the
% environment to fix the correct value to |left-margin|. The first argument of
% the following function is the name of the Lua function that will be applied to
% the second argument in order to count the number of lines.
%    \begin{macrocode}
\cs_new_protected:Npn \@@_compute_left_margin:nn #1 #2
  {
    \bool_lazy_and:nnT \l_@@_left_margin_auto_bool \l_@@_line_numbers_bool
       {
        \hbox_set:Nn \l_tmpa_box
          {
            \l_@@_line_numbers_format_tl 
            \bool_if:NTF \l_@@_skip_empty_lines_bool
              {
                \lua_now:n 
                  { piton.#1(token.scan_argument()) }
                  { #2 }
                \int_to_arabic:n 
                  { \g_@@_visual_line_int + \l_@@_nb_non_empty_lines_int }   
              }
              {
                \int_to_arabic:n 
                  { \g_@@_visual_line_int + \g_@@_nb_lines_int }  
              }
           }
         \dim_set:Nn \l_@@_left_margin_dim 
           { \box_wd:N \l_tmpa_box + \l_@@_numbers_sep_dim + 0.1 em }
       }
  }
\cs_generate_variant:Nn \@@_compute_left_margin:nn { n o }
%    \end{macrocode}
% 
%
% \bigskip
% The following command computes |\g_@@_width_dim| and it will be used
% when |max-width| or |width=min| is used.
%    \begin{macrocode}
\cs_new_protected:Npn \@@_compute_width:
  {
    \dim_gset:Nn \g_@@_width_dim { \box_wd:N \g_@@_output_box }
    \clist_if_empty:NTF \l_@@_bg_color_clist
      { \dim_gadd:Nn \g_@@_width_dim \l_@@_left_margin_dim }
      { 
        \dim_gadd:Nn \g_@@_width_dim { 0.5 em }
        \dim_compare:nNnTF \l_@@_left_margin_dim = \c_zero_dim 
          { \dim_gadd:Nn \g_@@_width_dim { 0.5 em } }
          { \dim_gadd:Nn \g_@@_width_dim \l_@@_left_margin_dim }
      }
  }          
%    \end{macrocode}
%
% \bigskip
% Whereas |\l_@@_with_dim| is the width of the environment as specified by the key
% |width| (except when |max-width| or |width=min| is used), |\l_@@_code_width_dim|
% is the width of the lines of code without the potential margins for the numbers
% of lines and the background.
%    \begin{macrocode}
\cs_new_protected:Npn \@@_compute_code_width:
  {
    \dim_set_eq:NN \l_@@_code_width_dim \l_@@_width_dim
    \clist_if_empty:NTF \l_@@_bg_color_clist
%    \end{macrocode}
% If there is no background, we only subtract the left margin. 
%    \begin{macrocode}
      { \dim_sub:Nn \l_@@_code_width_dim \l_@@_left_margin_dim }
%    \end{macrocode}
% If there is a background, we subtract 0.5~em for the margin on the right.
%    \begin{macrocode}
      { 
        \dim_sub:Nn \l_@@_code_width_dim { 0.5 em }
%    \end{macrocode}
% And we subtract also for the left margin. If the key |left-margin| has been
% used (with a numerical value or with the special value~|min|),
% |\l_@@_left_margin_dim| has a non-zero value\footnote{If the key
% |left-margin| has been used with the special value |min|, the
% actual value of \texttt{\textbackslash l_\@\@_left_margin_dim} has yet been
% computed when we use the current command.} and we use that value. Elsewhere,
% we use a value of 0.5~em.
%    \begin{macrocode}
        \dim_compare:nNnTF \l_@@_left_margin_dim = \c_zero_dim 
          { \dim_sub:Nn \l_@@_code_width_dim { 0.5 em } }
          { \dim_sub:Nn \l_@@_code_width_dim \l_@@_left_margin_dim }
      }
  }
%    \end{macrocode}
%
% 
% \bigskip
% For the following commands, the arguments are provided by curryfication.
%    \begin{macrocode}
\NewDocumentCommand { \NewPitonEnvironment } { }
  { \@@_DefinePitonEnvironment:nnnnn { New } }
%    \end{macrocode}
%
%    \begin{macrocode}
\NewDocumentCommand { \DeclarePitonEnvironment } { }
  { \@@_DefinePitonEnvironment:nnnnn { Declare } }
%    \end{macrocode}
%
%    \begin{macrocode}
\NewDocumentCommand { \RenewPitonEnvironment } { }
  { \@@_DefinePitonEnvironment:nnnnn { Renew } }
%    \end{macrocode}
%
%    \begin{macrocode}
\NewDocumentCommand { \ProvidePitonEnvironment } { }
  { \@@_DefinePitonEnvironment:nnnnn { Provide } }
%    \end{macrocode}
%
% \bigskip
% The first argument of the following macro is one of the four strings:
% |New|, |Renew|, |Provide| and |Declare|.
%    \begin{macrocode}
\cs_new_protected:Npn \@@_DefinePitonEnvironment:nnnnn #1 #2 #3 #4 #5
  { 
%    \end{macrocode}
% We construct a TeX macro which will catch as argument all the tokens until
% |\end{|\texttt{\textsl{name_env}}|}| with, in that
% |\end{|\texttt{\textsl{name_env}}|}|, the catcodes of |\|, |{| and |}| equal to
% 12 (``\texttt{other}''). The latter explains why the definition of that
% function is a bit complicated.
%    \begin{macrocode}
    \use:x 
      { 
        \cs_set_protected:Npn 
          \use:c { _@@_collect_ #2 :w } 
          ####1 
          \c_backslash_str end \c_left_brace_str #2 \c_right_brace_str
      }
          { 
            \group_end:
%    \end{macrocode}
% Maybe, we should deactivate all the ``shorthands'' of \pkg{babel} (when
% \pkg{babel} is loaded) with the following instruction:
%
% \verb|\IfPackageLoadedT { babel } { \languageshorthands { none } }|
%
% But we should be sure that there is no consequence in the LaTeX comments...
%
% \bigskip
%    \begin{macrocode}
            \tl_set:Nn \l_@@_listing_tl { ##1 }
            \@@_composition:
% \end{macrocode}
% 
% The following |\end{#2}| is only for the stack of environments of LaTeX.
%    \begin{macrocode}
            \end { #2 }
          }   
%    \end{macrocode}
% 
% 
% 
% \bigskip
% We can now define the new environment.
%
% We are still in the definition of the command |\NewPitonEnvironment|...
%    \begin{macrocode}
    \use:c { #1 DocumentEnvironment } { #2 } { #3 } 
      {
        \cs_set_eq:NN \PitonOptions \@@_fake_PitonOptions
        #4 
        \@@_pre_composition:
        \int_compare:nNnT { \l_@@_number_lines_start_int } > { \c_zero_int }
          { \int_gset:Nn \g_@@_visual_line_int { \l_@@_number_lines_start_int - 1 } }
        \group_begin:
        \tl_map_function:nN 
          { \ \\ \{ \} \$ \& \# \^ \_ \% \~ \^^I } 
          \char_set_catcode_other:N 
        \use:c { _@@_collect_ #2 :w }
      }
      { 
        #5 
        \ignorespacesafterend
      } 
%    \end{macrocode}
%
% \medskip
% The following code is for technical reasons. We want to change the catcode of
% |^^M| before catching the arguments of the new environment we are defining.
% Indeed, if not, we will have problems if there is a final optional argument in
% our environment (if that final argument is not used by the user in an
% instance of the environment, a spurious space is inserted, probably because
% the |^^M| is converted to space).
%    \begin{macrocode}
    \AddToHook { env / #2 / begin } { \char_set_catcode_other:N \^^M }
  }
%    \end{macrocode}
% This is the end of the definition of the command |\NewPitonEnvironment|.
%
% 
% \bigskip
%    \begin{macrocode}
\IfFormatAtLeastTF { 2025-06-01 }
  {
%    \end{macrocode}
% We will retreive the body of the environment in |\l_@@_listing_tl|.
%    \begin{macrocode}
    \cs_new_protected:Npn \@@_store_body:n #1 
      {
%    \end{macrocode}
% Now, we have to replace all the occurrences of |\obeyedline| by a character of
% end of line (|\r| in the strings of Lua).
%    \begin{macrocode}
        \tl_set:Ne \obeyedline { \char_generate:nn { 13 } { 11 } }
        \tl_set:Ne \l_@@_listing_tl { #1 }
        \tl_set_eq:NN \ProcessedArgument \l_@@_listing_tl
      }
%    \end{macrocode}      
% The first argument of the following macro is one of the four strings:
% |New|, |Renew|, |Provide| and |Declare|.
%    \begin{macrocode}
    \cs_set_protected:Npn \@@_DefinePitonEnvironment:nnnnn #1 #2 #3 #4 #5
      {
        \use:c { #1 DocumentEnvironment } { #2 } { #3 > { \@@_store_body:n } c } 
          {
            \cs_set_eq:NN \PitonOptions \@@_fake_PitonOptions
            #4
            \@@_pre_composition:
            \int_compare:nNnT { \l_@@_number_lines_start_int } > { \c_zero_int }
              { 
                \int_gset:Nn \g_@@_visual_line_int 
                  { \l_@@_number_lines_start_int - 1 } 
              }
%    \end{macrocode}
% Now, the main job.
%    \begin{macrocode}
            \bool_if:NT \g_@@_footnote_bool \savenotes % added
            \@@_composition:
            \bool_if:NT \g_@@_footnote_bool \endsavenotes % added 
            #5
          }
          { \ignorespacesafterend }  
      }    
  }
  { } 
%    \end{macrocode}
%
% 
%    \begin{macrocode}
\cs_new_protected:Npn \@@_composition:
  {
    \str_if_empty:NT \l_@@_box_str
      {
        \mode_if_vertical:F
          { \bool_if:NF \l_@@_in_PitonInputFile_bool { \newline } }
      }
%    \end{macrocode}
% The following line is only to compute |\l_@@_lines_int| which will be used
% only when both |left-margin=auto| and |skip-empty-lines = false| are in force.
% We should change that.
%    \begin{macrocode}
    \lua_now:e { piton.CountLines ( '\lua_escape:n{\l_@@_listing_tl}' ) }
%    \end{macrocode}
% The first argument of the following function is the name of the Lua function
% that will be applied to the second argument in order to count the number of lines.
%    \begin{macrocode}
    \@@_compute_left_margin:no { CountNonEmptyLines } { \l_@@_listing_tl }
    \lua_now:e
      {
        piton.join = "\l_@@_join_str"
        piton.write = "\l_@@_write_str"
        piton.path_write = "\l_@@_path_write_str"
      }
    \noindent
    \bool_if:NTF \l_@@_print_bool
      {
%    \end{macrocode}
% When |split-on-empty-lines| is in force, each chunk will be formated by an
% environment |{Piton}| (or the environment specified by |env-used-by-split|).
% Within each of these environments, we will come back here (but, of course,
% |split-on-empty-line| will have been set to |false|).
%    \begin{macrocode}
        \bool_if:NTF \l_@@_split_on_empty_lines_bool
          { \par \@@_retrieve_gobble_split_parse:o \l_@@_listing_tl }
          {
            \@@_create_output_box:
%    \end{macrocode}
% Now, the listing has been composed in |\g_@@_output_box| and |\g_@@_width_dim|
% contains the width of the listing (with the potential margin for the numbers of lines).
%    \begin{macrocode}
            \bool_if:NTF \l_@@_tcolorbox_bool
              {
                \str_if_empty:NTF \l_@@_box_str
                  { \@@_composition_iii: }
                  { \@@_composition_iv: }
              }
              {
                \str_if_empty:NTF \l_@@_box_str
                  { \@@_composition_i: }
                  { \@@_composition_ii: }
              }
         }
      }
      { \@@_gobble_parse_no_print:o \l_@@_listing_tl }
  }
%    \end{macrocode}
%
% \bigskip
% |\@@_composition_i:| is for the main case: the key |tcolorbox| is not used, nor
% the key |box|.
%
% We can't do a mere |\vbox_unpack:N \g_@@_output_box| because that would not work
% inside a list of LaTeX (|{itemize}| or |{enumerate}|).
%
% The composition in the box |\g_@@_output_box| was mandatory to be able to deal
% with the case of a conjunction of the keys |width=min| and |background-color=...|.
%    \begin{macrocode}
\cs_new_protected:Npn \@@_composition_i:
  { 
%    \end{macrocode}
% First, we ``reverse'' the box |\g_@@_output_box|: we put in the box
% |\g_tmpa_box| the boxes present in |\g_@@_output_box|, but in reversed order.
% The vertical spaces and the penalties are discarded.
%    \begin{macrocode}
    \box_clear:N \g_tmpa_box
%    \end{macrocode}
% The box |\g_@@_line_box| will be used as an auxiliary box.
%    \begin{macrocode}
    \box_clear_new:N \g_@@_line_box
%    \end{macrocode}
% We unpack |\g_@@_output_box| in |\l_tmpa_box| used as a scratched box.
%    \begin{macrocode}
    \vbox_set:Nn \l_tmpa_box 
      {
        \vbox_unpack_drop:N \g_@@_output_box
        \bool_gset_false:N \g_tmpa_bool
        \unskip \unskip
        \bool_gset_false:N \g_tmpa_bool
        \bool_do_until:nn \g_tmpa_bool
          {
            \unskip \unskip \unskip
            \unpenalty \unkern
            \box_set_to_last:N \l_@@_line_box
            \box_if_empty:NTF \l_@@_line_box
              { \bool_gset_true:N \g_tmpa_bool }
              {
                \vbox_gset:Nn \g_tmpa_box
                  {
                    \vbox_unpack:N \g_tmpa_box
                    \box_use:N \l_@@_line_box
                  }
              }
          }
      }
%    \end{macrocode}
% Now, we will loop over the boxes in |\g_tmpa_box| and compose the
% boxes in the TeX flow.
%    \begin{macrocode}
    \bool_gset_false:N \g_tmpa_bool
    \int_zero:N \g_@@_line_int
    \bool_do_until:nn \g_tmpa_bool
      {
%    \end{macrocode}
% We retrieve the last box of |\g_tmpa_box| (and store it in |\g_@@_line_box|)
% and keep the other boxes in |\g_tmpa_box|.
%    \begin{macrocode}
        \vbox_gset:Nn \g_tmpa_box
          {
            \vbox_unpack_drop:N \g_tmpa_box
            \box_gset_to_last:N \g_@@_line_box
          }
%    \end{macrocode}
% If the box that we have retrieved is void, that means that, in fact, there is no
% longer boxes in |\g_tmpa_bool| and we will exit the loop.
%    \begin{macrocode}
        \box_if_empty:NTF \g_@@_line_box
          { \bool_gset_true:N \g_tmpa_bool }
          {
            \box_use:N \g_@@_line_box
            \int_gincr:N \g_@@_line_int
            \par
            \kern -2.5 pt
%    \end{macrocode}
% We will determine the penalty by reading the Lua table |piton.lines_status|. That
% will use the current value of |\g_@@_line_int|.
%    \begin{macrocode}
            \@@_add_penalty_for_the_line:
%    \end{macrocode}
% We now add the instructions corresponding to the \emph{vertical detected commands}
% that are potentially used in the corresponding line of the listing.
%    \begin{macrocode}
            \cs_if_exist_use:cT { g_@@_after_line _ \int_use:N \g_@@_line_int _ tl }
              { \cs_undefine:c { g_@@_after_line _ \int_use:N \g_@@_line_int _ tl } }
            \mode_leave_vertical: 
          }
      }
  }      
%    \end{macrocode}
%
% \bigskip
% |\@@_composition_ii:| will be used when the key |box| is in force.
%    \begin{macrocode}
\cs_new_protected:Npn \@@_composition_ii:
  {
%    \end{macrocode}
% It will be possible to delete the |\exp_not:N| in TeXLive 2025
% because |\begin| is now protected by |\protected| (and not by |\protect|).
%    \begin{macrocode}
    \use:e { \exp_not:N \begin { minipage } [ \l_@@_box_str ] }
      { \g_@@_width_dim }
    \vbox_unpack:N \g_@@_output_box
    \end { minipage }
  }
%    \end{macrocode}  
%
% \bigskip
% |\@@_composition_iii:| will be used when the key |tcolorbox| is in force.
%    \begin{macrocode}
\cs_new_protected:Npn \@@_composition_iii:
  {
    \bool_if:NT \l_@@_minimize_width_bool 
      { \tcbset { text~width = \g_@@_width_dim } }
%    \end{macrocode}    
% Even though we use the key |breakable| of |{tcolorbox}|, our environment will be
% breakable only when the key |splittable| of \pkg{piton} is used.
%    \begin{macrocode}
    \begin { tcolorbox } [ breakable ]
    \par
    \vbox_unpack:N \g_@@_output_box
    \end { tcolorbox }
  }
%    \end{macrocode}
%
% \bigskip
% |\@@_composition_iv:| will be used when both keys |tcolorbox| and |box| are in force.
%    \begin{macrocode}
\cs_new_protected:Npn \@@_composition_iv:
  {
    \bool_if:NT \l_@@_minimize_width_bool 
      { \tcbset { text~width = \g_@@_width_dim } }
    \use:e
      {
        \begin { tcolorbox }
          [
            hbox ,
            nobeforeafter ,
            box~align =
              \str_case:Nn \l_@@_box_str 
                { 
                  t { top }
                  b { bottom }
                  c { center }
                  m { center }
                }
          ]
      }                                              
    \box_use:N \g_@@_output_box
    \end { tcolorbox }
  }                    
%    \end{macrocode}
% 
%
% \bigskip
% The following function will add the correct vertical penalty after
% a line of code in order to control the breaks of the pages.  We use the Lua table
% |piton.lines_status| which has been written by |piton.ComputeLinesStatus| for
% this aim. Each line has a ``status`` (equal to 0, 1 or 2) and that status
% directly says whether a break is allowed.
%    \begin{macrocode}
\cs_new_protected:Npn \@@_add_penalty_for_the_line:
  {
    \int_case:nn
      { 
        \lua_now:e 
          { 
            tex.sprint
              ( 
                luatexbase.catcodetables.expl ,
                tostring ( piton.lines_status [ \int_use:N \g_@@_line_int ] ) 
              ) 
          } 
      }
      { 1 { \penalty 100 } 2 \nobreak }
  }
%    \end{macrocode}
%
% \bigskip
%    \begin{macrocode}
\cs_new_protected:Npn \@@_create_output_box:
  {
    \@@_compute_code_width:
    \dim_gset_eq:NN \g_@@_width_dim \l_@@_width_dim
    \vbox_gset:Nn \g_@@_output_box
      { \@@_retrieve_gobble_parse:o \l_@@_listing_tl }
    \bool_if:NT \l_@@_minimize_width_bool { \@@_compute_width: }
    \clist_if_empty:NF \l_@@_bg_color_clist { \@@_add_backgrounds_to_output_box: }
  }
%    \end{macrocode}
%
%
% \bigskip
% We add the backgrounds after the composition of the box |\g_@@_output_box| by a
% loop over the lines in that box.
%    \begin{macrocode}
\cs_new_protected:Npn \@@_add_backgrounds_to_output_box:
  {
    \int_gset_eq:NN \g_@@_line_int \g_@@_nb_lines_int
%    \end{macrocode}
% |\l_tmpa_box| is only used to \emph{unpack} the vertical box |\g_@@_output_box|.
%    \begin{macrocode}
    \vbox_set:Nn \l_tmpa_box
      {
        \vbox_unpack_drop:N \g_@@_output_box
%    \end{macrocode}        
% We will raise |\g_tmpa_bool| to exit the loop |\bool_do_until:nn| below.
%    \begin{macrocode} 
        \bool_gset_false:N \g_tmpa_bool
        \unskip \unskip
%    \end{macrocode}
% We begin the loop.
%    \begin{macrocode}
        \bool_do_until:nn \g_tmpa_bool
          {
            \unskip \unskip \unskip
            \int_set_eq:NN \l_tmpa_int \lastpenalty
            \unpenalty \unkern
%    \end{macrocode}
% In standard TeX (not LuaTeX), the only way to loop over the sub-boxes of a given
% box is to use the TeX primitive |\lastbox| (via |\box_set_to_last:N| of L3).
% Of course, it would be interesting to replace that programmation by a programmation
% in Lua of LuaTeX...
%    \begin{macrocode}
            \box_set_to_last:N \l_@@_line_box
            \box_if_empty:NTF \l_@@_line_box
              { \bool_gset_true:N \g_tmpa_bool }
              {
%    \end{macrocode}
% |\g_@@_line_int| will be used in |\@@_add_background_to_line_and_use:|.
%    \begin{macrocode}
                \vbox_gset:Nn \g_@@_output_box
                  {
%    \end{macrocode}
% The command |\@@_add_background_to_line_and_use:| will add a background to
% the line (in |\l_@@_line_box|) but will also put the line in the current box.
%    \begin{macrocode}
                    \@@_add_background_to_line_and_use:
                    \kern -2.5 pt 
                    \penalty \l_tmpa_int
                    \vbox_unpack:N \g_@@_output_box
                  }
              }
            \int_gdecr:N \g_@@_line_int
          }
      }
  }      
%    \end{macrocode}
% 
% \bigskip
% The following will be used when the final user has user |print=false|.
%    \begin{macrocode}
\cs_new_protected:Npn \@@_gobble_parse_no_print:n
  {
    \lua_now:e
      {
        piton.GobbleParseNoPrint
         (
           '\l_piton_language_str' ,
           \int_use:N \l_@@_gobble_int ,
           token.scan_argument ( )
         )
      }
  }
\cs_generate_variant:Nn \@@_gobble_parse_no_print:n { o }
%    \end{macrocode}
% 
% \bigskip
% The following function will be used when the key |split-on-empty-lines| is not
% in force. It will retrieve the first empty line, gobble the spaces at the
% beginning of the lines and parse the code. The argument is provided by
% curryfication. 
%    \begin{macrocode}
\cs_new_protected:Npn \@@_retrieve_gobble_parse:n 
  {
    \lua_now:e 
      { 
        piton.RetrieveGobbleParse
          ( 
            '\l_piton_language_str' , 
            \int_use:N \l_@@_gobble_int ,
            \bool_if:NTF \l_@@_splittable_on_empty_lines_bool
              { \int_eval:n { - \l_@@_splittable_int } } 
              { \int_use:N \l_@@_splittable_int } ,
            token.scan_argument ( )
          ) 
      } 
  }
\cs_generate_variant:Nn \@@_retrieve_gobble_parse:n { o }
%    \end{macrocode}
%
% \bigskip
% The following function will be used when the key |split-on-empty-lines| is in
% force. It will gobble the spaces at the beginning of the lines (if the key
% |gobble| is in force), then split the code at the empty lines and, eventually,
% parse the code. The argument is provided by curryfication.
%    \begin{macrocode}
\cs_new_protected:Npn \@@_retrieve_gobble_split_parse:n 
  {
    \lua_now:e 
      { 
        piton.RetrieveGobbleSplitParse
          ( 
            '\l_piton_language_str' , 
            \int_use:N \l_@@_gobble_int ,
            \int_use:N \l_@@_splittable_int , 
            token.scan_argument ( )
          ) 
      } 
  }
\cs_generate_variant:Nn \@@_retrieve_gobble_split_parse:n { o }
%    \end{macrocode}
% 
% \bigskip
% Now, we define the environment |{Piton}|, which is the main environment
% provided by the package \pkg{piton}. Of course, you use
% |\NewPitonEnvironment|. 
%    \begin{macrocode} 
\bool_if:NTF \g_@@_beamer_bool
  {
    \NewPitonEnvironment { Piton } { d < > O { } }
      { 
        \keys_set:nn { PitonOptions } { #2 } 
        \tl_if_novalue:nTF { #1 }
          { \begin { uncoverenv } }
          { \begin { uncoverenv } < #1 > }
      }
      { \end { uncoverenv } } 
  }
  { 
    \NewPitonEnvironment { Piton } { O { } } 
      { \keys_set:nn { PitonOptions } { #1 } } 
      { } 
  }
%    \end{macrocode}
%
%
% \bigskip
% The code of the command |\PitonInputFile| is somewhat similar to the code of
% the environment |{Piton}|.
%    \begin{macrocode}
\NewDocumentCommand { \PitonInputFileTF } { d < > O { } m m m }
  {
    \group_begin:
    \seq_concat:NNN 
      \l_file_search_path_seq 
      \l_@@_path_seq
      \l_file_search_path_seq  
    \file_get_full_name:nNTF { #3 } \l_@@_file_name_str
      {
        \@@_input_file:nn { #1 } { #2 }
        #4
      }
      { #5 }
    \group_end:
  }
%    \end{macrocode}
%
% \medskip
%    \begin{macrocode}
\cs_new_protected:Npn \@@_unknown_file:n #1 
  { \msg_error:nnn { piton } { Unknown~file } { #1 } }
%    \end{macrocode}
% 
%    \begin{macrocode}
\NewDocumentCommand { \PitonInputFile } { d < > O { } m }
  {
    \PitonInputFileTF < #1 > [ #2 ] { #3 } { }
      {
%    \end{macrocode}
% The following line is for |latexmk| (suggestion of Y. Salmon).
%    \begin{macrocode}
        \iow_log:n { No~file~#3 }
        \@@_unknown_file:n { #3 }
      }
  }
\NewDocumentCommand { \PitonInputFileT } { d < > O { } m m }
  {
    \PitonInputFileTF < #1 >  [ #2 ] { #3 } { #4 }
      {
%    \end{macrocode}
% The following line is for |latexmk| (suggestion of Y. Salmon).
%    \begin{macrocode}
        \iow_log:n { No~file~#3 }
        \@@_unknown_file:n { #3 }
      }
  }
\NewDocumentCommand { \PitonInputFileF } { d < > O { } m m }
  { \PitonInputFileTF < #1 >  [ #2 ] { #3 } { } { #4 } }
%    \end{macrocode}
% 
% The following command uses as implicit argument the name of the file in
% |\l_@@_file_name_str|. 
%    \begin{macrocode}
\cs_new_protected:Npn \@@_input_file:nn #1 #2 
  {
%    \end{macrocode}
% We recall that, if we are in Beamer, the command |\PitonInputFile| is
% ``overlay-aware'' and that's why there is an optional argument between angular
% brackets (|<| and |>|).
%    \begin{macrocode}
    \tl_if_novalue:nF { #1 }
      {
        \bool_if:NTF \g_@@_beamer_bool
          { \begin { uncoverenv } < #1 > }     
          { \@@_error_or_warning:n { overlay~without~beamer } }
      }
    \group_begin:
% The following line is to allow tools such as |latexmk| to be aware that the
% file (read by |\PitonInputFile|) is loaded during the compilation of the LaTeX
% document. 
%    \begin{macrocode}
      \iow_log:e { (\l_@@_file_name_str) }
      \int_zero_new:N \l_@@_first_line_int
      \int_zero_new:N \l_@@_last_line_int 
      \int_set_eq:NN \l_@@_last_line_int \c_max_int
      \bool_set_true:N \l_@@_in_PitonInputFile_bool
      \keys_set:nn { PitonOptions } { #2 }
      \bool_if:NT \l_@@_line_numbers_absolute_bool
        { \bool_set_false:N \l_@@_skip_empty_lines_bool }
      \bool_if:nTF
        {
          ( 
            \int_compare_p:nNn \l_@@_first_line_int > \c_zero_int
            || \int_compare_p:nNn \l_@@_last_line_int < \c_max_int 
          )
          && ! \str_if_empty_p:N \l_@@_begin_range_str
        }
        { 
          \@@_error_or_warning:n { bad~range~specification } 
          \int_zero:N \l_@@_first_line_int
          \int_set_eq:NN \l_@@_last_line_int \c_max_int
        }
        { 
          \str_if_empty:NF \l_@@_begin_range_str 
            { 
              \@@_compute_range: 
              \bool_lazy_or:nnT 
                \l_@@_marker_include_lines_bool
                { ! \str_if_eq_p:NN \l_@@_begin_range_str \l_@@_end_range_str } 
                {
                  \int_decr:N \l_@@_first_line_int
                  \int_incr:N \l_@@_last_line_int
                }
            }
        }
      \@@_pre_composition:
      \bool_if:NT \l_@@_line_numbers_absolute_bool
        { \int_gset:Nn \g_@@_visual_line_int { \l_@@_first_line_int - 1 } } 
      \int_compare:nNnT \l_@@_number_lines_start_int > \c_zero_int 
        { 
          \int_gset:Nn \g_@@_visual_line_int 
            { \l_@@_number_lines_start_int - 1 } 
        }
%    \end{macrocode}
% The following case arises when the code |line-numbers/absolute| is in force
% without the use of a marked range.
%    \begin{macrocode}
      \int_compare:nNnT \g_@@_visual_line_int < \c_zero_int 
        { \int_gzero:N \g_@@_visual_line_int }
%    \end{macrocode}
% We count with Lua the number of lines of the argument. The result will be
% stored by Lua in |\g_@@_nb_lines_int|. 
%    \begin{macrocode}
      \lua_now:e { piton.CountLinesFile ( '\l_@@_file_name_str' ) } 
%    \end{macrocode}
% The first argument of the following function is the name of the Lua function
% that will be applied to the second argument in order to count the number of lines.
%    \begin{macrocode}
      \@@_compute_left_margin:no 
        { CountNonEmptyLinesFile } 
        { \l_@@_file_name_str }
      \lua_now:e
        {
%    \end{macrocode}
% The following command will store the content of the file (or only a part of that file)
% in |\l_@@_listing_tl|.
%    \begin{macrocode}
          piton.ReadFile(
           '\l_@@_file_name_str' ,
           \int_use:N \l_@@_first_line_int , 
           \int_use:N \l_@@_last_line_int )
        }
      \@@_composition:
    \group_end:
%    \end{macrocode}
%
% 
% We recall that, if we are in Beamer, the command |\PitonInputFile| is
% ``overlay-aware'' and that's why we close now an environment |{uncoverenv}|
% that we have opened at the beginning of the command.
%    \begin{macrocode}
    \tl_if_novalue:nF { #1 }
      { \bool_if:NT \g_@@_beamer_bool { \end { uncoverenv } } }
  }
%    \end{macrocode}
% 
% \bigskip
% The following command computes the values of |\l_@@_first_line_int| and
% |\l_@@_last_line_int| when |\PitonInputFile| is used with textual markers. 
%    \begin{macrocode}
\cs_new_protected:Npn \@@_compute_range: 
  {
%    \end{macrocode}
% We store the markers in L3 strings (|str|) in order to do safely the following
% replacement of |\#|.
%    \begin{macrocode}
    \str_set:Ne \l_tmpa_str { \@@_marker_beginning:n { \l_@@_begin_range_str } }
    \str_set:Ne \l_tmpb_str { \@@_marker_end:n { \l_@@_end_range_str } }
%    \end{macrocode}
% We replace the sequences |\#| which may be present in the prefixes and
% suffixes added to the markers by the functions |\@@_marker_beginning:n| and
% |\@@_marker_end:n|. 
%    \begin{macrocode}
    \tl_replace_all:Nee \l_tmpa_str { \c_backslash_str \c_hash_str } \c_hash_str
    \tl_replace_all:Nee \l_tmpb_str { \c_backslash_str \c_hash_str } \c_hash_str
%    \end{macrocode}
% 
%    \begin{macrocode}
    \lua_now:e
      { 
        piton.ComputeRange
          ( '\l_tmpa_str' , '\l_tmpb_str' , '\l_@@_file_name_str' ) 
      } 
  }
%    \end{macrocode}
% 
% \bigskip
% \subsubsection{The styles}
% 
% \medskip
% The following command is fundamental: it will be used by the Lua code.
%    \begin{macrocode}
\NewDocumentCommand { \PitonStyle } { m } 
  { 
    \cs_if_exist_use:cF { pitonStyle _ \l_piton_language_str  _ #1 }
      { \use:c { pitonStyle _ #1 } }
  }
%    \end{macrocode}
%
% \medskip
%    \begin{macrocode}
\NewDocumentCommand { \SetPitonStyle } { O { } m } 
  { 
    \str_clear_new:N \l_@@_SetPitonStyle_option_str
    \str_set:Ne \l_@@_SetPitonStyle_option_str { \str_lowercase:n { #1 } }
    \str_if_eq:onT { \l_@@_SetPitonStyle_option_str } { current-language }
      { \str_set_eq:NN \l_@@_SetPitonStyle_option_str \l_piton_language_str }
    \keys_set:nn { piton / Styles } { #2 } 
  } 
%    \end{macrocode}
% 
% \medskip
%    \begin{macrocode}
\cs_new_protected:Npn \@@_math_scantokens:n #1 
  { \normalfont \scantextokens { \begin{math} #1 \end{math} } }
%    \end{macrocode}
% 
% \medskip
%    \begin{macrocode}
\clist_new:N \g_@@_styles_clist
\clist_gset:Nn \g_@@_styles_clist 
  {
    Comment ,
    Comment.Internal , 
    Comment.LaTeX ,
    Discard ,
    Exception ,
    FormattingType ,    
    Identifier.Internal ,
    Identifier ,
    InitialValues ,
    Interpol.Inside ,
    Keyword ,
    Keyword.Governing ,
    Keyword.Constant ,
    Keyword2 ,
    Keyword3 ,
    Keyword4 ,
    Keyword5 ,
    Keyword6 ,
    Keyword7 ,
    Keyword8 ,
    Keyword9 ,
    Name.Builtin ,      
    Name.Class ,
    Name.Constructor ,
    Name.Decorator ,
    Name.Field ,
    Name.Function ,
    Name.Module ,
    Name.Namespace ,
    Name.Table , 
    Name.Type ,
    Number ,
    Number.Internal ,
    Operator ,
    Operator.Word ,
    Preproc ,
    Prompt ,
    String.Doc ,
    String.Doc.Internal ,
    String.Interpol ,
    String.Long ,
    String.Long.Internal ,
    String.Short ,    
    String.Short.Internal ,    
    Tag ,  
    TypeParameter ,    
    UserFunction ,
%    \end{macrocode}
% |TypeExpression| is an internal style for expressions which defines types in OCaml.
%    \begin{macrocode}
    TypeExpression ,
%    \end{macrocode}
% Now, specific styles for the languages created with |\NewPitonLanguage| with
% the syntax of \pkg{listings}.
%    \begin{macrocode}
    Directive 
  }  

\clist_map_inline:Nn \g_@@_styles_clist  
  {
    \keys_define:nn { piton / Styles }
      {
        #1 .value_required:n = true ,
        #1 .code:n = 
          \tl_set:cn 
            {
              pitonStyle _
              \str_if_empty:NF \l_@@_SetPitonStyle_option_str 
                { \l_@@_SetPitonStyle_option_str _ }
              #1
            }
            { ##1 }
      }
  }

\keys_define:nn { piton / Styles }
  {
    String       .meta:n = { String.Long = #1 , String.Short = #1 } ,
    Comment.Math .tl_set:c = pitonStyle _ Comment.Math  ,
    unknown      .code:n = 
      \@@_error:n { Unknown~key~for~SetPitonStyle }
  }
%    \end{macrocode}
%
% \bigskip
%    \begin{macrocode}
\SetPitonStyle[OCaml]
  {
    TypeExpression = 
      { 
        \SetPitonStyle [ OCaml ] { Identifier = \PitonStyle { Name.Type } } 
        \@@_piton:n  
      }
  }
%    \end{macrocode}
% 
% \bigskip
% We add the word |String| to the list of the styles because we will use that
% list in the error message for an unknown key in |\SetPitonStyle|.
%
%    \begin{macrocode}
\clist_gput_left:Nn \g_@@_styles_clist { String }
%    \end{macrocode}
%
% \bigskip
% Of course, we sort that clist.
%    \begin{macrocode}
\clist_gsort:Nn \g_@@_styles_clist
  { 
    \str_compare:nNnTF { #1 } < { #2 } 
      \sort_return_same:    
      \sort_return_swapped:  
  } 
%    \end{macrocode}
% 
% \bigskip
%    \begin{macrocode}
\cs_set_eq:NN \@@_break_strings_anywhere:n \prg_do_nothing:

\cs_set_eq:NN \@@_break_numbers_anywhere:n \prg_do_nothing:

\cs_new_protected:Npn \@@_actually_break_anywhere:n #1
  {
    \tl_set:Nn \l_tmpa_tl { #1 }
%    \end{macrocode}
% We have to begin by a substitution for the spaces. Otherwise, they would be
% gobbled in the |\tl_map_inline:Nn|.
%    \begin{macrocode}
    \tl_replace_all:NVn \l_tmpa_tl \c_catcode_other_space_tl \space 
    \seq_clear:N \l_tmpa_seq % added 2025/03/03
    \tl_map_inline:Nn \l_tmpa_tl { \seq_put_right:Nn \l_tmpa_seq { ##1 } } 
    \seq_use:Nn \l_tmpa_seq { \- }
  }
%    \end{macrocode}
%
% \bigskip
%    \begin{macrocode}
\cs_new_protected:Npn \@@_comment:n #1
  {
    \PitonStyle { Comment }
      {
        \bool_if:NTF \l_@@_break_lines_in_Piton_bool
          {
            \tl_set:Nn \l_tmpa_tl { #1 }
            \tl_replace_all:NVn \l_tmpa_tl 
              \c_catcode_other_space_tl 
              \@@_breakable_space:
            \l_tmpa_tl
          }
          { #1 }
      }
  }
%    \end{macrocode}
%
% \bigskip
%    \begin{macrocode}
\cs_new_protected:Npn \@@_string_long:n #1
  {
    \PitonStyle { String.Long } 
      { 
        \bool_if:NTF \l_@@_break_strings_anywhere_bool
          { \@@_actually_break_anywhere:n { #1 } }
          { 
%    \end{macrocode}
% We have, when |break-lines-in-Piton| is in force, to replace the spaces by
% |\@@_breakable_space:| because, when we have done a similar job in
% |\@@_replace_spaces:n| used in |\@@_begin_line:|, that job was not able to do
% the replacement in the brace group |{...}| of |\PitonStyle{String.Long}{...}|
% because we used a |\tl_replace_all:NVn|. At that time, it would have been
% possible to use a |\tl_regex_replace_all:Nnn| but it is notoriously slow.
%    \begin{macrocode}
            \bool_if:NTF \l_@@_break_lines_in_Piton_bool
              {
                \tl_set:Nn \l_tmpa_tl { #1 }
                \tl_replace_all:NVn \l_tmpa_tl 
                  \c_catcode_other_space_tl 
                  \@@_breakable_space:
                \l_tmpa_tl
              }
              { #1 }
          } 
      }
  }
\cs_new_protected:Npn \@@_string_short:n #1
  {
    \PitonStyle { String.Short } 
      { 
        \bool_if:NT \l_@@_break_strings_anywhere_bool
          { \@@_actually_break_anywhere:n }
        { #1 } 
      }
  }
\cs_new_protected:Npn \@@_string_doc:n #1
  {
    \PitonStyle { String.Doc } 
      { 
        \bool_if:NTF \l_@@_break_lines_in_Piton_bool
          {
            \tl_set:Nn \l_tmpa_tl { #1 }
            \tl_replace_all:NVn \l_tmpa_tl 
              \c_catcode_other_space_tl 
              \@@_breakable_space:
            \l_tmpa_tl
          }
          { #1 }
      }
  }
\cs_new_protected:Npn \@@_number:n #1 
  {
    \PitonStyle { Number } 
      { 
        \bool_if:NT \l_@@_break_numbers_anywhere_bool
          { \@@_actually_break_anywhere:n }
        { #1 } 
      }
  }
%    \end{macrocode}
% 
% \bigskip
% \subsubsection{The initial styles}
%
% The initial styles are inspired by the style ``manni'' of Pygments.
% 
% \medskip
%    \begin{macrocode}
\SetPitonStyle
  {                                                       
    Comment               = \color [ HTML ] { 0099FF } \itshape , 
    Comment.Internal      = \@@_comment:n , 
    Exception             = \color [ HTML ] { CC0000 } ,
    Keyword               = \color [ HTML ] { 006699 } \bfseries , 
    Keyword.Governing     = \color [ HTML ] { 006699 } \bfseries , 
    Keyword.Constant      = \color [ HTML ] { 006699 } \bfseries ,               
    Name.Builtin          = \color [ HTML ] { 336666 } ,
    Name.Decorator        = \color [ HTML ] { 9999FF }, 
    Name.Class            = \color [ HTML ] { 00AA88 } \bfseries ,
    Name.Function         = \color [ HTML ] { CC00FF } , 
    Name.Namespace        = \color [ HTML ] { 00CCFF } , 
    Name.Constructor      = \color [ HTML ] { 006000 } \bfseries , 
    Name.Field            = \color [ HTML ] { AA6600 } , 
    Name.Module           = \color [ HTML ] { 0060A0 } \bfseries ,
    Name.Table            = \color [ HTML ] { 309030 } ,
    Number                = \color [ HTML ] { FF6600 } ,
    Number.Internal       = \@@_number:n ,
    Operator              = \color [ HTML ] { 555555 } ,
    Operator.Word         = \bfseries ,
    String                = \color [ HTML ] { CC3300 } ,
    String.Long.Internal  = \@@_string_long:n , 
    String.Short.Internal = \@@_string_short:n , 
    String.Doc.Internal   = \@@_string_doc:n , 
    String.Doc            = \color [ HTML ] { CC3300 } \itshape , 
    String.Interpol       = \color [ HTML ] { AA0000 } ,
    Comment.LaTeX         = \normalfont \color [ rgb ] { .468, .532, .6 } , 
    Name.Type             = \color [ HTML ] { 336666 } ,
    InitialValues         = \@@_piton:n ,
    Interpol.Inside       = { \l_@@_font_command_tl \@@_piton:n } ,
    TypeParameter         = \color [ HTML ] { 336666} \itshape ,
    Preproc               = \color [ HTML ] { AA6600} \slshape ,
%    \end{macrocode}
% We need the command |\@@_identifier:n| because of the command
% |\SetPitonIdentifier|. The command |\@@_identifier:n| will potentially call
% the style |Identifier| (which is a user-style, not an internal style).
%    \begin{macrocode}
    Identifier.Internal   = \@@_identifier:n , 
    Identifier            = ,
    Directive             = \color [ HTML ] { AA6600} ,
    Tag                   = \colorbox { gray!10 } , 
    UserFunction          = \PitonStyle { Identifier } ,
    Prompt                = , 
    Discard               = \use_none:n 
  }
%    \end{macrocode}
%
% \medskip 
% If the key |math-comments| has been used in the preamble of the LaTeX
% document, we change the style |Comment.Math| which should be considered only
% at an ``internal style''. However, maybe we will document in a future version
% the possibility to write change the style \emph{locally} in a document)].
%    \begin{macrocode}
\hook_gput_code:nnn { begindocument } { . } 
  { 
    \bool_if:NT \g_@@_math_comments_bool 
      { \SetPitonStyle { Comment.Math = \@@_math_scantokens:n } } 
  }
%    \end{macrocode}
% 
% \bigskip
%
% \bigskip
% \subsubsection{Highlighting some identifiers}
%
%
% \medskip
%    \begin{macrocode}
\NewDocumentCommand { \SetPitonIdentifier } { o m m } 
  { 
    \clist_set:Nn \l_tmpa_clist { #2 }
    \tl_if_novalue:nTF { #1 }
      { 
        \clist_map_inline:Nn \l_tmpa_clist 
          { \cs_set:cpn { PitonIdentifier _ ##1 } { #3 } }
      }
      {
        \str_set:Ne \l_tmpa_str { \str_lowercase:n { #1 } }
        \str_if_eq:onT \l_tmpa_str { current-language }
          { \str_set_eq:NN \l_tmpa_str \l_piton_language_str }
        \clist_map_inline:Nn \l_tmpa_clist 
          { \cs_set:cpn { PitonIdentifier _ \l_tmpa_str _ ##1 } { #3 } }
      }
  } 
%    \end{macrocode}
% 
%    \begin{macrocode}
\cs_new_protected:Npn \@@_identifier:n #1
  { 
    \cs_if_exist_use:cF { PitonIdentifier _ \l_piton_language_str _ #1 } 
      { 
        \cs_if_exist_use:cF { PitonIdentifier _ #1 } 
          { \PitonStyle { Identifier } } 
      }
    { #1 } 
  } 
%    \end{macrocode}
%
%
% \bigskip
% In particular, we have an highlighting of the identifiers which are the
% names of Python functions previously defined by the user. Indeed, when a
% Python function is defined, the style |Name.Function.Internal| is applied to
% that name. We define now that style (you define it directly and you short-cut
% the function |\SetPitonStyle|). 
%    \begin{macrocode}
\cs_new_protected:cpn { pitonStyle _ Name.Function.Internal } #1
  { 
%    \end{macrocode}
% First, the element is composed in the TeX flow with the style |Name.Function|
% which is provided to the final user.
%    \begin{macrocode}
    { \PitonStyle { Name.Function } { #1 } }
%    \end{macrocode}
% Now, we specify that the name of the new Python function is a known identifier
% that will be formatted with the Piton style |UserFunction|. Of course,
% here the affectation is global because we have to exit many groups and even
% the environments |{Piton}|).
%    \begin{macrocode}
    \cs_gset_protected:cpn { PitonIdentifier _ \l_piton_language_str _ #1 } 
      { \PitonStyle { UserFunction } }
%    \end{macrocode}
% Now, we put the name of that new user function in the dedicated sequence
% (specific of the current language). {\bfseries That sequence will be used only
% by |\PitonClearUserFunctions|}.
%    \begin{macrocode}
    \seq_if_exist:cF { g_@@_functions _ \l_piton_language_str _ seq }
      { \seq_new:c { g_@@_functions _ \l_piton_language_str _ seq } }
    \seq_gput_right:cn { g_@@_functions _ \l_piton_language_str _ seq } { #1 }
%    \end{macrocode}
% We update |\g_@@_languages_seq| which is used only by the command
% |\PitonClearUserFunctions| when it's used without its optional argument.
%    \begin{macrocode}
    \seq_if_in:NoF \g_@@_languages_seq { \l_piton_language_str }
      { \seq_gput_left:No \g_@@_languages_seq { \l_piton_language_str } }
  }
%    \end{macrocode}
%
% \bigskip
%    \begin{macrocode}
\NewDocumentCommand \PitonClearUserFunctions { ! o } 
  { 
    \tl_if_novalue:nTF { #1 }
%    \end{macrocode}
% If the command is used without its optional argument, we will deleted the
% user language for all the computer languages.
%    \begin{macrocode}
      { \@@_clear_all_functions: }
      { \@@_clear_list_functions:n { #1 } }
  }
%    \end{macrocode}
%
% \bigskip
%    \begin{macrocode}
\cs_new_protected:Npn \@@_clear_list_functions:n #1
  {
    \clist_set:Nn \l_tmpa_clist { #1 }
    \clist_map_function:NN \l_tmpa_clist \@@_clear_functions_i:n 
    \clist_map_inline:nn { #1 }
      { \seq_gremove_all:Nn \g_@@_languages_seq { ##1 } }
  }
%    \end{macrocode}
% 
% \bigskip
%    \begin{macrocode}
\cs_new_protected:Npn \@@_clear_functions_i:n #1
  { \@@_clear_functions_ii:n { \str_lowercase:n { #1 } } }
%    \end{macrocode}
% 
% The following command clears the list of the user-defined functions for the
% language provided in argument (mandatory in lower case). 
%    \begin{macrocode}
\cs_new_protected:Npn \@@_clear_functions_ii:n #1 
  {
    \seq_if_exist:cT { g_@@_functions _ #1 _ seq }
      {
        \seq_map_inline:cn { g_@@_functions _ #1 _ seq }
          { \cs_undefine:c { PitonIdentifier _ #1 _ ##1} }
        \seq_gclear:c { g_@@_functions _ #1 _ seq }
      }
  }
\cs_generate_variant:Nn \@@_clear_functions_ii:n { e }
%    \end{macrocode}
%  
% \bigskip
%    \begin{macrocode}
\cs_new_protected:Npn \@@_clear_functions:n #1 
  {
    \@@_clear_functions_i:n { #1 }
    \seq_gremove_all:Nn \g_@@_languages_seq { #1 }
  }
%    \end{macrocode}
% 
% \bigskip
% The following command clears all the user-defined functions for all the
% computer languages.
%    \begin{macrocode}
\cs_new_protected:Npn \@@_clear_all_functions:
  { 
    \seq_map_function:NN \g_@@_languages_seq \@@_clear_functions_i:n 
    \seq_gclear:N \g_@@_languages_seq
  }
%    \end{macrocode}
%
%
% \bigskip
%    \begin{macrocode}
\AtEndDocument
  { \lua_now:n { piton.join_and_write_files() } }
%    \end{macrocode}
%
% \bigskip
% \subsubsection{Security}
%
%     \begin{macrocode}
\AddToHook { env / piton / begin } 
  { \@@_fatal:n { No~environment~piton } }

\msg_new:nnn { piton } { No~environment~piton }
  { 
    There~is~no~environment~piton!\\
    There~is~an~environment~{Piton}~and~a~command~
    \token_to_str:N \piton\ but~there~is~no~environment~
    {piton}.~This~error~is~fatal.
  }
%    \end{macrocode}
%
% 
% \bigskip
% \subsubsection{The error messages of the package}
%
%
%    \begin{macrocode}
\@@_msg_new:nn { rounded-corners~without~Tikz }
  {
    TikZ~not~used \\
    You~can't~use~the~key~'rounded-corners'~because~
    you~have~not~loaded~the~package~TikZ. \\
    If~you~go~on,~that~key~will~be~ignored. \\
    You~won't~have~similar~error~till~the~end~of~the~document.
  }
%    \end{macrocode}
% 
%    \begin{macrocode}
\@@_msg_new:nn { tcolorbox~not~loaded }
  {
    tcolorbox~not~loaded \\
    You~can't~use~the~key~'tcolorbox'~because~
    you~have~not~loaded~the~package~tcolorbox. \\
    Use~\token_to_str:N \usepackage[breakable]{tcolorbox}. \\
    If~you~go~on,~that~key~will~be~ignored.
  }
%    \end{macrocode}
%
%    \begin{macrocode}
\@@_msg_new:nn { library~breakable~not~loaded }
  {
    breakable~not~loaded \\
    You~can't~use~the~key~'tcolorbox'~because~
    you~have~not~loaded~the~library~'breakable'~of~tcolorbox'. \\
    Use~\token_to_str:N \tcbuselibrary{breakable}. \\
    If~you~go~on,~that~key~will~be~ignored.
  }
%    \end{macrocode}
%
%    \begin{macrocode}
\@@_msg_new:nn { Language~not~defined }
  { 
    Language~not~defined \\
    The~language~'\l_tmpa_tl'~has~not~been~defined~previously.\\
    If~you~go~on,~your~command~\token_to_str:N \NewPitonLanguage\ 
    will~be~ignored.
  }
%    \end{macrocode}
%
% 
%    \begin{macrocode}
\@@_msg_new:nn { bad~version~of~piton.lua }
  {
    Bad~number~version~of~'piton.lua'\\
    The~file~'piton.lua'~loaded~has~not~the~same~number~of~
    version~as~the~file~'piton.sty'.~You~can~go~on~but~you~should~
    address~that~issue.
  }
%    \end{macrocode}
%
%
%    \begin{macrocode}
\@@_msg_new:nn { Unknown~key~NewPitonLanguage } 
  {
    Unknown~key~for~\token_to_str:N \NewPitonLanguage.\\
    The~key~'\l_keys_key_str'~is~unknown.\\
    This~key~will~be~ignored.\\
  }
%    \end{macrocode}
% 
%    \begin{macrocode}
\@@_msg_new:nn { Unknown~key~for~SetPitonStyle } 
  {
    The~style~'\l_keys_key_str'~is~unknown.\\
    This~key~will~be~ignored.\\
    The~available~styles~are~(in~alphabetic~order):~
    \clist_use:Nnnn \g_@@_styles_clist { ~and~ } { ,~ } { ~and~ }.
  }
%    \end{macrocode}
%
%    \begin{macrocode}
\@@_msg_new:nn { Invalid~key }
  {
    Wrong~use~of~key.\\
    You~can't~use~the~key~'\l_keys_key_str'~here.\\
    That~key~will~be~ignored.
  }
%    \end{macrocode}
%
%    \begin{macrocode}
\@@_msg_new:nn { Unknown~key~for~line-numbers } 
  { 
    Unknown~key. \\
    The~key~'line-numbers / \l_keys_key_str'~is~unknown.\\
    The~available~keys~of~the~family~'line-numbers'~are~(in~
    alphabetic~order):~
    absolute,~false,~label-empty-lines,~resume,~skip-empty-lines,~
    sep,~start~and~true.\\
    That~key~will~be~ignored.
  }
%    \end{macrocode}
%
%    \begin{macrocode}
\@@_msg_new:nn { Unknown~key~for~marker } 
  { 
    Unknown~key. \\
    The~key~'marker / \l_keys_key_str'~is~unknown.\\
    The~available~keys~of~the~family~'marker'~are~(in~
    alphabetic~order):~ beginning,~end~and~include-lines.\\
    That~key~will~be~ignored.
  }
%    \end{macrocode}
% 
%    \begin{macrocode}
\@@_msg_new:nn { bad~range~specification } 
  {
    Incompatible~keys.\\
    You~can't~specify~the~range~of~lines~to~include~by~using~both~
    markers~and~explicit~number~of~lines.\\
    Your~whole~file~'\l_@@_file_name_str'~will~be~included.
  }
%    \end{macrocode}
%
%    \begin{macrocode}
\cs_new_nopar:Nn \@@_thepage:
  {
    \thepage
    \cs_if_exist:NT \insertframenumber
      {
        ~(frame~\insertframenumber
        \cs_if_exist:NT \beamer@slidenumber { ,~slide~\insertslidenumber }
        )
      } 
  }
%    \end{macrocode}
% 
% We don't give the name |syntax error| for the following error because you
% should not give a name with a space because such space could be replaced by
% U+2423 when the key |show-spaces| is in force in the command |\piton|.
%   \begin{macrocode}
\@@_msg_new:nn { SyntaxError }
  {
    Syntax~Error~on~page~\@@_thepage:.\\
    Your~code~of~the~language~'\l_piton_language_str'~is~not~
    syntactically~correct.\\  
    It~won't~be~printed~in~the~PDF~file.
  }
%    \end{macrocode}
%
%    \begin{macrocode}
\@@_msg_new:nn { FileError }
  {
    File~Error.\\
    It's~not~possible~to~write~on~the~file~'#1' \\
    \sys_if_shell_unrestricted:F 
      { (try~to~compile~with~'lualatex~-shell-escape').\\ } 
    If~you~go~on,~nothing~will~be~written~on~that~file.
  }
%    \end{macrocode}
%
%    \begin{macrocode}
\@@_msg_new:nn { InexistentDirectory }
  {
    Inexistent~directory.\\
    The~directory~'\l_@@_path_write_str'~
    given~in~the~key~'path-write'~does~not~exist.\\
    Nothing~will~be~written~on~'\l_@@_write_str'.
  }
%    \end{macrocode}
% 
%    \begin{macrocode}
\@@_msg_new:nn { begin~marker~not~found }
  {
    Marker~not~found.\\
    The~range~'\l_@@_begin_range_str'~provided~to~the~  
    command~\token_to_str:N \PitonInputFile\ has~not~been~found.~
    The~whole~file~'\l_@@_file_name_str'~will~be~inserted.
  }
%    \end{macrocode}
%
% 
%    \begin{macrocode}
\@@_msg_new:nn { end~marker~not~found }
  {
    Marker~not~found.\\
    The~marker~of~end~of~the~range~'\l_@@_end_range_str'~
    provided~to~the~command~\token_to_str:N \PitonInputFile\ 
    has~not~been~found.~The~file~'\l_@@_file_name_str'~will~
    be~inserted~till~the~end.
  }
%    \end{macrocode}
% 
% 
%    \begin{macrocode}
\@@_msg_new:nn { Unknown~file }
  {
    Unknown~file. \\
    The~file~'#1'~is~unknown.\\
    Your~command~\token_to_str:N \PitonInputFile\ will~be~discarded.
  }
%    \end{macrocode}
%
%    \begin{macrocode} 
\cs_new_protected:Npn \@@_error_if_not_in_beamer:
  { 
    \bool_if:NF \g_@@_beamer_bool
      { \@@_error_or_warning:n { Without~beamer } }
  }
%    \end{macrocode}
%
%    \begin{macrocode}
\@@_msg_new:nn { Without~beamer }
  {
    Key~'\l_keys_key_str'~without~Beamer.\\
    You~should~not~use~the~key~'\l_keys_key_str'~since~you~
    are~not~in~Beamer.\\
    However,~you~can~go~on.
  }
%    \end{macrocode}
% 
%    \begin{macrocode}
\@@_msg_new:nnn { Unknown~key~for~PitonOptions }
  { 
    Unknown~key. \\
    The~key~'\l_keys_key_str'~is~unknown~for~\token_to_str:N \PitonOptions.~
    It~will~be~ignored.\\
    For~a~list~of~the~available~keys,~type~H~<return>.
  }
  {
    The~available~keys~are~(in~alphabetic~order):~
    auto-gobble,~
    background-color,~
    begin-range,~
    box,~
    break-lines,~
    break-lines-in-piton,~
    break-lines-in-Piton,~
    break-numbers-anywhere,~
    break-strings-anywhere,~
    continuation-symbol,~ 
    continuation-symbol-on-indentation,~
    detected-beamer-commands,~
    detected-beamer-environments,~
    detected-commands,~
    end-of-broken-line,~
    end-range,~
    env-gobble,~
    env-used-by-split,~
    font-command,~
    gobble,~
    indent-broken-lines,~
    join,~
    language,~
    left-margin,~
    line-numbers/,~
    marker/,~
    math-comments,~
    path,~
    path-write,~
    print,~
    prompt-background-color,~
    raw-detected-commands,~
    resume,~
    rounded-corners,~
    show-spaces,~
    show-spaces-in-strings,~
    splittable,~
    splittable-on-empty-lines,~
    split-on-empty-lines,~
    split-separation,~
    tabs-auto-gobble,~
    tab-size,~
    tcolorbox,~
    varwidth,~
    vertical-detected-commands,~
    width~and~write.
  }
%    \end{macrocode}
%
% \bigskip
%    \begin{macrocode}
\@@_msg_new:nn { label~with~lines~numbers } 
  {
    You~can't~use~the~command~\token_to_str:N \label\
    because~the~key~'line-numbers'~is~not~active.\\
    If~you~go~on,~that~command~will~ignored.
  }
%    \end{macrocode}
% 
%
% \bigskip
%    \begin{macrocode}
\@@_msg_new:nn { overlay~without~beamer }
  {
    You~can't~use~an~argument~<...>~for~your~command~
    \token_to_str:N \PitonInputFile\ because~you~are~not~
    in~Beamer.\\
    If~you~go~on,~that~argument~will~be~ignored.
  }
%    \end{macrocode}
%
% 
% \bigskip
% \subsubsection{We load piton.lua}
%
%
% \bigskip
%    \begin{macrocode}
\cs_new_protected:Npn \@@_test_version:n #1
  { 
    \str_if_eq:onF \PitonFileVersion { #1 } 
      { \@@_error:n { bad~version~of~piton.lua } }
  }
%    \end{macrocode}
%
% \bigskip
%    \begin{macrocode}
\hook_gput_code:nnn { begindocument } { . }
  { 
    \lua_load_module:n { piton }
    \lua_now:n 
      { 
        tex.sprint ( luatexbase.catcodetables.expl ,
                     [[\@@_test_version:n {]] .. piton_version ..  "}" ) 
      } 
  }
%    \end{macrocode}
% 
%   
%</STY>
% 
% 
% \bigskip
% \subsection{The Lua part of the implementation}
%
% \bigskip
% The Lua code will be loaded via a |{luacode*}| environment. The environment
% is by itself a Lua block and the local declarations will be local to that
% block. All the global functions (used by the L3 parts of the implementation)
% will be put in a Lua table called |piton|.
%
% 
%    \begin{macrocode}
%<*LUA>
piton.comment_latex = piton.comment_latex or ">" 
piton.comment_latex = "#" .. piton.comment_latex 
%    \end{macrocode}
%
% \bigskip
% The table |piton.write_files| will contain the contents of all the files that we
% will write on the disk in the |\AtEndDocument| (if the user has used the key
% |write-file|). The table |piton.join_files| is similar for the key |join|.
%    \begin{macrocode}
piton.write_files = { }
piton.join_files = { }
%    \end{macrocode}
%
% \bigskip
%    \begin{macrocode}
local sprintL3
function sprintL3 ( s ) 
  tex.sprint ( luatexbase.catcodetables.expl , s ) 
end 
%    \end{macrocode}
% 
% \bigskip
% \subsubsection{Special functions dealing with LPEG}
%
% \medskip
% We will use the Lua library \pkg{lpeg} which is built in LuaTeX. That's why we
% define first aliases for several functions of that library.
%    \begin{macrocode}
local P, S, V, C, Ct, Cc = lpeg.P, lpeg.S, lpeg.V, lpeg.C, lpeg.Ct, lpeg.Cc
local Cg , Cmt , Cb = lpeg.Cg , lpeg.Cmt , lpeg.Cb 
local B , R = lpeg.B , lpeg.R
%    \end{macrocode}
%
% \vspace{1cm}
% The following line is mandatory.
%    \begin{macrocode}
lpeg.locale(lpeg)
%    \end{macrocode}
%
%
% \subsubsection{The functions Q, K, WithStyle, etc.}
%
% \bigskip
% The function |Q| takes in as argument a pattern and returns a \textsc{lpeg}
% \emph{which does a capture} of the pattern. That capture will be sent to LaTeX
% with the catcode ``other'' for all the characters: it's suitable for elements
% of the informatic listings that \pkg{piton} will typeset verbatim (thanks to the
% catcode ``other''). 
%    \begin{macrocode}
local Q
function Q ( pattern )
  return Ct ( Cc ( luatexbase.catcodetables.CatcodeTableOther ) * C ( pattern ) )
end 
%    \end{macrocode}
%
%
% \bigskip
% The function |L| takes in as argument a pattern and returns a \textsc{lpeg}
% \emph{which does a capture} of the pattern. That capture will be sent to LaTeX
% with standard LaTeX catcodes for all the characters: the elements captured
% will be formatted as normal LaTeX codes. It's suitable for the ``LaTeX
% comments'' in the environments |{Piton}| and the elements between 
% |begin-escape| and |end-escape|. That function won't be much used.
%    \begin{macrocode}
local L
function L ( pattern ) return 
  Ct ( C ( pattern ) ) 
end
%    \end{macrocode}
%
% \bigskip
% The function |Lc| (the c is for \emph{constant}) takes in as argument a string
% and returns a \textsc{lpeg} \emph{with does a constant capture} which returns
% that string. The elements captured will be formatted as L3 code. It will be
% used to send to LaTeX all the formatting LaTeX instructions we have to insert
% in order to do the syntactic highlighting (that's the main job of
% \pkg{piton}). That function, unlike the previous one, will be widely used.
%    \begin{macrocode}
local Lc
function Lc ( string ) return 
  Cc ( { luatexbase.catcodetables.expl , string } ) 
end 
%    \end{macrocode}
% 
% \bigskip
% The function |K| creates a \textsc{lpeg} which will return as capture the
% whole LaTeX code corresponding to a Python chunk (that is to say with the
% LaTeX formatting instructions corresponding to the syntactic nature of that
% Python chunk). The first argument is a Lua string corresponding to the name of
% a \pkg{piton} style and the second element is a pattern (that is to say a
% \textsc{lpeg} without capture)
%    \begin{macrocode}e
local K
function K ( style , pattern ) return
  Lc ( [[ {\PitonStyle{ ]] .. style .. "}{" )
  * Q ( pattern )
  * Lc "}}"
end
%    \end{macrocode}
% The formatting commands in a given \pkg{piton} style (eg. the style |Keyword|)
% may be semi-global declarations (such as |\bfseries| or |\slshape|) or LaTeX
% macros with an argument (such as |\fbox| or |\colorbox{yellow}|). In order to
% deal with both syntaxes, we have used two pairs of braces: 
% |{\PitonStyle{Keyword}{|\texttt{\slshape text to format}|}}|.
% 
%
% \bigskip
% The following function |WithStyle| is similar to the function |K| but should
% be used for multi-lines elements. 
%    \begin{macrocode}
local WithStyle
function WithStyle ( style , pattern ) return 
    Ct ( Cc "Open" * Cc ( [[{\PitonStyle{]] .. style .. "}{" ) * Cc "}}" ) 
  * pattern 
  * Ct ( Cc "Close" ) 
end
%    \end{macrocode}
% 
% \bigskip
% The following \textsc{lpeg} catches the Python chunks which are in LaTeX
% escapes (and that chunks will be considered as normal LaTeX constructions).
%    \begin{macrocode}
Escape = P ( false ) 
EscapeClean = P ( false ) 
if piton.begin_escape then 
  Escape = 
    P ( piton.begin_escape )
    * L ( ( 1 - P ( piton.end_escape ) ) ^ 1 ) 
    * P ( piton.end_escape )
%    \end{macrocode}
% The LPEG |EscapeClean| will be used in the LPEG Clean (and that LPEG is used
% to ``clean'' the code by removing the formatting elements).
%    \begin{macrocode}
  EscapeClean = 
    P ( piton.begin_escape )
    * ( 1 - P ( piton.end_escape ) ) ^ 1 
    * P ( piton.end_escape )
end
%    \end{macrocode}
% 
%    \begin{macrocode}
EscapeMath = P ( false ) 
if piton.begin_escape_math then 
  EscapeMath = 
    P ( piton.begin_escape_math )
    * Lc "$"
    * L ( ( 1 - P(piton.end_escape_math) ) ^ 1 )
    * Lc "$" 
    * P ( piton.end_escape_math )
end
%    \end{macrocode}
% 
%
% \bigskip
% \paragraph{The basic syntactic LPEG}
%
%    \begin{macrocode}
local alpha , digit = lpeg.alpha , lpeg.digit
local space = P " " 
%    \end{macrocode}
%
% Remember that, for \textsc{lpeg}, the Unicode characters such as |à|, |â|,
% |ç|, etc. are in fact strings of length 2 (2 bytes) because \pkg{lpeg} is not
% Unicode-aware. 
%    \begin{macrocode}
local letter = alpha + "_" + "â" + "à" + "ç" + "é" + "è" + "ê" + "ë" + "ï" + "î"
                    + "ô" + "û" + "ü" + "Â" + "À" + "Ç" + "É" + "È" + "Ê" + "Ë"
                    + "Ï" + "Î" + "Ô" + "Û" + "Ü"

local alphanum = letter + digit
%    \end{macrocode}
% 
% \bigskip
% The following \textsc{lpeg} |identifier| is a mere pattern (that is to say
% more or less a regular expression) which matches the Python identifiers (hence
% the name).
%    \begin{macrocode}
local identifier = letter * alphanum ^ 0
%    \end{macrocode}
% 
% \medskip
% On the other hand, the \textsc{lpeg} |Identifier| (with a capital) also returns
% a \emph{capture}.
%    \begin{macrocode}
local Identifier = K ( 'Identifier.Internal' , identifier )
%    \end{macrocode}
%
% \bigskip
% By convention, we will use names with an initial capital for \textsc{lpeg}
% which return captures.
%
% 
% \bigskip
% Here is the first use of our function~|K|. That function will be used to
% construct \textsc{lpeg} which capture Python chunks for which we have a
% dedicated \pkg{piton} style. For example, for the numbers, \pkg{piton}
% provides a style which is called |Number|. The name of the style is provided
% as a Lua string in the second argument of the function~|K|. By convention, we
% use single quotes for delimiting the Lua strings which are names of
% \pkg{piton} styles (but this is only a convention).
%    \begin{macrocode}
local Number =
  K ( 'Number.Internal' ,
      ( digit ^ 1 * P "." * # ( 1 - P "." ) * digit ^ 0 
        + digit ^ 0 * P "." * digit ^ 1 
        + digit ^ 1 )
      * ( S "eE" * S "+-" ^ -1 * digit ^ 1 ) ^ -1
      + digit ^ 1 
    ) 
%    \end{macrocode}
%
% \bigskip
% We will now define the LPEG |Word|.
%
% We have a problem in the following LPEG because, obviously, we should adjust
% the list of symbols with the delimiters of the current language (no?).
%    \begin{macrocode}
local lpeg_central = 1 - S " '\"\r[({})]" - digit
%    \end{macrocode}
% We recall that |piton.begin_escape| and |piton_end_escape| are Lua strings
% corresponding to the keys |begin-escape| and |end-escape|. 
%    \begin{macrocode}
if piton.begin_escape then 
  lpeg_central = lpeg_central - piton.begin_escape 
end
if piton.begin_escape_math then 
  lpeg_central = lpeg_central - piton.begin_escape_math 
end
local Word = Q ( lpeg_central ^ 1 ) 
%    \end{macrocode}
%
% \bigskip
%    \begin{macrocode}
local Space = Q " " ^ 1

local SkipSpace = Q " " ^ 0 

local Punct = Q ( S ".,:;!" )

local Tab = "\t" * Lc [[ \@@_tab: ]]
%    \end{macrocode}
% 
% \bigskip
% Remember that |\@@_leading_space:| does \emph{not} create a space, only an
% incrementation of the counter |\g_@@_indentation_int|.
%    \begin{macrocode}
local SpaceIndentation = Lc [[ \@@_leading_space: ]] * Q " "
%    \end{macrocode}
%
% \bigskip
%    \begin{macrocode}
local Delim = Q ( S "[({})]" )
%    \end{macrocode}
% 
% \bigskip 
% The following \textsc{lpeg} catches a space (U+0020) and replace it by
% |\l_@@_space_in_string_tl|. It will be used in the strings. Usually,
% |\l_@@_space_in_string_tl| will contain a space and therefore there won't be any
% difference. However, when the key |show-spaces-in-strings| is in force,
% |\\l_@@_space_in_string_tl| will contain ␣ (U+2423) in order to visualize the
% spaces.
%    \begin{macrocode}
local SpaceInString = space * Lc [[ \l_@@_space_in_string_tl ]]
%    \end{macrocode}
%
%
% \bigskip
% \subsubsection{The option 'detected-commands' and al.}
%
% We create four Lua tables called |detected_commands|, |raw_detected_commands|,
% |beamer_commands| and |beamer_environments|.
%
% On the TeX side, the corresponding data have first been stored as clists. 

% Then, in a |\AtBeginDocument|, they have been converted in ``toks registers''
% of TeX.
%
% Now, on the Lua side, we are able to access to those ``toks registers'' with
% the special pseudo-table |tex.toks| of LuaTeX.
%
% Remark that we can safely use |explode(',')| to convert such ``toks
% registers'' in Lua tables since, in a clist of L3, there is no empty
% component and, for each component, there is no space on both sides (the
% |explode| of the Lua of LuaTeX is unable to do itself such purification of the
% components). 
%    \begin{macrocode}
local detected_commands = tex.toks.PitonDetectedCommands : explode ( ',' )
local raw_detected_commands = tex.toks.PitonRawDetectedCommands : explode ( ',' )
local beamer_commands = tex.toks.PitonBeamerCommands : explode ( ',' )  
local beamer_environments = tex.toks.PitonBeamerEnvironments : explode ( ',' ) 
%    \end{macrocode}
%
% We will also create some \textsc{lpeg}.
%
% According to our conventions, a \textsc{lpeg} with a name in camelCase is a
% \textsc{lpeg} which doesn't do any capture.
%    \begin{macrocode}
local detectedCommands = P ( false ) 
for _ , x in ipairs ( detected_commands ) do
  detectedCommands = detectedCommands + P ( "\\" .. x ) 
end 
%    \end{macrocode}
% Further, we will have a \textsc{lpeg} called |DetectedCommands| (in
% PascalCase) which will be a \textsc{lpeg} \emph{with} captures.
%
% \bigskip
%    \begin{macrocode}
local rawDetectedCommands = P ( false ) 
for _ , x in ipairs ( raw_detected_commands ) do
  rawDetectedCommands = rawDetectedCommands + P ( "\\" .. x ) 
end 
%    \end{macrocode}
%
%
%    \begin{macrocode}
local beamerCommands = P ( false ) 
for _ , x in ipairs ( beamer_commands ) do
  beamerCommands = beamerCommands + P ( "\\" .. x ) 
end 
%    \end{macrocode}
%
%    \begin{macrocode}
local beamerEnvironments = P ( false ) 
for _ , x in ipairs ( beamer_environments ) do
  beamerEnvironments = beamerEnvironments + P ( x ) 
end 
%    \end{macrocode}
%
%    \begin{macrocode}
local beamerBeginEnvironments = 
    ( space ^ 0 * 
      L 
        ( 
          P [[\begin{]] * beamerEnvironments * "}"
          * ( "<" * ( 1 - P ">" ) ^ 0 * ">" ) ^ -1
        )
      * "\r" 
    ) ^ 0 
%    \end{macrocode}
%
%    \begin{macrocode}
local beamerEndEnvironments = 
    ( space ^ 0 * 
      L ( P [[\end{]] * beamerEnvironments * "}" ) 
      * "\r"
    ) ^ 0 
%    \end{macrocode}
% 
%
% \bigskip
% \paragraph{Several tools for the construction of the main LPEG}
%
%    \begin{macrocode}
local LPEG0 = { }
local LPEG1 = { }
local LPEG2 = { }
local LPEG_cleaner = { }
%    \end{macrocode}
%
% \bigskip
% For each language, we will need a pattern to match expressions with balanced
% braces. Those balanced braces must \emph{not} take into account the braces
% present in strings of the language. However, the syntax for the strings is
% language-dependent. That's why we write a Lua function |Compute_braces| which
% will compute the pattern by taking in as argument a pattern for the strings of
% the language (at least the shorts strings). The argument of |Compute_braces|
% must be a pattern \emph{which does no catching}.
%    \begin{macrocode}
local Compute_braces
function Compute_braces ( lpeg_string ) return 
  P { "E" ,
       E = 
           (
             "{" * V "E" * "}" 
             + 
             lpeg_string
             + 
             ( 1 - S "{}" ) 
           ) ^ 0 
    }
end 
%    \end{macrocode}
%
% 
%         
% \bigskip
% The following Lua function will compute the \text{lpeg} |DetectedCommands|
% which is a \textsc{lpeg} with captures.
%    \begin{macrocode}
local Compute_DetectedCommands
function Compute_DetectedCommands ( lang , braces ) return 
  Ct ( 
       Cc "Open" 
        * C ( detectedCommands * space ^ 0 * P "{" ) 
        * Cc "}" 
     ) 
   * ( braces 
       / ( function ( s ) 
             if s ~= '' then return 
               LPEG1[lang] : match ( s ) 
             end
           end ) 
     )
   * P "}" 
   * Ct ( Cc "Close" ) 
end
%    \end{macrocode}
%
%    \begin{macrocode}
local Compute_RawDetectedCommands
function Compute_RawDetectedCommands ( lang , braces ) return 
  Ct ( C ( rawDetectedCommands * space ^ 0 * P "{" * braces * P "}" ) )
end
%    \end{macrocode}
%
% \bigskip
%    \begin{macrocode}
local Compute_LPEG_cleaner
function Compute_LPEG_cleaner ( lang , braces ) return 
  Ct ( ( ( detectedCommands + rawDetectedCommands ) * "{" 
          * ( braces 
              / ( function ( s ) 
                    if s ~= '' then return 
                      LPEG_cleaner[lang] : match ( s ) 
                    end  
                  end ) 
            )
          * "}" 
         + EscapeClean
         +  C ( P ( 1 ) )
        ) ^ 0 ) / table.concat 
end
%    \end{macrocode}
% 
% \bigskip
% The following function |ParseAgain| will be used in the definitions of the
% LPEG of the different computer languages when we will need to
% \emph{parse again} a small chunk of code. It's a way to avoid the use of a actual
% \emph{grammar} of LPEG (in a sens, a recursive regular expression). 
%
% Remark that there is no \pkg{piton} style associated to a chunk of code which
% is analyzed by |ParseAgain|. If we wish a \pkg{piton} style available to the
% final user (if he wish to format that element with a uniform font instead of
% an analyze by |ParseAgain|), we have to use |\@@_piton:n|. 
%    \begin{macrocode}
local ParseAgain 
function ParseAgain ( code ) 
  if code ~= '' then return 
%    \end{macrocode}
% The variable |piton.language| is set in the function |piton.Parse|.
%    \begin{macrocode}
    LPEG1[piton.language] : match ( code ) 
  end 
end 
%    \end{macrocode}
% 
%
%
% \bigskip
% \paragraph{Constructions for Beamer}
%
% \bigskip
% If the class Beamer is used, some environments and commands of Beamer are
% automatically detected in the listings of \pkg{piton}.
%
%    \begin{macrocode}
local Beamer = P ( false ) 
%    \end{macrocode}
% 
% \bigskip
% The following Lua function will be used to compute the \textsc{lpeg} |Beamer|
% for each computer language.
%
% According to our conventions, the \textsc{lpeg} |Beamer|, with its name in
% PascalCase does captures.
%    \begin{macrocode}
local Compute_Beamer
function Compute_Beamer ( lang , braces ) 
%    \end{macrocode}
%
% \smallskip
% We will compute in |lpeg| the \textsc{lpeg} that we will return.
%    \begin{macrocode}
  local lpeg = L ( P [[\pause]] * ( "[" * ( 1 - P "]" ) ^ 0 * "]" ) ^ -1 ) 
  lpeg = lpeg + 
      Ct ( Cc "Open" 
            * C ( beamerCommands
                  * ( "<" * ( 1 - P ">" ) ^ 0 * ">" ) ^ -1 
                  * P "{" 
                ) 
            * Cc "}" 
         ) 
       * ( braces / 
           ( function ( s ) if s ~= '' then return LPEG1[lang] : match ( s ) end end ) )
       * "}" 
       * Ct ( Cc "Close" ) 
%    \end{macrocode}
% 
%
% \bigskip
% For the command |\alt|, the specification of the overlays (between angular
% brackets) is mandatory. 
%    \begin{macrocode}
  lpeg = lpeg +
    L ( P [[\alt]] * "<" * ( 1 - P ">" ) ^ 0 * ">{" )
     * ( braces / 
         ( function ( s ) if s ~= '' then return LPEG1[lang] : match ( s ) end end ) )
     * L ( P "}{" )
     * ( braces / 
         ( function ( s ) if s ~= '' then return LPEG1[lang] : match ( s ) end end ) )
     * L ( P "}" )
%    \end{macrocode}
%
% \bigskip
% For |\temporal|, the specification of the overlays (between angular brackets)
% is mandatory. 
%    \begin{macrocode}
  lpeg = lpeg +
      L ( P [[\temporal]] * "<" * ( 1 - P ">" ) ^ 0 * ">{" )
      * ( braces 
          / ( function ( s ) 
              if s ~= '' then return LPEG1[lang] : match ( s ) end end ) )
      * L ( P "}{" )
      * ( braces 
          / ( function ( s ) 
              if s ~= '' then return LPEG1[lang] : match ( s ) end end ) )
      * L ( P "}{" )
      * ( braces 
          / ( function ( s ) 
              if s ~= '' then return LPEG1[lang] : match ( s ) end end ) )
      * L ( P "}" )
%    \end{macrocode}
%
% \bigskip
% Now, the environments of Beamer.
%    \begin{macrocode}
  for _ , x in ipairs ( beamer_environments ) do
    lpeg = lpeg + 
          Ct ( Cc "Open" 
               * C ( 
                      P ( [[\begin{]] .. x .. "}" )
                      * ( "<" * ( 1 - P ">") ^ 0 * ">" ) ^ -1 
                    ) 
               * Cc ( [[\end{]] .. x ..  "}" )
              ) 
          * ( 
              ( ( 1 - P ( [[\end{]] .. x .. "}" ) ) ^ 0 ) 
                  / ( function ( s ) 
                        if s ~= '' then return 
                          LPEG1[lang] : match ( s ) 
                        end 
                      end ) 
            )
          * P ( [[\end{]] .. x .. "}" ) 
          * Ct ( Cc "Close" ) 
  end 
%    \end{macrocode}
%
% \bigskip
% Now, you can return the value we have computed.
%    \begin{macrocode}
  return lpeg 
end
%    \end{macrocode}
%
% 
% \bigskip
% The following LPEG is in relation with the key |math-comments|. It will be
% used in all the languages.
%    \begin{macrocode}
local CommentMath = 
  P "$" * K ( 'Comment.Math' , ( 1 - S "$\r" ) ^ 1  ) * P "$" -- $
%    \end{macrocode}
% 
%
% \bigskip
% \paragraph{EOL}
%
% \bigskip
% The following LPEG will detect the Python prompts when the user is typesetting
% an interactive session of Python (directly or through |{pyconsole}| of
% \pkg{pyluatex}). We have to detect that prompt twice. The first detection
% (called \emph{hasty detection}) will be before the |\@@_begin_line:| because
% you want to trigger a special background color for that row (and, after the
% |\@@_begin_line:|, it's too late to change de background).
%    \begin{macrocode}
local PromptHastyDetection = 
  ( # ( P ">>>" + "..." ) * Lc [[ \@@_prompt: ]] ) ^ -1 
%    \end{macrocode}
% We remind that the marker |#| of \textsc{lpeg} specifies that the pattern will be
% detected but won't consume any character.
%
% \medskip
% With the following \textsc{lpeg}, a style will actually be applied to the
% prompt (for instance, it's possible to decide to discard these prompts).
%    \begin{macrocode}
local Prompt = 
  K ( 'Prompt' , ( ( P ">>>" + "..." ) * P " " ^ -1 + P ( true ) ) ) ^ -1  
%    \end{macrocode}
% The |P ( true )| at the end is mandatory because we want the style to be
% \emph{always} applied, even with an empty argument, in order, for example to
% add a ``false'' prompt marker with the tuning:
% 
% \begin{Verbatim}
% \SetPitonStyle{ Prompt = >>>\space }
% \end{Verbatim}
% 
%
%
% \bigskip
% The following \textsc{lpeg} |EOL| is for the end of lines.
%    \begin{macrocode}
local EOL = 
  P "\r" 
  *
  (
    space ^ 0 * -1 
    + 
%    \end{macrocode}
% We recall that each line of the informatic code we have to parse will be sent
% back to LaTeX between a pair |\@@_begin_line:| --
% |\@@_end_line:|\footnote{Remember that the \texttt{\textbackslash
% @@\_end\_line:} must be explicit because it will be used as marker in order to
% delimit the argument of the command \texttt{\textbackslash @@\_begin\_line:}}.
%    \begin{macrocode}
    Ct ( 
         Cc "EOL"
         * 
         Ct ( Lc [[ \@@_end_line: ]]
              * beamerEndEnvironments 
              *
                (
%    \end{macrocode}
% If the last line of the listing is the end of an environment of Beamer (eg.
% |\end{uncoverenv}|), then, we don't open a new line. A token |\@@_end_line:|
% will be added at the end of the environment but it will be no-op since we have
% defined the macro |\@@_end_line:| to be no-op (even though it is also used as
% a marker for the TeX delimited macro |\@@_begin_line:|).
%    \begin{macrocode}
                    -1 
                  + 
                    beamerBeginEnvironments 
                  * PromptHastyDetection
                  * Lc [[ \@@_par:\@@_begin_line: ]]
                  * Prompt
                )
            )
       )
  ) 
  * ( SpaceIndentation ^ 0 * # ( 1 - S " \r" ) ) ^ -1
%    \end{macrocode}
%
% 
% \bigskip
% The following \textsc{lpeg} |CommentLaTeX| is for what is called in that
% document the ``LaTeX comments''. Since the elements that will be caught must
% be sent to LaTeX with standard LaTeX catcodes, we put the capture (done by
% the function~|C|) in a table (by using~|Ct|, which is an alias for |lpeg.Ct|).
%    \begin{macrocode}
local CommentLaTeX =
  P ( piton.comment_latex ) 
  * Lc [[{\PitonStyle{Comment.LaTeX}{\ignorespaces]]
  * L ( ( 1 - P "\r" ) ^ 0 ) 
  * Lc "}}"
  * ( EOL + -1 )  
%    \end{macrocode}
% 
% 
% \bigskip
% \subsubsection{The language Python}
% 
% We open a Lua local scope for the language Python (of course, there will be
% also global definitions).
%    \begin{macrocode}
--python Python
do 
%    \end{macrocode}
%
% \bigskip
% Some strings of length 2 are explicit because we want the corresponding
% ligatures available in some fonts such as \emph{Fira Code} to be active.
%    \begin{macrocode}
  local Operator = 
    K ( 'Operator' ,
        P "!=" + "<>" + "==" + "<<" + ">>" + "<=" + ">=" + ":=" + "//" + "**" 
        + S "-~+/*%=<>&.@|" )

  local OperatorWord = 
    K ( 'Operator.Word' , P "in" + "is" + "and" + "or" + "not" )
%    \end{macrocode}
%
% \smallskip
% The keyword |in| in a construction such as ``|for i in range(n)|'' must be
% formatted as a keyword and not as an |Operator.Word| and that's why we write
% the following LPEG |For|.
%    \begin{macrocode}
  local For = K ( 'Keyword' , P "for" ) 
              * Space
              * Identifier
              * Space 
              * K ( 'Keyword' , P "in" ) 

  local Keyword = 
    K ( 'Keyword' ,
        P  "assert" + "as" + "break" + "case" + "class" + "continue" + "def" +
        "del" + "elif" + "else" + "except" + "exec" + "finally" + "for" + "from" +
        "global" + "if" + "import" + "lambda" + "non local" + "pass" + "return" +
        "try" + "while" + "with" + "yield" + "yield from" )
    + K ( 'Keyword.Constant' , P "True" + "False" + "None" ) 

  local Builtin = 
    K ( 'Name.Builtin' ,
        P "__import__" + "abs" + "all" + "any" + "bin" + "bool" + "bytearray" +
        "bytes" + "chr" + "classmethod" + "compile" + "complex" + "delattr" +
        "dict" + "dir" + "divmod" + "enumerate" + "eval" + "filter" + "float" +
        "format" + "frozenset" + "getattr" + "globals" + "hasattr" + "hash" +
        "hex" + "id" + "input" + "int" + "isinstance" + "issubclass" + "iter" +
        "len" + "list" + "locals" + "map" + "max" + "memoryview" + "min" + "next"
        + "object" + "oct" + "open" + "ord" + "pow" + "print" + "property" +
        "range" + "repr" + "reversed" + "round" + "set" + "setattr" + "slice" +
        "sorted" + "staticmethod" + "str" + "sum" + "super" + "tuple" + "type" +
        "vars" + "zip" ) 

  local Exception =
    K ( 'Exception' ,
        P "ArithmeticError" + "AssertionError" + "AttributeError" +
        "BaseException" + "BufferError" + "BytesWarning" + "DeprecationWarning" +
        "EOFError" + "EnvironmentError" + "Exception" + "FloatingPointError" +
        "FutureWarning" + "GeneratorExit" + "IOError" + "ImportError" +
        "ImportWarning" + "IndentationError" + "IndexError" + "KeyError" +
        "KeyboardInterrupt" + "LookupError" + "MemoryError" + "NameError" +
        "NotImplementedError" + "OSError" + "OverflowError" +
        "PendingDeprecationWarning" + "ReferenceError" + "ResourceWarning" +
        "RuntimeError" + "RuntimeWarning" + "StopIteration" + "SyntaxError" +
        "SyntaxWarning" + "SystemError" + "SystemExit" + "TabError" + "TypeError"
        + "UnboundLocalError" + "UnicodeDecodeError" + "UnicodeEncodeError" +
        "UnicodeError" + "UnicodeTranslateError" + "UnicodeWarning" +
        "UserWarning" + "ValueError" + "VMSError" + "Warning" + "WindowsError" +
        "ZeroDivisionError" + "BlockingIOError" + "ChildProcessError" +
        "ConnectionError" + "BrokenPipeError" + "ConnectionAbortedError" +
        "ConnectionRefusedError" + "ConnectionResetError" + "FileExistsError" +
        "FileNotFoundError" + "InterruptedError" + "IsADirectoryError" +
        "NotADirectoryError" + "PermissionError" + "ProcessLookupError" +
        "TimeoutError" + "StopAsyncIteration" + "ModuleNotFoundError" +
        "RecursionError" ) 

  local RaiseException = K ( 'Keyword' , P "raise" ) * SkipSpace * Exception * Q "("  
%    \end{macrocode}
%
% \bigskip
% In Python, a ``decorator'' is a statement whose begins by |@| which patches
% the function defined in the following statement.
%    \begin{macrocode}
  local Decorator = K ( 'Name.Decorator' , P "@" * letter ^ 1  ) 
%    \end{macrocode}
% 
% \bigskip
% The following \textsc{lpeg} |DefClass| will be used to detect the definition of a
% new class (the name of that new class will be formatted with the \pkg{piton}
% style |Name.Class|). 
%
% \smallskip
% Example:\enskip \piton{class myclass:}
%    \begin{macrocode}
  local DefClass = 
    K ( 'Keyword' , "class" ) * Space * K ( 'Name.Class' , identifier ) 
%    \end{macrocode}
% 
% If the word |class| is not followed by a identifier, it will be caught as
% keyword by the \textsc{lpeg} |Keyword| (useful if we want to type a
% list of keywords).
%
% \bigskip
% The following \textsc{lpeg} |ImportAs| is used for the lines beginning by |import|.
% % We have to detect the potential keyword |as| because both the name of the
% module and its alias must be formatted with the \pkg{piton} style |Name.Namespace|.
%
% \smallskip
% Example:\enskip \piton{import numpy as np}
%
% \smallskip
% Moreover, after the keyword |import|, it's possible to have a comma-separated
% list of modules (if the keyword |as| is not used).
%
% \smallskip
% Example:\enskip \piton{import math, numpy}
%    \begin{macrocode}
  local ImportAs = 
    K ( 'Keyword' , "import" )
     * Space 
     * K ( 'Name.Namespace' , identifier * ( "." * identifier ) ^ 0 )
     * ( 
         ( Space * K ( 'Keyword' , "as" ) * Space 
            * K ( 'Name.Namespace' , identifier ) )
         + 
         ( SkipSpace * Q "," * SkipSpace 
            * K ( 'Name.Namespace' , identifier ) ) ^ 0 
       ) 
%    \end{macrocode}
% Be careful: there is no commutativity of |+| in the previous expression.
%
% \bigskip
% The \textsc{lpeg} |FromImport| is used for the lines beginning by |from|. We
% need a special treatment because the identifier following the keyword |from|
% must be formatted with the \pkg{piton} style |Name.Namespace| and the
% following keyword |import| must be formatted with the \pkg{piton} style
% |Keyword| and must \emph{not} be caught by the \textsc{lpeg} |ImportAs|.
%
% \smallskip
% Example:\enskip \piton{from math import pi}
%
% \smallskip
%    \begin{macrocode}
  local FromImport =
    K ( 'Keyword' , "from" ) 
      * Space * K ( 'Name.Namespace' , identifier )
      * Space * K ( 'Keyword' , "import" ) 
%    \end{macrocode}
%
% \bigskip
% \paragraph{The strings of Python}
%
% For the strings in Python, there are four categories of delimiters (without
% counting the prefixes for f-strings and raw strings). We will use, in the
% names of our \textsc{lpeg}, prefixes to distinguish the \textsc{lpeg} dealing
% with that categories of strings, as presented in the following tabular.
% \begin{center}
% \begin{tabular}{@{}lcc@{}}
% \toprule
%         & |Single| & |Double| \\
% \midrule
% |Short| & |'text'|     & |"text"| \\
% |Long|  & |'''test'''| & |"""text"""| \\
% \bottomrule
% \end{tabular}
% \end{center}
%
% 
% \bigskip
% We have also to deal with the interpolations in the f-strings. Here
% is an example of a f-string with an interpolation and a format
% instruction\footnote{There is no special \pkg{piton} style for the formatting
% instruction (after the colon): the style which will be applied will be the
% style of the encompassing string, that is to say |String.Short| or
% |String.Long|.} in that interpolation:
%
% |\piton{f'Total price: {total+1:.2f} €'}|
%
%
% \bigskip
% The interpolations beginning by |%| (even though there is more modern
% techniques now in Python).
%    \begin{macrocode}
  local PercentInterpol =  
    K ( 'String.Interpol' , 
        P "%" 
        * ( "(" * alphanum ^ 1 * ")" ) ^ -1 
        * ( S "-#0 +" ) ^ 0 
        * ( digit ^ 1 + "*" ) ^ -1 
        * ( "." * ( digit ^ 1 + "*" ) ) ^ -1 
        * ( S "HlL" ) ^ -1
        * S "sdfFeExXorgiGauc%" 
      ) 
%    \end{macrocode}
% 
% \bigskip
% We can now define the \textsc{lpeg} for the four kinds of strings. It's not
% possible to use our function~|K| because of the interpolations which must be
% formatted with another \pkg{piton} style that the rest of the
% string.\footnote{The interpolations are formatted with the \pkg{piton} style
% |Interpol.Inside|. The initial value of that style is \texttt{\textbackslash
% @@\_piton:n} which means that the interpolations are parsed once again by \pkg{piton}.}
%    \begin{macrocode}
  local SingleShortString =
    WithStyle ( 'String.Short.Internal' ,
%    \end{macrocode}
% First, we deal with the f-strings of Python, which are prefixed by |f| or |F|.
%    \begin{macrocode}
           Q ( P "f'" + "F'" ) 
           * ( 
               K ( 'String.Interpol' , "{" )
                * K ( 'Interpol.Inside' , ( 1 - S "}':" ) ^ 0  )
                * Q ( P ":" * ( 1 - S "}:'" ) ^ 0 ) ^ -1
                * K ( 'String.Interpol' , "}" )
               + 
               SpaceInString 
               + 
               Q ( ( P "\\'" + "\\\\" + "{{" + "}}" + 1 - S " {}'" ) ^ 1 )
             ) ^ 0 
           * Q "'" 
         + 
%    \end{macrocode}
% Now, we deal with the standard strings of Python, but also the ``raw strings''.
%    \begin{macrocode}
           Q ( P "'" + "r'" + "R'" ) 
           * ( Q ( ( P "\\'" + "\\\\" + 1 - S " '\r%" ) ^ 1 ) 
               + SpaceInString 
               + PercentInterpol 
               + Q "%" 
             ) ^ 0 
           * Q "'" )
%    \end{macrocode}
%
%    \begin{macrocode}
  local DoubleShortString =
    WithStyle ( 'String.Short.Internal' , 
           Q ( P "f\"" + "F\"" ) 
           * ( 
               K ( 'String.Interpol' , "{" )
                 * K ( 'Interpol.Inside' , ( 1 - S "}\":" ) ^ 0 )
                 * ( K ( 'String.Interpol' , ":" ) * Q ( (1 - S "}:\"") ^ 0 ) ) ^ -1
                 * K ( 'String.Interpol' , "}" )
               + 
               SpaceInString 
               + 
               Q ( ( P "\\\"" + "\\\\" + "{{" + "}}" + 1 - S " {}\"" ) ^ 1 ) 
              ) ^ 0
           * Q "\"" 
         +
           Q ( P "\"" + "r\"" + "R\"" ) 
           * ( Q ( ( P "\\\"" + "\\\\" + 1 - S " \"\r%" ) ^ 1 ) 
               + SpaceInString 
               + PercentInterpol 
               + Q "%"
             ) ^ 0 
           * Q "\""  )

  local ShortString = SingleShortString + DoubleShortString
%    \end{macrocode}
%
% \bigskip
% \paragraph{Beamer}
%
% The argument of |Compute_braces| must be a pattern \emph{which does no
% catching} corresponding to the strings of the language.
%
%    \begin{macrocode}
  local braces = 
    Compute_braces 
     (
         ( P "\"" + "r\"" + "R\"" + "f\"" + "F\"" ) 
             * ( P '\\\"' + 1 - S "\"" ) ^ 0 * "\""  
       + 
         ( P '\'' + 'r\'' + 'R\'' + 'f\'' + 'F\'' ) 
             * ( P '\\\'' + 1 - S '\'' ) ^ 0 * '\''  
     )

  if piton.beamer then Beamer = Compute_Beamer ( 'python' , braces ) end
%    \end{macrocode}
%
% \bigskip
% \paragraph{Detected commands}
%
%    \begin{macrocode}
  DetectedCommands = Compute_DetectedCommands ( 'python' , braces ) 
       + Compute_RawDetectedCommands ( 'python' , braces )
%    \end{macrocode}
%
% \bigskip
%
% \paragraph{LPEG_cleaner}
%
%    \begin{macrocode}
  LPEG_cleaner.python = Compute_LPEG_cleaner ( 'python' , braces ) 
%    \end{macrocode}
% 
% \bigskip
% \paragraph{The long strings}
% 
% 
%    \begin{macrocode}
  local SingleLongString =
    WithStyle ( 'String.Long.Internal' , 
       ( Q ( S "fF" * P "'''" )
           * (
               K ( 'String.Interpol' , "{" )
                 * K ( 'Interpol.Inside' , ( 1 - S "}:\r" - "'''" ) ^ 0  )
                 * Q ( P ":" * (1 - S "}:\r" - "'''" ) ^ 0 ) ^ -1
                 * K ( 'String.Interpol' , "}"  )
               + 
               Q ( ( 1 - P "'''" - S "{}'\r" ) ^ 1 )
               + 
               EOL
             ) ^ 0 
         +
           Q ( ( S "rR" ) ^ -1  * "'''" )
           * (
               Q ( ( 1 - P "'''" - S "\r%" ) ^ 1 )  
               + 
               PercentInterpol
               +
               P "%"
               +
               EOL
             ) ^ 0 
        )
        * Q "'''"  ) 
%    \end{macrocode}
%   
%    \begin{macrocode}
  local DoubleLongString =
    WithStyle ( 'String.Long.Internal' ,
       (
          Q ( S "fF" * "\"\"\"" )
          * (
              K ( 'String.Interpol', "{"  )
                * K ( 'Interpol.Inside' , ( 1 - S "}:\r" - "\"\"\"" ) ^ 0 )
                * Q ( ":" * (1 - S "}:\r" - "\"\"\"" ) ^ 0 ) ^ -1
                * K ( 'String.Interpol' , "}"  )
              + 
              Q ( ( 1 - S "{}\"\r" - "\"\"\"" ) ^ 1 ) 
              + 
              EOL
            ) ^ 0 
        +
          Q ( S "rR" ^ -1  * "\"\"\"" )
          * (
              Q ( ( 1 - P "\"\"\"" - S "%\r" ) ^ 1 )  
              + 
              PercentInterpol 
              + 
              P "%"
              + 
              EOL
            ) ^ 0 
       )
       * Q "\"\"\"" 
    ) 
%    \end{macrocode}
%
%    \begin{macrocode}
  local LongString = SingleLongString + DoubleLongString
%    \end{macrocode}
%
% \bigskip
% We have a \textsc{lpeg} for the Python docstrings. That \textsc{lpeg} will
% be used in the \textsc{lpeg} |DefFunction| which deals with the whole preamble
% of a function definition (which begins with |def|).
%    \begin{macrocode}
  local StringDoc = 
      K ( 'String.Doc.Internal' , P "r" ^ -1 * "\"\"\"" )
        * ( K ( 'String.Doc.Internal' , (1 - P "\"\"\"" - "\r" ) ^ 0  ) * EOL
            * Tab ^ 0 
          ) ^ 0
        * K ( 'String.Doc.Internal' , ( 1 - P "\"\"\"" - "\r" ) ^ 0 * "\"\"\"" )
%    \end{macrocode}
%
% \bigskip
% \paragraph{The comments in the Python listings}
%
% We define different \textsc{lpeg} dealing with comments in the Python
% listings.
%    \begin{macrocode}
  local Comment = 
    WithStyle 
     ( 'Comment.Internal' ,
       Q "#" * ( CommentMath + Q ( ( 1 - S "$\r" ) ^ 1 ) ) ^ 0  -- $
     ) 
    * ( EOL + -1 )
%    \end{macrocode}
% 
% 
% \bigskip
% \paragraph{DefFunction}
%
% The following \textsc{lpeg} |expression| will be used for the parameters in
% the \emph{argspec} of a Python function. It's necessary to use a \emph{grammar}
% because that pattern mainly checks the correct nesting of the delimiters
% (and it's known in the theory of formal languages that this can't be done with
% regular expressions \emph{stricto sensu} only). 
%    \begin{macrocode}
  local expression =
    P { "E" ,
         E = ( "'" * ( P "\\'" + 1 - S "'\r" ) ^ 0 * "'" 
               + "\"" * ( P "\\\"" + 1 - S "\"\r" ) ^ 0 * "\""
               + "{" * V "F" * "}"
               + "(" * V "F" * ")"
               + "[" * V "F" * "]" 
               + ( 1 - S "{}()[]\r," ) ) ^ 0 ,
         F = (   "{" * V "F" * "}"
               + "(" * V "F" * ")"
               + "[" * V "F" * "]"
               + ( 1 - S "{}()[]\r\"'" ) ) ^ 0
      }
%    \end{macrocode}
%
% \bigskip 
% We will now define a \textsc{lpeg} |Params| that will catch the list of
% parameters (that is to say the \emph{argspec}) in the definition of a Python
% function. For example, in the line of code
% \begin{center}
% \piton{def MyFunction(a,b,x=10,n:int): return n}
% \end{center}
% the \textsc{lpeg} |Params| will be used to catch the chunk\enskip |a,b,x=10,n:int|.
%
% \medskip
%    \begin{macrocode}
  local Params = 
    P { "E" , 
         E = ( V "F" * ( Q "," * V "F" ) ^ 0 ) ^ -1 ,
         F = SkipSpace * ( Identifier + Q "*args" + Q "**kwargs" ) * SkipSpace
             * ( 
                   K ( 'InitialValues' , "=" * expression )
                 + Q ":" * SkipSpace * K ( 'Name.Type' , identifier ) 
               ) ^ -1
      }
%    \end{macrocode}
% 
% 
% \bigskip
% The following \textsc{lpeg} |DefFunction| catches a keyword |def| and the
% following name of function \emph{but also everything else until a potential
% docstring}. That's why this definition of \textsc{lpeg} must occur (in the file
% |piton.sty|) after the definition of several other \textsc{lpeg} such as
% |Comment|, |CommentLaTeX|, |Params|, |StringDoc|...
%    \begin{macrocode}
  local DefFunction =
    K ( 'Keyword' , "def" )
    * Space
    * K ( 'Name.Function.Internal' , identifier ) 
    * SkipSpace 
    * Q "("  * Params * Q ")" 
    * SkipSpace
    * ( Q "->" * SkipSpace * K ( 'Name.Type' , identifier ) ) ^ -1
%    \end{macrocode}
% 
%    \begin{macrocode}
    * ( C ( ( 1 - S ":\r" ) ^ 0 ) / ParseAgain )
    * Q ":" 
    * ( SkipSpace
        * ( EOL + CommentLaTeX + Comment ) -- in all cases, that contains an EOL
        * Tab ^ 0 
        * SkipSpace
        * StringDoc ^ 0 -- there may be additional docstrings 
      ) ^ -1
%    \end{macrocode}
% Remark that, in the previous code, |CommentLaTeX| \emph{must} appear
% before |Comment|: there is no commutativity of the addition for the
% \emph{parsing expression grammars} (\textsc{peg}).
% 
% \smallskip 
% If the word |def| is not followed by an identifier and parenthesis, it will be
% caught as keyword by the \textsc{lpeg} |Keyword| (useful if, for example, the
% final user wants to speak of the keyword \piton{def}).
%
%
% \paragraph{Miscellaneous}
% 
%    \begin{macrocode}
  local ExceptionInConsole = Exception *  Q ( ( 1 - P "\r" ) ^ 0 ) * EOL 
%    \end{macrocode}
% 
% 
% \bigskip
% \paragraph{The main LPEG for the language Python}
%
%    \begin{macrocode}
  local EndKeyword 
    = Space + Punct + Delim + EOL + Beamer + DetectedCommands + Escape +
    EscapeMath + -1 
%    \end{macrocode}
%
% First, the main loop :
%    \begin{macrocode}
  local Main = 
       space ^ 0 * EOL -- faut-il le mettre en commentaire ?
       + Space 
       + Tab
       + Escape + EscapeMath
       + CommentLaTeX
       + Beamer
       + DetectedCommands 
       + LongString 
       + Comment
       + ExceptionInConsole
       + Delim
       + Operator
       + OperatorWord * EndKeyword
       + ShortString
       + Punct
       + FromImport
       + RaiseException 
       + DefFunction
       + DefClass 
       + For
       + Keyword * EndKeyword
       + Decorator
       + Builtin * EndKeyword
       + Identifier 
       + Number
       + Word
%    \end{macrocode}
%
% \bigskip
% Here, we must not put |local|, of course.
%    \begin{macrocode}
  LPEG1.python = Main ^ 0 
%    \end{macrocode}
%
% \bigskip
% We recall that each line in the Python code to parse will be sent back to
% LaTeX between a pair |\@@_begin_line:| -- |\@@_end_line:|\footnote{Remember
% that the \texttt{\textbackslash @@\_end\_line:} must be explicit because it
% will be used as marker in order to delimit the argument of the command
% \texttt{\textbackslash @@\_begin\_line:}}.
%    \begin{macrocode}
  LPEG2.python = 
    Ct (
         ( space ^ 0 * "\r" ) ^ -1 
         * beamerBeginEnvironments 
         * PromptHastyDetection
         * Lc [[ \@@_begin_line: ]]
         * Prompt 
         * SpaceIndentation ^ 0 
         * ( space ^ 1 * -1 + space ^ 0 * EOL + Main ) ^ 0 
         * -1 
         * Lc [[ \@@_end_line: ]] 
       )
%    \end{macrocode}
%
% \bigskip
% End of the Lua scope for the language Python.
%    \begin{macrocode}
end
%    \end{macrocode}
%
% \bigskip
% \subsubsection{The language Ocaml}
%
% We open a Lua local scope for the language OCaml (of course, there will be also
% global definitions).
%    \begin{macrocode}
--ocaml Ocaml OCaml
do 
%    \end{macrocode}
%
% \bigskip
%    \begin{macrocode}
  local SkipSpace = ( Q " " + EOL ) ^ 0
  local Space = ( Q " " + EOL ) ^ 1
%    \end{macrocode}
%
% \bigskip
%     \begin{macrocode}
  local braces = Compute_braces ( '\"' * ( 1 - S "\"" ) ^ 0 * '\"' )
%     \end{macrocode}
%
% \bigskip
%    \begin{macrocode}
  if piton.beamer then Beamer = Compute_Beamer ( 'ocaml' , braces ) end
  DetectedCommands = 
    Compute_DetectedCommands ( 'ocaml' , braces )
    + Compute_RawDetectedCommands ( 'ocaml' , braces )
  local Q
%    \end{macrocode}
% Usually, the following version of the function |Q| will be used without the
% second arguemnt (|strict|), that is to say in a loosy way. However, in some
% circunstancies, we will a need the ``strict'' version, for instance in
% |DefFunction|. 
%    \begin{macrocode}
  function Q ( pattern, strict ) 
    if strict ~= nil then
      return Ct ( Cc ( luatexbase.catcodetables.CatcodeTableOther ) * C ( pattern ) )
    else
      return Ct ( Cc ( luatexbase.catcodetables.CatcodeTableOther ) * C ( pattern ) )
          + Beamer + DetectedCommands + EscapeMath + Escape
    end     
  end
%    \end{macrocode}
%
% \bigskip
%    \begin{macrocode}
  local K
  function K ( style , pattern, strict ) return
    Lc ( [[ {\PitonStyle{ ]] .. style .. "}{" )
    * Q ( pattern, strict )
    * Lc "}}"
  end
%    \end{macrocode}
%
% \bigskip
%    \begin{macrocode}
  local WithStyle
  function WithStyle ( style , pattern ) return
      Ct ( Cc "Open" * Cc ( [[{\PitonStyle{]] .. style .. "}{" ) * Cc "}}" )
    * (pattern + Beamer + DetectedCommands + EscapeMath + Escape)
    * Ct ( Cc "Close" )
  end
%    \end{macrocode}
% 
% \bigskip
% The following LPEG corresponds to the balanced expressions (balanced according
% to the parenthesis). Of course, we must write |(1 - S "()")| with outer parenthesis.
%    \begin{macrocode}
  local balanced_parens =
    P { "E" , E = ( "(" * V "E" * ")" + ( 1 - S "()" ) ) ^ 0 }
%    \end{macrocode}
%
% \paragraph{The strings of OCaml}
%    \begin{macrocode}
  local ocaml_string =
    P "\""
  * (
      P " " 
      +
      P ( ( 1 - S " \"\r" ) ^ 1 )
      +
      EOL -- ?
    ) ^ 0
  * P "\""
%    \end{macrocode}
%
%    \begin{macrocode}
  local String = 
    WithStyle 
      ( 'String.Long.Internal' , 
          Q "\""
        * (
            SpaceInString
            +
            Q ( ( 1 - S " \"\r" ) ^ 1 )
            +
            EOL
          ) ^ 0
        * Q "\""
      )
%    \end{macrocode}
%
% \bigskip
% Now, the ``quoted strings'' of OCaml (for example \verb+{ext|Essai|ext}+). 
%
% For those strings, we will do two consecutive analysis. First an analysis to
% determine the whole string and, then, an analysis for the potential visual
% spaces and the EOL in the string.
% 
% The first analysis require a match-time capture. For explanations about that
% programmation, see the paragraphe \emph{Lua's long strings} in
% |www.inf.puc-rio.br/~roberto/lpeg|. 
%
%    \begin{macrocode}
  local ext = ( R "az" + "_" ) ^ 0
  local open = "{" * Cg ( ext , 'init' ) * "|"
  local close = "|" * C ( ext ) * "}"
  local closeeq =
    Cmt ( close * Cb ( 'init' ) ,
          function ( s , i , a , b ) return a == b end )
%    \end{macrocode}
%
% \medskip
% The \textsc{lpeg} |QuotedStringBis| will do the second analysis. 
%    \begin{macrocode}
  local QuotedStringBis =
    WithStyle ( 'String.Long.Internal' ,
        (
          Space
          +
          Q ( ( 1 - S " \r" ) ^ 1 )
          +
          EOL
        ) ^ 0  )
%    \end{macrocode}
% 
% \medskip
% We use a ``function capture'' (as called in the official documentation of the
% \textsc{lpeg}) in order to do the second analysis on the result of the first one.
%    \begin{macrocode}
  local QuotedString =
    C ( open * ( 1 - closeeq ) ^ 0  * close ) /
    ( function ( s ) return QuotedStringBis : match ( s ) end )
%    \end{macrocode}
%
% \bigskip
% In OCaml, the delimiters for the comments are |(*| and |*)|. There are
% unsymmetrical and OCaml allows those comments to be nested. That's why we need a
% grammar.
% 
% In these comments, we embed the math comments (between |$| and |$|) and we
% embed also a treatment for the end of lines (since the comments may be multi-lines).
% 
%    \begin{macrocode}
  local comment = 
      P {
          "A" ,
          A = Q "(*"
              * ( V "A"
                  + Q ( ( 1 - S "\r$\"" - "(*" - "*)" ) ^ 1 ) -- $
                  + ocaml_string
                  + "$" * K ( 'Comment.Math' , ( 1 - S "$\r" ) ^ 1 ) * "$" -- $
                  + EOL
                ) ^ 0
              * Q "*)"
        }   
  local Comment = WithStyle ( 'Comment.Internal' , comment ) 
%    \end{macrocode}
%
% 
% \paragraph{Some standard LPEG}
%
%    \begin{macrocode}
  local Delim = Q ( P "[|" + "|]" + S "[()]" )
  local Punct = Q ( S ",:;!" )
%    \end{macrocode}
%
% \bigskip
% The identifiers caught by |cap_identifier| begin with a capital. In OCaml,
% it's used for the constructors of types and for the names of the modules.
%
%    \begin{macrocode}
  local cap_identifier = R "AZ" * ( R "az" + R "AZ" + S "_'" + digit ) ^ 0
%    \end{macrocode}
%
% \bigskip
%    \begin{macrocode}
  local Constructor = 
    K ( 'Name.Constructor' , 
        Q "`" ^ -1 * cap_identifier 
%    \end{macrocode}
% We consider |::|  and |[]| as constructors (of the lists) as does the Tuareg
% mode of Emacs.
%    \begin{macrocode}
        + Q "::" 
        + Q ( "[" , true ) * SkipSpace * Q ( "]" , true) )
%    \end{macrocode}
%
% \bigskip
%    \begin{macrocode}
  local ModuleType = K ( 'Name.Type' , cap_identifier )
%    \end{macrocode}
%
% \bigskip
%    \begin{macrocode}
  local OperatorWord =
    K ( 'Operator.Word' ,
        P "asr" + "land" + "lor" + "lsl" + "lxor" + "mod" + "or" + "not" )
%    \end{macrocode}
% 
% \bigskip
% In OCaml, some keywords are considered as \emph{governing keywords} with some
% special syntactic characteristics.
%    \begin{macrocode}
  local governing_keyword = P "and" + "begin" + "class" + "constraint" +
        "end" + "external" + "functor" + "include" + "inherit" + "initializer" +
        "in" + "let" + "method" + "module" + "object" + "open" + "rec" + "sig" +
        "struct" + "type" + "val"
%    \end{macrocode}
%   
% \bigskip
%    \begin{macrocode}
  local Keyword =
    K ( 'Keyword' ,
        P "assert" + "as" + "done" + "downto" + "do" + "else" + "exception"
        + "for" + "function"  + "fun" + "if" + "lazy" + "match" + "mutable"
        + "new" + "of" + "private" + "raise" + "then" + "to" + "try"
        + "virtual" + "when" + "while" + "with" )
    + K ( 'Keyword.Constant' , P "true" + "false" )
    + K ( 'Keyword.Governing', governing_keyword )
%    \end{macrocode}
%
% \bigskip
%    \begin{macrocode}
  local EndKeyword 
    = Space + Punct + Delim + EOL + Beamer + DetectedCommands + Escape 
       + EscapeMath + -1 
%    \end{macrocode}
%
% \bigskip
% Now, the identifier. Recall that we have also a LPEG |cap_identifier| for the
% indentifiers beginning with a capital letter.
%    \begin{macrocode}
  local identifier = ( R "az" + "_" ) * ( R "az" + R "AZ" + S "_'" + digit ) ^ 0 
                     - ( OperatorWord + Keyword ) * EndKeyword
%    \end{macrocode}
%
% \bigskip
% We have the internal style |Identifier.Internal| in order to be able to
% implement the mechanism |\SetPitonIdentifier|. The final user has access to a
% style called |Identifier|.
%    \begin{macrocode}
  local Identifier = K ( 'Identifier.Internal' , identifier )
%    \end{macrocode}
%
% \bigskip
% In OCmal, \emph{character} is a type different of the type |string|.
%    \begin{macrocode}
  local ocaml_char = 
      P "'" *
      (
        ( 1 - S "'\\" )
        + "\\"
          * ( S "\\'ntbr \""
              + digit * digit * digit
              + P "x" * ( digit + R "af" + R "AF" )
                      * ( digit + R "af" + R "AF" )
                      * ( digit + R "af" + R "AF" )
              + P "o" * R "03" * R "07" * R "07" )
      )
      * "'"
  local Char =
    K ( 'String.Short.Internal', ocaml_char ) 
%    \end{macrocode}
%
% \bigskip
% For the parameter of the types (for example : |`\a| as in |`a list|).
%    \begin{macrocode}
  local TypeParameter =
    K ( 'TypeParameter' ,
        "'" * Q "_" ^ -1 * alpha ^ 1 * digit ^ 0 * ( # ( 1 - P "'" ) + -1 ) )
%    \end{macrocode}
%
% \paragraph{DotNotation}
% 
% Now, we deal with the notations with points (eg: |List.length|). In OCaml,
% such notation is used for the fields of the records and for the modules.
%    \begin{macrocode}
local DotNotation =
    (
        K ( 'Name.Module' , cap_identifier )
          * Q "."
          * ( Identifier + Constructor + Q "(" + Q "[" + Q "{" ) ^ -1
        +
         Identifier
          * Q "."
          * K ( 'Name.Field' , identifier )
    )
    * ( Q "." * K ( 'Name.Field' , identifier ) ) ^ 0    
%    \end{macrocode}
%
% \paragraph{The records}
%
%    \begin{macrocode}
  local expression_for_fields_type =
    P { "E" ,
        E =  (  "{" * V "F" * "}"
              + "(" * V "F" * ")"
              + TypeParameter
              + ( 1 - S "{}()[]\r;" ) ) ^ 0 ,
        F = (    "{" * V "F" * "}"
              + "(" * V "F" * ")"
              + ( 1 - S "{}()[]\r\"'" ) + TypeParameter ) ^ 0
      }
%    \end{macrocode}
%
% \bigskip
%    \begin{macrocode}
  local expression_for_fields_value =
    P { "E" ,
        E =  (   "{" * V "F" * "}"
              + "(" * V "F" * ")"
              + "[" * V "F" * "]"
              + ocaml_string + ocaml_char
              + ( 1 - S "{}()[];" ) ) ^ 0 ,
        F = (    "{" * V "F" * "}"
              + "(" * V "F" * ")"
              + "[" * V "F" * "]"
              + ocaml_string + ocaml_char
              + ( 1 - S "{}()[]\"'" )) ^ 0
      }
%    \end{macrocode}
%
% \bigskip
%    \begin{macrocode}
  local OneFieldDefinition =
      ( K ( 'Keyword' , "mutable" ) * SkipSpace ) ^ -1
    * K ( 'Name.Field' , identifier ) * SkipSpace
    * Q ":" * SkipSpace
    * K ( 'TypeExpression' , expression_for_fields_type )
    * SkipSpace
%    \end{macrocode}
%
% \bigskip
%    \begin{macrocode}
  local OneField =
      K ( 'Name.Field' , identifier ) * SkipSpace
    * Q "=" * SkipSpace
%    \end{macrocode}
% Don't forget the parentheses!
%    \begin{macrocode}
    * ( C ( expression_for_fields_value ) / ParseAgain ) 
    * SkipSpace
%    \end{macrocode}
%
% \bigskip
% The \emph{records}.
%    \begin{macrocode}
  local RecordVal =
    Q "{" * SkipSpace
    *
      (
        (Identifier + DotNotation) * Space * K('Keyword', "with") * Space
      ) ^-1
    *
      (
        OneField * ( Q ";" * SkipSpace * ( Comment * SkipSpace ) ^ 0 * OneField ) ^ 0
      )
    * SkipSpace
    * Q ";" ^ -1
    * SkipSpace
    * Comment ^ -1
    * SkipSpace
    * Q "}"
  local RecordType =
    Q "{" * SkipSpace
    *
      (
        OneFieldDefinition
        * ( Q ";" * SkipSpace * ( Comment * SkipSpace ) ^ 0 * OneFieldDefinition ) ^ 0
      )
    * SkipSpace
    * Q ";" ^ -1
    * SkipSpace
    * Comment ^ -1
    * SkipSpace
    * Q "}"
  local Record = RecordType + RecordVal
%    \end{macrocode}
%
% \paragraph{DotNotation}
%
% Now, we deal with the notations with points (eg: |List.length|). In OCaml,
% such notation is used for the fields of the records and for the modules.
%    \begin{macrocode}
  local DotNotation =
    (
        K ( 'Name.Module' , cap_identifier )
          * Q "."
          * ( Identifier + Constructor + Q "(" + Q "[" + Q "{" ) ^ -1 
        +
         Identifier
          * Q "."
          * K ( 'Name.Field' , identifier )
    )
    * ( Q "." * K ( 'Name.Field' , identifier ) ) ^ 0
%    \end{macrocode}
%
% \bigskip
%    \begin{macrocode}
  local Operator =
    K ( 'Operator' ,
        P "!=" + "<>" + "==" + "<<" + ">>" + "<=" + ">=" + ":=" + "||" + "&&" +
        "//" + "**" + ";;" + "->" + "+." + "-." + "*." + "/."
        + S "-~+/*%=<>&@|" )
%    \end{macrocode}
%
% \bigskip
%    \begin{macrocode}
  local Builtin =
    K ( 'Name.Builtin' , P "incr" + "decr" + "fst" + "snd" + "ref" )
%    \end{macrocode}
%
% \bigskip
%    \begin{macrocode}
  local Exception =
    K (   'Exception' ,
        P "Division_by_zero" + "End_of_File" + "Failure" + "Invalid_argument" +
        "Match_failure" + "Not_found" + "Out_of_memory" + "Stack_overflow" +
        "Sys_blocked_io" + "Sys_error" + "Undefined_recursive_module" )
%    \end{macrocode}
%
% \bigskip
%    \begin{macrocode}
  LPEG_cleaner.ocaml = Compute_LPEG_cleaner ( 'ocaml' , braces )
%    \end{macrocode}
%
% \bigskip
% An argument in the definition of a OCaml function may be of the form
% |(pattern:type)|. |pattern| may be a single identifier but it's not mandatory.
% First instance, it's possible to write in OCaml:
%
% |let head (a::q) = a|
%
% First, we write a pattern (in the LPEG sens!) to match what will be the
% pattern (in the OCaml sens).
%    \begin{macrocode}
  local pattern_part = 
    ( P "(" * balanced_parens * ")" + ( 1 - S ":()" ) + P "::" ) ^ 0 
%    \end{macrocode}
% For the ``type'' part, the LPEG-pattern will merely be |balanced_parens|.
% 
% \bigskip
% We can now write a LPEG |Argument| which catches a argument of function (in
% the definition of the function).
%    \begin{macrocode}
  local Argument =
%    \end{macrocode}
% The following line is for the labels of the labeled arguments. Maybe we will,
% in the future, create a style for those elements.
%    \begin{macrocode}
    (  Q "~" * Identifier * Q ":" * SkipSpace ) ^ -1
    *
%    \end{macrocode}
% Now, the argument itself, either a single identifier, or a construction
% between parentheses
%    \begin{macrocode}
    ( 
        K ( 'Identifier.Internal' , identifier )
      + 
        Q "(" * SkipSpace
        * ( C ( pattern_part ) / ParseAgain ) 
        * SkipSpace
%    \end{macrocode}
% Of course, the specification of type is optional.
%    \begin{macrocode}
       * ( Q ":" * #(1- P"=")
           * K ( 'TypeExpression' , balanced_parens ) * SkipSpace
         ) ^ -1
        * Q ")"
    )
%    \end{macrocode}
%
% \bigskip
% Despite its name, then \textsc{lpeg} |DefFunction| deals also with |let open|
% which opens locally a module.
%    \begin{macrocode}
  local DefFunction =
    K ( 'Keyword.Governing' , "let open" )
    * Space
    * K ( 'Name.Module' , cap_identifier )
    +
    K ( 'Keyword.Governing' , P "let rec" + "let" + "and" )
      * Space
      * K ( 'Name.Function.Internal' , identifier )
      * Space
      * (
%    \end{macrocode}
% You use here the argument |strict| in order to allow a correct analyse of 
% |let x = \uncover<2->{y}| (elsewhere, it's interpreted as a definition of a OCaml
% function). 
%    \begin{macrocode}
          Q "=" * SkipSpace * K ( 'Keyword' , "function" , true )
          +
          Argument * ( SkipSpace * Argument ) ^ 0
          * (
              SkipSpace
              * Q ":" * # ( 1 - P "=" )
              * K ( 'TypeExpression' , ( 1 - P "=" ) ^ 0 )
            ) ^ -1
        )
%    \end{macrocode}
%
% \paragraph{DefModule}
%
%    \begin{macrocode}
  local DefModule =
    K ( 'Keyword.Governing' , "module" ) * Space
    *
      (
            K ( 'Keyword.Governing' , "type" ) * Space
          * K ( 'Name.Type' , cap_identifier )
        +
          K ( 'Name.Module' , cap_identifier ) * SkipSpace
          *
            (
              Q "(" * SkipSpace
                * K ( 'Name.Module' , cap_identifier ) * SkipSpace
                * Q ":" * # ( 1 - P "=" ) * SkipSpace
                * K ( 'Name.Type' , cap_identifier ) * SkipSpace
                *
                  (
                    Q "," * SkipSpace
                      * K ( 'Name.Module' , cap_identifier ) * SkipSpace
                      * Q ":" * # ( 1 - P "=" ) * SkipSpace
                      * K ( 'Name.Type' , cap_identifier ) * SkipSpace
                  ) ^ 0
                * Q ")"
            ) ^ -1
          *
            (
              Q "=" * SkipSpace
              * K ( 'Name.Module' , cap_identifier )  * SkipSpace
              * Q "("
              * K ( 'Name.Module' , cap_identifier ) * SkipSpace
                *
                (
                  Q ","
                  *
                  K ( 'Name.Module' , cap_identifier ) * SkipSpace
                ) ^ 0
              * Q ")"
            ) ^ -1
      )
    +
    K ( 'Keyword.Governing' , P "include" + "open" )
    * Space 
    * K ( 'Name.Module' , cap_identifier )
%    \end{macrocode}
%
% \paragraph{DefType}
%
%    \begin{macrocode}
  local DefType =
    K ( 'Keyword.Governing' , "type" )
    * Space
    * K ( 'TypeExpression' , Q ( 1 - P "=" - P "+=" ) ^ 1 )
    * SkipSpace
    * ( Q "+=" + Q "=" )
    * SkipSpace
    * (
        RecordType
        +
%    \end{macrocode}
% The following lines are a suggestion of Y. Salmon. 
%    \begin{macrocode}
        WithStyle
         (
           'TypeExpression' ,
           (
             (  
               EOL 
               + comment 
               +  Q ( 1 
                      - P ";;"
                      - P "type"
                      - ( ( Space + EOL ) * governing_keyword * EndKeyword )
                    ) 
             ) ^ 0  
             * 
             ( 
               # ( P "type" + ( Space + EOL ) * governing_keyword * EndKeyword )
               + Q ";;"
               + -1 
             ) 
           )
         )
      )
%    \end{macrocode}
%
% \bigskip
%    \begin{macrocode}
  local prompt =
    Q "utop[" * digit^1 * Q "]> "
  local start_of_line = P(function(subject, position)
  if position == 1 or subject:sub(position - 1, position - 1) == "\r" then
    return position
  end
  return nil
end)
  local Prompt = #start_of_line * K( 'Prompt', prompt)
  local Answer = #start_of_line * (Q"-" + Q "val" * Space * Identifier )
                 * SkipSpace * Q ":" * #(1- P"=") * SkipSpace
                 * (K ( 'TypeExpression' , Q ( 1 - P "=") ^ 1 ) ) * SkipSpace * Q "="
%    \end{macrocode}
% 
% \bigskip
% \paragraph{The main LPEG for the language OCaml}
%
%    \begin{macrocode}
  local Main =
      space ^ 0 * EOL
      + Space
      + Tab
      + Escape + EscapeMath
      + Beamer
      + DetectedCommands
      + TypeParameter
      + String + QuotedString + Char
      + Comment
      + Prompt + Answer 
%    \end{macrocode}
% For the labels (maybe we will write in the future a dedicated LPEG pour those tokens).
%    \begin{macrocode}
      + Q "~" * Identifier * ( Q ":" ) ^ -1
      + Q ":" * # (1 - P ":") * SkipSpace
          * K ( 'TypeExpression' , balanced_parens ) * SkipSpace * Q ")"
      + Exception
      + DefType
      + DefFunction
      + DefModule
      + Record
      + Keyword * EndKeyword
      + OperatorWord * EndKeyword
      + Builtin * EndKeyword
      + DotNotation
      + Constructor
      + Identifier
      + Punct
      + Delim -- Delim is before Operator for a correct analysis of [| et |]
      + Operator
      + Number
      + Word
%    \end{macrocode}
%
% \bigskip
% % Here, we must not put |local|, of course.
%    \begin{macrocode}
  LPEG1.ocaml = Main ^ 0
%    \end{macrocode}
%
% \bigskip
%    \begin{macrocode}
  LPEG2.ocaml =
    Ct (
%    \end{macrocode}
% The following lines are in order to allow, in |\piton| (and not in |{Piton}|),
% judgments of type (such as |f : my_type -> 'a list|) or single expressions of
% type such as |my_type -> 'a list| (in that case, the argument of |\piton|
% \emph{must} begin by a colon).
%    \begin{macrocode}
        ( P ":" + (K ( 'Name.Module' , cap_identifier ) * Q ".") ^-1
          * Identifier * SkipSpace * Q ":" )
          * # ( 1 - S ":=" )
          * SkipSpace
          * K ( 'TypeExpression' , ( 1 - P "\r" ) ^ 0 )
        +
        ( space ^ 0 * "\r" ) ^ -1
        * beamerBeginEnvironments
        * Lc [[ \@@_begin_line: ]]
        * SpaceIndentation ^ 0
        * ( ( space * Lc [[ \@@_trailing_space: ]] ) ^ 1 * -1 
              + space ^ 0 * EOL 
              + Main 
          ) ^ 0
        * -1
        * Lc [[ \@@_end_line: ]]
      )
%    \end{macrocode}
% 
% \bigskip
% End of the Lua scope for the language OCaml.
%    \begin{macrocode}
end
%    \end{macrocode}
% 
% 
% \bigskip
% \subsubsection{The language C}
% 
% We open a Lua local scope for the language C (of course, there will be also
% global definitions).
%    \begin{macrocode}
--c C c++ C++
do 
%    \end{macrocode}
%
% \bigskip
%    \begin{macrocode}
  local Delim = Q ( S "{[()]}" )
%    \end{macrocode}
%
%    \begin{macrocode}
  local Punct = Q ( S ",:;!" )
%    \end{macrocode}
%
% \bigskip
% Some strings of length 2 are explicit because we want the corresponding
% ligatures available in some fonts such as \emph{Fira Code} to be active.
%    \begin{macrocode}
  local identifier = letter * alphanum ^ 0

  local Operator = 
    K ( 'Operator' ,
        P "!=" + "==" + "<<" + ">>" + "<=" + ">=" + "||" + "&&" 
          + S "-~+/*%=<>&.@|!" )

  local Keyword = 
    K ( 'Keyword' ,
        P "alignas" + "asm" + "auto" + "break" + "case" + "catch" + "class" +
        "const" + "constexpr" + "continue" + "decltype" + "do" + "else" + "enum" +
        "extern" + "for" + "goto" + "if" + "nexcept" + "private" + "public" +
        "register" + "restricted" + "return" + "static" + "static_assert" +
        "struct" + "switch" + "thread_local" + "throw" + "try" + "typedef" +
        "union" + "using" + "virtual" + "volatile" + "while" 
      )
    + K ( 'Keyword.Constant' , P "default" + "false" + "NULL" + "nullptr" + "true" ) 

  local Builtin = 
    K ( 'Name.Builtin' ,
        P "alignof" + "malloc" + "printf" + "scanf" + "sizeof" )

  local Type = 
    K ( 'Name.Type' ,
        P "bool" + "char" + "char16_t" + "char32_t" + "double" + "float" + "int" +
        "int8_t" + "int16_t" + "int32_t" + "int64_t" + "long" + "short" + "signed"
        + "unsigned" + "void" + "wchar_t" ) * Q "*" ^ 0 

  local DefFunction = 
    Type 
    * Space 
    * Q "*" ^ -1
    * K ( 'Name.Function.Internal' , identifier ) 
    * SkipSpace 
    * # P "("
%    \end{macrocode}
% We remind that the marker |#| of \textsc{lpeg} specifies that the pattern will be
% detected but won't consume any character.
%
% \bigskip
% The following \textsc{lpeg} |DefClass| will be used to detect the definition of a
% new class (the name of that new class will be formatted with the \pkg{piton}
% style |Name.Class|). 
%
% \smallskip
% Example:\enskip \piton{class myclass:}
%    \begin{macrocode}
  local DefClass = 
    K ( 'Keyword' , "class" ) * Space * K ( 'Name.Class' , identifier ) 
%    \end{macrocode}
% 
% If the word |class| is not followed by a identifier, it will be caught as
% keyword by the \textsc{lpeg} |Keyword| (useful if we want to type a
% list of keywords).
%
% \bigskip
%    \begin{macrocode}
  local Character =
    K ( 'String.Short' ,
        P [['\'']] + P "'" * ( 1 - P "'" ) ^ 0 * P "'" )  
%    \end{macrocode}
% 
% \bigskip
% \paragraph{The strings of C}
%
%    \begin{macrocode}
  String = 
    WithStyle ( 'String.Long.Internal' ,
        Q "\"" 
        * ( SpaceInString 
            + K ( 'String.Interpol' , 
                  "%" * ( S "difcspxXou" + "ld" + "li" + "hd" + "hi" )
                ) 
            + Q ( ( P "\\\"" + 1 - S " \"" ) ^ 1 ) 
          ) ^ 0 
        * Q "\""
      )
%    \end{macrocode}
% 
% \bigskip
% \paragraph{Beamer}
%
% \bigskip
% The argument of |Compute_braces| must be a pattern \emph{which does no
% catching} corresponding to the strings of the language.
%    \begin{macrocode}
  local braces = Compute_braces ( "\"" * ( 1 - S "\"" ) ^ 0 * "\"" ) 
  if piton.beamer then Beamer = Compute_Beamer ( 'c' , braces ) end
%    \end{macrocode}
%
%    \begin{macrocode}
  DetectedCommands = 
    Compute_DetectedCommands ( 'c' , braces ) 
    + Compute_RawDetectedCommands ( 'c' , braces )
%    \end{macrocode}
%
%    \begin{macrocode}
  LPEG_cleaner.c = Compute_LPEG_cleaner ( 'c' , braces ) 
%    \end{macrocode}
% 
% \bigskip
% \paragraph{The directives of the preprocessor}
%
%    \begin{macrocode}
  local Preproc = K ( 'Preproc' , "#" * ( 1 - P "\r" ) ^ 0  ) * ( EOL + -1 )
%    \end{macrocode}
% 
%
% \bigskip
% \paragraph{The comments in the C listings}
%
% We define different \textsc{lpeg} dealing with comments in the C listings.
%    \begin{macrocode}
  local Comment = 
    WithStyle ( 'Comment.Internal' ,
       Q "//" * ( CommentMath + Q ( ( 1 - S "$\r" ) ^ 1 ) ) ^ 0 ) -- $
              * ( EOL + -1 )

  local LongComment = 
    WithStyle ( 'Comment.Internal' , 
                 Q "/*" 
                 * ( CommentMath + Q ( ( 1 - P "*/" - S "$\r" ) ^ 1 ) + EOL ) ^ 0 
                 * Q "*/" 
              ) -- $
%    \end{macrocode}
%
%
%
% 
% \bigskip
% \paragraph{The main LPEG for the language C}
%
%    \begin{macrocode}
  local EndKeyword 
    = Space + Punct + Delim + EOL + Beamer + DetectedCommands + Escape +
    EscapeMath  + -1 
%    \end{macrocode}
%
% First, the main loop :
%    \begin{macrocode}
  local Main = 
       space ^ 0 * EOL
       + Space 
       + Tab
       + Escape + EscapeMath 
       + CommentLaTeX
       + Beamer
       + DetectedCommands
       + Preproc
       + Comment + LongComment
       + Delim
       + Operator
       + Character
       + String
       + Punct
       + DefFunction
       + DefClass 
       + Type * ( Q "*" ^ -1 + EndKeyword ) 
       + Keyword * EndKeyword
       + Builtin * EndKeyword
       + Identifier 
       + Number
       + Word
%    \end{macrocode}
%
% \bigskip
% Here, we must not put |local|, of course.
%    \begin{macrocode}
  LPEG1.c = Main ^ 0 
%    \end{macrocode}
%
% \bigskip
% We recall that each line in the C code to parse will be sent back to
% LaTeX between a pair |\@@_begin_line:| -- |\@@_end_line:|\footnote{Remember
% that the \texttt{\textbackslash @@\_end\_line:} must be explicit because it
% will be used as marker in order to delimit the argument of the command
% \texttt{\textbackslash @@\_begin\_line:}}.
%    \begin{macrocode}
  LPEG2.c = 
    Ct (
         ( space ^ 0 * P "\r" ) ^ -1 
         * beamerBeginEnvironments 
         * Lc [[ \@@_begin_line: ]]
         * SpaceIndentation ^ 0 
         * ( space ^ 1 * -1 + space ^ 0 * EOL + Main ) ^ 0 
         * -1 
         * Lc [[ \@@_end_line: ]] 
       )
%    \end{macrocode}
%
% \bigskip
% End of the Lua scope for the language C.
%    \begin{macrocode}
end
%    \end{macrocode}
%
% \bigskip
% \subsubsection{The language SQL}
% 
% We open a Lua local scope for the language SQL (of course, there will be also
% global definitions).
%    \begin{macrocode}
--sql SQL
do 
%    \end{macrocode}
%
% \bigskip
%    \begin{macrocode}
  local LuaKeyword
  function LuaKeyword ( name ) return 
    Lc [[ {\PitonStyle{Keyword}{ ]] 
    * Q ( Cmt ( 
                C ( letter * alphanum ^ 0 ) , 
                function ( s , i , a ) return string.upper ( a ) == name end 
              ) 
        ) 
    * Lc "}}" 
  end 
%    \end{macrocode}
%
% \bigskip
% In the identifiers, we will be able to catch those contening spaces, that is
% to say like |"last name"|. 
%    \begin{macrocode}
  local identifier = 
    letter * ( alphanum + "-" ) ^ 0 
    + P '"' * ( ( 1 - P '"' ) ^ 1 ) * '"' 
%    \end{macrocode}
% 
%    \begin{macrocode}
  local Operator = 
    K ( 'Operator' , P "=" + "!=" + "<>" + ">=" + ">" + "<=" + "<"  + S "*+/" )
%    \end{macrocode}
%
% In SQL, the keywords are case-insensitive. That's why we have a little
% complication. We will catch the keywords with the identifiers and, then,
% distinguish the keywords with a Lua function. However, some keywords will be
% caught in special LPEG because we want to detect the names of the SQL tables.
% 
% \bigskip
% The following function converts a comma-separated list in a ``set'', that is
% to say a Lua table with a fast way to test whether a string belongs to that
% set (eventually, the indexation of the components of the table is no longer
% done by integers but by the strings themselves). 
%    \begin{macrocode}
  local Set
  function Set ( list )
    local set = { }
    for _ , l in ipairs ( list ) do set[l] = true end
    return set
  end
%    \end{macrocode}
% 
% \bigskip
% We now use the previous function |Set| to creates the ``sets'' |set_keywords|
% and |set_builtin|. That list of keywords comes from
% \url{https://sqlite.org/lang_keywords.html}.
%    \begin{macrocode}
  local set_keywords = Set
   { 
     "ABORT", "ACTION", "ADD", "AFTER", "ALL", "ALTER", "ALWAYS", "ANALYZE",
     "AND", "AS", "ASC", "ATTACH", "AUTOINCREMENT", "BEFORE", "BEGIN", "BETWEEN",
     "BY", "CASCADE", "CASE", "CAST", "CHECK", "COLLATE", "COLUMN", "COMMIT",
     "CONFLICT", "CONSTRAINT", "CREATE", "CROSS", "CURRENT", "CURRENT_DATE",
     "CURRENT_TIME", "CURRENT_TIMESTAMP", "DATABASE", "DEFAULT", "DEFERRABLE",
     "DEFERRED", "DELETE", "DESC", "DETACH", "DISTINCT", "DO", "DROP", "EACH",
     "ELSE", "END", "ESCAPE", "EXCEPT", "EXCLUDE", "EXCLUSIVE", "EXISTS",
     "EXPLAIN", "FAIL", "FILTER", "FIRST", "FOLLOWING", "FOR", "FOREIGN", "FROM",
     "FULL", "GENERATED", "GLOB", "GROUP", "GROUPS", "HAVING", "IF", "IGNORE",
     "IMMEDIATE", "IN", "INDEX", "INDEXED", "INITIALLY", "INNER", "INSERT",
     "INSTEAD", "INTERSECT", "INTO", "IS", "ISNULL", "JOIN", "KEY", "LAST",
     "LEFT", "LIKE", "LIMIT", "MATCH", "MATERIALIZED", "NATURAL", "NO", "NOT",
     "NOTHING", "NOTNULL", "NULL", "NULLS", "OF", "OFFSET", "ON", "OR", "ORDER",
     "OTHERS", "OUTER", "OVER", "PARTITION", "PLAN", "PRAGMA", "PRECEDING",
     "PRIMARY", "QUERY", "RAISE", "RANGE", "RECURSIVE", "REFERENCES", "REGEXP",
     "REINDEX", "RELEASE", "RENAME", "REPLACE", "RESTRICT", "RETURNING", "RIGHT",
     "ROLLBACK", "ROW", "ROWS", "SAVEPOINT", "SELECT", "SET", "TABLE", "TEMP",
     "TEMPORARY", "THEN", "TIES", "TO", "TRANSACTION", "TRIGGER", "UNBOUNDED",
     "UNION", "UNIQUE", "UPDATE", "USING", "VACUUM", "VALUES", "VIEW", "VIRTUAL",
     "WHEN", "WHERE", "WINDOW", "WITH", "WITHOUT"
   }
%    \end{macrocode}
% 
%    \begin{macrocode}
  local set_builtins = Set
   { 
     "AVG" , "COUNT" , "CHAR_LENGTH" , "CONCAT" , "CURDATE" , "CURRENT_DATE" ,
     "DATE_FORMAT" , "DAY" , "LOWER" , "LTRIM" , "MAX" , "MIN" , "MONTH" , "NOW" ,
     "RANK" , "ROUND" , "RTRIM" , "SUBSTRING" , "SUM" , "UPPER" , "YEAR" 
   }
%    \end{macrocode}
%
% The \textsc{lpeg} |Identifier| will catch the identifiers of the fields  
% but also the keywords and the built-in functions of SQL. If will \emph{not}
% catch the names of the SQL tables.
%    \begin{macrocode}
  local Identifier = 
    C ( identifier ) /
    ( 
      function ( s ) 
          if set_keywords[string.upper(s)] then return
%    \end{macrocode}
% Remind that, in Lua, it's possible to return \emph{several} values. 
%    \begin{macrocode}
            { [[{\PitonStyle{Keyword}{]] } ,
            { luatexbase.catcodetables.other , s } ,
            { "}}" }
          else 
            if set_builtins[string.upper(s)] then return
              { [[{\PitonStyle{Name.Builtin}{]] } ,
              { luatexbase.catcodetables.other , s } ,
              { "}}" }
            else return
              { [[{\PitonStyle{Name.Field}{]] } ,
              { luatexbase.catcodetables.other , s } ,
              { "}}" }
            end
          end 
      end
    ) 
%    \end{macrocode}
% 
% \bigskip
% \paragraph{The strings of SQL}
%
%    \begin{macrocode}
  local String = K ( 'String.Long.Internal' , "'" * ( 1 - P "'" ) ^ 1 * "'" ) 
%    \end{macrocode}
% 
% \bigskip
% \paragraph{Beamer}
%
% \bigskip
% The argument of |Compute_braces| must be a pattern \emph{which does no
% catching} corresponding to the strings of the language.
%
%    \begin{macrocode}
  local braces = Compute_braces ( "'" * ( 1 - P "'" ) ^ 1 * "'" ) 
  if piton.beamer then Beamer = Compute_Beamer ( 'sql' , braces ) end
%    \end{macrocode}
% 
%    \begin{macrocode}
  DetectedCommands = 
    Compute_DetectedCommands ( 'sql' , braces ) 
    + Compute_RawDetectedCommands ( 'sql' , braces )
%    \end{macrocode}
%
%    \begin{macrocode}
  LPEG_cleaner.sql = Compute_LPEG_cleaner ( 'sql' , braces ) 
%    \end{macrocode}
%
% 
%
% \bigskip
% \paragraph{The comments in the SQL listings}
%
% We define different \textsc{lpeg} dealing with comments in the SQL listings.
%    \begin{macrocode}
  local Comment = 
    WithStyle ( 'Comment.Internal' ,
       Q "--"   -- syntax of SQL92
       * ( CommentMath + Q ( ( 1 - S "$\r" ) ^ 1 ) ) ^ 0 ) -- $
    * ( EOL + -1 )

  local LongComment = 
    WithStyle ( 'Comment.Internal' , 
                 Q "/*" 
                 * ( CommentMath + Q ( ( 1 - P "*/" - S "$\r" ) ^ 1 ) + EOL ) ^ 0 
                 * Q "*/" 
              ) -- $
%    \end{macrocode}
%
%
% 
% \bigskip
% \paragraph{The main LPEG for the language SQL}
%
%    \begin{macrocode}
  local EndKeyword 
    = Space + Punct + Delim + EOL + Beamer + DetectedCommands + Escape + 
      EscapeMath + -1 
%    \end{macrocode}
%
%    \begin{macrocode}
  local TableField = 
         K ( 'Name.Table' , identifier ) 
       * Q "." 
       * ( DetectedCommands + ( K ( 'Name.Field' , identifier ) ) ^ 0 )

  local OneField = 
    ( 
      Q ( "(" * ( 1 - P ")" ) ^ 0 * ")" )
      + 
          K ( 'Name.Table' , identifier ) 
        * Q "." 
        * K ( 'Name.Field' , identifier ) 
      + 
      K ( 'Name.Field' , identifier ) 
    )
    * ( 
        Space * LuaKeyword "AS" * Space * K ( 'Name.Field' , identifier ) 
      ) ^ -1
    * ( Space * ( LuaKeyword "ASC" + LuaKeyword "DESC" ) ) ^ -1

  local OneTable = 
       K ( 'Name.Table' , identifier ) 
     * ( 
         Space 
         * LuaKeyword "AS"  
         * Space 
         * K ( 'Name.Table' , identifier ) 
       ) ^ -1 

  local WeCatchTableNames = 
       LuaKeyword "FROM" 
     * ( Space + EOL ) 
     * OneTable * ( SkipSpace * Q "," * SkipSpace * OneTable ) ^ 0 
    + ( 
        LuaKeyword "JOIN" + LuaKeyword "INTO" + LuaKeyword "UPDATE" 
        + LuaKeyword "TABLE" 
      ) 
      * ( Space + EOL ) * OneTable 
%    \end{macrocode}
% 
%    \begin{macrocode}
  local EndKeyword 
    = Space + Punct + Delim + EOL + Beamer 
        + DetectedCommands + Escape + EscapeMath + -1 
%    \end{macrocode}
%
%
% First, the main loop :
%    \begin{macrocode}
  local Main = 
       space ^ 0 * EOL
       + Space 
       + Tab
       + Escape + EscapeMath 
       + CommentLaTeX
       + Beamer
       + DetectedCommands
       + Comment + LongComment
       + Delim
       + Operator
       + String
       + Punct
       + WeCatchTableNames
       + ( TableField + Identifier ) * ( Space + Operator + Punct + Delim + EOL + -1 )  
       + Number
       + Word
%    \end{macrocode}
%
% \bigskip
% Here, we must not put |local|, of course.
%    \begin{macrocode}
  LPEG1.sql = Main ^ 0 
%    \end{macrocode}
%
% \bigskip
% We recall that each line in the code to parse will be sent back to
% LaTeX between a pair |\@@_begin_line:| -- |\@@_end_line:|\footnote{Remember
% that the \texttt{\textbackslash @@\_end\_line:} must be explicit because it
% will be used as marker in order to delimit the argument of the command
% \texttt{\textbackslash @@\_begin\_line:}}.
%    \begin{macrocode}
  LPEG2.sql =
    Ct (
         ( space ^ 0 * "\r" ) ^ -1 
         * beamerBeginEnvironments 
         * Lc [[ \@@_begin_line: ]]
         * SpaceIndentation ^ 0 
         * ( space ^ 1 * -1 + space ^ 0 * EOL + Main ) ^ 0 
         * -1 
         * Lc [[ \@@_end_line: ]]
       )
%    \end{macrocode}
%
% \bigskip
% End of the Lua scope for the language SQL.
%    \begin{macrocode}
end
%    \end{macrocode}
% 
% \subsubsection{The language ``Minimal''}
% 
% We open a Lua local scope for the language ``Minimal'' (of course, there
% will be also global definitions).
%    \begin{macrocode}
--minimal Minimal
do 
%    \end{macrocode}
%
%    \begin{macrocode}
  local Punct = Q ( S ",:;!\\" )

  local Comment = 
    WithStyle ( 'Comment.Internal' ,
                Q "#" 
                * ( CommentMath + Q ( ( 1 - S "$\r" ) ^ 1 ) ) ^ 0 -- $
              )
       * ( EOL + -1 )

  local String = 
    WithStyle ( 'String.Short.Internal' ,
                Q "\"" 
                * ( SpaceInString 
                    + Q ( ( P [[\"]] + 1 - S " \"" ) ^ 1 ) 
                  ) ^ 0 
                * Q "\""
              )
%    \end{macrocode}
%
% \bigskip
% The argument of |Compute_braces| must be a pattern \emph{which does no
% catching} corresponding to the strings of the language.
%    \begin{macrocode}
  local braces = Compute_braces ( P "\"" * ( P "\\\"" + 1 - P "\"" ) ^ 1 * "\"" ) 

  if piton.beamer then Beamer = Compute_Beamer ( 'minimal' , braces ) end

  DetectedCommands = 
    Compute_DetectedCommands ( 'minimal' , braces ) 
    + Compute_RawDetectedCommands ( 'minimal' , braces )

  LPEG_cleaner.minimal = Compute_LPEG_cleaner ( 'minimal' , braces ) 

  local identifier = letter * alphanum ^ 0

  local Identifier = K ( 'Identifier.Internal' , identifier )

  local Delim = Q ( S "{[()]}" )

  local Main = 
       space ^ 0 * EOL
       + Space 
       + Tab
       + Escape + EscapeMath 
       + CommentLaTeX
       + Beamer
       + DetectedCommands
       + Comment
       + Delim
       + String
       + Punct
       + Identifier 
       + Number
       + Word
%    \end{macrocode}
%
% \bigskip
% Here, we must not put |local|, of course.
%    \begin{macrocode}
  LPEG1.minimal = Main ^ 0

  LPEG2.minimal = 
    Ct (
         ( space ^ 0 * "\r" ) ^ -1 
         * beamerBeginEnvironments 
         * Lc [[ \@@_begin_line: ]]
         * SpaceIndentation ^ 0 
         * ( space ^ 1 * -1 + space ^ 0 * EOL + Main ) ^ 0 
         * -1 
         * Lc [[ \@@_end_line: ]]
       )
%    \end{macrocode}
% 
%
% \bigskip
% End of the Lua scope for the language ``Minimal''.
%    \begin{macrocode}
end
%    \end{macrocode}
%
% \subsubsection{The language ``Verbatim''}
% 
% We open a Lua local scope for the language ``Verbatim'' (of course, there
% will be also global definitions).
%    \begin{macrocode}
--verbatim Verbatim
do 
%    \end{macrocode}

% \bigskip
% Here, we don't use |braces| as done with the other languages because we don't
% have have to take into account the strings (there is no string in the langage
% ``Verbatim''). 
%    \begin{macrocode}
  local braces = 
      P { "E" ,
           E = ( "{" * V "E" * "}" + ( 1 - S "{}" ) ) ^ 0 
        }

  if piton.beamer then Beamer = Compute_Beamer ( 'verbatim' , braces ) end

  DetectedCommands = 
    Compute_DetectedCommands ( 'verbatim' , braces ) 
    + Compute_RawDetectedCommands ( 'verbatim' , braces )

  LPEG_cleaner.verbatim = Compute_LPEG_cleaner ( 'verbatim' , braces ) 
%    \end{macrocode}
% 
% Now, you will construct the LPEG Word.
%    \begin{macrocode}
  local lpeg_central = 1 - S " \\\r"
  if piton.begin_escape then 
    lpeg_central = lpeg_central - piton.begin_escape 
  end
  if piton.begin_escape_math then 
    lpeg_central = lpeg_central - piton.begin_escape_math 
  end
  local Word = Q ( lpeg_central ^ 1 ) 

  local Main = 
       space ^ 0 * EOL
       + Space 
       + Tab
       + Escape + EscapeMath 
       + Beamer
       + DetectedCommands
       + Q [[\]]
       + Word
%    \end{macrocode}
%
% \bigskip
% Here, we must not put |local|, of course.
%    \begin{macrocode}
  LPEG1.verbatim = Main ^ 0

  LPEG2.verbatim = 
    Ct (
         ( space ^ 0 * "\r" ) ^ -1 
         * beamerBeginEnvironments 
         * Lc [[ \@@_begin_line: ]]
         * SpaceIndentation ^ 0 
         * ( space ^ 1 * -1 + space ^ 0 * EOL + Main ) ^ 0 
         * -1 
         * Lc [[ \@@_end_line: ]]
       )
%    \end{macrocode}
% 
%
% \bigskip
% End of the Lua scope for the language ``|verbatim|''.
%    \begin{macrocode}
end
%    \end{macrocode}
%
% \bigskip
% \subsubsection{The function Parse}
%
% \medskip
% The function |Parse| is the main function of the package \pkg{piton}. It
% parses its argument and sends back to LaTeX the code with interlaced
% formatting LaTeX instructions. In fact, everything is done by the
% \textsc{lpeg} corresponding to the considered language (|LPEG2[language]|)
% which returns as capture a Lua table containing data to send to LaTeX.
% 
% \bigskip
%    \begin{macrocode}
function piton.Parse ( language , code )
%    \end{macrocode}
% The variable |piton.language| will be used by the function |ParseAgain|.
%    \begin{macrocode}
  piton.language = language
  local t = LPEG2[language] : match ( code ) 
  if t == nil then 
    sprintL3 [[ \@@_error_or_warning:n { SyntaxError } ]] 
    return -- to exit in force the function
  end 
  local left_stack = {}
  local right_stack = {}
  for _ , one_item in ipairs ( t ) do 
    if one_item[1] == "EOL" then 
      for _ , s in ipairs ( right_stack ) do 
        tex.sprint ( s ) 
      end
      for _ , s in ipairs ( one_item[2] ) do 
        tex.tprint ( s )
      end
      for _ , s in ipairs ( left_stack ) do 
        tex.sprint ( s ) 
      end
    else 
%    \end{macrocode}
%
% Here is an example of an item beginning with |"Open"|.
%
% |{ "Open" , "\begin{uncover}<2>" , "\end{uncover}" }|
%
% In order to deal with the ends of lines, we have to close the environment
% (|{uncover}| in this example) at the end of each line and reopen it at the
% beginning of the new line. That's why we use two Lua stacks, called
% |left_stack| and |right_stack|. |left_stack| will be for the elements like
% |\begin{uncover}<2>| and |right_stack| will be for the elements like
% |\end{uncover}|. 
%    \begin{macrocode}
      if one_item[1] == "Open" then
        tex.sprint ( one_item[2] ) 
        table.insert ( left_stack , one_item[2] )
        table.insert ( right_stack , one_item[3] )
      else 
        if one_item[1] == "Close" then
          tex.sprint ( right_stack[#right_stack] ) 
          left_stack[#left_stack] = nil
          right_stack[#right_stack] = nil
        else 
          tex.tprint ( one_item )  
        end 
      end 
    end 
  end 
end
%    \end{macrocode}
%
% \bigskip
% There is the problem of the conventions of end of lines (|\n| in Unix and
% Linux but |\r\n| in Windows). The function |cr_file_lines| will read a file
% line by line after replacement of the |\r\n| by |\n|.
%    \begin{macrocode}
local cr_file_lines
function cr_file_lines ( filename )
    local f = io.open ( filename , 'rb' )
    local s = f : read ( '*a' )
    f : close ( )
    return ( s .. '\n' ) : gsub( '\r\n?' , '\n') : gmatch ( '(.-)\n' ) 
end
%    \end{macrocode}
% 
%
% \bigskip
%    \begin{macrocode}
function piton.ReadFile ( name , first_line , last_line )
  local s = ''
  local i = 0 
  for line in cr_file_lines ( name ) do 
    i = i + 1 
    if i >= first_line then 
      s = s .. '\r' .. line 
    end
    if i >= last_line then break end 
  end
%    \end{macrocode}
%
% We extract the BOM of utf-8, if present.
%    \begin{macrocode}
  if string.byte ( s , 1 ) == 13 then 
    if string.byte ( s , 2 ) == 239 then 
      if string.byte ( s , 3 ) == 187 then 
        if string.byte ( s , 4 ) == 191 then 
          s = string.sub ( s , 5 , -1 )
        end 
      end
    end
  end
  sprintL3 ( [[ \tl_set:Nn \l_@@_listing_tl { ]]) 
  tex.sprint ( luatexbase.catcodetables.CatcodeTableOther , s ) 
  sprintL3 ( [[ } ]] ) 
end
%    \end{macrocode}
%
% \bigskip
%    \begin{macrocode}
function piton.RetrieveGobbleParse ( lang , n , splittable , code )
  local s 
  s = ( ( P " " ^ 0 * "\r" ) ^ -1 * C ( P ( 1 ) ^ 0 ) * -1 ) : match ( code )  
  piton.GobbleParse ( lang , n , splittable , s )
end 
%    \end{macrocode}
% 
% \bigskip
% \subsubsection{Two variants of  the function Parse with integrated preprocessors}
%
% The following command will be used by the user command |\piton|.
% For that command, we have to undo the duplication of the symbols |#|.
%    \begin{macrocode}
function piton.ParseBis ( lang , code )
  return piton.Parse ( lang , code : gsub ( '##' , '#' ) )
end
%    \end{macrocode}
% 
% \bigskip
% The following command will be used when we have to parse some small chunks of
% code that have yet been parsed. They are re-scanned by LaTeX because it has
% been required by |\@@_piton:n| in the \pkg{piton} style of the syntaxic
% element. In that case, you have to remove the potential \verb|\@@_breakable_space:|
% that have been inserted when the key |break-lines| is in force.
%    \begin{macrocode}
function piton.ParseTer ( lang , code )
%    \end{macrocode}
% Be careful: we have to write |[[\@@_breakable_space: ]]| with a space after
% the name of the LaTeX command |\@@_breakable_space:|. Remember that
% |\@@_leading_space:| does not create a space, only an incrementation of the
% counter |\g_@@_indentation_int|. That's why we don't replace it by a space...
%    \begin{macrocode}
  return piton.Parse 
          ( 
            lang , 
            code : gsub ( [[\@@_breakable_space: ]] , ' ' )
                 : gsub ( [[\@@_leading_space: ]] , '' )
          )
end
%    \end{macrocode}
% 
%
% \bigskip
% \subsubsection{Preprocessors of the function Parse for gobble}
%
% We deal now with preprocessors of the function |Parse| which are needed when
% the ``gobble mechanism'' is used.
% 
% 
% \bigskip
% The following \textsc{lpeg} returns as capture the minimal number of spaces at
% the beginning of the lines of code. 
%    \begin{macrocode}
local AutoGobbleLPEG =
      (  (
           P " " ^ 0 * "\r"
           +
           Ct ( C " " ^ 0 ) / table.getn
           * ( 1 - P " " ) * ( 1 - P "\r" ) ^ 0 * "\r"
         ) ^ 0
         * ( Ct ( C " " ^ 0 ) / table.getn
              * ( 1 - P " " ) * ( 1 - P "\r" ) ^ 0 ) ^ -1 
       ) / math.min
%    \end{macrocode}
%
% \bigskip
% The following \textsc{lpeg} is similar but works with the tabulations.
%    \begin{macrocode}
local TabsAutoGobbleLPEG =
       (
         (
           P "\t" ^ 0 * "\r"
           +
           Ct ( C "\t" ^ 0 ) / table.getn
           * ( 1 - P "\t" ) * ( 1 - P "\r" ) ^ 0 * "\r"
         ) ^ 0
         * ( Ct ( C "\t" ^ 0 ) / table.getn
             * ( 1 - P "\t" ) * ( 1 - P "\r" ) ^ 0 ) ^ -1 
       ) / math.min
%    \end{macrocode}
%
% \bigskip
% The following \textsc{lpeg} returns as capture the number of spaces at the
% last line, that is to say before the |\end{Piton}| (and usually it's also the
% number of spaces before the corresponding |\begin{Piton}| because that's the
% traditional way to indent in LaTeX). 
%    \begin{macrocode}
local EnvGobbleLPEG =
      ( ( 1 - P "\r" ) ^ 0 * "\r" ) ^ 0
    * Ct ( C " " ^ 0 * -1 ) / table.getn 
%    \end{macrocode}
%
%    \begin{macrocode}
local remove_before_cr
function remove_before_cr ( input_string )
  local match_result = ( P "\r" ) : match ( input_string )
  if match_result then return 
    string.sub ( input_string , match_result ) 
  else return 
    input_string 
  end
end
%    \end{macrocode}
%
% \bigskip
% The function |gobble| gobbles $n$ characters on the left of the code. The
% negative values of $n$ have special significations.
%    \begin{macrocode}
function piton.Gobble ( n , code )
  code = remove_before_cr ( code ) 
  if n == 0 then return 
    code
  else 
    if n == -1 then 
      n = AutoGobbleLPEG : match ( code )
%    \end{macrocode}
% for the cas of an empty environment (only blank lines)
%    \begin{macrocode}
      if tonumber(n) then else n = 0 end
    else 
      if n == -2 then 
        n = EnvGobbleLPEG : match ( code )
      else 
        if n == -3 then 
          n = TabsAutoGobbleLPEG : match ( code )
          if tonumber(n) then else n = 0 end
        end
      end
    end
%    \end{macrocode}
% We have a second test |if n == 0| because, even if the key like
% |auto-gobble| is in force, it's possible that, in fact, there is no space to
% gobble... 
%    \begin{macrocode}
    if n == 0 then return 
      code
    else return 
%    \end{macrocode}
% We will now use a \textsc{lpeg} that we have to compute dynamically because it
% depends on the value of~$n$.
%    \begin{macrocode}
      ( Ct ( 
             ( 1 - P "\r" ) ^ (-n) * C ( ( 1 - P "\r" ) ^ 0 ) 
               * ( C "\r" * ( 1 - P "\r" ) ^ (-n) * C ( ( 1 - P "\r" ) ^ 0 )
           ) ^ 0 ) 
        / table.concat 
      ) : match ( code ) 
    end 
  end
end
%    \end{macrocode}
%
% 
% \bigskip
%
% \bigskip
% In the following code, |n| is the value of |\l_@@_gobble_int|.
%
% |splittable| is the value of |\l_@@_splittable_int|.
%    \begin{macrocode}
function piton.GobbleParse ( lang , n , splittable , code )
  piton.ComputeLinesStatus ( code , splittable ) 
  piton.last_code = piton.Gobble ( n , code )
  piton.last_language = lang
%    \end{macrocode}
% We count the number of lines of the informatic code. The result will be stored
% by Lua in |\g_@@_nb_lines_int|. 
%    \begin{macrocode}
  piton.CountLines ( piton.last_code ) 
  piton.Parse ( lang , piton.last_code )
  sprintL3 [[ \vspace{2.5pt} ]]
%    \end{macrocode}
% We finish the paragraph (each line of the listing is composed in a TeX box
% --- with potentially several lines when |break-lines-in-Piton| is in force ---
% put alone in a paragraph.
%    \begin{macrocode}
  sprintL3 [[ \par ]]
  piton.join_and_write ( ) 
end
%    \end{macrocode}
%
% \bigskip
% The following function will be used when the final user has used the key |join|
% or the key |write|.
% The value of the key |join| has been written in the Lua variable |piton.join|.
%    \begin{macrocode}
function piton.join_and_write ( )
  if piton.join ~= '' then
    if piton.join_files [ piton.join ] == nil then
      piton.join_files [ piton.join ] = piton.get_last_code ( )
    else
      piton.join_files [ piton.join ] =
      piton.join_files [ piton.join ] .. "\r\n" .. piton.get_last_code ( )
    end
  end 
%     \end{macrocode}
%
% Now, if the final user has used the key |write| to write the listing of the
% environment on an external file (on the disk).
%
% We have written the values of the keys |write| and |path-write| in the Lua
% variables |piton.write| and |piton.path-write|.
%
% If |piton.write| is not empty, that means that the key |write| has been used
% for the current environment and, hence, we have to write the content of the
% listing on the corresponding external file.
%    \begin{macrocode}
  if piton.write ~= '' then 
%    \end{macrocode}
% We will write on |file_name| the full name (with the path) of the file in
% which we will write.
%    \begin{macrocode}
    local file_name = ''
    if piton.path_write == '' then
      file_name = piton.write 
    else
%    \end{macrocode}
% If |piton.path-write| is not empty, that means that we will not write on a
% file in the current directory but in another directory. First, we verify that that
% directory actually exists. 
%    \begin{macrocode}
      local attr = lfs.attributes ( piton.path_write )
      if attr and attr.mode == "directory" then
        file_name = piton.path_write .. "/" .. piton.write
      else
%    \end{macrocode}
% If the directory does \emph{not} exist, you raise an (non-fatal) error since
% TeX is not able to create a new directory.
%    \begin{macrocode}
        sprintL3 [[ \@@_error_or_warning:n { InexistentDirectory } ]] 
      end
    end
    if file_name ~= '' then
%    \end{macrocode}
% Now, |file_name| contains the complete name of the file on which we will have
% to write. Maybe the file does not exist but we are sure that the directory
% exist. 
%
% The Lua table |piton.write_files| is a table of Lua strings corresponding to all the
%   files that we will write on the disk in the |\AtEndDocument|. They
%   correspond to the use of the key |write| (and |path-write|).
%    \begin{macrocode}
      if piton.write_files [ file_name ] == nil then
        piton.write_files [ file_name ] = piton.get_last_code ( )
      else
        piton.write_files [ file_name ] =
        piton.write_files [ file_name ] .. "\n" .. piton.get_last_code ( ) 
      end 
    end
  end
end 
%    \end{macrocode}
%
% \bigskip
% The following command will be used when the final user has set |print=false|.
%    \begin{macrocode}
function piton.GobbleParseNoPrint ( lang , n , code )
  piton.last_code = piton.Gobble ( n , code )
  piton.last_language = lang
  piton.join_and_write ( )
end
%    \end{macrocode}
%
% \bigskip
% The following function will be used when the key |split-on-empty-lines| is in
% force. With that key, the informatic code is split in chunks at the empty
% lines (usually between the informatic functions defined in the informatic
% code). LaTeX will be able to change the page between the chunks. The second
% argument |n| corresponds to the value of the key |gobble| (number of spaces to
% gobble). 
%    \begin{macrocode}
function piton.GobbleSplitParse ( lang , n , splittable , code )
  local chunks 
  chunks =
     ( 
       Ct ( 
            ( 
              P " " ^ 0 * "\r" 
              + 
              C ( ( ( 1 - P "\r" ) ^ 1 * ( P "\r" + -1 )  
                    - ( P " " ^ 0 * ( P "\r" + -1 ) ) 
                  ) ^ 1 
                ) 
            ) ^ 0 
          ) 
     ) : match ( piton.Gobble ( n , code ) )
  sprintL3 [[ \begingroup ]] 
  sprintL3 
    ( 
      [[ \PitonOptions { split-on-empty-lines = false, gobble = 0, ]]
      .. "language = " .. lang .. "," 
      .. "splittable = " .. splittable .. "}"
    )
  for k , v in pairs ( chunks ) do 
    if k > 1 then 
      sprintL3 ( [[ \l_@@_split_separation_tl ]] ) 
    end 
    tex.print
      ( 
        [[\begin{]] .. piton.env_used_by_split .. "}\r" 
        .. v 
        .. [[\end{]] .. piton.env_used_by_split .. "}\r" -- previously: }%\r
      )  
  end
  sprintL3 [[ \endgroup ]] 
end 
%    \end{macrocode}
%
%  \bigskip 
%    \begin{macrocode}
function piton.RetrieveGobbleSplitParse ( lang , n , splittable , code )
  local s 
  s = ( ( P " " ^ 0 * "\r" ) ^ -1 * C ( P ( 1 ) ^ 0 ) * -1 ) : match ( code )  
  piton.GobbleSplitParse ( lang , n , splittable , s )
end 
%    \end{macrocode}
% 
% \bigskip
% The following Lua string will be inserted between the chunks of code created
% when the key |split-on-empty-lines| is in force. It's used only once: you have
% given a name to that Lua string only for legibility. The token list
% |\l_@@_split_separation_tl| corresponds to the key |split-separation|. That
% token list must contain elements inserted in \emph{vertical mode} of TeX.
%    \begin{macrocode}
piton.string_between_chunks = 
 [[ \par \l_@@_split_separation_tl \mode_leave_vertical: ]]
 .. [[ \int_gzero:N \g_@@_line_int ]]
%    \end{macrocode}
% The counter |\g_@@_line_int| will be used to control the points where the code
% may be broken by a change of page (see the key |splittable|).
% 
% \bigskip
% The following public Lua function is provided to the developer. 
%    \begin{macrocode}
function piton.get_last_code ( ) 
  return LPEG_cleaner[piton.last_language] : match ( piton.last_code )
         : gsub('\r\n','\n') : gsub('\r','\n')
end
%    \end{macrocode}
% 
% \subsubsection{To count the number of lines}
%
%    \begin{macrocode}
function piton.CountLines ( code )
  local count = 0
  count =
     ( Ct ( ( ( 1 - P "\r" ) ^ 0 * C "\r" ) ^ 0
            * ( ( 1 - P "\r" ) ^ 1 * Cc "\r" ) ^ -1
            * -1
          ) / table.getn
     ) : match ( code )
  sprintL3 ( string.format ( [[ \int_gset:Nn \g_@@_nb_lines_int { %i } ]] , count ) ) 
end 
%    \end{macrocode}
%
% \bigskip
% The following function is only used once (in |piton.GobbleParse|). We have
% written an autonomous function only for legibility. The number of lines of the
% code will be stored in |\l_@@_nb_non_empty_lines_int|. It will be used to
% compute the largest number of lines to write (when |line-numbers| is in force).
%    \begin{macrocode}
function piton.CountNonEmptyLines ( code )
  local count = 0
  count =
     ( Ct ( ( P " " ^ 0 * "\r"
              + ( 1 - P "\r" ) ^ 0 * C "\r" ) ^ 0
            * ( 1 - P "\r" ) ^ 0 
            * -1
          ) / table.getn
     ) : match ( code )
  sprintL3 
   ( string.format ( [[ \int_set:Nn  \l_@@_nb_non_empty_lines_int { %i } ]] , count ) ) 
end 
%    \end{macrocode}
%
% \bigskip
%    \begin{macrocode}
function piton.CountLinesFile ( name )
  local count = 0 
  for line in io.lines ( name ) do count = count + 1 end
  sprintL3 
   ( string.format ( [[ \int_gset:Nn \g_@@_nb_lines_int { %i } ]], count ) )
end 
%    \end{macrocode}
%
%
% \bigskip
%    \begin{macrocode}
function piton.CountNonEmptyLinesFile ( name )
  local count = 0 
  for line in io.lines ( name ) do 
    if not ( ( P " " ^ 0 * -1 ) : match ( line ) ) then
       count = count + 1 
    end
  end
  sprintL3 
   ( string.format ( [[ \int_set:Nn \l_@@_nb_non_empty_lines_int { % i } ]] , count ) ) 
end 
%    \end{macrocode}
%
% 
% \bigskip
% The following function stores in |\l_@@_first_line_int| and
% |\l_@@_last_line_int| the numbers of lines of the file |file_name|
% corresponding to the strings |marker_beginning| and |marker_end|.
%
% |s| is the marker of the beginning and |t| is the marker of the end.
%    \begin{macrocode}
function piton.ComputeRange(s,t,file_name)
  local first_line = -1
  local count = 0
  local last_found = false
  for line in io.lines ( file_name ) do 
    if first_line == -1 then 
      if string.sub ( line , 1 , #s ) == s then 
        first_line = count 
      end 
    else 
      if string.sub ( line , 1 , #t ) == t then 
        last_found = true 
        break 
      end 
    end 
    count = count + 1 
  end
  if first_line == -1 then 
    sprintL3 [[ \@@_error_or_warning:n { begin~marker~not~found } ]] 
  else 
    if last_found == false then 
      sprintL3 [[ \@@_error_or_warning:n { end~marker~not~found } ]] 
    end
  end
  sprintL3 ( 
      [[ \int_set:Nn \l_@@_first_line_int { ]] .. first_line .. ' + 2 }' 
      .. [[ \int_set:Nn \l_@@_last_line_int { ]] .. count .. ' }' )
end
%    \end{macrocode}
%
% \bigskip
% \subsubsection{To determine the empty lines of the listings}
%
% Despite its name, the Lua function |ComputeLinesStatus| computes
% |piton.lines_status| but also |piton.empty_lines|.
%
% \medskip
% In |piton.empty_lines|, a line will have the number 0 if it's a empty line (in
% fact a blank line, with only spaces) and 1 elsewhere.
% 
% \medskip
% In |piton.lines_status|, each line will have a status with regard the
% breaking points allowed (for the changes of pages).
% \begin{itemize}
% \item 0 if the line is empty and a page break is allowed;
% \item 1 if the line is not empty but a page break is allowed after that line;
% \item 2 if a page break is \emph{not} allowed after that line (empty or not empty).
% \end{itemize}
%    
% \medskip
% |splittable| is the value of |\l_@@_splittable_int|.
% However, if |splittable-on-empty-lines| is in force, |splittable| is the
% opposite of |\l_@@_splittable_int|.
%    \begin{macrocode}
function piton.ComputeLinesStatus ( code , splittable ) 
%    \end{macrocode}
% The lines in the listings which correspond to the beginning or the end of an
% environment of Beamer (eg. |\begin{uncoverenv}|) must be retrieved (those
% lines have \emph{no} number and therefore, \emph{no} status).
%    \begin{macrocode}
  local lpeg_line_beamer
  if piton.beamer then
    lpeg_line_beamer = 
       space ^ 0 
        * P [[\begin{]] * beamerEnvironments * "}"
        * ( "<" * ( 1 - P ">" ) ^ 0 * ">" ) ^ -1 
       +
       space ^ 0 
        * P [[\end{]] * beamerEnvironments * "}" 
  else 
    lpeg_line_beamer = P ( false ) 
  end 
%    \end{macrocode}
%    \begin{macrocode}
  local lpeg_empty_lines =
    Ct (
         ( lpeg_line_beamer * "\r" 
           +
           P " " ^ 0 * "\r" * Cc ( 0 )
           + 
           ( 1 - P "\r" ) ^ 0 * "\r" * Cc ( 1 )
         ) ^ 0
         * 
         ( lpeg_line_beamer + ( 1 - P "\r" ) ^ 1 * Cc ( 1 ) ) ^ -1
       )
    * -1
%    \end{macrocode}
%    \begin{macrocode}
  local lpeg_all_lines =    
    Ct ( 
         ( lpeg_line_beamer * "\r" 
           +
           ( 1 - P "\r" ) ^ 0 * "\r" * Cc ( 1 )
         ) ^ 0 
         * 
         ( lpeg_line_beamer + ( 1 - P "\r" ) ^ 1 * Cc ( 1 ) ) ^ -1
       )
    * -1
%    \end{macrocode}
% We begin with the computation of |piton.empty_lines|. It will be used in
% conjonction with |line-numbers|. 
%    \begin{macrocode}
  piton.empty_lines = lpeg_empty_lines : match ( code ) 
%    \end{macrocode}
% 
% Now, we compute |piton.lines_status|. It will be used in conjonction with
% |splittable| and |splittable-on-empty-lines|. 
%
% Now, we will take into account the current value of |\l_@@_splittable_int|
% (provided by the \emph{absolute value} of the argument |splittable|).
%    \begin{macrocode}
  local lines_status
  local s = splittable
  if splittable < 0 then s = - splittable end 
%    \end{macrocode}
% 
%    \begin{macrocode}
  if splittable > 0 then  
    lines_status = lpeg_all_lines : match ( code ) 
  else
%    \end{macrocode}
% Here, we should try to copy |piton.empty_lines| but it's not easy.
%    \begin{macrocode}
    lines_status = lpeg_empty_lines : match ( code ) 
    for i , x in ipairs ( lines_status ) do
      if x == 0 then
        for j = 1 , s - 1 do
          if i + j > #lines_status then break end 
          if lines_status[i+j] == 0 then break end 
            lines_status[i+j] = 2 
        end
        for j = 1 , s - 1 do
          if i - j == 1 then break end 
          if lines_status[i-j-1] == 0 then break end 
          lines_status[i-j-1] = 2 
        end
      end 
    end 
  end 
%    \end{macrocode}
% 
% In all cases (whatever is the value of |splittable-on-empty-lines|) we have to
% deal with both extremities of the listing to format.
%
% First from the beginning of the code.
%    \begin{macrocode}
  for j = 1 , s - 1 do
    if j > #lines_status then break end
    if lines_status[j] == 0 then break end
    lines_status[j] = 2
  end
%    \end{macrocode}
% Now, from the end of the code.
%    \begin{macrocode}
  for j = 1 , s - 1 do 
    if #lines_status - j == 0 then break end 
    if lines_status[#lines_status - j] == 0 then break end
    lines_status[#lines_status - j] = 2
  end 
%    \end{macrocode}
% 
% \bigskip
%    \begin{macrocode}
  piton.lines_status = lines_status
end 
%    \end{macrocode}
% 

% 
% \bigskip
% \subsubsection{To create new languages with the syntax of listings}
%
%    \begin{macrocode}
function piton.new_language ( lang , definition )
  lang = string.lower ( lang )
%    \end{macrocode}
%
% \bigskip
%    \begin{macrocode}
  local alpha , digit = lpeg.alpha , lpeg.digit 
  local extra_letters = { "@" , "_" , "$" } -- $
%    \end{macrocode}
%
% 
% \bigskip
% The command |add_to_letter| (triggered by the key ||) don't write right away
% in the \textsc{lpeg} pattern of the letters in an intermediate |extra_letters|
% because we may have to retrieve letters from that ``list'' if there appear in
% a key |alsoother|.
%    \begin{macrocode}
  function add_to_letter ( c )
    if c ~= " " then table.insert ( extra_letters , c ) end
  end 
%    \end{macrocode}
%
% For the digits, it's straitforward.
%    \begin{macrocode}
  function add_to_digit ( c )
    if c ~= " " then digit = digit + c end 
  end 
%    \end{macrocode}
%
% \bigskip
% The main use of the key |alsoother| is, for the language LaTeX, when you have
% to retrieve some characters from the list of letters, in particular |@| and
% |_| (which, by default, are not allowed in the name of a control sequence in
% TeX). 
%
% \medskip
% (In the following \textsc{lpeg} we have a problem when we try to add |{| and
% |}|). 
%    \begin{macrocode}
  local other = S ":_@+-*/<>!?;.()[]~^=#&\"\'\\$" -- $
  local extra_others = { } 
%    \end{macrocode}
%
%    \begin{macrocode}
  function add_to_other ( c )
    if c ~= " " then 
%    \end{macrocode}
% We will use |extra_others| to retrieve further these characters from the list
% of the letters. 
%    \begin{macrocode}
      extra_others[c] = true 
%    \end{macrocode}
% The \textsc{lpeg} pattern |other| will be used in conjunction with the key
% |tag| (mainly for languages such as \textsc{html} and \textsc{xml}) for the
% character |/| in the closing tags |</....>|).
%    \begin{macrocode}
      other = other + P ( c )
    end 
  end 
%    \end{macrocode}
% 
%
% \bigskip
% Now, the first transformation of the definition of the language, as provided
% by the final user in the argument |definition| of |piton.new_language|.
%    \begin{macrocode}
  local def_table 
  if ( S ", " ^ 0 * -1 ) : match ( definition ) then 
    def_table = {} 
  else
    local strict_braces  =
      P { "E" , 
          E = ( "{" * V "F" * "}" + ( 1 - S ",{}" ) ) ^ 0  ,
          F = ( "{" * V "F" * "}" + ( 1 - S "{}" ) ) ^ 0 
        }
    local cut_definition =
      P { "E" ,
          E = Ct ( V "F" * ( "," * V "F" ) ^ 0 ) ,
          F = Ct ( space ^ 0 * C ( alpha ^ 1 ) * space ^ 0 
                  * ( "=" * space ^ 0 * C ( strict_braces ) ) ^ -1 )
        }
    def_table = cut_definition : match ( definition )
  end 
%    \end{macrocode}
% The definition of the language, provided by the final user of \pkg{piton} is
% now in the Lua table |def_table|. We will use it \emph{several times}.
%
% \medskip
% The following \textsc{lpeg} will be used to extract arguments in the values of
% the keys (|morekeywords|, |morecomment|, |morestring|, etc.).
%    \begin{macrocode}
  local tex_braced_arg = "{" * C ( ( 1 - P "}" ) ^ 0 ) * "}" 
  local tex_arg = tex_braced_arg + C ( 1 ) 
  local tex_option_arg =  "[" * C ( ( 1 - P "]" ) ^ 0 ) * "]" + Cc ( nil ) 
%    \end{macrocode}
%
%    \begin{macrocode}
  local args_for_tag 
    =  tex_option_arg
       * space ^ 0 
       * tex_arg
       * space ^ 0
       * tex_arg
%    \end{macrocode}
%
%    \begin{macrocode}
  local args_for_morekeywords 
    = "[" * C ( ( 1 - P "]" ) ^ 0 ) * "]" 
       * space ^ 0 
       * tex_option_arg
       * space ^ 0
       * tex_arg
       * space ^ 0
       * ( tex_braced_arg + Cc ( nil ) ) 
%    \end{macrocode}
%
%    \begin{macrocode}
  local args_for_moredelims
    = ( C ( P "*" ^ -2 ) + Cc ( nil ) ) * space ^ 0 
      * args_for_morekeywords
%    \end{macrocode}
%
%    \begin{macrocode}
  local args_for_morecomment
    = "[" * C ( ( 1 - P "]" ) ^ 0 ) * "]" 
       * space ^ 0 
       * tex_option_arg 
       * space ^ 0
       * C ( P ( 1 ) ^ 0 * -1 )
%    \end{macrocode}
%
%
% \bigskip
% We scan the definition of the language (i.e. the table |def_table|) in order
% to detect the potential key |sensitive|. Indeed, we have to catch that key
% before the treatment of the keywords of the language. We will also look for
% the potential keys |alsodigit|, |alsoletter| and |tag|. 
%    \begin{macrocode}
  local sensitive = true 
  local style_tag , left_tag , right_tag 
  for _ , x in ipairs ( def_table ) do
    if x[1] == "sensitive" then 
      if x[2] == nil or ( P "true" ) : match ( x[2] ) then 
        sensitive = true 
      else 
        if ( P "false" + P "f" ) : match ( x[2] ) then sensitive = false end
      end
    end
    if x[1] == "alsodigit" then x[2] : gsub ( "." , add_to_digit ) end 
    if x[1] == "alsoletter" then x[2] : gsub ( "." , add_to_letter ) end 
    if x[1] == "alsoother" then x[2] : gsub ( "." , add_to_other ) end 
    if x[1] == "tag" then
      style_tag , left_tag , right_tag = args_for_tag : match ( x[2] ) 
      style_tag = style_tag or [[\PitonStyle{Tag}]]
    end 
  end
%    \end{macrocode}
% Now, the \textsc{lpeg} for the numbers. Of course, it uses |digit| previously
% computed. 
%    \begin{macrocode}
  local Number =
    K ( 'Number.Internal' ,
        ( digit ^ 1 * "." * # ( 1 - P "." ) * digit ^ 0 
          + digit ^ 0 * "." * digit ^ 1 
          + digit ^ 1 )
        * ( S "eE" * S "+-" ^ -1 * digit ^ 1 ) ^ -1
        + digit ^ 1 
      ) 
%    \end{macrocode}
%
%    \begin{macrocode}
  local string_extra_letters = ""
  for _ , x in ipairs ( extra_letters ) do 
    if not ( extra_others[x] ) then
     string_extra_letters = string_extra_letters .. x      
    end 
  end 
  local letter = alpha + S ( string_extra_letters ) 
                  + P "â" + "à" + "ç" + "é" + "è" + "ê" + "ë" + "ï" + "î"
                    + "ô" + "û" + "ü" + "Â" + "À" + "Ç" + "É" + "È" + "Ê" + "Ë"
                    + "Ï" + "Î" + "Ô" + "Û" + "Ü"
%    \end{macrocode}
% 
%    \begin{macrocode}
  local alphanum = letter + digit
  local identifier = letter * alphanum ^ 0
  local Identifier = K ( 'Identifier.Internal' , identifier )
%    \end{macrocode}
%
% 
% \bigskip
% Now, we scan the definition of the language (i.e. the table |def_table|) for
% the keywords.
%
%
% The following LPEG does \emph{not} catch the optional argument between square
% brackets in first position.
%    \begin{macrocode}
  local split_clist = 
    P { "E" ,
         E = ( "[" * ( 1 - P "]" ) ^ 0 * "]" ) ^ -1  
             * ( P "{" ) ^ 1
             * Ct ( V "F" * ( "," * V "F" ) ^ 0 ) 
             * ( P "}" ) ^ 1 * space ^ 0 ,
         F = space ^ 0 * C ( letter * alphanum ^ 0 + other ^ 1 ) * space ^ 0 
      }
%    \end{macrocode}
% The following function will be used if the keywords are not case-sensitive.
%    \begin{macrocode}
  local keyword_to_lpeg
  function keyword_to_lpeg ( name ) return 
    Q ( Cmt ( 
              C ( identifier ) , 
              function ( s , i , a ) return 
                string.upper ( a ) == string.upper ( name )  
              end 
            ) 
      ) 
  end 
  local Keyword = P ( false )  
  local PrefixedKeyword = P ( false )  
%    \end{macrocode}
% Now, we actually treat all the keywords and also the key |moredirectives|.
%    \begin{macrocode}
  for _ , x in ipairs ( def_table )
  do if x[1] == "morekeywords" 
        or x[1] == "otherkeywords" 
        or x[1] == "moredirectives" 
        or x[1] == "moretexcs"
     then
        local keywords = P ( false )
%    \end{macrocode}
%    \begin{macrocode}
        local style = [[\PitonStyle{Keyword}]]
        if x[1] == "moredirectives" then style = [[\PitonStyle{Directive}]] end
        style =  tex_option_arg : match ( x[2] ) or style
        local n = tonumber ( style )
        if n then 
          if n > 1 then style = [[\PitonStyle{Keyword]] .. style .. "}" end
        end 
%    \end{macrocode}
%    \begin{macrocode}
        for _ , word in ipairs ( split_clist : match ( x[2] ) ) do
          if x[1] == "moretexcs" then
            keywords = Q ( [[\]] .. word ) + keywords
          else 
            if sensitive 
%    \end{macrocode}
% The documentation of \pkg{lstlistings} specifies that, for the key
% |morekeywords|, if a keyword is a prefix of another keyword, then the prefix
% must appear first. However, for the \text{lpeg}, it's rather the contrary.
% That's why, here, we add the new element \emph{on the left}.
%    \begin{macrocode}
            then keywords = Q ( word  ) + keywords
            else keywords = keyword_to_lpeg ( word ) + keywords
            end
          end 
        end
        Keyword = Keyword + 
           Lc ( "{" .. style .. "{" ) * keywords * Lc "}}" 
     end 
%    \end{macrocode}
% Of course, the feature with the key |keywordsprefix| is designed for the
% languages TeX, LaTeX, et \emph{al}. In that case, there is two kinds of
% keywords (= control sequences).
% \begin{itemize}
% \item those beginning with |\| and a sequence of characters of catcode
% ``|letter|''; 
% \item those beginning by |\| followed by one character of catcode ``|other|''.
% \end{itemize}
% The following code addresses both cases. Of course, the \textsc{lpeg} pattern
% |letter| must catch only characters of catcode ``|letter|''. That's why we
% have a key |alsoletter| to add new characters in that category (e.g. |:| when
% we want to format L3 code). However, the \textsc{lpeg} pattern is allowed to
% catch \emph{more} than only the characters of catcode ``other'' in TeX.
%    \begin{macrocode}
     if x[1] == "keywordsprefix" then
       local prefix = ( ( C ( 1 - P " " ) ^ 1 ) * P " " ^ 0 ) : match ( x[2] ) 
       PrefixedKeyword = PrefixedKeyword 
          + K ( 'Keyword' , P ( prefix ) * ( letter ^ 1 + other ) )
     end 
  end    
%    \end{macrocode}
%
% 
% \bigskip
% Now, we scan the definition of the language (i.e. the table |def_table|) for
% the strings.
%    \begin{macrocode}
  local long_string  = P ( false )
  local Long_string = P ( false ) 
  local LongString = P (false ) 
  local central_pattern = P ( false ) 
  for _ , x in ipairs ( def_table ) do
    if x[1] == "morestring" then
      arg1 , arg2 , arg3 , arg4 = args_for_morekeywords : match ( x[2] ) 
      arg2 = arg2 or [[\PitonStyle{String.Long}]]
      if arg1 ~= "s" then 
        arg4 = arg3 
      end 
      central_pattern = 1 - S ( " \r" .. arg4 ) 
      if arg1 : match "b" then 
        central_pattern = P ( [[\]] .. arg3 ) + central_pattern
      end
%    \end{macrocode}
% In fact, the specifier |d| is point-less: when it is not in force, it's still
% possible to double the delimiter with a correct behaviour of \pkg{piton}
% since, in that case, \pkg{piton} will compose \emph{two} contiguous strings...
%    \begin{macrocode}
      if arg1 : match "d" or arg1 == "m" then 
        central_pattern = P ( arg3 .. arg3 ) + central_pattern
      end
      if arg1 == "m" 
      then prefix = B ( 1 - letter - ")" - "]" ) 
      else prefix = P ( true ) 
      end
%    \end{macrocode}
% First, a pattern \emph{without captures} (needed to compute |braces|).
%    \begin{macrocode}
     long_string = long_string + 
         prefix 
         * arg3  
         * ( space + central_pattern ) ^ 0 
         * arg4  
%    \end{macrocode}
% Now a pattern \emph{with captures}.
%    \begin{macrocode}
     local pattern = 
         prefix 
         * Q ( arg3 ) 
         * ( SpaceInString + Q ( central_pattern ^ 1 ) + EOL ) ^ 0 
         * Q ( arg4 ) 
%    \end{macrocode}
% We will need |Long_string| in the nested comments.
%    \begin{macrocode}
      Long_string = Long_string + pattern 
      LongString = LongString +
         Ct ( Cc "Open" * Cc ( "{" ..  arg2 .. "{" ) * Cc "}}" )
         * pattern
         * Ct ( Cc "Close" ) 
    end 
  end
%    \end{macrocode}
% The argument of |Compute_braces| must be a pattern \emph{which does no
% catching} corresponding to the strings of the language.
%    \begin{macrocode}
  local braces = Compute_braces ( long_string ) 
  if piton.beamer then Beamer = Compute_Beamer ( lang , braces ) end

  DetectedCommands = 
    Compute_DetectedCommands ( lang , braces ) 
    + Compute_RawDetectedCommands ( lang , braces )

  LPEG_cleaner[lang] = Compute_LPEG_cleaner ( lang , braces ) 
%    \end{macrocode}
%
% \bigskip
% Now, we deal with the comments and the delims.
%    \begin{macrocode}
  local CommentDelim = P ( false ) 

  for _ , x in ipairs ( def_table ) do
    if x[1] == "morecomment" then 
      local arg1 , arg2 , other_args = args_for_morecomment : match ( x[2] ) 
      arg2 = arg2 or [[\PitonStyle{Comment}]]
%    \end{macrocode}
% If the letter |i| is present in the first argument (eg: 
% |morecomment = [si]{(*}{*)}|, then the corresponding comments are discarded. 
%    \begin{macrocode}
      if arg1 : match "i" then arg2 = [[\PitonStyle{Discard}]] end
      if arg1 : match "l" then 
        local arg3 = ( tex_braced_arg + C ( P ( 1 ) ^ 0 * -1 ) )
                     : match ( other_args ) 
        if arg3 == [[\#]] then arg3 = "#" end -- mandatory
        if arg3 == [[\%]] then arg3 = "%" end -- mandatory¨
        CommentDelim = CommentDelim + 
            Ct ( Cc "Open" 
                 * Cc ( "{" .. arg2 .. "{" ) * Cc "}}" ) 
                 * Q ( arg3 ) 
                 * ( CommentMath + Q ( ( 1 - S "$\r" ) ^ 1 ) ) ^ 0 -- $
            * Ct ( Cc "Close" ) 
            * ( EOL + -1 )
      else 
        local arg3 , arg4 = 
          ( tex_arg * space ^ 0 * tex_arg ) : match ( other_args )
        if arg1 : match "s" then 
          CommentDelim = CommentDelim +
              Ct ( Cc "Open" * Cc ( "{" .. arg2 .. "{" ) * Cc "}}" ) 
              * Q ( arg3 ) 
              * ( 
                  CommentMath 
                  + Q ( ( 1 - P ( arg4 ) - S "$\r" ) ^ 1 ) -- $
                  + EOL 
                ) ^ 0  
              * Q ( arg4 ) 
              * Ct ( Cc "Close" ) 
        end
        if arg1 : match "n" then 
          CommentDelim = CommentDelim +
            Ct ( Cc "Open" * Cc ( "{" .. arg2 .. "{" ) * Cc "}}" ) 
             * P { "A" ,
                  A = Q ( arg3 ) 
                      * ( V "A" 
                          + Q ( ( 1 - P ( arg3 ) - P ( arg4 ) 
                                  - S "\r$\"" ) ^ 1 ) -- $
                          + long_string
                          +   "$" -- $
                              * K ( 'Comment.Math' , ( 1 - S "$\r" ) ^ 1 ) --$
                              * "$" -- $
                          + EOL
                        ) ^ 0 
                      * Q ( arg4 )
                 }  
            * Ct ( Cc "Close" ) 
        end
      end
    end 
%    \end{macrocode}
% For the keys |moredelim|, we have to add another argument in first position,
% equal to |*| or |**|.
%    \begin{macrocode}
    if x[1] == "moredelim" then 
      local arg1 , arg2 , arg3 , arg4 , arg5 
        = args_for_moredelims : match ( x[2] )  
      local MyFun = Q 
      if arg1 == "*" or arg1 == "**" then
        function MyFun ( x ) 
          if x ~= '' then return
            LPEG1[lang] : match ( x ) 
          end 
        end
      end
      local left_delim 
      if arg2 : match "i" then
        left_delim = P ( arg4 ) 
      else 
        left_delim = Q ( arg4 )
      end 
      if arg2 : match "l" then 
        CommentDelim = CommentDelim +
            Ct ( Cc "Open" * Cc ( "{" .. arg3 .. "{" ) * Cc "}}" ) 
            * left_delim
            * ( MyFun ( ( 1 - P "\r" ) ^ 1 ) ) ^ 0  
            * Ct ( Cc "Close" ) 
            * ( EOL + -1 ) 
      end
      if arg2 : match "s" then 
        local right_delim 
        if arg2 : match "i" then
          right_delim = P ( arg5 )
        else 
          right_delim = Q ( arg5 )
        end 
        CommentDelim = CommentDelim +
            Ct ( Cc "Open" * Cc ( "{" .. arg3 .. "{" ) * Cc "}}" ) 
            * left_delim
            * ( MyFun ( ( 1 - P ( arg5 ) - "\r" ) ^ 1 ) + EOL ) ^ 0  
            * right_delim
            * Ct ( Cc "Close" ) 
      end
    end 
  end 

  local Delim = Q ( S "{[()]}" )
  local Punct = Q ( S "=,:;!\\'\"" )
%    \end{macrocode}
%
% 
%    \begin{macrocode}
  local Main = 
       space ^ 0 * EOL
       + Space 
       + Tab
       + Escape + EscapeMath 
       + CommentLaTeX
       + Beamer
       + DetectedCommands
       + CommentDelim
%    \end{macrocode}
% We must put |LongString| before |Delim| because, in PostScript, the strings
% are delimited by parenthesis and those parenthesis would be caught by |Delim|.
%    \begin{macrocode}
       + LongString
       + Delim
       + PrefixedKeyword
       + Keyword * ( -1 + # ( 1 - alphanum ) ) 
       + Punct 
       + K ( 'Identifier.Internal' , letter * alphanum ^ 0 )
       + Number
       + Word
%    \end{macrocode}
%
% The \textsc{lpeg} |LPEG1[lang]| is used to reformat small elements, for
% example the arguments of the ``detected commands''.
%
% Of course, here, we must not put |local|, of course.
%   \begin{macrocode}
  LPEG1[lang] = Main ^ 0 
%    \end{macrocode}
%
% 
% The \textsc{lpeg} |LPEG2[lang]| is used to format general chunks of code.
%    \begin{macrocode}
  LPEG2[lang] = 
    Ct (
         ( space ^ 0 * P "\r" ) ^ -1 
         * beamerBeginEnvironments 
         * Lc [[ \@@_begin_line: ]]
         * SpaceIndentation ^ 0 
         * ( space ^ 1 * -1 + space ^ 0 * EOL + Main ) ^ 0 
         * -1 
         * Lc [[ \@@_end_line: ]]
       )
%    \end{macrocode}
%
% If the key |tag| has been used. Of course, this feature is designed for the
% languages such as \textsc{html} and \textsc{xml}.
%    \begin{macrocode}
  if left_tag then
    local Tag = Ct ( Cc "Open" * Cc ( "{" .. style_tag .. "{" ) * Cc "}}" ) 
                * Q ( left_tag * other ^ 0 ) -- $
                * ( ( ( 1 - P ( right_tag ) ) ^ 0 )
                  / ( function ( x ) return LPEG0[lang] : match ( x ) end ) )
                * Q ( right_tag ) 
                * Ct ( Cc "Close" ) 
    MainWithoutTag
            = space ^ 1 * -1
            + space ^ 0 * EOL 
            + Space 
            + Tab
            + Escape + EscapeMath 
            + CommentLaTeX
            + Beamer
            + DetectedCommands
            + CommentDelim
            + Delim
            + LongString
            + PrefixedKeyword
            + Keyword * ( -1 + # ( 1 - alphanum ) ) 
            + Punct 
            + K ( 'Identifier.Internal' , letter * alphanum ^ 0 )
            + Number
            + Word 
    LPEG0[lang] = MainWithoutTag ^ 0 
    local LPEGaux = Tab + Escape + EscapeMath + CommentLaTeX 
                    + Beamer + DetectedCommands + CommentDelim + Tag 
    MainWithTag 
            = space ^ 1 * -1
            + space ^ 0 * EOL 
            + Space 
            + LPEGaux 
            + Q ( ( 1 - EOL - LPEGaux ) ^ 1 )
    LPEG1[lang] = MainWithTag ^ 0 
    LPEG2[lang] = 
      Ct (
           ( space ^ 0 * P "\r" ) ^ -1 
           * beamerBeginEnvironments 
           * Lc [[ \@@_begin_line: ]]
           * SpaceIndentation ^ 0 
           * LPEG1[lang]
           * -1 
           * Lc [[ \@@_end_line: ]]
         )
  end
end
%    \end{macrocode}
%
% \subsubsection{We write the files (key 'write') and join the files in the PDF (key 'join')}
%
%    \begin{macrocode}
function piton.join_and_write_files ( )
  for file_name , file_content in pairs ( piton.write_files ) do
    local file = io.open ( file_name , "w" ) 
    if file then
      file : write ( file_content )
      file : close ( ) 
    else
      sprintL3
        ( [[ \@@_error_or_warning:nn { FileError } { ]] .. file_name .. [[ } ]] )
    end 
  end
%    \end{macrocode}
%
%    \begin{macrocode}
  for file_name , file_content in pairs ( piton.join_files ) do
    pdf.immediateobj("stream", file_content)
    tex.print
      (
        [[ \pdfextension annot width 0pt height 0pt depth 0pt ]]
        ..
%    \end{macrocode}
% The entry |/F| in the \textsc{pdf} dictionnary of the annotation is an
% unsigned 32-bit integer containing flags specifying various characteristics of
% the annotation. The bit in position 2 means \emph{Hidden}.
% However, despite that bit which means \emph{Hidden}, some \textsc{pdf} readers
% show the annotation. That's why we have used |width 0pt height 0pt depth 0pt|.
%    \begin{macrocode}
        [[ { /Subtype /FileAttachment /F 2 /Name /Paperclip]]
        ..
        [[ /Contents (File included by the key 'join' of piton) ]]
        ..
%    \end{macrocode}
% We recall that the value of |file_name| comes from the key |join|,
% and that we have converted immediatly the value of the key in utf16
% (with the \text{bom} big endian) written in hexadecimal.
% It's the suitable form for insertion as value of the key |/UF| between angular
% brackets |<| and |>|.
%    \begin{macrocode} 
        [[ /FS << /Type /Filespec /UF <]] .. file_name .. [[>]]
        ..
        [[ /EF << /F \pdffeedback lastobj 0 R >> >> }  ]]
      )
  end
end


%</LUA>
%    \end{macrocode}
%
%
% 
% \vspace{1cm}
% \section{History}
%
% The successive versions of the file |piton.sty| provided by TeXLive are available on the
% \textsc{svn} server of TeXLive:
%
% \nobreak
% \smallskip
% {
% \small
% \nolinkurl{https://tug.org/svn/texlive/trunk/Master/texmf-dist/tex/lualatex/piton/piton.sty}
% }
%
% \medskip
% The development of the extension \pkg{piton} is done on the following GitHub
% repository:
%
% \verb|https://github.com/fpantigny/piton|
%
% \subsection*{Changes between versions 4.6 and 4.7}
%
% New key |rounded-corners|
%
% \subsection*{Changes between versions 4.5 and 4.6}
%
% New keys |tcolorbox|, |box|, |max-width| and |vertical-detected-commands|
%
% New special color: |none|
%
% \subsection*{Changes between versions 4.4 and 4.5}
%
% New key |print|
%
% |\RenewPitonEnvironment|,
% |\DeclarePitonEnvironment| and |\ProvidePitonEnvironment| have been added.
%
% \subsection*{Changes between versions 4.3 and 4.4}
%
% New key |join| which generates files embedded in the \textsc{pdf} as \emph{joined files}.
%
% \subsection*{Changes between versions 4.2 and 4.3}
%
% New key |raw-detected-commands|
%
% The key |old-PitonInputFile| has been deleted.
% 
% \subsection*{Changes between versions 4.1 and 4.2}
%
% New key |break-numbers-anywhere|.
% 
% \subsection*{Changes between versions 4.0 and 4.1}
%
% New language |verbatim|.
%
% New key |break-strings-anywhere|.
% 
% \subsection*{Changes between versions 3.1 and 4.0}
%
% This version introduces an incompatibility: the syntax for the relative and
% absolute paths in |\PitonInputFile| and the key |path| has been changed to be
% conform to usual conventions. An temporary key |old-PitonInputFile|, available
% at load-time, has been added for backward compatibility.
%
% New keys |font-command|, |splittable-on-empty-lines| and |env-used-by-split|.
%
%
% \subsection*{Changes between versions 3.0 and 3.1}
%
% Keys |line-numbers/format|, |detected-beamer-commands| and
% |detected-beamer-environments|. 
%
%
% \subsection*{Changes between versions 2.8 and 3.0}
%
% New command |\NewPitonLanguage|. Thanks to that command, it's now possible to
% define new computer languages with the syntax used by \pkg{listings}.
% Therefore, it's possible to say that virtually all the computer languages
% are now supported by \pkg{piton}.
%
% \subsection*{Changes between versions 2.7 and 2.8}
%
% The key |path| now accepts a \emph{list} of paths where the files to include
% will be searched.
%
% New commands |\PitonInputFileT|, |\PitonInputFileF| and |\PitonInputFileTF|.
%
% \subsection*{Changes between versions 2.6 and 2.7}
%
% New keys |split-on-empty-lines| and |split-separation|
%
% \subsection*{Changes between versions 2.5 and 2.6}
%
% API: |piton.last_code| and |\g_piton_last_code_tl| are provided.
%
% \subsection*{Changes between versions 2.4 and 2.5}
%
% New key |path-write|
%
% \subsection*{Changes between versions 2.3 and 2.4}
%
% The key |identifiers| of the command |\PitonOptions| is now deprecated and
% replaced by the new command |\SetPitonIdentifier|.
%
% A new special language called ``minimal'' has been added.
%
% New key |detected-commands|.
%
% \subsection*{Changes between versions 2.2 and 2.3}
%
% New key |detected-commands|
% 
% The variable |\l_piton_language_str| is now public.
% 
%
% New key |write|.
% 
% \subsection*{Changes between versions 2.1 and 2.2}
%
% New key |path| for |\PitonOptions|.
%
% New language SQL.
%
% It's now possible to define styles locally to a given language (with the
% optional argument of |\SetPitonStyle|). 
%
% \subsection*{Changes between versions 2.0 and 2.1}
%
% The key |line-numbers| has now subkeys |line-numbers/skip-empty-lines|,
% |line-numbers/label-empty-lines|, etc. 
%
% The key |all-line-numbers| is deprecated: use
% |line-numbers/skip-empty-lines=false|. 
%
% New system to import, with |\PitonInputFile|, only a part (of the file)
% delimited by textual markers. 
%
% New keys |begin-escape|, |end-escape|, |begin-escape-math| and |end-escape-math|.
%
% The key |escape-inside| is deprecated: use |begin-escape| and |end-escape|.
%
%
% \section*{Acknowledgments}
%
% Acknowledgments to Yann Salmon for its numerous suggestions of improvments.
%
% \tableofcontents
%
% \end{document}
% 
% Local Variables:
% fill-column: 80
% End:
