% -*- coding: utf-8 ; -*-
\documentclass[dvipsnames]{article}

\usepackage{fontspec}

\usepackage[french]{babel}
\frenchsetup{og = « , fg = »}

\usepackage[footnotehyper]{piton}
\PitonOptions
  { 
    splittable = 4 ,
    math-comments,
    begin-escape = ! ,
    end-escape = ! ,
    begin-escape-math = \( , 
    end-escape-math = \) 
  }



\usepackage[executable=python.exe]{pyluatex}

\usepackage{xcolor}

\usepackage{geometry}
\geometry{left=2.8cm,right=2.8cm,top=2.5cm,bottom=2.5cm,papersize={21cm,29.7cm}}

\usepackage{enumitem}
\usepackage{verbatim}
\usepackage{amsmath}
\usepackage{tabularx}
\usepackage{booktabs}
\usepackage{tcolorbox}
\usepackage{luacolor,lua-ul}

\usepackage{caption}

% We use \MakeShortVerb of shortvrb and not \DefineShortVerb of fancyvrb
% because we don't want the contents of short verbatim colored in gray
\usepackage{shortvrb}
\MakeShortVerb{\|}

\usepackage{varwidth}

\skip\footins = 2\bigskipamount

\usepackage{fancyvrb}
\fvset{commandchars=\~\#\@,formatcom=\color{gray}}
\def\emphase{\bgroup\color{RoyalPurple}\let\next=}


\usepackage{titlesec}
\titlespacing*{\section}{0pt}{6.5ex plus 1ex minus .2ex}{4.3ex plus .2ex}
\titlespacing*{\subsection}{0pt}{4.5ex plus 1ex minus .2ex}{2ex plus .2ex}

\def\interitem{\vspace{7mm plus 2 mm minus 3mm}}          


\usepackage[hyperfootnotes = false]{hyperref}

\hypersetup
  {
    pdfinfo = 
      {
        Title = L’extension piton ,
        Subject = Une extension LaTeX ,
        Author = F. Pantigny 
      }
  } 


\NewDocumentCommand {\pkg} {m} {\textsf{#1}}
\NewDocumentCommand {\cls} {m} {\textsf{#1}}

\setlength{\parindent}{0pt}

\def\CC{{C\nolinebreak[4]\hspace{-.05em}\raisebox{.4ex}{\tiny\bfseries ++}}}

\begin{document}



\VerbatimFootnotes


\title{L'extension LaTeX \pkg{piton}\thanks{Ce document correspond à la 
version~\myfileversion\space de \pkg{piton}, à la date du~\myfiledate.}} 
\author{F. Pantigny \\ \texttt{fpantigny@wanadoo.fr}}

\maketitle

\begin{abstract}
L'extension \pkg{piton} propose des outils pour composer des codes informatiques en Python, OCaml et C avec une
coloration syntaxique en utilisant la bibliothèque Lua LPEG. L'extension \pkg{piton} nécessite l'emploi de
LuaLaTeX.
\end{abstract}



\section{Présentation}

L'extension \pkg{piton} utilise la librairie Lua nommée LPEG\footnote{LPEG est une librairie de capture de motifs
  (\emph{pattern-matching} en anglais) pour Lua, écrite en C, fondée sur les PEG (\emph{parsing expression
    grammars}): \url{http://www.inf.puc-rio.br/~roberto/lpeg/}} pour «parser» le code Python, OCaml ou C et le
composer avec un coloriage syntaxique. Comme elle utilise du code Lua, elle fonctionne uniquement avec |lualatex|
(et ne va pas fonctionner avec les autres moteurs de compilation LaTeX, que ce soit |latex|, |pdflatex| ou
|xelatex|). Elle n'utilise aucun programme extérieur et la compilation ne requiert donc pas |--shell-escape|. La
compilation est très rapide puisque tout le travail du parseur est fait par la librairie LPEG, écrite en C.

\bigskip
Voici un exemple de code Python composé avec l'environnement |{Piton}| proposé par \pkg{piton}.


\bigskip
\begin{Piton}
from math import pi

def arctan(x,n=10):
    """Calcule la valeur mathématique de arctan(x)

    n est le nombre de termes de la somme
    """
    if x < 0:
        return -arctan(-x) # appel récursif
    elif x > 1: 
        return pi/2 - arctan(1/x) 
        #> (on a utilisé le fait que $\arctan(x)+\arctan(1/x)=\pi/2$ pour $x>0$)\footnote{Cet échappement vers LaTeX a été obtenu en débutant par \ttfamily\#>.} 
    else: 
        s = 0
        for k in range(n):
            s += (-1)**k/(2*k+1)*x**(2*k+1)
        return s 
\end{Piton}



\section{Installation}

L'extension \pkg{piton} est composée de deux fichiers : |piton.sty| et |piton.lua| (le fichier LaTeX |piton.sty|
chargé par |\usepackage| va à son tour charger le fichier |piton.lua|). Les deux fichiers doivent être présents
dans un répertoire où LaTeX pourra les trouver, de préférence dans une arborescence |texmf|. Le mieux reste néanmoins
d'installer \pkg{piton} avec une distribution TeX comme MiKTeX, TeX~Live ou MacTeX.



\section{Utilisation de l'extension}

\subsection{Choix du langage}

L'extension \pkg{piton} prend en charge trois langages informatiques : Python, OCaml et C, ou plutôt \CC.

\smallskip
Par défaut, le language est Python.

\smallskip
On peut changer de langage avec la clé |language| de |\PitonOptions| : 

\smallskip
|\PitonOptions{language = C}|

\smallskip
Dans la suite de ce document, on parlera de Python mais les fonctionnalités s'appliquent aussi aux autres langages.



\subsection{Chargement de l'extension}

L'extension \pkg{piton} se charge simplement avec la commande |\usepackage| : |\usepackage{piton}|.

On a cependant deux remarques à formuler :
\begin{itemize}
\item l'extension \pkg{piton} utilise l'extension \pkg{xcolor} (mais \pkg{piton} ne charge pas \pkg{xcolor} : si
\pkg{xcolor} n'est pas chargée avant le |\begin{document}|, une erreur fatale sera levée) ;

\item l'extension \pkg{piton} n'est utilisable qu'avec LuaLaTeX : si un autre moteur de compilation (comme |latex|,
|pdflatex| ou |xelatex|) est utilisé, une erreur fatale sera levée.
\end{itemize}

\subsection{Les commandes et environnements à la disposition de l'utilisateur}

L'extension \pkg{piton} fournit plusieurs outils pour composer du code Python: les commandes |\piton|,
l'environnement |{Piton}| et la commande |\PitonInputFile|. 

\begin{itemize} \setlength{\fboxsep}{1pt}
\item La commande \colorbox{gray!20}{\ttfamily \textbackslash piton} doit être utilisée pour composer de petits éléments de code à l'intérieur d'un
paragraphe. Par exemple : 

{\color{gray}\verb|\piton{def carré(x): return x*x}|}\qquad 
\piton{def carré(x): return x*x}

La syntaxe et les particularités de la commande sont détaillées ci-après.


\item L'environnement \colorbox{gray!20}{\ttfamily \{Piton\}} doit être utilisé pour composer des codes de
plusieurs lignes. Comme cet environnement prend son argument selon un mode verbatim, il ne peut pas être utilisé
dans l'argument d'une commande LaTeX. Pour les besoins de personnalisation, il est possible de définir de nouveaux
environnements similaires à |{Piton}| en utilisant la commande |\NewPitonEnvironment| : cf. partie
\ref{NewPitonEnvironment} p.~\pageref{NewPitonEnvironment}.

\item La commande \colorbox{gray!20}{\ttfamily \textbackslash PitonInputFile} doit être utilisée pour insérer et
composer un fichier extérieur.

Il est possible de n'insérer qu'une partie du fichier : cf. partie~\ref{part-of-a-file}, p.~\pageref{part-of-a-file}.
\end{itemize}


\subsection{La syntaxe de la commande \textbackslash piton}


La commande |\piton| possède en fait une syntaxe double. Elle est peut être utilisée comme une commande standard de
LaTeX prenant son argument entre accolades (|\piton{...}|), ou bien selon la syntaxe de la commande |\verb| où l'argument est
délimité entre deux caractères identiques (par ex. : \verb!\piton|...|!). On détaille maintenant ces deux syntaxes.

\bigskip
\begin{itemize}
\item {\color{blue} \textsf{Syntaxe} \verb|\piton{...}|}\par\nobreak

Quand son argument est donné entre accolades, la commande |\piton| ne prend pas son argument en mode verbatim. Les
points suivants doivent être remarqués :

\begin{itemize}
\item plusieurs espaces successives sont remplacées par une unique espace, 

{\color{cyan} mais la commande |\|␣ est fournie pour forcer l'insertion d'une espace} ;

\item il n'est pas possible d'utiliser le caractère |%| à l'intérieur, 

{\color{cyan} mais la commande |\%| est fournie pour insérer un |%|} ;

\item les accolades doivent apparaître par paires correctement imbriquées, 

{\color{cyan} mais les commandes |\{| et |\}| sont aussi fournies pour insérer des accolades individuelles} ;

\item les commandes LaTeX\footnote{Cela s'applique aux commandes commençant par une contre-oblique |\| mais
  également aux caractères actifs, c'est-à-dire ceux de catcode~13.} sont complètement développées sans être
exécutées

{\color{cyan} et on peut donc utiliser |\\| pour insérer une contre-oblique}.
\end{itemize}

Les autres caractères (y compris |#|, |^|, |_|, |&|, |$| % $
et |@|) doivent être insérés sans contre-oblique.

\bigskip
\begin{tabular}{>{\color{gray}}w{l}{75mm}@{\hspace*{1cm}}l}
\omit Exemples : \hfil \\
\noalign{\vskip1mm}
\verb|\piton{ma_chaîne = '\\n'}| & 
\piton{ma_chaîne = '\\n' } \\
\verb|\piton{def pair(n): return n\%2==0}| & 
\piton{def pair(n): return n\%2==0 } \\
\verb|\piton{c="#"    # une affectation }| & 
\piton{c="#"     # une affectation } \\
\verb|\piton{c="#" \ \ \ # une affectation }| & 
\piton{c="#" \ \ \ # une affectation } \\
\verb|\piton{my_dict = {'a': 3, 'b': 4}}| &
\piton{my_dict = {'a': 3, 'b': 4}}
\end{tabular}

\bigskip
La commande |\piton| avec son argument entre accolades peut être utilisée dans les arguments des autres commandes LaTeX.\footnote{La commande
  |\piton| peut par exemple être
  utilisée dans une note de bas de page. Exemple : \piton{s = 'Une chaîne'}.}

\bigskip
\item {\color{blue} \textsf{Syntaxe} \verb!\piton|...|!}\par\nobreak

Quand la commande |\piton| prend son argument entre deux caractères identiques, cet argument est pris \emph{en mode
  verbatim}. De ce fait, avec cette syntaxe, la commande |\piton| ne peut \emph{pas} être utilisée dans l'argument d'une
autre fonction.  

\medskip
\begin{tabular}{>{\color{gray}}w{l}{75mm}@{\hspace*{1cm}}l}
\omit Exemples : \hfil \\
\noalign{\vskip1mm}
\verb!\piton|ma_chaîne = '\n'|! & 
\piton|ma_chaîne = '\n'| \\
\verb|\piton!def pair(n): return n%2==0!| & 
\piton!def pair(n): return n%2==0! \\
\verb|\piton+c="#"    # une affectation +| & 
\piton+c="#"     # une affectation + \\
\verb|\piton?my_dict = {'a': 3, 'b': 4}?| &
\piton!my_dict = {'a': 3, 'b': 4}!
\end{tabular}

\end{itemize}



\section{Personnalisation}

Concernant la fonte de caractères utilisée dans les listings produits par l'extension \pkg{piton}, il s'agit
simplement de la fonte mono-chasse courante (\pkg{piton} utilise simplement en interne la commande LaTeX standard
|\ttfamily|). Pour la changer, le mieux est d'utiliser |\setmonofont| de \pkg{fontspec}.

\subsection{Les clés de la commande \textbackslash PitonOptions}

\NewDocumentCommand{\Definition}{m}
  {{\setlength{\fboxsep}{1pt}\colorbox{gray!20}{\ttfamily \vphantom{gl}#1}}}


La commande |\PitonOptions| prend en argument une liste de couples \textsl{clé=valeur}. La portée des réglages
effectués par cette commande est le groupe TeX courant.\footnote{On rappelle que tout environnement LaTeX est, en
  particulier, un groupe.}

Ces clés peuvent aussi être appliquées à un environnement |{Piton}| individuel (entre crochets).

\begin{itemize}
\item La clé \Definition{language} spécifie le langage informatique considéré (la casse n'est pas prise en compte).
Trois valeurs sont possibles : |Python|, |OCaml| et |C|. La valeur initiale est |Python|.

\item La clé \Definition{gobble} prend comme valeur un entier positif $n$ : les $n$ premiers caractères de chaque
ligne sont alors retirés (avant formatage du code) dans les environnements |{Piton}|. Ces $n$ caractères ne sont
pas nécessairement des espaces.

\item Quand la clé \Definition{auto-gobble} est activée, l'extension \pkg{piton} détermine la valeur minimale $n$
du nombre d'espaces successifs débutant chaque ligne (non vide) de l'environnement |{Piton}| et applique |gobble|
avec cette valeur de~$n$.

\item Quand la clé \Definition{env-gobble} est activée, \pkg{piton} analyse la dernière ligne de l'environnement,
c'est-à-dire celle qui contient le |\end{Piton}| et détermine si cette ligne ne comporte que des espaces suivis par
|\end{Piton}|. Si c'est le cas, \pkg{piton} calcule le nombre $n$ de ces espaces et applique |gobble| avec cette
valeur de~$n$. Le nom de cette clé vient de \emph{environment gobble}: le nombre d'espaces à retirer ne dépend que
de la position des délimiteurs |\begin{Piton}| et |\end{Piton}| de l'environnement.

\item La clé \Definition{line-numbers} active la numérotation des lignes (en débordement à gauche) dans les
environnements \verb|{Piton}| et dans les listings produits par la commande |\PitonInputFile|.

\colorbox{yellow!50}{\textbf{Nouveau 2.1}}\enskip Cette clé propose en fait plusieurs sous-clés.
\begin{itemize}
\item La clé \Definition{line-numbers/skip-empty-lines} demande que les lignes vides soient considérées comme non existantes en
ce qui concerne la numérotation des lignes (si la clé \verb|/absolute| est active, la clé \verb|/skip-empty-lines|
n'a pas d'effet dans \verb|\PitonInputFile|). La valeur initiale de cette clé est \verb|true| (et non
\verb|false|).\footnote{Avec le language Python, les lignes vides des \emph{docstrings} sont prises en compte.}

\item La clé \Definition{line/numbers/label-empty-lines} demande que les labels (c'est-à-dire les numéros) des lignes vides
soient affichés. Si la clé \verb|/skip-empty-lines| est active, la clé \verb|/label-empty-lines| est sans effet. La
valeur initiale de cette clé est \verb|true|.

\item La clé \Definition{line-numbers/absolute} demande, pour les listings générés par \verb|\PitonInputFile|, que les numéros de
lignes affichés soient absolus (c'est-à-dire ceux du fichier d'origine). Elle n'a d'intérêt que si on insère qu'une
partie du fichier (cf. part~\ref{part-of-a-file}, p.~\pageref{part-of-a-file}). La clé |/absolute| est sans effet dans les environnements \verb|{Piton}|.

\item La clé \Definition{line-numbers/resume} reprend la numérotation là où elle avait été laissée au dernier listing. En fait,
la clé \verb|line-numbers/resume| a un alias, qui est \verb|resume| tout court. 

\item La clé \Definition{line-numbers/start} impose que la numérotation commence à ce numéro. Elle n'est pas disponible dans
\verb|\PitonOptions|. 

\item La clé \Definition{sep} est la distance horizontale entre les numéros de lignes (insérés par |line-numbers|)
et les lignes du code informatique. La valeur initiale est 0.7~em.
\end{itemize}

Pour la commmodité, un dispositif de factorisation du préfixe |line-numbers| est disponible, c'est-à-dire que l'on
peut écrire :

\begin{Verbatim}
\PitonOptions
  {
    line-numbers = 
      { 
        skip-empty-lines = false ,
        label-empty-lines = false ,
        sep = 1 em
      }
  }
\end{Verbatim}


\item La clé \Definition{left-margin} fixe une marge sur la gauche. Cette clé peut être utile, en particulier, en
conjonction avec la clé |line-numbers| si on ne souhaite pas que les numéros de ligne soient dans une position en
débordement sur la gauche.

Il est possible de donner à la clé |left-margin| la valeur spéciale~|auto|. Avec cette valeur, une marge est
insérée automatiquement pour les numéros de ligne quand la clé |line-numbers| est utilisée. Voir un exemple à la
partie \ref{example-numbering} p.~\pageref{example-numbering}.

\item La clé \Definition{background-color} fixe la couleur de fond des environnements |{Piton}| et des listings
produits par |\PitonInputFile| (ce fond a une largeur que l'on peut fixer avec la clé |width| décrite ci-dessous).
La clé |background-color| accepte une couleur définie «à la volée», c'est-à-dire que l'on peut écrire par exemple
|background-color = [cmyk]{0.1,0.05,0,0}|

\smallskip
La clé |background-color| accepte aussi en argument une \emph{liste} de couleurs. Les lignes sont alors coloriées
de manière cyclique avec ces couleurs.

\emph{Exemple} : |\PitonOptions{background-color = {gray!5,white}}|

\item Avec la clé \Definition{prompt-background-color}, \pkg{piton} ajoute un fond coloré aux lignes débutant par
le prompt «|>>>|» (et sa continuation «|...|») caractéristique des consoles Python avec boucle \textsc{repl}
(\emph{read-eval-print loop}). Pour un exemple d'utilisation de cette clé, voir la partie \ref{pythonrepl}
p.~\pageref{pythonrepl}.

\item La clé \Definition{width} fixe la largeur du listing produit. Cette largeur s'applique aux fonds colorés
spécifiés par les clés |background-color| et |prompt-background-color| et également quand une coupure automatique
des lignes est demandée par |break-lines| (cf.~\ref{line-breaks}, p.~\pageref{line-breaks}).

Cette clé peut prendre comme valeur une longueur explicite mais aussi la valeur spéciale~|min|. Avec cette valeur,
la largeur sera calculée à partir de la largeur maximale des lignes de code. Attention : l'usage de cette valeur
spéciale~|min| requiert deux compilations LuaLaTeX\footnote{La largeur maximale est calculée lors de la première
compilation, écrite sur le fichier~|aux|, puis réutilisée lors de la compilation suivante. Certains outils comme
|latexmk| (utilisé par Overleaf) effectuent automatiquement un nombre suffisant de compilations.}.

Pour un exemple d'utilisation de |width=min|, voir la partie~\ref{example-comments} sur les exemples, p.~\pageref{example-comments}.

\item En activant la clé \Definition{show-spaces-in-strings}, les espaces dans les chaînes courtes (c'est-à-dire
celles délimitées par |'| ou |"|) sont matérialisés par le caractère ␣ (U+2423 : \textsc{open box}). Bien sûr, le
caractère U+2423 doit être présent dans la fonte mono-chasse utilisée.\footnote{L'extension \pkg{piton} utilise
  simplement la fonte mono-chasse courante. Pour la changer, le mieux est d'utiliser |\setmonofont| de
  \pkg{fontspec}.}\par\nobreak
%
\begingroup
\PitonOptions{show-spaces-in-strings}
Exemple : \piton{my_string = 'Très bonne réponse'}
\endgroup

\smallskip
Avec la clé \Definition{show-spaces}, tous les espaces sont matérialisés (et aucune coupure de ligne ne peut plus
intervenir sur ces espaces matérialisés, même si la clé |break-lines|\footnote{cf. \ref{line-breaks} p.~\pageref{line-breaks}.} est active).
\end{itemize}

\bigskip

\begingroup
\fvset{commandchars=\~\&\@,formatcom=\small\color{gray}}
\begin{Verbatim}
~emphase&\begin{Piton}[language=C,line-numbers,auto-gobble,background-color = gray!15]@
    void bubbleSort(int arr[], int n) {
        int temp;
        int swapped;
        for (int i = 0; i < n-1; i++) {
            swapped = 0;
            for (int j = 0; j < n - i - 1; j++) {
                if (arr[j] > arr[j + 1]) {
                    temp = arr[j];
                    arr[j] = arr[j + 1];
                    arr[j + 1] = temp;
                    swapped = 1; 
                }
            }
            if (!swapped) break;
        }
    }   
\end{Piton}
\end{Verbatim}
\endgroup


\begin{Piton}[language=C,line-numbers,auto-gobble,background-color = gray!15]
    void bubbleSort(int arr[], int n) {
        int temp;
        int swapped;
        for (int i = 0; i < n-1; i++) {
            swapped = 0;
            for (int j = 0; j < n - i - 1; j++) {
                if (arr[j] > arr[j + 1]) {
                    temp = arr[j];
                    arr[j] = arr[j + 1];
                    arr[j + 1] = temp;
                    swapped = 1; 
                }
            }
            if (!swapped) break;
        }
    }   
\end{Piton}




\bigskip
La commande |\PitonOptions| propose d'autres clés qui seront décrites plus loin (voir en particulier la coupure des
pages et des lignes p.~\pageref{breakable}).


\subsection{Les styles}

\label{styles}

L'extension \pkg{piton} fournit la commande |\SetPitonStyle| pour personnaliser les différents styles utilisés pour
formater les éléments syntaxiques des listings Python. Ces personnalisations ont une portée qui correspond au
groupe TeX courant.\footnote{On rappelle que tout environnement LaTeX est, en particulier, un groupe.}

\bigskip
La commande |\SetPitonStyle| prend en argument une liste de couples \textsl{clé=valeur}. Les clés sont les noms des
styles et les valeurs sont les instructions LaTeX de formatage correspondantes.

\bigskip
Ces instructions LaTeX doivent être des instructions de formatage du type de |\bfseries|, |\slshape|,
|\color{...}|, etc. (les commandes de ce type sont parfois qualifiées de \emph{semi-globales}). Il est aussi
possible de mettre, \emph{à la fin de la liste d'instructions}, une commande LaTeX prenant exactement un argument.

\bigskip
Voici un exemple qui change le style utilisé pour le nom d'une fonction Python, au moment de sa définition
(c'est-à-dire après le mot-clé |def|). Elle utilise la commande |\highLight| de \pkg{lua-ul} (qui nécessite
lui-même le chargement de \pkg{luacolor}).

\begin{Verbatim}
\SetPitonStyle
  { Name.Function = \bfseries \hightLight[red!50] }
\end{Verbatim}

Ici, |\highLight[red!50]| doit être considéré comme le nom d'une fonction LaTeX qui prend exactement un argument,
puisque, habituellement, elle est utilisée avec |\highLight[red!50]{|\texttt{\slshape text}|}|.

\medskip
\begingroup
\SetPitonStyle 
  { Name.Function = \bfseries \highLight[red!50] }
Avec ce réglage, on obtient : \piton{def cube(x) : return x * x * x }
\endgroup


\bigskip
Les différents styles sont décrits dans la table \ref{Semantic}. Les réglages initiaux effectués par \pkg{piton}
dans |piton.sty| sont inspirés par le style \pkg{manni} de Pygments.\footnote{Voir
  \url{https://pygments.org/styles/}. À remarquer que, par défaut, Pygments propose pour le style \pkg{manni} un
  fond coloré dont la couleur est la couleur HTML |#F0F3F3|. Il est possible d'avoir la même couleur dans |{Piton}|
  avec l'instruction : |\PitonOptions{background-color = [HTML]{F0F3F3}}|}


\bigskip
La commande |\PitonStyle| prend en argument le nom d'un style et permet de récupérer la valeur (en tant que liste
d'instructions LaTeX) de ce style.

\smallskip
Par exemple, on peut écrire, dans le texte courant, |{\PitonStyle{Keyword}{function}}| et on aura 
le mot {\PitonStyle{Keyword}{function}} formaté comme un mot-clé.

\smallskip
La syntaxe |{\PitonStyle{|\textsl{\texttt{style}}|}{...}}| est nécessaire pour pouvoir tenir compte à la fois des commandes
semi-globales et des commandes à argument présentes dans la valeur courante du style \texttt{\textsl{style}}.


\bigskip
Il existe un style spécial nommé |UserFunction|. Ce style s'applique aux noms des fonctions précédemment définies 
par l'utilisateur via une instruction Python \piton{def} dans l'un des listings précédents. La valeur initiale de
ce style est nulle (=vide), ce qui fait que ces noms de fonctions sont formatés comme du texte courant (en noir).
Néanmoins, il est possible de changer la valeur de ce style, comme tous les autres styles, avec la commande
|\SetPitonStyle|.


\medskip
Dans l'exemple suivant, on prend comme valeur de ce style |UserFunction| la valeur initiale du style
|Name.Function| (qui s'applique au nom des fonctions \emph{au moment de leur définition}).

\begingroup

\begin{Verbatim}
\SetPitonStyle{~emphase#UserFunction@ = \color[HTML]{CC00FF}}
\end{Verbatim}

\SetPitonStyle{UserFunction = \color[HTML]{CC00FF}}

\begin{Piton}
def transpose(v,i,j):
    x = v[i]
    v[i] = v[j]
    v[j] = x 

def passe(v):
    for in in range(0,len(v)-1):
        if v[i] > v[i+1]:
            transpose(v,i,i+1)
\end{Piton}

\endgroup

\medskip
Comme on le voit, le nom |transpose| a été mis en évidence comme étant le nom d'une fonction définie précédemment
par l'utilisateur (d'où le nom |UserFunction| pour ce style).

\PitonClearUserFunctions

\bigskip
Bien sûr, la liste des noms de fonctions Python précédemment définies est gardée en mémoire de LuaLaTeX (de manière
globale, c'est-à-dire indépendamment des groupes TeX). L'extension \pkg{piton} fournit une commande qui permet de
vider cette liste : c'est la commande |\PitonClearUserFunctions|.


\subsection{Définition de nouveaux environnements}

\label{NewPitonEnvironment}

Comme l'environnement |{Piton}| a besoin d'absorber son contenu d'une manière spéciale (à peu près comme du texte
verbatim), il n'est pas possible de définir de nouveaux environnements directement au-dessus de l'environnement
|{Piton}| avec les commandes classiques |\newenvironment| et |\NewDocumentEnvironment|.

C'est pourquoi \pkg{piton} propose une commande |\NewPitonEnvironment|. Cette commande a la même syntaxe que la
commande classique |\NewDocumentEnvironment|.

\bigskip
Par exemple, avec l'instruction suivante, un nouvel environnement |{Python}| sera défini avec le même comportement
que l'environnement |{Piton}|:

{\color{gray}\verb|\NewPitonEnvironment{Python}{O{}}{\PitonOptions{#1}}{}|}

\bigskip
Si on souhaite un environnement |{Python}| qui compose le code inclus dans une boîte de \pkg{tcolorbox}, on peut
écrire (à condition, bien entendu, d'avoir chargé l'extension \pkg{tcolorbox}):

\begin{Verbatim}
\NewPitonEnvironment{Python}{}
  {\begin{tcolorbox}}
  {\end{tcolorbox}}
\end{Verbatim}

\bigskip
Avec ce nouvel environnement |{Python}|, on peut écrire : 

\begin{Verbatim}
~emphase#\begin{Python}@
def carré(x):
    """Calcule le carré d'un nombre"""
    return x*x
~emphase#\end{Python}@
\end{Verbatim}

\NewPitonEnvironment{Python}{}
  {\begin{tcolorbox}}
  {\end{tcolorbox}}

\begin{Python}
def carré(x):
    """Calcule le carré d'un nombre"""
    return x*x
\end{Python}


\section{Fonctionnalités avancées}


\subsection{Coupure des pages et des lignes}

\label{breakable}

\subsubsection{Coupure des pages}

Par défaut les listings produits par l'environnement |{Piton}| et par la commande |\PitonInputFile| sont
insécables.

Néanmoins, la commande |\PitonOptions| propose la clé \Definition{splittable} pour autoriser de telles coupures.

\begin{itemize}
\item Si la clé |splittable| est utilisée sans valeur, les listings sont sécables n'importe où.

\item Si la clé |splittable| est utilisée avec une valeur numérique $n$ (qui doit être un entier naturel non nul),
alors les listings seront sécables mais aucune coupure ne pourra avoir lieu entre les $n$~premières lignes, ni
entre les $n$~dernières. De ce fait, |splittable=1| est équivalent à |splittable|.
\end{itemize}

\medskip
\emph{Remarque}\par\nobreak

Même avec une couleur de fond (fixée avec |background-color|), les sauts de page sont possibles, à partir du moment
où la clé |splittable| est utilisée.\footnote{Avec la clé |splittable|, un environnement |{Piton}| est sécable même dans un environnement de
  \pkg{tcolorbox} (à partir du moment où la clé |breakable| de \pkg{tcolorbox} est utilisée). On précise cela parce
  que, en revanche, un environnement de \pkg{tcolorbox} inclus dans un autre environnement de \pkg{tcolorbox} n'est
  pas sécable, même quand les deux utilisent la clé |breakable|.}

\subsubsection{Coupure des lignes}

\label{line-breaks}

Par défaut, les éléments produits par \pkg{piton} ne peuvent pas être coupés par une fin de ligne. Il existe
néanmoins des clés pour autoriser de telles coupures (les points de coupure possibles sont les espaces, y compris
les espaces dans les chaînes Python).
\begin{itemize}
\item Avec la clé \Definition{break-lines-in-piton}, les coupures de ligne sont autorisées dans la commande
|\piton{...}| (mais pas dans la commande \verb+\piton|...|+, c'est-à-dire avec la syntaxe verbatim). 

\item Avec la clé \Definition{break-lines-in-Piton}, les coupures de ligne sont autorisées dans l'environnement
|{Piton}| (d'où la lettre |P| capitale dans le nom) et les listings produits par |\PitonInputFile|.

\item La clé \Definition{break-lines} est la conjonction des deux clés précédentes.
\end{itemize}

\medskip
L'extension \pkg{piton} fournit aussi plusieurs clés pour contrôler l'apparence des coupures de ligne autorisées par |break-lines-in-Piton|.

\begin{itemize}
\item Avec la clé \Definition{indent-broken-lines}, l'indentation de la ligne coupée est respectée à chaque retour
à la ligne. 

\item La clé \Definition{end-of-broken-line} correspond au symbole placé à la fin d'une ligne coupée. Sa valeur initiale est : 
|\hspace*{0.5em}\textbackslash|.

\item La clé \Definition{continuation-symbol} correspond au symbole placé à chaque retour de ligne dans la marge
gauche. Sa valeur initiale est : |+\;| (la commande |\;| insère un petit espace horizontal). 

\item La clé \Definition{continuation-symbol-on-indentation} correspond au symbole placé à chaque retour de ligne
au niveau de l'indentation (uniquement dans le cas où la clé |indent-broken-lines| est active). Sa valeur initiale
est : |$\hookrightarrow\;$|.
\end{itemize}

\bigskip
Le code suivant a été composé avec le réglage suivant :  

\begin{Verbatim}
\PitonOptions{width=12cm,break-lines,indent-broken-lines,background-color=gray!15}
\end{Verbatim}

\begin{center}
\PitonOptions{width=12cm,break-lines,indent-broken-lines,background-color=gray!15}
\begin{Piton}
def dict_of_liste(liste):
    """Convertit une liste de subrs et de descriptions de glyphes en dictionnaire"""
    dict = {}
    for liste_lettre in liste:
        if (liste_lettre[0][0:3] == 'dup'): # si c'est un subr
            nom = liste_lettre[0][4:-3]
            print("On traite le subr de numéro " + nom)
        else:
            nom = liste_lettre[0][1:-3] # si c'est un glyphe
            print("On traite le glyphe du caractère " + nom)
        dict[nom] = [traite_ligne_Postscript(k) for k in liste_lettre[1:-1]]
    return dict
\end{Piton}
\end{center}


\bigskip
\subsection{Insertion d'une partie d'un fichier}

\label{part-of-a-file}

La commande |\PitonInputFile| permet d'insérer (avec formatage) le contenu d'un fichier. En fait, il existe des
mécanismes permettant de n'insérer qu'une partie du fichier en question.
\begin{itemize}
\item On peut spécifier la partie à insérer par les numéros de lignes (dans le fichier d'origine).
\item \colorbox{yellow!50}{\textbf{Nouveau 2.1}}\enskip On peut aussi spécifier la partie à insérer par des marqueurs textuels.
\end{itemize}
Dans les deux cas, si on souhaite numéroter les lignes avec les numéros des lignes du fichier d'origine, il
convient d'utiliser la clé |line-numbers/absolute|.

\subsubsection{Avec les numéros de lignes absolus}

La commande |\PitonInputFile| propose les clés \Definition{first-line} et \Definition{last-line} pour n'insérer que
la partie du fichier comprise entre les lignes correspondantes. Ne pas confondre avec la clé |line-numbers/start|
qui demande un numérotage des lignes commençant à la valeur donnée à cette clé (en un sens |line-numbers/start|
concerne la sortie alors que |first-line| et |last-line| concernent l'entrée).


\subsubsection{Avec des marqueurs textuels}

\colorbox{yellow!50}{\textbf{Nouveau 2.1}}

Pour utiliser cette technique, il convient d'abord de spécifier le format des marqueurs marquant le début et la fin
de la partie du fichier à inclure. Cela se fait avec les deux clés \Definition{marker/beginning} et
\Definition{marker/end} (usuellement dans la commande |\PitonOptions|). 

\medskip
Prenons d'abord un exemple.

\medskip
Supposons que le fichier à inclure contienne des solutions à des exercices de programmation sur le modèle suivant :  

\begin{Verbatim}[formatcom=\small\color{gray}]
~#[Exercice 1] Version itérative
def fibo(n):
    if n==0: return 0 
    else:
        u=0
        v=1
        for i in range(n-1):
            w = u+v
            u = v
            v = w
        return v
~#<Exercice 1>
\end{Verbatim}

Les marqueurs de début de début et de fin sont les chaînes |#[Exercice 1]| et |#<Exercice 1>|. La chaîne
«|Exercice 1|» sera appelée le \emph{label} de l'exercice (ou de la partie du fichier à inclure).

Pour spécifier des marqueurs de cette sorte dans \pkg{piton}, on utilisera les clés |marker/beginning| et |marker/end|
de la manière suivante (le caractère |#| des commentaires de Python doit être inséré sous la forme échappée |\#|). 

\begin{Verbatim}
\PitonOptions{ ~emphase#marker/beginning@ = \~#[~#1] , ~emphase#marker/end@ = \~#<~#1> } 
\end{Verbatim}

Comme on le voit, |marker/beginning| est une expression correspondant à la fonction mathématique qui, au nom du
label (par exemple |Exercice 1|), associe le marqueur de début (dans l'exemple |#[Exercice 1]|). La chaîne |#1|
correspond aux occurrences de l'argument de cette fonction (c'est la syntaxe habituelle de TeX). De même pour
|marker/end|.


\bigskip
Pour insérer une partie marquée d'un fichier, il suffit alors d'utiliser la clé \Definition{range} de
|\PitonInputFile|. 

\smallskip
\begin{Verbatim}
\PitonInputFile[~emphase#range = Exercice 1@]{~textsl#nom_du_fichier@}
\end{Verbatim}

\medskip
\begin{Piton}
def fibo(n):
    if n==0: return 0 
    else:
        u=0
        v=1
        for i in range(n-1):
            w = u+v
            u = v
            v = w
        return v
\end{Piton}

\vspace{1cm}
La clé \Definition{marker/include-line} demande que les lignes contenant les marqueurs soient également insérées.

\begin{Verbatim}
\PitonInputFile[~emphase#marker/include-lines@,range = Exercice 1]{~textsl#nom_du_fichier@}
\end{Verbatim}

\begin{Piton}
#[Exercice 1] Version itérative
def fibo(n):
    if n==0: return 0 
    else:
        u=0
        v=1
        for i in range(n-1):
            w = u+v
            u = v
            v = w
        return v
#<Exercice 1>
\end{Piton}


\bigskip
Il existe en fait aussi les clés \Definition{begin-range} et \Definition{end-range} pour insérer plusieurs
contenus marqués simultanément. 

Par exemple, pour insérer les solutions des exercices 3 à 5, on pourra écrire (à condition que le fichier soit
structuré correctement!):

\begin{Verbatim}
\PitonInputFile[~emphase#begin-range = Exercice 3, end-range = Exercice 5@]{~textsl#nom_du_fichier@}
\end{Verbatim}




\subsection{Mise en évidence d'identificateurs}

On peut demander le changement de formatage de certains identificateurs avec la clé |identifiers| de
|\PitonOptions|.  

\smallskip
Cette clé prend en argument une valeur au format suivant : 

\quad |{ names = |\textsl{\ttfamily noms}|, style = |\textsl{\ttfamily instructions}| }|

\begin{itemize}
\item \textsl{\ttfamily noms} est une liste de noms d'identificateurs (séparés par des virgules) ; 

\item \textsl{\ttfamily instructions} est une liste d'instructions LaTeX de formatage du même type que pour les
styles précédemment définis (cf.~\ref{styles} p.~\pageref{styles}).
\end{itemize}

\emph{Attention} : Seuls les identifiants peuvent voir leur formatage affecté. Les mots-clés et les noms de
fonctions prédéfinies ne seront pas affectés, même s'ils figurent dans la liste \textsl{\ttfamily noms}.

\begin{Verbatim}
\PitonOptions 
  {
    ~emphase#identifiers@ = 
     {
       ~emphase#names@ = { l1 , l2 } , 
       ~emphase#style@ = \color{red} 
     }
  }

\begin{Piton}
def tri(l):
    """Tri par segmentation"""
    if len(l) <= 1:
        return l
    else:
        a = l[0]
        l1 = [ x for x in l[1:] if x < a  ]
        l2 = [ x for x in l[1:] if x >= a ]
        return tri(l1) + [a] + tri(l2)
\end{Piton}
\end{Verbatim}


\bigskip

\begingroup

\PitonOptions 
  {
    identifiers = 
     {
       names = { l1 , l2 } , 
       style = \color{red} 
     }
  }

\begin{Piton}
def tri(l):
    """Tri par segmentation"""
    if len(l) <= 1:
        return l
    else:
        a = l[0]
        l1 = [ x for x in l[1:] if x < a  ]
        l2 = [ x for x in l[1:] if x >= a ]
        return tri(l1) + [a] + tri(l2)
\end{Piton}

\endgroup

\bigskip
En utilisant la clé |identifiers|, on peut ajouter de nouvelles fonctions prédéfinies (ou de nouveaux mots-clés,
etc.) qui seront détectées par \pkg{piton}.


\begin{Verbatim}
\PitonOptions 
  {
    ~emphase#identifiers@ = 
     {
       names = { cos, sin, tan, floor, ceil, trunc, pow, exp, ln, factorial } , 
       style = \PitonStyle{Name.Builtin}
     }
  }

\begin{Piton}
from math import *
cos(pi/2) 
factorial(5)
ceil(-2.3) 
floor(5.4) 
\end{Piton}
\end{Verbatim}

\begingroup

\PitonOptions 
  {
    identifiers = 
     {
       names = { cos, sin, tan, floor, ceil, trunc, pow, exp, ln, factorial } , 
       style = \PitonStyle{Name.Builtin}
     }
  }

\begin{Piton}
from math import *
cos(pi/2) 
factorial(5)
ceil(-2.3) 
floor(5.4) 
\end{Piton}


\endgroup

\subsection{Les échappements vers LaTeX}

L'extension \pkg{piton} propose plusieurs mécanismes d'échappement vers LaTeX :
\begin{itemize}
\item Il est possible d'avoir des commentaires entièrement composés en LaTeX.
\item Il est possible d'avoir, dans les commentaires Python, les éléments entre \texttt{\$} composés en mode
mathématique de LaTeX.
\item Il est possible d'insérer du code LaTeX à n'importe quel endroit d'un listing Python.
\end{itemize}
Ces mécanismes vont être détaillés dans les sous-parties de cette partie.

\smallskip
À remarquer également que, dans le cas où \pkg{piton} est utilisée dans la classe \cls{beamer}, \pkg{piton} détecte
la plupart des commandes et environnements de Beamer : voir la sous-section \ref{beamer}, p.~\pageref{beamer}.


\subsubsection{Les «commentaires LaTeX»}

Dans ce document, on appelle «commentaire LaTeX» des commentaires qui débutent par |#>|. Tout ce qui suit ces deux 
caractères, et jusqu'à la fin de la ligne, sera composé comme du code LaTeX standard.

Il y a deux outils pour personnaliser ces commentaires.

\begin{itemize}
\item Il est possible de changer le marquage syntaxique utilisé (qui vaut initialement~|#>|). Pour ce faire, il
existe une clé |comment-latex|, disponible uniquement dans le préambule du document, qui permet de choisir les
caractères qui (précédés par~|#|) serviront de marqueur syntaxique.

Par exemple, avec le réglage suivant (fait dans le préambule du document) :

\verb|\PitonOptions{comment-latex = LaTeX}|

les commentaires LaTeX commenceront par~|#LaTeX|.

Si on donne la valeur nulle à la clé |comment-latex|, tous les commentaires Python (débutant par~|#|) seront en
fait des «commentaires LaTeX».

\smallskip
\item Il est possible de changer le formatage du commentaire LaTeX lui-même en changeant le style \pkg{piton}
|Comment.LaTeX|.

Par exemple, avec |\SetPitonStyle{Comment.LaTeX = \normalfont\color{blue}}|, les commentaires LaTeX seront composés
en bleu.

Si on souhaite qu'un croisillon (|#|) soit affiché en début de commentaire dans le \textsc{pdf}, on peut régler
|Comment.LaTeX| de la manière suivante :

\begin{Verbatim}
\SetPitonStyle{Comment.LaTeX = \color{gray}\~#\normalfont\space }
\end{Verbatim}

Pour d'autres exemples de personnalisation des commentaires LaTeX, voir la partie \ref{example-comments}
p.~\pageref{example-comments}. 
\end{itemize}


\bigskip 
Si l'utilisateur a demandé l'affichage des numéros de ligne avec |line-numbers|, il est possible de faire référence
à ce numéro de ligne avec la commande |\label| placée dans un commentaire LaTeX.\footnote{Cette fonctionnalité est
  implémentée en redéfinissant, dans les environnements |{Piton}|, la commande |\label|. Il peut donc y avoir des
  incompatibilités avec les extensions qui redéfinissent (globalement) cette commande |\label| (comme
  \pkg{varioref}, \pkg{refcheck}, \pkg{showlabels}, etc.)}




\subsubsection{La clé «math-comments»}

Il est possible de demander que, dans les commentaires Python normaux, c'est-à-dire débutant par~|#| (et non par
|#>|), les éléments placés entre symboles \texttt{\$} soient composés en mode mathématique de LaTeX (le reste du
commentaire restant composé en verbatim).

La clé \Definition{math-comments} (qui ne peut être activée que dans le préambule du document) active ce
comportement.

\bigskip
Dans l'exemple suivant, on suppose que |\PitonOptions{math-comments}| a été utilisé dans le préambule du document.

\begin{Verbatim}
\begin{Piton}
def carré(x):
    return x*x ~# renvoie $x^2$
\end{Piton}
\end{Verbatim}

\begin{Piton}
def carré(x):
    return x*x # renvoie $x^2$
\end{Piton}


\subsubsection{Le mécanisme «escape»}

Il est aussi possible de surcharger les listings Python pour y insérer du code LaTeX à peu près n'importe où (mais
entre deux lexèmes, bien entendu). Cette fonctionnalité n'est pas activée par défaut par \pkg{piton}. Pour
l'utiliser, il faut spécifier les deux délimiteurs marquant l'échappement (le premier le commençant et le deuxième
le terminant) en utilisant les clés \Definition{begin-escape} et \Definition{end-escape} (qui ne sont accessibles
que dans le préambule du document). Les deux délimiteurs peuvent être identiques.


\bigskip
Dans l'exemple suivant, on suppose que le préambule du document contient l'instruction :

\begin{Verbatim}
\PitonOptions{~emphase#begin-escape=!,end-escape=!@}
\end{Verbatim}

\medskip
Dans le code suivant, qui est une programmation récursive de la factorielle, on décide de surligner en jaune
l'instruction qui contient l'appel récursif. Cet exemple utilise la commande |\highLight| de \pkg{lua-ul} (cette
extension requiert aussi l'extension \pkg{luacolor}).
\begin{Verbatim}
\begin{Piton}
def fact(n):
    if n==0:
        return 1
    else:
        ~emphase#!\highLight{!@return n*fact(n-1)~emphase#!}!@
\end{Piton}
\end{Verbatim}

\begin{Piton}
def fact(n):
    if n==0:
        return 1
    else:
        !\highLight{!return n*fact(n-1)!}!
\end{Piton}

\bigskip
En fait, dans le cas présent, il serait sans doute plus habile d'utiliser la commande |\@highLight| de
\pkg{lua-ul}: cette commande bascule vers un fond jaune jusqu'à la fin du groupe TeX courant. Comme le nom de cette
commande contient le caractère |@|, il convient de lui donner un synonyme sans caractère |@| pour pouvoir
l'utiliser directement dans |{Piton}|.

\begingroup
\fvset{commandchars=\~\#\+,formatcom=\color{gray}}
\begin{Verbatim}
\makeatletter
~emphase#\let\Jaune\@highLight+
\makeatother
\end{Verbatim}
\endgroup

\begin{Verbatim}
\begin{Piton}
def fact(n):
    if n==0:
        return 1
    else:
        ~emphase#!\Jaune!@return n*fact(n-1)
\end{Piton}
\end{Verbatim}

\makeatletter
\let\Jaune\@highLight
\makeatother


\begin{Piton}
def fact(n):
    if n==0:
        return 1
    else:
        !\Jaune!return n*fact(n-1)
\end{Piton}



\bigskip
\emph{Attention} : L'échappement vers LaTeX permis par les clés |begin-escape| et |end-escape| n'est pas actif dans
les chaînes de caractères ni dans les commentaires (pour avoir un commentaire entièrement en échappement vers
LaTeX, c'est-à-dire ce qui est appelé dans ce document «commentaire LaTeX», il suffit de le faire débuter par
|#>|).


\subsubsection{Le mécanisme «escape-math»}

Le mécanisme «|escape-math|» est très similaire au mécanisme «|escape|» puisque la seule différence est que les
éléments en échappement LaTeX y sont composés en mode mathématique.

On active ce mécanisme avec les clés \Definition{begin-escape-math} et \Definition{end-escape-math} (qui ne sont
accessibles que dans le préambule du document).

\medskip
Malgré la proximité technique, les usages du mécanisme «|escape-math|» sont en fait assez différents de ceux du
mécanisme «|escape|». En effet, comme le contenu en échappement est composé en mode mathématique, il est en
particulier composé dans un groupe TeX et ne pourra donc pas servir à changer le formatage d'autres unités lexicales.

\medskip
Dans les langages où le caractère \verb|$| ne joue pas un rôle syntaxique important, on peut assez naturellement
vouloir activer le mécanisme «|escape-math|» avec le caractère \verb|$|:
\begin{Verbatim}
\PitonOptions{~emphase#begin-escape-math=$,end-escape-math=$@}
\end{Verbatim}
Remarquer que le caractère \verb|$| ne doit \emph{pas} être protégé par une contre-oblique.

\bigskip
Néanmoins, il est sans doute plus prudent d'utiliser |\(| et |\)|.
\begin{Verbatim}
\PitonOptions{~emphase#begin-escape-math=\(,end-escape-math=\)@}
\end{Verbatim}

\bigskip
Voici un exemple d'utilisation typique :

\medskip
\begin{Verbatim}
\begin{Piton}[line-numbers]
def arctan(x,n=10):
    if ~emphase#\(x < 0\)@ :
        return ~emphase#\(-\arctan(-x)\)@ 
    elif ~emphase#\(x > 1\)@ : 
        return ~emphase#\(\pi/2 - \arctan(1/x)\)@ 
    else: 
        s = ~emphase#\(0\)@
        for ~emphase#\(k\)@ in range(~emphase#\(n\)@): s += ~emphase#\(\smash{\frac{(-1)^k}{2k+1} x^{2k+1}}\)@
        return s
\end{Piton}
\end{Verbatim}


\bigskip

\begin{Piton}[line-numbers]
def arctan(x,n=10):
    if \(x < 0\) :
        return \(-\arctan(-x)\) 
    elif \(x > 1\) : 
        return \(\pi/2 - \arctan(1/x)\) 
    else: 
        s = \(0\)
        for \(k\) in range(\(n\)): s += \(\smash{\frac{(-1)^k}{2k+1} x^{2k+1}}\)
        return s
\end{Piton}



\subsection{Comportement dans la classe Beamer}

\label{beamer}

\emph{Première remarque}\par\nobreak
Remarquons que, comme l'environnement |{Piton}| prend son argument selon un mode verbatim, il convient, ce qui
n'est pas surprenant, de l'utiliser dans des environnements |{frame}| de Beamer protégés par la clé |fragile|,
c'est-à-dire débutant par |\begin{frame}[fragile]|.\footnote{On rappelle que pour un environnement |{frame}| de Beamer qui
  utilise la clé |fragile|, l'instruction |\end{frame}| doit être seule sur une ligne (à l'exception d'éventuels
  espaces en début de ligne).}


\medskip
Quand l'extension \pkg{piton} est utilisée dans la classe \cls{beamer}\footnote{L'extension \pkg{piton} détecte la
  classe \cls{beamer} et l'extension \pkg{beamerarticle} si elle est chargée précédemment, mais il est aussi
  possible, si le besoin s'en faisait sentir, d'activer ce comportement avec la clé |beamer| au chargement de
  \pkg{piton} : |\usepackage[beamer]{piton}|}, le comportement de \pkg{piton} est légèrement modifié, comme décrit
maintenant.

\subsubsection{\{Piton\} et \textbackslash PitonInputFile sont ``overlay-aware''}

Quand \pkg{piton} est utilisé avec Beamer, l'environnement |{Piton}| et la commande |\PitonInputFile| acceptent
l'argument optionnel |<...>| de Beamer pour indiquer les «\emph{overlays}» concernés.

On peut par exemple écrire :

\begin{Verbatim}
\begin{Piton}~emphase#<2-5>@
...
\end{Piton}
\end{Verbatim}

ou aussi 

\begin{Verbatim}
\PitonInputFile~emphase#<2-5>@{mon_fichier.py}
\end{Verbatim}


\subsubsection{Commandes de Beamer reconnues dans \{Piton\} et \textbackslash PitonInputFile}

Quand \pkg{piton} est utilisé dans la classe \cls{beamer}, les commandes suivantes de \cls{beamer} (classées selon
leur nombre d'arguments obligatoires) sont directement reconnues dans les environnements |{Piton}| (ainsi que dans
les listings composés par la commande |\PitonInputFile|, même si c'est sans doute moins utile).
%
\begin{itemize}
\item aucun argument obligatoire : |\pause|\footnote{On remarquera que, bien sûr, on peut aussi utiliser |\pause|
  dans un «commentaire LaTeX», c'est-à-dire en écrivant |#> \pause|. Ainsi, si le code Python est copié, il est
  interprétable par Python.} ;
\item un argument obligatoire : |\action|, |\alert|, |\invisible|, |\only|, |\uncover| et |\visible| ;
\item deux arguments obligatoires : |\alt| ; 
\item trois arguments obligatoires : |\temporal|.
\end{itemize}

\medskip
Les accolades dans les arguments obligatoires de ces commandes doivent être équilibrées (cependant, les accolades
présentes dans des chaînes courtes\footnote{Les chaînes courtes de Python sont les chaînes (string) délimitées par
  les caractères \texttt{'} ou \texttt{"} non triplés. En Python, les chaînes de caractères courtes ne peuvent pas
  s'étendre sur plusieurs lignes de code.} de Python ne sont pas prises en compte).

\medskip
Concernant les fonctions |\alt| et |\temporal|, aucun retour à la ligne ne doit se trouver dans les arguments de ces fonctions.


\medskip
Voici un exemple complet de fichier :

\begin{Verbatim}[formatcom = \small\color{gray}]
\documentclass{beamer}
\usepackage{piton}
\begin{document}
\begin{frame}[fragile]
\begin{Piton}
def string_of_list(l):
    """Convertit une liste de nombres en chaîne"""
~emphase#    \only<2->{s = "{" + str(l[0])}@
~emphase#    \only<3->{for x in l[1:]: s = s + "," + str(x)}@
~emphase#    \only<4->{s = s + "}"}@
    return s
\end{Piton}
\end{frame}
\end{document}
\end{Verbatim}

Dans l'exemple précédent, les accolades des deux chaînes de caractères Python |"{"| et |"}"| sont correctement
interprétées (sans aucun caractère d'échappement).



\bigskip
\subsubsection{Environnements de Beamer reconnus dans \{Piton\} et \textbackslash PitonInputFile}

Quand \pkg{piton} est utilisé dans la classe \pkg{beamer}, les environnements suivants de Beamer sont directement
reconnus dans les environnements |{Piton}| (ainsi que dans les listings composés par la commande |\PitonInputFile|
même si c'est sans doute moins utile) : |{actionenv}|, |{alertenv}|, |{invisibleenv}|, |{onlyenv}|, |{uncoverenv}|
et |{visibleenv}|.


\medskip
Il y a néanmoins une restriction : ces environnements doivent englober des \emph{lignes entières de code Python}.


\medskip
On peut par exemple écrire :

\begin{Verbatim}[formatcom = \small\color{gray}]
\documentclass{beamer}
\usepackage{piton}
\begin{document}
\begin{frame}[fragile]
\begin{Piton}
def carré(x):
    """Calcule le carré de l'argument"""
    ~emphase#\begin{uncoverenv}<2>@
    return x*x
    ~emphase#\end{uncoverenv}@
\end{Piton}
\end{frame}
\end{document}
\end{Verbatim}



\vspace{1cm}
\textbf{Remarque à propos de la commande \textbackslash alert et de l'environnement \{alertenv\} de Beamer}\par\nobreak

\smallskip
Beamer propose un moyen aisé de changer la couleur utilisée par l'environnement |{alertenv}| (et par la commande
|\alert| qui s'appuie dessus). Par exemple, on peut écrire:

\begin{Verbatim}
\setbeamercolor{~emphase#alerted text@}{fg=blue} 
\end{Verbatim}

Néanmoins, dans le cas d'une utilisation à l'intérieur d'un environnement |{Piton}| un tel réglage n'est sans doute
pas pertinent, puisque, justement, \pkg{piton} va (le plus souvent) changer la couleur des élements selon leur
valeur lexicale. On préfèrera sans doute un environnement |{alertenv}| qui change la couleur de fond des
éléments à mettre en évidence.

\smallskip
Voici un code qui effectuera ce travail en mettant un fond jaune. Ce code utilise la commande |\@highLight| de
l'extension \pkg{lua-ul} (cette extension nécessite elle-même l'extension \pkg{luacolor}).

\begingroup
\fvset{commandchars=\~\#\+,formatcom=\color{gray}}
\begin{Verbatim}
\setbeamercolor{alerted text}{bg=yellow!50}
\makeatletter
\AddToHook{env/Piton/begin}
  {\renewenvironment<>{alertenv}{\only~#1{~emphase#\@highLight+[alerted text.bg]}}{}}
\makeatother
\end{Verbatim}
\endgroup

Ce code redéfinit localement l'environnement |{alertenv}| à l'intérieur de l'environnement |{Piton}| (on rappelle
que la commande |\alert| s'appuie sur cet environnement |{alertenv}|). 



\bigskip


\bigskip
\subsection{Notes de pied de page dans les environnements de piton}

\label{footnote}

Si vous voulez mettre des notes de pied de page dans un environnement de \pkg{piton} (ou bien dans un listing
produit par |\PitonInputFile|, bien que cela paraisse moins pertinent dans ce cas-là) vous pouvez utiliser une
paire |\footnotemark|--|\footnotetext|. 

\smallskip
Néanmoins, il est également possible d'extraire les notes de pieds de page avec l'extension \pkg{footnote} ou bien
l'extension \pkg{footnotehyper}.

\smallskip
Si \pkg{piton} est chargée avec l'option |footnote| (avec |\usepackage[footnote]{piton}|) l'extension
\pkg{footnote} est chargée (si elle ne l'est pas déjà) et elle est utilisée pour extraire les notes de pied de
page.

\smallskip
Si \pkg{piton} est chargée avec l'option |footnotehyper|, l'extension \pkg{footnotehyper} est chargée (si
elle ne l'est pas déjà) et elle est utilisée pour extraire les notes de pied de page.

\smallskip
Attention : Les extensions \pkg{footnote} et \pkg{footnotehyper} sont incompatibles.
L'extension \pkg{footnotehyper} est le successeur de l'extension \pkg{footnote}
et devrait être utilisée préférentiellement. L'extension \pkg{footnote} a quelques défauts ;
en particulier, elle doit être chargée après l'extension \pkg{xcolor} et elle n'est pas parfaitement
compatible avec \pkg{hyperref}.

\medskip
Dans ce document, l'extension \pkg{piton} a été chargée avec l'option |footnotehyper| et c'est pourquoi des notes
peuvent être mises dans les environnements |{Piton}| : voir un exemple sur la première page de ce document. 

\subsection{Tabulations}

\smallskip
Même s'il est recommandé d'indenter les listings Python avec des espaces (cf. PEP~8), \pkg{piton} accepte les
caractères de tabulations (U+0009) en début de ligne. Chaque caractère U+0009 est remplacé par $n$ espaces. La
valeur initiale de~$n$ est~4 mais on peut la changer avec la clé |tab-size| de |\PitonOptions|.

\smallskip
Il existe aussi une clé |tabs-auto-gobble| qui détermine le nombre minimal de caractères U+0009 débutant chaque
ligne (non vide) de l'environnement |{Piton}| et applique |gobble| avec cette valeur (avant le remplacement des
caractères U+0009 par des espaces, bien entendu). Cette clé est donc similaire à la clé |auto-gobble| mais agit sur
des caractères U+0009 au lieu de caractères U+0020 (espaces).

\section{Exemples}


\subsection{Numérotation des lignes}

\label{example-numbering}

On rappelle que l'on peut demander la numérotation des lignes des listings avec la clé |line-numbers|.

Par défaut, les numéros de ligne sont composés par \pkg{piton} en débordement à gauche (en utilisant en interne la commande |\llap| de LaTeX).

Si on ne veut pas de débordement, on peut utiliser l'option |left-margin=auto| qui va insérer une marge adaptée aux
numéros qui seront insérés (elle est plus large quand les numéros dépassent 10).


\begingroup
\fvset{commandchars=\~\&\@,formatcom=\small\color{gray}}
\begin{Verbatim}
~emphase&\PitonOptions{background-color=gray!10, left-margin = auto, line-numbers}@
\begin{Piton}
def arctan(x,n=10):
    if x < 0:
        return -arctan(-x)        #> (appel récursif)
    elif x > 1: 
        return pi/2 - arctan(1/x) #> (autre appel récursif) 
    else: 
        return sum( (-1)**k/(2*k+1)*x**(2*k+1) for k in range(n) ) 
\end{Piton}
\end{Verbatim}
\endgroup



\begingroup
\PitonOptions{background-color=gray!10,left-margin = auto, line-numbers}
\begin{Piton}
def arctan(x,n=10):
    if x < 0:
        return -arctan(-x)        #> (appel récursif)
    elif x > 1: 
        return pi/2 - arctan(1/x) #> (autre appel récursif) 
    else: 
        return sum( (-1)**k/(2*k+1)*x**(2*k+1) for k in range(n) ) 
\end{Piton}
\endgroup


\bigskip

\subsection{Formatage des commentaires LaTeX}

\label{example-comments}

On peut modifier le style |Comment.LaTeX| (avec |\SetPitonStyle|) pour faire afficher les commentaires 
LaTeX (qui débutent par |#>|) en butée à droite.

\begingroup
\fvset{commandchars=\~\&\@,formatcom=\small\color{gray}}
\begin{Verbatim}
\PitonOptions{background-color=gray!10}
~emphase&\SetPitonStyle{Comment.LaTeX = \hfill \normalfont\color{gray}}@
\begin{Piton}
def arctan(x,n=10):
    if x < 0:
        return -arctan(-x)        #> appel récursif
    elif x > 1: 
        return pi/2 - arctan(1/x) #> autre appel récursif 
    else: 
        return sum( (-1)**k/(2*k+1)*x**(2*k+1) for k in range(n) ) 
\end{Piton}
\end{Verbatim}
\endgroup

\begingroup
\PitonOptions{background-color=gray!10}
\SetPitonStyle{Comment.LaTeX = \hfill \normalfont\color{gray}}
\begin{Piton}
def arctan(x,n=10):
    if x < 0:
        return -arctan(-x)        #> appel récursif
    elif x > 1: 
        return pi/2 - arctan(1/x) #> autre appel récursif 
    else: 
        return sum( (-1)**k/(2*k+1)*x**(2*k+1) for k in range(n) ) 
\end{Piton}
\endgroup


\vspace{1cm}
On peut aussi faire afficher les commentaires dans une deuxième colonne à droite si on limite la largeur du code
proprement dit avec la clé |width|. Dans l'exemple qui suit, on utilise la clé |width| avec la valeur
spéciale~|min|. 


\begingroup
\fvset{commandchars=\~\&\@,formatcom=\small\color{gray}}
\begin{Verbatim}
\PitonOptions{width=min, background-color=gray!10}
~emphase&\NewDocumentCommand{\MyLaTeXCommand}{m}{\hfill \normalfont\itshape\rlap{\quad #1}}@
~emphase&\SetPitonStyle{Comment.LaTeX = \MyLaTeXCommand}@
\begin{Piton}
def arctan(x,n=10):
    if x < 0:
        return -arctan(-x)       #> appel récursif
    elif x > 1: 
        return pi/2 - arctan(1/x) #> autre appel récursif 
    else: 
        s = 0
        for k in range(n):
             s += (-1)**k/(2*k+1)*x**(2*k+1)
        return s 
\end{Piton}
\end{Verbatim}
\endgroup



\begingroup
\PitonOptions{width = min, background-color=gray!10}
\NewDocumentCommand{\MyLaTeXCommand}{m}{\hfill \normalfont\itshape\rlap{\quad #1}}
\SetPitonStyle{Comment.LaTeX = \MyLaTeXCommand}
\begin{Piton}
def arctan(x,n=10):
    if x < 0:
        return -arctan(-x)       #> appel récursif
    elif x > 1: 
        return pi/2 - arctan(1/x) #> autre appel récursif 
    else: 
        s = 0
        for k in range(n):
             s += (-1)**k/(2*k+1)*x**(2*k+1)
        return s 
\end{Piton}
\endgroup

\bigskip

\subsection{Notes dans les listings}

Pour pouvoir extraire les notes (introduites par |\footnote|), l'extension |piton| doit être chargée, soit avec la
clé |footnote|, soit avec la clé |footnotehyper|, comme expliqué à la section \ref{footnote} p.~\pageref{footnote}.
Dans le présent document, l'extension \pkg{piton} a été chargée par la clé |footnotehyper|.

Bien entendu, une commande |\footnote| ne peut apparaître que dans un commentaire LaTeX (qui débute par |#>|). Un
tel commentaire peut se limiter à cette unique commande |\footnote|, comme dans l'exemple suivant.

\begingroup
\fvset{commandchars=\~\&\@,formatcom=\small\color{gray}}
\begin{Verbatim}
\PitonOptions{background-color=gray!10}
\begin{Piton}
def arctan(x,n=10):
    if x < 0:
        return -arctan(-x)~emphase&#>\footnote{Un premier appel récursif.}]@
    elif x > 1: 
        return pi/2 - arctan(1/x)~emphase&#>\footnote{Un deuxième appel récursif.}@
    else: 
        return sum( (-1)**k/(2*k+1)*x**(2*k+1) for k in range(n) ) 
\end{Piton}
\end{Verbatim}
\endgroup

\begingroup
\PitonOptions{background-color=gray!10}
\begin{Piton}
def arctan(x,n=10):
    if x < 0:
        return -arctan(-x)#>\footnote{Un premier appel récursif.}
    elif x > 1: 
        return pi/2 - arctan(1/x)#>\footnote{Un deuxième appel récursif.}
    else: 
        return sum( (-1)**k/(2*k+1)*x**(2*k+1) for k in range(n) ) 
\end{Piton}
\endgroup


\vspace{1cm}

Si on utilise l'environnement |{Piton}| dans un environnement |{minipage}| de LaTeX, les notes sont, bien entendu,
composées au bas de l'environnement |{minipage}|. Rappelons qu'une telle |{minipage}| ne peut être coupée par un
saut de page. 


\begingroup
\fvset{commandchars=\~\&\@,formatcom=\small\color{gray}}
\begin{Verbatim}
\PitonOptions{background-color=gray!10}
\emphase\begin{minipage}{\linewidth}
\begin{Piton}
def arctan(x,n=10):
    if x < 0:
        return -arctan(-x)~emphase&#>\footnote{Un premier appel récursif.}@
    elif x > 1: 
        return pi/2 - arctan(1/x)~emphase&#>\footnote{Un deuxième appel récursif.}@
    else: 
        return sum( (-1)**k/(2*k+1)*x**(2*k+1) for k in range(n) ) 
\end{Piton}
\end{minipage}
\end{Verbatim}
\endgroup

\begingroup
\PitonOptions{background-color=gray!10}
\begin{minipage}{\linewidth}
\begin{Piton}
def arctan(x,n=10):
    if x < 0:
        return -arctan(-x)#>\footnote{Un premier appel récursif.}
    elif x > 1: 
        return pi/2 - arctan(1/x)#>\footnote{Un deuxième appel récursif.}
    else: 
        return sum( (-1)**k/(2*k+1)*x**(2*k+1) for k in range(n) ) 
\end{Piton}
\end{minipage}
\endgroup


\bigskip

\subsection{Un exemple de réglage des styles}

Les styles graphiques ont été présentés à la partie \ref{styles}, p.~\pageref{styles}.

\smallskip
On présente ici un réglage de ces styles adapté pour les documents en noir et blanc. On l'utilise avec la fonte
\emph{DejaVu Sans Mono}\footnote{Voir: \url{https://dejavu-fonts.github.io}} spécifiée avec la commande
  |\setmonofont| de \pkg{fontspec}.

Ce réglage utilise la commande |\highLight| de \pkg{lua-ul} (cette extension nécessite elle-même l'extension \pkg{luacolor}).


\begin{Verbatim}
\setmonofont[Scale=0.85]{DejaVu Sans Mono}

\SetPitonStyle
  {
    Number = ,
    String = \itshape , 
    String.Doc = \color{gray} \itshape ,
    Operator = , 
    Operator.Word = \bfseries ,
    Name.Builtin = ,
    Name.Function = \bfseries \highLight[gray!20] ,
    Comment = \color{gray} ,
    Comment.LaTeX = \normalfont \color{gray},
    Keyword = \bfseries ,
    Name.Namespace = ,
    Name.Class = , 
    Name.Type = ,
    InitialValues = \color{gray}
  }
\end{Verbatim}


\begingroup


\setmonofont[Scale=0.85]{DejaVu Sans Mono}

\SetPitonStyle
  {
    Number = ,
    String = \itshape , 
    String.Doc = \color{gray} \itshape ,
    Operator = , 
    Operator.Word = \bfseries ,
    Name.Builtin = ,
    Name.Function = \bfseries \highLight[gray!20] ,
    Comment = \color{gray} ,
    Comment.LaTeX = \normalfont \color{gray},
    Keyword = \bfseries ,
    Name.Namespace = ,
    Name.Class = , 
    Name.Type = ,
    InitialValues = \color{gray}
  }


\bigskip 

\begin{Piton}
from math import pi

def arctan(x,n=10):
    """Compute the mathematical value of arctan(x)

    n is the number of terms in the sum
    """
    if x < 0:
        return -arctan(-x) # appel récursif
    elif x > 1: 
        return pi/2 - arctan(1/x) 
        #> (on a utilisé le fait que $\arctan(x)+\arctan(1/x)=\pi/2$ pour $x>0$)
    else: 
        s = 0
        for k in range(n):
            s += (-1)**k/(2*k+1)*x**(2*k+1)
        return s 
\end{Piton}


\endgroup

\section{Utilisation avec pyluatex}

\subsection{Utilisation standard de pyluatex}

L'extension \pkg{pyluatex} est une extension qui permet l'exécution de code Python à partir de |lualatex| (pourvu
que Python soit installé sur la machine et que la compilation soit effectuée avec |lualatex| et |--shell-escape|).

Voici, à titre d'exemple, un environnement |{PitonExecute}| qui formate un listing Python (avec \pkg{piton}) et qui
affiche également dessous le résultat de l'exécution de ce code avec Python.


\begin{Verbatim}
\ExplSyntaxOn
\NewDocumentEnvironment { ~emphase#PitonExecute@ } { ! O { } }  % le ! est obligatoire 
  { 
    \PyLTVerbatimEnv
    \begin{pythonq} 
  }
  {
    \end{pythonq}
    \directlua
      {
        tex.print("\\PitonOptions{~#1}")
        tex.print("\\begin{Piton}")
        tex.print(pyluatex.get_last_code())
        tex.print("\\end{Piton}")
        tex.print("") 
      }
    \begin{center}
      \directlua{tex.print(pyluatex.get_last_output())}
    \end{center}
  }
\ExplSyntaxOff
\end{Verbatim}

\ExplSyntaxOn
\NewDocumentEnvironment { PitonExecute } { ! O { } } 
  { 
    \PyLTVerbatimEnv
    \begin{pythonq} 
  }
  {
    \end{pythonq}
    \directlua
      {
        tex.print("\\PitonOptions{#1}")
        tex.print("\\begin{Piton}")
        tex.print(pyluatex.get_last_code())
        tex.print("\\end{Piton}")
        tex.print("") 
      }
    \begin{center}
      \directlua{tex.print(pyluatex.get_last_output())}
    \end{center}
  }
\ExplSyntaxOff

Cet environnement |{PitonExecute}| prend en argument optionnel (entre crochets) les options proposées par la
commande |\PitonOptions|.

\bigskip
Voici un exemple d'utilisation de cet environnement |{PitonExecute}| :

\begin{Verbatim}
\begin{~emphase#PitonExecute@}[background-color=gray!15]
def square(x):
    return x*x
print(f'Le carré de 12 est {square(12)}.')
\end{~emphase#PitonExecute@}
\end{Verbatim}


\medskip
\begin{PitonExecute}[background-color=gray!15]
def square(x):
    return x*x
print(f'Le carré de 12 est {square(12)}.')
\end{PitonExecute}

\bigskip

\subsection{Utilisation de l'environnement \{pythonrepl\} de pyluatex}

\label{pythonrepl}

L'environnement |{pythonrepl}| de \pkg{pyluatex} passe son contenu à Python et renvoie ce que l'on obtient quand on
fournit ce code à une boucle \textsc{repl} (\emph{read-eval-print loop}) de Python. On obtient un entrelacement
d'instructions précédées par le prompt |>>>| de Python et des valeurs renvoyées par Python (et de ce qui a été
demandé d'être affiché avec des \piton{print} de Python).

\medskip
Il est ensuite possible de passer cela à un environnement |{Piton}| qui va faire un coloriage syntaxique 
et mettre sur fond grisé les lignes correspondant aux instructions fournies à l'interpréteur Python (gràce à la clé
|prompt-background-color| de |\PitonOptions|).


\medskip
Voici la programmation d'un environnement |{PitonREPL}| qui effectue ce travail (pour des raisons techniques, le
|!| est ici obligatoire dans la signature de l'environnement).

\begin{Verbatim}
\NewDocumentEnvironment { PitonREPL } { ! O { } } % le ! est obligatoire
 { 
   \PitonOptions
     { 
       background-color=white,
       ~emphase#prompt-background-color = gray!10@,
       ~#1
     }
   \PyLTVerbatimEnv
   \begin{pythonrepl}
 }
 {
   \directlua
     {
       tex.print("\\begin{Piton}")
       tex.print(pyluatex.get_last_output())
       tex.print("\\end{Piton}")
       tex.print("") 
     }
   \end{pythonrepl}
 }
\end{Verbatim}

\medskip
Voici un exemple d'utilisation de ce nouvel environnement |{PitonREPL}|.

\medskip
\begin{Verbatim}
~emphase#\begin{PitonREPL}@
    def valeur_absolue(x):
        "Renvoie la valeur absolue de x"
        if x > 0:
           return x
        else:
           return -x

    valeur_absolue(-3)
    valeur_absolue(0)
    valeur_absolue(5)
~emphase#\end{PitonREPL}@
\end{Verbatim}

\bigskip


\NewDocumentEnvironment { PitonREPL } { }
 { 
   \PitonOptions{background-color=white,prompt-background-color = gray!10}
   \PyLTVerbatimEnv
   \begin{pythonrepl}
 }
 {
   \directlua
     {
       tex.print("\\begin{Piton}")
       tex.print(pyluatex.get_last_output())
       tex.print("\\end{Piton}")
       tex.print("") 
     }
   \end{pythonrepl}
 }


\begin{PitonREPL}
    def valeur_absolue(x):
        "Renvoie la valeur absolue de x"
        if x > 0:
           return x
        else:
           return -x

    valeur_absolue(-3)
    valeur_absolue(0)
    valeur_absolue(5)
\end{PitonREPL}


\bigskip
En fait, il est possible de ne pas faire afficher les prompts eux-mêmes (c'est-à-dire les chaînes de caractères
|>>>| et |...|). En effet, \pkg{piton} propose un style pour ces éléments, qui est appelé |Prompt|. Par défaut, la
valeur de ce style est vide, ce qui fait qu'aucune action n'est exécutée sur ces éléments qui sont donc affichés
tels quels. En fournissant comme valeur une fonction qui se contente de gober son argument, on peut demander à ce
qu'ils ne soient pas affichés.

\bigskip
\begin{savenotes}
\begin{Verbatim}
\NewDocumentCommand{\Gobe}{m}{}#~color#black@~footnote#On a défini ici une fonction ~texttt#~string~Gobe@ mais, en fait, elle existe déjà en L3 sous le nom ~texttt#~string~use~_none:n@.@@
\SetPitonStyle{ ~emphase#Prompt = \Gobe@ }
\end{Verbatim}
\end{savenotes}




\NewDocumentCommand{\Gobe}{m}{}
\ExplSyntaxOn
\SetPitonStyle{ Prompt = \use_none:n }
\ExplSyntaxOff

\bigskip
L'exemple précédent donne alors :

\medskip
\begin{Verbatim}
~emphase#\begin{PitonREPL}@
    def valeur_absolue(x):
        "Renvoie la valeur absolue de x"
        if x > 0:
           return x
        else:
           return -x

    valeur_absolue(-3)
    valeur_absolue(0)
    valeur_absolue(5)
~emphase#\end{PitonREPL}@
\end{Verbatim}



\begin{PitonREPL}
    def valeur_absolue(x):
        "Renvoie la valeur absolue de x"
        if x > 0:
           return x
        else:
           return -x

    valeur_absolue(-3)
    valeur_absolue(0)
    valeur_absolue(5)
\end{PitonREPL}


\newpage

\section{Les styles pour les différents langages informatiques}

\label{Semantic}


\subsection{Le langage Python}

Le langage par défaut de l'extension \pkg{piton} est Python. Si besoin est, on peut revenir au langage Python avec
|\PitonOptions{language=Python}|. 

\vspace{1cm}

\begin{center}
\begin{tabularx}{0.9\textwidth}{@{}>{\ttfamily}l>{\raggedright\arraybackslash}X@{}}
  \toprule
  \normalfont Style  & Usage \\
  \midrule
  Number & les nombres \\
  String.Short & les chaînes de caractères courtes (entre \texttt{'} ou \texttt{"}) \\
  String.Long & les chaînes de caractères longues (entre \texttt{'''} ou \texttt{"""}) sauf les chaînes de
                documentation (qui sont gérées par |String.Doc|)\\
  String & cette clé fixe à la fois |String.Short| et |String.Long| \\
  String.Doc & les chaînes de documentation (seulement entre |"""| suivant PEP~257) \\
  String.Interpol & les éléments syntaxiques des champs des f-strings (c'est-à-dire les caractères \texttt{\{} et
                    \texttt{\}}) ; ce style hérite des styles |String.Short| et |String.Long| (suivant la chaîne où
                    apparaît l'interpolation)\\
  Interpol.Inside & le contenu des interpolations dans les f-strings (c'est-à-dire les éléments qui se trouvent entre
                    \texttt{\{} et~\texttt{\}} ; si l'utilisateur n'a pas fixé ce style, ces éléments sont analysés et formatés par \pkg{piton}
                    au même titre que le reste du code. \\
                    Operator & les opérateurs suivants : \texttt{!= == << >> - \~{} + / * \% = < > \& .} \verb+|+ \verb|@| \\
  Operator.Word & les opérateurs suivants : |in|, |is|, |and|, |or| et |not| \\
  Name.Builtin & la plupart des fonctions prédéfinies par Python \\
  Name.Decorator & les décorateurs (instructions débutant par \verb|@|) \\
  Name.Namespace & le nom des modules (= bibliothèques extérieures) \\
  Name.Class & le nom des classes au moment de leur définition, c'est-à-dire après le mot-clé \verb|class| \\
  UserFunction & le nom des fonctions précédemment définies par l'utilisateur (la valeur initiale de ce paramètre est
                 vide et ces éléments sont affichés en noir --- ou plutôt dans la couleur courante) \\
  Exception & les exceptions prédéfinies (ex.: \texttt{SyntaxError}) \\
  InitialValues & les valeurs initiales (et le symbole |=| qui précède) des arguments optionnels dans les définitions
                  de fonctions ; si l'utilisateur n'a pas fixé ce style, ces éléments sont analysés et formatés par \pkg{piton}
                  au même titre que le reste du code.\\
  Comment & les commentaires commençant par \texttt{\#} \\
  Comment.LaTeX & les commentaires commençant par \texttt{\#>} qui sont composés par \pkg{piton} comme du code LaTeX (et
                  appelés simplement «commentaires LaTeX» dans ce document) \\
  Keyword.Constant & |True|, |False| et |None| \\
  Keyword & les mots-clés suivants :
            \ttfamily assert, break, case, continue, del,
            elif, else, except, exec, finally, for, from, 
            global, if, import, lambda, non local,
            pass, raise, return, try, while,
            with, yield et yield from.\\
  \bottomrule
\end{tabularx}
\end{center}


\newpage

\subsection{Le langage OCaml}

On peut basculer vers le langage |OCaml| avec |\PitonOptions{language = OCaml}|

\bigskip
On peut aussi choisir le langage |OCaml| pour un environnement |{Piton}| individuel :
%
\begin{Verbatim}
\begin{Piton}~emphase#[language=OCaml]@
...
\end{Piton}
\end{Verbatim}


\vspace{1cm}


\begin{center}
\begin{tabularx}{0.9\textwidth}{@{}>{\ttfamily}l>{\raggedright\arraybackslash}X@{}}
  \toprule
  \normalfont Style  & Usage \\
  \midrule
  Number & les nombres \\
  String.Short & les caractères (entre \texttt{'}) \\
  String.Long & les chaînes de caractères, entre |"| mais aussi les \emph{quoted-strings} \\
  String & cette clé fixe à la fois |String.Short| et |String.Long| \\
  Operator & les opérateurs, en particulier |+|, |-|, |/|, |*|, |@|, |!=|, |==|, |&&| \\
  Operator.Word & les opérateurs suivants : |and|, |asr|, |land|, |lor|, |lsl|, |lxor|, |mod| et |or| \\
  Name.Builtin & les fonctions |not|, |incr|, |decr|, |fst| et |snd| \\
  Name.Type & le nom des types OCaml \\
  Name.Field & le nom d'un champ de module \\
  Name.Constructor & le nom des constructeurs de types (qui débutent par une majuscule) \\
  Name.Module & le nom des modules \\
  Name.Function & le nom des fonctions définies par l'utilisateur \emph{au moment de leur définition} (après le
                  mot-clé |let|) \\
  UserFunction & le nom des fonctions précédemment définies par l'utilisateur 
                 (la valeur initiale de ce paramètre est
                 vide et ces éléments sont affichés en noir --- ou plutôt dans la couleur courante) \\
  Exception & les exceptions prédéfinies (ex. : |End_of_File|) \\
  TypeParameter & les paramétreurs de type \\
  Comment & les commentaires, entre |(*| et |*)| ; ces commentaires peuvent être imbriqués \\
  Keyword.Constant & |true| et |false| \\
  Keyword & les mots-clés suivants :
            |assert|, |as|, |begin|, |class|, |constraint|, |done|,
            |downto|, |do|, |else|, |end|, |exception|, |external|,
            |for|, |function|, |functor|, |fun| , |if| 
            |include|, |inherit|, |initializer|, |in| , |lazy|, |let|,
            |match|, |method|, |module|, |mutable|, |new|, |object|,
            |of|, |open|, |private|, |raise|, |rec|, |sig|,
            |struct|, |then|, |to|, |try|, |type|, 
            |value|, |val|, |virtual|, |when|, |while| et |with| \\
  \bottomrule
\end{tabularx}
\end{center}

\newpage

\subsection[Le langage C (et C++)]{Le langage C (et \CC)}


On peut basculer vers le langage |C| avec |\PitonOptions{language = C}|

\bigskip
On peut aussi choisir le langage |C| pour un environnement |{Piton}| individuel :
%
\begin{Verbatim}
\begin{Piton}~emphase#[language=C]@
...
\end{Piton}
\end{Verbatim}


\vspace{1cm}

\begin{center}
\begin{tabularx}{0.9\textwidth}{@{}>{\ttfamily}l>{\raggedright\arraybackslash}X@{}}
\toprule
\normalfont Style  & Usage \\
\midrule
Number & les nombres \\
String.Long & les chaînes de caractères (entre \texttt{"}) \\
String.Interpol &  les éléments \texttt{\%d}, \texttt{\%i}, \texttt{\%f}, \texttt{\%c}, etc. dans les chaînes de
                  caractères ; ce style hérite du style |String.Long| \\
Operator & les opérateurs suivants : \texttt{!= == << >> - \~{} + / * \% = < > \& .} \verb+|+ \verb|@| \\
Name.Type & les types prédéfinis suivants :
  |bool|, |char|, |char16_t|, |char32_t|, |double|, |float|, |int|, |int8_t|, |int16_t|, |int32_t|, 
  |int64_t|, |long|, |short|, |signed|, |unsigned|, |void| et |wchar_t| \\ 
Name.Builtin & les fonctions prédéfinies suivantes : |printf|, |scanf|, |malloc|, |sizeof| et  |alignof|  \\
  Name.Class & le nom des classes au moment de leur définition, c'est-à-dire après le mot-clé \verb|class| \\
Name.Function & le nom des fonctions définies par l'utilisateur \emph{au moment de leur définition} \\
UserFunction & le nom des fonctions précédemment définies par l'utilisateur (la valeur initiale de ce paramètre est
               vide et ces éléments sont affichés en noir --- ou plutôt dans la couleur courante) \\
Preproc & les instructions du préprocesseur (commençant par |#|) \\
Comment & les commentaires (commençant par \texttt{//} ou entre |/*| et |*/|) \\
Comment.LaTeX & les commentaires commençant par \texttt{//>} qui sont composés par \pkg{piton} comme du code LaTeX (et
                appelés simplement «commentaires LaTeX» dans ce document) \\
Keyword.Constant & |default|, |false|, |NULL|, |nullptr| et |true| \\
Keyword & les mots-clés suivants :
|alignas|, |asm|, |auto|, |break|, |case|, |catch|, |class|, 
|constexpr|, |const|, |continue|, |decltype|, |do|, |else|, |enum|,
|extern|, |for|, |goto|, |if|, |nexcept|, |private|, |public|, |register|, |restricted|, |try|, 
|return|, |static|, |static_assert|, |struct|, |switch|, |thread_local|, |throw|, 
|typedef|, |union|, |using|, |virtual|, |volatile| et |while|
\\
\bottomrule
\end{tabularx}
\end{center}

\newpage

\section*{Autre documentation}

Le document |piton.pdf| (fourni avec l'extension \pkg{piton}) contient une traduction anglaise de la
documentation ici présente, ainsi que le code source commenté et un historique des versions. 


\medskip
Les versions successives du fichier |piton.sty| fournies par
TeXLive sont disponibles sur le serveur \textsc{svn} de TeXLive :

\smallskip
{
\small
\nolinkurl{https://tug.org/svn/texlive/trunk/Master/texmf-dist/tex/lualatex/piton/piton.sty}
}

\medskip
Le développement de l'extension \pkg{piton} se fait sur le dépôt GitHub suivant :

\smallskip
\verb|https://github.com/fpantigny/piton|

\tableofcontents



\end{document}